
\documentclass[dvipsnames,table]{beamer}
%\documentclass{beamer}
%\usepackage{beamerthemesplit} 
%\usetheme{Berkeley}
%\usecolortheme{dolphin}
\usetheme{Szeged}

\usepackage{amsfonts}
\usepackage[spanish]{babel}
\usepackage[latin1]{inputenc}
%\usepackage[utf8]{inputenc}
%\usepackage[dvips]{graphicx}
\usepackage{cancel}
%\usepackage{bm}
\usepackage{ae,aecompl,amsmath,amsbsy}

\beamertemplateballitem

\usepackage{tikz}
\usepackage{pbox}
%\usepackage{subfigure}
\usepackage{subcaption}

\newtheorem{mydef}{Definici�n}
\newcommand{\peq}[1]{{\scriptscriptstyle{#1}}} 
\newcommand{\rp}[1]{\left(#1\right)}

\usetikzlibrary{babel,decorations.pathreplacing,decorations.markings}

\title{EAE105A \\ Introducci�n a la Econom�a}
\subtitle{II. Microeconom�a: Teor�a de la Oferta} 
\author{Pinjas Albagli}
\institute{Instituto de Econom�a \\ Pontificia Universidad Cat�lica de Chile}
\date{Segundo Semestre de 2017}

\begin{document}

		\begin{frame}
			\frametitle{Elasticidad unitaria y oferta lineal}
			Recordar que 
			\begin{align*}
				\frac{\Delta\%q^\peq{s}}{\Delta\%p}&=\frac{\Delta q^\peq{s}/q^\peq{s}}{\Delta p/p} \\
																					 &=\frac{1}{\Delta p / \Delta q^\peq{s}}\cdot\frac{p}{q^\peq{s}}								
			\end{align*}
		\end{frame}

		\begin{frame}
			\frametitle{Elasticidad unitaria y oferta lineal}
			Notar que
			\begin{itemize}
				\item Si la oferta es lineal, su pendiente ($\frac{\Delta p}{\Delta q^\peq{s}}$) es constante.
				\item Si adem�s pasa por el origen ($a=0$ en $p=a+b\cdot q^\peq{s}$), la raz�n $\frac{p}{q^\peq{s}}$ es constante: $$\frac{p}{q^\peq{s}}=\frac{b\cdot q^\peq{s}}{q^\peq{s}}=b$$
				\item Luego, en este caso, la elasticidad precio de la oferta es constante (e igual a 1).
			\end{itemize}
		\end{frame}

		\begin{frame}
			\frametitle{Elasticidad unitaria y oferta lineal}
			Esto significa que lo que dije en clase no es precisamente correcto. Deb� decir que
			\begin{itemize}
				\item Si la oferta es lineal \textbf{y no pasa por el origen}, la elasticidad no puede ser constante.
			\end{itemize}
			Por lo tanto, no hay ninguna inconsistencia entre esta afirmaci�n y la relaci�n entre el intercepto y la elasticidad precio de la oferta lineal que les mostr� en clase.
		\end{frame}

		\begin{frame}
			\frametitle{Excedente del productor y ganancia}
			Para ordenar lo que discutimos al final de la clase:
			\begin{itemize}
				\item Si el precio sube de $p_\peq{0}$ a $p_\peq{1}$, el cambio en la ganancia de la empresa est� dado por $$
					\begin{array}{ll}
					\Delta \pi &= \pi\rp{q^\peq{s}_\peq{1}} - \pi\rp{q^\peq{s}_\peq{0}} \\
										 &= IT\rp{q^\peq{s}_\peq{1}}-CV\rp{q^\peq{s}_\peq{1}}-CF-\rp{IT\rp{q^\peq{s}_\peq{0}}-CV\rp{q^\peq{s}_\peq{0}}-CF}\\
										 &= IT\rp{q^\peq{s}_\peq{1}}-CV\rp{q^\peq{s}_\peq{1}} - \rp{IT\rp{q^\peq{s}_\peq{0}}-CV\rp{q^\peq{s}_\peq{0}}} \\
										 &= EP\rp{q^\peq{s}_\peq{1}}-EP\rp{q^\peq{s}_\peq{0}}\\
										 &= \Delta EP
					\end{array}$$
			\end{itemize}
		\end{frame}

\end{document}