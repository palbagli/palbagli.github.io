
\documentclass[dvipsnames,table,leqno]{beamer}
%\documentclass{beamer}
%\usepackage{beamerthemesplit} 
%\usetheme{Berkeley}
%\usecolortheme{dolphin}
\usetheme{Szeged}

\usepackage{amsfonts}
\usepackage{txfonts}
\usepackage[spanish]{babel}
%\usepackage[latin1]{inputenc}
\usepackage[utf8]{inputenc}
%\usepackage[dvips]{graphicx}
\usepackage{cancel}
%\usepackage{bm}
\usepackage{ae,aecompl,amsmath,amsbsy}
\setbeamertemplate{navigation symbols}{}

\beamertemplateballitem

\usepackage{tikz}
\usepackage{pbox}
%\usepackage{subfigure}
\usepackage{subcaption}
\usepackage{centernot}
\usepackage{etoolbox}
\usepackage{pgfplots}
\AtBeginEnvironment{align}{\setcounter{equation}{0}}

\usetikzlibrary{babel,decorations.pathreplacing,decorations.markings}
\decimalpoint

\newtheorem{mydef}{Definición}
\newcommand{\peq}[1]{{\scriptscriptstyle{#1}}} 
\newcommand{\rp}[1]{\left(#1\right)}
\newcommand{\sqp}[1]{\left[#1\right]}

\title{EAE105A \\ Introducción a la Economía}
\subtitle{II. Microeconomía: Fallas de Mercado} 
\author{Pinjas Albagli}
\institute{Instituto de Economía \\ Pontificia Universidad Católica de Chile}
\date{Primer Semestre de 2018}

\begin{document}

	\maketitle 
	
		\begin{frame}
			\frametitle{Fallas de Mercado}
			En esta sección estudiaremos situaciones en las que la asignación de equilibrio del mercado no es eficiente y cómo la política económica puede producir una asignación eficiente.
		\end{frame}

		\begin{frame}
			\frametitle{Fallas de Mercado}
			\begin{mydef}
				\textbf{Falla de mercado:} Situación en la cual el mercado, por sí solo, no asigna los recursos eficientemente.
					\begin{itemize}
						\item \textbf{Poder de mercado:} Capacidad que tiene un solo actor económico (o un grupo pequeño de actores económicos) de ejercer influencia considerable en los precios del mercado.
						\item \textbf{Externalidad:} Impacto de las acciones de una persona sobre el bienestar de otras.
					\end{itemize}
			\end{mydef}
		\end{frame}
	
	\section{Poder de Mercado}

		\begin{frame}
			\frametitle{Poder de Mercado}
			\centering
				\begin{tikzpicture}[scale=1.5,font=\footnotesize]
					\tikzset{
					solid node/.style={rectangle,draw,inner sep=1.5,fill=teal,opacity=.8},
					hollow node/.style={rectangle,draw,inner sep=1.5}
					}
					\tikzstyle{level 1}=[level distance=30mm,sibling distance=25mm]
					\tikzstyle{level 2}=[level distance=10mm,sibling distance=15mm]
					\node(0)[solid node]{\textbf{Cantidad de empresas}}
					child{node(1)[hollow node]{\pbox{\textwidth}{Monopolio}}
					edge from parent node[left,xshift=-3]{\pbox{\textwidth}{Una\\empresa}}
					}
					child{node(2)[hollow node]{\pbox{\textwidth}{Oligopolio}}
					edge from parent node[left]{\pbox{\textwidth}{Pocas\\empresas}}
					}
					child{node(3)[solid node,yshift=1.35cm]{\textbf{Grado de diferenciación}}
					child{node[hollow node]{\pbox{\textwidth}{Competencia\\monopolística}} edge from parent node[left,xshift=-.2cm]{\pbox{\textwidth}{Productos\\diferenciados}}}
					child{node[hollow node]{\pbox{\textwidth}{Competencia\\perfecta}} edge from parent node[right,xshift=.2cm]{\pbox{\textwidth}{Productos\\idénticos}}}
					edge from parent node[right,xshift=-.3cm,yshift=.5cm]{\pbox{\textwidth}{Muchas\\empresas}}
					};
			\end{tikzpicture}
		\end{frame}

		\begin{frame}
			\frametitle{Monopolio}
			\begin{mydef}
				\textbf{Monopolio:} Empresa que es la única vendedora de un producto que no tiene sustitutos cercanos.
			\end{mydef}
		\end{frame}

		\begin{frame}
			\frametitle{Monopolio}
			Causa principal: barreras a la entrada. Pueden producirse, por ejemplo, por...
			\begin{itemize}
				\item Un recurso clave para la producción es propiedad de una sola empresa.
					\begin{itemize}
						\item Ejemplo: diamantes.
					\end{itemize}
				\item Gobierno concede a una sola empresa el derecho exclusivo de vender un bien.
					\begin{itemize}
						\item Ej: patentes y derechos de propiedad intelectual.
					\end{itemize}
					\item Características del proceso productivo: economías de escala.
						\begin{itemize}
							\item Ejemplo: distribución de agua potable.
						\end{itemize}
			\end{itemize}
		\end{frame}

		\begin{frame}
			\frametitle{Monopolio}
			\begin{mydef}
				\textbf{Monopolio natural:} Monopolio que surge cuando una sola empresa ofrece un bien o servicio al mercado completo a un costo menor del que tendrían varias empresas.
			\end{mydef}
			
			\vspace{.1cm}
			Esto ocurre cuando hay economías de escala en el rango relevante de producción.
		\end{frame}

		\begin{frame}
			\frametitle{Monopolio}
			\centering
			\begin{tikzpicture}[scale=0.05]
				\draw [ultra thick,teal] plot [domain=9.5:100](\x,{1000/\x});
				\draw [dashed,help lines] (50,0) -- (50,20) -- (0,20);
				\draw [dashed,help lines] (0,100) -- (10,100) -- (10,0);
				\draw [<->,ultra thick] (0,110) -- (0,0) -- (130,0);
				\node [below] at (130,0) {$q$};
				\node [left] at (0,110) {$p$};
				\node [right] at (100,10) {$CMeT\rp{q}$};
				\node [below] at (50,0) {$q_\peq{1}$};
				\node [below] at (10,0) {$q_\peq{0}$};
				\node [left] at (0,20) {$CMeT\left(q_\peq{1}\right)$};
				\node [left] at (0,100) {$CMeT\left(q_\peq{0}\right)$};
				\node [below] at (0,0) {\tiny 0};
			\end{tikzpicture}
		\end{frame}

		\begin{frame}
			\frametitle{Monopolio}
			¿Cómo decide $q^\peq{s}$ y $p^\peq{s}$ el monopolio?
			\begin{itemize}
				\item Ya vimos que para maximizar sus ganancias una empresa escoge la cantidad que iguala $$CMg\rp{q}=IMg\rp{q}$$
				\item La diferencia entre los comportamientos de la empresa competitiva y del monopolio radica en la forma de su ingreso marginal.
			\end{itemize}
		\end{frame}

		\begin{frame}
			\frametitle{Monopolio}
			\begin{itemize}
				\item La firma competitiva es tomadora de precios y su ingreso marginal es constante.
				\item En esencia, enfrenta una demanda perfectamente elástica (en el precio de mercado).
			\end{itemize}
		\end{frame}

		\begin{frame}
			\frametitle{Monopolio}
			\begin{itemize}
				\item El monopolio enfrenta la demanda del mercado, que suele tener pendiente negativa.
				\item Su ingreso marginal no es constante
			\end{itemize}
		\end{frame}

		\begin{frame}
			\frametitle{Monopolio}
			\begin{figure}[hbtp!]
				\centering
				\begin{subfigure}[b]{0.49\textwidth}
					\begin{tikzpicture}[scale=1]
						\draw[ultra thick,teal] (0,3)--(4.5,3);
						\node [right,teal] at (4.5,3) {$D$};
						\draw [<->,ultra thick] (0,5)--(0,0)--(5,0);
						\node [below] at (5,0) {$q$};
						\node [left] at (0,5) {$p$};
						\node at (2.5,5) {\underline{Firma competitiva}};
						\node [below] at (0,0) {\tiny 0};
					\end{tikzpicture}
				\end{subfigure}
				\begin{subfigure}[b]{0.49\textwidth}
					\begin{tikzpicture}[scale=1]
						\draw[ultra thick,teal] plot [domain=0:4] (\x,{4-\x});
						\node [above right,teal] at (4,0) {$D$};
						\draw [<->,ultra thick] (0,5)--(0,0)--(5,0);
						\node [below] at (5,0) {$q$};
						\node [left] at (0,5) {$p$};
						\node at (2.5,5) {\underline{Monopolio}};
						\node [below] at (0,0) {\tiny 0};
					\end{tikzpicture}
				\end{subfigure}
			\end{figure}
		\end{frame}

		\begin{frame}
			\frametitle{Monopolio}
			Cuando el monopolio aumenta la producción se producen dos efectos sobre su ingreso total:
			\begin{itemize}
				\item Efecto producto (por la unidad marginal): $$\Delta^\peq{+}q\implies\Delta^\peq{+}\rp{p\cdot q}$$
				\item Efecto precio (por las unidades inframarginales): $$\Delta^\peq{+}q\implies\Delta^\peq{-}p^\peq{d}\rp{q}\implies\Delta^\peq{-}\rp{p\cdot q}$$  
			\end{itemize}
		\end{frame}

		\begin{frame}
			\frametitle{Monopolio}
			\begin{itemize}
				\item Empresa competitiva: $$IMg\rp{q}=p$$
				\item Monopolio: $$IMg\rp{q}<p\rp{q}$$ (ver pizarra)
			\end{itemize}
		\end{frame}
	
		\begin{frame}
			\frametitle{Monopolio}
			Ejemplo:
			\begin{table}[htbp!]
				\centering
				%\resizebox{8.5cm}{!}{
					\begin{tabular}{|l|c|c|c|c|}\hline
						$q$&	$p$	&$IT$&$IMe$&$IMg$\\ [1ex] \hline 
						0u &\$11/u&\$0 &\$11/u& --	 \\ \hline
						1u &\$10/u&\$10&\$10/u&\$10/u \\ \hline
						2u &\$9/u	&\$18&\$9/u	&\$8/u\\ \hline
						3u &\$8/u &\$24&\$8/u &\$6/u\\ \hline
						4u &\$7/u	&\$28&\$7/u	&\$4/u\\ \hline
						5u &\$6/u	&\$30&\$6/u	&\$2/u\\ \hline
						6u &\$5/u	&\$30&\$5/u	&\$0/u\\ \hline
						7u &\$4/u	&\$28&\$4/u &\$-2/u\\ \hline
						8u &\$3/u	&\$24&\$3/u	&\$-4/u\\ \hline
					\end{tabular}%}
			\end{table}
		\end{frame}

		\begin{frame}
			\frametitle{Monopolio}
			\begin{figure}[htbp!]
				\centering
					\begin{tikzpicture}[scale=0.425]
						\draw [<->,ultra thick] (0,12) -- (0,0) -- (12,0);
						\draw [ultra thick,teal] (0,11)--(1,10)--(2,9)--(3,8)--(4,7)--(5,6)--(6,5)--(7,4)--(8,3);
						\draw [fill,teal] (0,11) circle [radius=.2cm];
						\draw [fill,teal] (1,10) circle [radius=.2cm];
						\draw [fill,teal] (2,9) circle [radius=.2cm];
						\draw [fill,teal] (3,8) circle [radius=.2cm];
						\draw [fill,teal] (4,7) circle [radius=.2cm];
						\draw [fill,teal] (5,6) circle [radius=.2cm];
						\draw [fill,teal] (6,5) circle [radius=.2cm];
						\draw [fill,teal] (7,4) circle [radius=.2cm];
						\draw [fill,teal] (8,3) circle [radius=.2cm];
						\node [below right, teal] at (8,3) {$D=IMe$};
						\draw [ultra thick,purple] (0,11)--(.5,10)--(1.5,8)--(2.5,6)--(3.5,4)--(4.5,2)--(5.5,0)--(6.5,-2)--(7.5,-4);
						\draw [fill,purple] (.5,10) circle [radius=.2cm];
						\draw [fill,purple] (1.5,8) circle [radius=.2cm];
						\draw [fill,purple] (2.5,6) circle [radius=.2cm];
						\draw [fill,purple] (3.5,4) circle [radius=.2cm];
						\draw [fill,purple] (4.5,2) circle [radius=.2cm];
						\draw [fill,purple] (5.5,0) circle [radius=.2cm];
						\draw [fill,purple] (6.5,-2) circle [radius=.2cm];
						\draw [fill,purple] (7.5,-4) circle [radius=.2cm];
						\node [above right, purple] at (8,-4) {$IMg$};
						\node [below] at (0,0) {\tiny 0};
					\end{tikzpicture}
			\end{figure}	
		\end{frame}			

		\begin{frame}
			\frametitle{Monopolio}
			En general:
			\begin{figure}[htbp!]
				\centering
					\begin{tikzpicture}[scale=0.0125]
						%\draw [ultra thick,teal] plot [domain=0:400] (\x,{400-\x}); % D
						%\draw [ultra thick,purple] plot [domain=0:220] (\x,{400-2*\x}); % IMg
						\draw [ultra thick,teal] plot [domain=50:290] (\x,{(.0072*\x-1.2)*\x+(.06*\x-10)^2+50}); % CMg
						\draw [ultra thick,blue] plot [domain=50:400] (\x,{(.06*\x-10)^2+50+10000/\x}); %CMeT
						\draw [<->,ultra thick] (0,450) -- (0,0) -- (450,0);
						\node [below] at (450,0) {$q$};
						\node [below] at (0,0) {\tiny 0};
						%\node [right,purple] at (220,-40) {$IMg$};
						\node [left] at (0,450) {$p$};
						\node [right,teal] at (290,362.28) {$CMg$};
						\node [right,blue] at (400,271) {$CMeT$};
					\end{tikzpicture}
			\end{figure}	
		\end{frame}			

		\begin{frame}
			\frametitle{Monopolio}
			\begin{figure}[htbp!]
				\centering
					\begin{tikzpicture}[scale=0.0125]
						\draw [ultra thick,teal] plot [domain=0:400] (\x,{400-\x}); % D
						%\draw [ultra thick,purple] plot [domain=0:220] (\x,{400-2*\x}); % IMg
						\draw [ultra thick,teal] plot [domain=50:290] (\x,{(.0072*\x-1.2)*\x+(.06*\x-10)^2+50}); % CMg
						\draw [ultra thick,blue] plot [domain=50:400] (\x,{(.06*\x-10)^2+50+10000/\x}); %CMeT
						\draw [<->,ultra thick] (0,450) -- (0,0) -- (450,0);
						\node [below] at (450,0) {$q$};
						\node [below] at (0,0) {\tiny 0};
						%\node [right,purple] at (220,-40) {$IMg$};
						\node [left] at (0,450) {$p$};
						\node [right,teal] at (290,362.28) {$CMg$};
						\node [right,blue] at (400,271) {$CMeT$};
						\node [above right,teal] at (400,0) {$D$};
					\end{tikzpicture}
			\end{figure}	
		\end{frame}			

		\begin{frame}
			\frametitle{Monopolio}
			\begin{figure}[htbp!]
				\centering
					\begin{tikzpicture}[scale=0.0125]
						\draw [ultra thick,teal] plot [domain=0:400] (\x,{400-\x}); % D
						\draw [ultra thick,purple] plot [domain=0:220] (\x,{400-2*\x}); % IMg
						\draw [ultra thick,teal] plot [domain=50:290] (\x,{(.0072*\x-1.2)*\x+(.06*\x-10)^2+50}); % CMg
						\draw [ultra thick,blue] plot [domain=50:400] (\x,{(.06*\x-10)^2+50+10000/\x}); %CMeT
						\draw [<->,ultra thick] (0,450) -- (0,0) -- (450,0);
						\node [below] at (450,0) {$q$};
						\node [below] at (0,0) {\tiny 0};
						\node [right,purple] at (220,-40) {$IMg$};
						\node [left] at (0,450) {$p$};
						\node [right,teal] at (290,362.28) {$CMg$};
						\node [right,blue] at (400,271) {$CMeT$};
						\node [above right,teal] at (400,0) {$D$};
					\end{tikzpicture}
			\end{figure}	
		\end{frame}			

		\begin{frame}
			\frametitle{Monopolio}
			\begin{figure}[htbp!]
				\centering
					\begin{tikzpicture}[scale=0.0125]
						%\draw [dashed,help lines] (0,228.2134689)--(171.7865311,228.2134689)--(171.7865311,0);
						\draw [dashed,help lines] (171.7865311,0)--(171.7865311,56.4269378);
						\draw [ultra thick,teal] plot [domain=00:400] (\x,{400-\x}); % D
						\draw [ultra thick,purple] plot [domain=00:220] (\x,{400-2*\x}); % IMg
						\draw [ultra thick,teal] plot [domain=50:290] (\x,{(.0072*\x-1.2)*\x+(.06*\x-10)^2+50}); % CMg
						\draw [ultra thick,blue] plot [domain=50:400] (\x,{(.06*\x-10)^2+50+10000/\x}); %CMeT
						\draw [<->,ultra thick] (0,450) -- (0,0) -- (450,0);
						\node [below] at (450,0) {$q$};
						\node [below] at (0,0) {\tiny 0};
						\node [right,purple] at (220,-40) {$IMg$};
						\node [left] at (0,450) {$p$};
						\node [right,teal] at (290,362.28) {$CMg$};
						\node [right,blue] at (400,271) {$CMeT$};
						\draw [fill,blue] (171.7865311,56.4269378) circle [radius=8];
						%\draw [fill,blue] (171.7865311,228.2134689) circle [radius=8];
						\node [below] at (171.7865311,0) {\tiny $q^\peq{*}_\peq{M}$};
						\node [above right,teal] at (400,0) {$D$};
					\end{tikzpicture}
			\end{figure}	
		\end{frame}			

		\begin{frame}
			\frametitle{Monopolio}
			\begin{figure}[htbp!]
				\centering
					\begin{tikzpicture}[scale=0.0125]
						\draw [dashed,help lines] (0,228.2134689)--(171.7865311,228.2134689)--(171.7865311,0);
						%\draw [dashed,help lines] (171.7865311,0)--(171.7865311,56.4269378);
						\draw [ultra thick,teal] plot [domain=00:400] (\x,{400-\x}); % D
						\draw [ultra thick,purple] plot [domain=00:220] (\x,{400-2*\x}); % IMg
						\draw [ultra thick,teal] plot [domain=50:290] (\x,{(.0072*\x-1.2)*\x+(.06*\x-10)^2+50}); % CMg
						\draw [ultra thick,blue] plot [domain=50:400] (\x,{(.06*\x-10)^2+50+10000/\x}); %CMeT
						\draw [<->,ultra thick] (0,450) -- (0,0) -- (450,0);
						\node [below] at (450,0) {$q$};
						\node [below] at (0,0) {\tiny 0};
						\node [right,purple] at (220,-40) {$IMg$};
						\node [left] at (0,450) {$p$};
						\node [right,teal] at (290,362.28) {$CMg$};
						\node [right,blue] at (400,271) {$CMeT$};
						\draw [fill,blue] (171.7865311,56.4269378) circle [radius=8];
						\draw [fill,blue] (171.7865311,228.2134689) circle [radius=8];
						\node [below] at (171.7865311,0) {\tiny $q^\peq{*}_\peq{M}$};
						\node [left] at (0,228.2134689) {\tiny $p^\peq{*}_\peq{M}$};
						\node [above right,teal] at (400,0) {$D$};
					\end{tikzpicture}
			\end{figure}	
		\end{frame}			
	
		\begin{frame}
			\frametitle{Monopolio}
			\begin{figure}[htbp!]
				\centering
					\begin{tikzpicture}[scale=0.0125]
						\draw [fill,opacity=.3,blue] (0,228.2134689)--(171.7865311,228.2134689)--(171.7865311,108.3061483)--(0,108.3061483);
						\draw [dashed,help lines] (0,228.2134689)--(171.7865311,228.2134689)--(171.7865311,0);
						\draw [dashed,help lines] (0,108.3061483)--(171.7865311,108.3061483);
						\draw [ultra thick,teal] plot [domain=0:400] (\x,{400-\x}); % D
						\draw [ultra thick,purple] plot [domain=0:220] (\x,{400-2*\x}); % IMg
						\draw [ultra thick,teal] plot [domain=50:290] (\x,{(.0072*\x-1.2)*\x+(.06*\x-10)^2+50}); % CMg
						\draw [ultra thick,blue] plot [domain=50:400] (\x,{(.06*\x-10)^2+50+10000/\x}); %CMeT
						\draw [<->,ultra thick] (0,450) -- (0,0) -- (450,0);
						\node [below] at (450,0) {$q$};
						\node [below] at (0,0) {\tiny 0};
						\node [right,purple] at (220,-40) {$IMg$};
						\node [left] at (0,450) {$p$};
						\node [right,teal] at (290,362.28) {$CMg$};
						\node [right,blue] at (400,271) {$CMeT$};
						\draw [fill,blue] (171.7865311,56.4269378) circle [radius=8];
						\draw [fill,blue] (171.7865311,228.2134689) circle [radius=8];
						\draw [fill,blue] (171.7865311,108.3061483) circle [radius=8];
						\node [below] at (171.7865311,0) {\tiny $q^\peq{*}_\peq{M}$};
						\node [left] at (0,228.2134689) {\tiny $p^\peq{*}_\peq{M}$};
						\node [left] at (0,108.3061483) {\tiny $CMeT^\peq{*}_\peq{M}$};
						\node [above right,teal] at (400,0) {$D$};
					\end{tikzpicture}
			\end{figure}	
		\end{frame}				
	
		\begin{frame}
			\frametitle{Monopolio}
			\begin{figure}[htbp!]
				\centering
					\begin{tikzpicture}[scale=0.0125]
						\draw [fill,opacity=.3,black] (171.7865311,56.4269378)--(171.7865311,228.2134689)--(230.19,169.81)--plot [domain=171.7865311:230.19] (\x,{(.0072*\x-1.2)*\x+(.06*\x-10)^2+50})--cycle;
						\draw [dashed,help lines] (0,228.2134689)--(171.7865311,228.2134689)--(171.7865311,0);
						\draw [dashed,help lines] (0,56.4269378)--(171.7865311,56.4269378);
						\draw [dashed,help lines] (230.19,0)--(230.19,169.81)--(0,169.81);
						\draw [ultra thick,teal] plot [domain=0:400] (\x,{400-\x}); % D
						\draw [ultra thick,purple] plot [domain=0:220] (\x,{400-2*\x}); % IMg
						\draw [ultra thick,teal] plot [domain=50:290] (\x,{(.0072*\x-1.2)*\x+(.06*\x-10)^2+50}); % CMg
						\draw [ultra thick,blue] plot [domain=50:400] (\x,{(.06*\x-10)^2+50+10000/\x}); %CMeT
						\draw [<->,ultra thick] (0,450) -- (0,0) -- (450,0);
						\node [below] at (450,0) {$q$};
						\node [below] at (0,0) {\tiny 0};
						\node [right,purple] at (220,-40) {$IMg$};
						\node [left] at (0,450) {$p$};
						\node [right,teal] at (290,362.28) {$CMg$};
						\node [right,blue] at (400,271) {$CMeT$};
						\draw [fill,blue] (171.7865311,56.4269378) circle [radius=8];
						\draw [fill,blue] (171.7865311,228.2134689) circle [radius=8];
						%\draw [fill,blue] (171.7865311,108.3061483) circle [radius=8];
						\draw [fill,blue] (230.19,169.81) circle [radius=8];
						\node [below] at (171.7865311,0) {\tiny $q^\peq{*}_\peq{M}$};
						\node [below] at (230.19,0) {\tiny $q^\peq{*}_\peq{C}$};
						\node [left] at (0,228.2134689) {\tiny $p^\peq{*}_\peq{M}$};
						\node [left] at (0,169.81) {\tiny $p^\peq{*}_\peq{C}$};
						\node [left] at (0,56.4269378) {\tiny $CMg^\peq{*}_\peq{M}$};
						\node [above right,teal] at (400,0) {$D$};
					\end{tikzpicture}
			\end{figure}	
		\end{frame}					

		\begin{frame}
			\frametitle{Monopolio}
			Notar que el monopolio \textbf{no} tiene curva de oferta
			\begin{itemize}
				\item La curva de oferta responde qué cantidad querría ofrecer la firma para cada precio. Esta pregunta sólo tiene sentido en el caso de una firma tomadora de precios.
				\item De hecho no podemos separar la decisión de producción del monopolio de la curva de demanda que enfrenta.
			\end{itemize}
		\end{frame}		
	
		\begin{frame}
			\frametitle{Monopolio}
			Ejemplo: costo marginal constante
			\begin{itemize}
				\item Demanda: $p=a-bq\implies IMg\rp{q}=a-2bq$
				\item Costo marginal: $CMg\rp{q}=c$
				\item Este ejemplo podría representar a la industria farmaceútica en presencia de patentes.
			\end{itemize}
		\end{frame}							

		\begin{frame}
			\frametitle{Monopolio}
			Resolviendo:
			\begin{itemize}
				\item Asignación eficiente: $p=CMg$ $$q^\peq{*}_\peq{C}=\frac{a-c}{b}$$ $$p^\peq{*}_\peq{C}=c$$
				\item Equilibrio monopólico: $IMg=CMg$ $$q^\peq{*}_\peq{M}=\frac{a-c}{2b}<q^\peq{*}_\peq{C}$$ $$p^\peq{*}_\peq{M}=\frac{a+c}{2}>p^\peq{*}_\peq{C} \text{ }\rp{a>c}$$
			\end{itemize}
		\end{frame}		

		\begin{frame}
			\frametitle{Monopolio}
			\centering
			\begin{tikzpicture}[scale=1]
				\draw [dashed,help lines] (0,2.5)--(1.5,2.5)--(1.5,0);
				\draw [dashed,help lines] (3,1)--(3,0);
				\draw [ultra thick,teal] plot [domain=0:4] (\x,{1});
				\draw [ultra thick,teal] plot [domain=0:4] (\x,{4-\x});
				\draw [ultra thick,purple] plot [domain=0:2.25] (\x,{4-2*\x});
				\draw [<->,ultra thick] (5,0)--(0,0)--(0,5);
				\draw [fill,blue] (1.5,1) circle [radius=.08];
				\draw [fill,blue] (1.5,2.5) circle [radius=.08];
				\draw [fill,blue] (3,1) circle [radius=.08];
				\node [below] at (1.5,0) {\tiny $q^\peq{*}_\peq{M}$};
				\node [below] at (3,0) {\tiny $q^\peq{*}_\peq{C}$};
				\node [left] at (0,2.5) {\tiny $p^\peq{*}_\peq{M}$};
				\node [left] at (0,1) {\tiny $p^\peq{*}_\peq{C}=c$};
				\node [left] at (0,5) {$p$};
				\node [below] at (5,0) {$q$};
				\node [right,teal] at (4,1) {\small $CMg\rp{q}$};
				\node [below right,purple] at (2.25,-.5) {\small $IMg\rp{q}$};
				\node [above right,teal] at (4,0) {\small $D$};
				\node [below] at (0,0) {\tiny 0};
			\end{tikzpicture}
		\end{frame}	

		\begin{frame}
			\frametitle{Monopolio}
			\begin{mydef}
				\textbf{Discriminación de precios:} Práctica de negocios que consiste en cobrar precios distintos por el mismo bien a clientes distintos (o por cantidades distintas).
			\end{mydef}
		\end{frame}	

		\begin{frame}
			\frametitle{Monopolio}
			La discriminación de precios
			\begin{itemize}
				\item Requiere algún grado de poder de mercado $\implies$ no es posible en competencia perfecta.
				\item Es una estrategia racional para un monopolio: si lo hace es porque así maximiza ganancias.
			\end{itemize}
		\end{frame}	

		\begin{frame}
			\frametitle{Monopolio}
			\begin{itemize}
				\item Requiere la capacidad de separar a los clientes según su disposición a pagar (geográficamente, por edad, por ingresos, etc.)
				\item No es factible si hay posibilidades de arbitraje.
				\begin{mydef}
					\textbf{Arbitraje:} Comprar en un mercado a precio bajo y vender en otro mercado a precio alto.
				\end{mydef}
			\end{itemize}
		\end{frame}	

		\begin{frame}
			\frametitle{Monopolio}
			\begin{mydef}
				\textbf{Discriminación perfecta:} El monopolio conoce la disposición a pagar de cada cliente (o por cada unidad), lo que le permite cobrar a cada cliente (o por cada unidad) exactamente su disposición a pagar.
			\end{mydef}
		\end{frame}	

		\begin{frame}
			\frametitle{Monopolio}
			Ejemplo: $CT\rp{q}=cq\implies CMg\rp{q}=CMeT\rp{q}=c$
			\begin{figure}[htbp!]
				\centering
				\begin{subfigure}[b]{0.49\textwidth}
					\begin{tikzpicture}[scale=1]
						\draw [fill,opacity=.2,blue] (0,4)--(0,2.5)--(1.5,2.5);
						\draw [fill,opacity=.2,green] (0,2.5)--(1.5,2.5)--(1.5,1)--(0,1);
						\draw [fill,opacity=.2,black] (1.5,2.5)--(1.5,1)--(3,1);
						\draw [dashed,help lines] (0,2.5)--(1.5,2.5)--(1.5,0);
						\draw [ultra thick,teal] plot [domain=0:4] (\x,{1});
						\draw [ultra thick,teal] plot [domain=0:4] (\x,{4-\x});
						\draw [ultra thick,purple] plot [domain=0:2.25] (\x,{4-2*\x});
						\draw [<->,ultra thick] (5,0)--(0,0)--(0,5);
						\draw [fill,blue] (1.5,1) circle [radius=.08];
						\draw [fill,blue] (1.5,2.5) circle [radius=.08];
						\node [below] at (1.5,0) {\tiny $q^\peq{*}_\peq{M}$};
						\node [left] at (0,2.5) {\tiny $p^\peq{*}_\peq{M}$};
						\node [left] at (0,1) {\tiny $c$};
						\node [above] at (0.5,2.5) { $\mathbf{EC}$}; 
						\node at (0.5,1.75) {$\mathbf{\pi}$}; 
						\node at (1.8,1.75) {$\mathbf{PS}$};
						\node [left] at (0,5) {$p$};
						\node [below] at (5,0) {$q$};
						\node [below] at (0,0) {\tiny 0};
					\end{tikzpicture}
				\end{subfigure}
				\begin{subfigure}[b]{0.49\textwidth}
					\begin{tikzpicture}[scale=1]
						\draw [ultra thick,white] plot [domain=0:2.25] (\x,{4-2*\x});
						\draw [fill,opacity=.2,green] (0,4)--(0,1)--(3,1);
						\draw [dashed,help lines] (3,1)--(3,0);
						\draw [ultra thick,teal] plot [domain=0:4] (\x,{1});
						\draw [ultra thick,teal] plot [domain=0:4] (\x,{4-\x});
						\draw [<->,ultra thick] (5,0)--(0,0)--(0,5);
						\draw [fill,blue] (3,1) circle [radius=.08];
						\node [below] at (3,0) {\tiny $q^\peq{*}_\peq{M}$};
						\node [left] at (0,1) {\tiny $c$};
						\node at (1.2,1.75) {$\mathbf{\pi}$};
						\node [left] at (0,5) {$p$};
						\node [below] at (5,0) {$q$};
						\node [below] at (0,0) {\tiny 0};
					\end{tikzpicture}
				\end{subfigure}
			\end{figure}	
		\end{frame}			

		\begin{frame}
			\frametitle{Monopolio}
			\begin{itemize}
				\item En el caso de discriminación de precios perfecta, el resultado es eficiente. El monopolio extrae todo el excedente a los consumidores y produce la cantidad eficiente.
				\item En general es difícil que se dé la discriminación de precios perfecta.
			\end{itemize}
		\end{frame}	

		\begin{frame}
			\frametitle{Monopolio}
			\begin{itemize}
				\item Cuando la discriminación de precios no es perfecta el análisis es más complejo.
					\begin{itemize}
						\item $ET$ puede aumentar, disminuir o no cambiar respecto del equilibrio sin discriminación de precios.
						\item Lo que sí sabemos es que $\pi$ debe aumentar. De lo contrario el monopolio no discriminaría.
					\end{itemize}
			\end{itemize}
		\end{frame}	

		\begin{frame}
			\frametitle{Monopolio}
			Algunos ejemplos de discriminación de precios:
			\begin{itemize}
				\item Boletos de cine
				\item Pasajes aéreos
				\item Cupones de descuento
				\item Ayuda financiera
				\item Descuentos por volumen
			\end{itemize}
		\end{frame}	

		\begin{frame}
			\frametitle{Monopolio}
			Política pública: 4 enfoques
			\begin{itemize}
				\item Leyes antimonopolio
				\item Regulación
				\item Propiedad pública
				\item No hacer nada
			\end{itemize}
		\end{frame}	

		\begin{frame}
			\frametitle{Monopolio}
			\begin{mydef}
				\textbf{Ley antimonopolio:} Cuerpo legal que busca promover y proteger la libre competencia.
			\end{mydef}
		\end{frame}	

		\begin{frame}
			\frametitle{Monopolio}
			Institucionalidad de la libre competencia en Chile
			\begin{itemize}
				\item Modelo judicial dividido:
				\begin{itemize}
					\item Órgano administrativo investiga (Fiscalía Nacional Económica)
					\item Tribunal especializado resuelve (Tribunal de Defensa de la Libre Competencia)
				\end{itemize}
			\end{itemize}
		\end{frame}	

		\begin{frame}
			\frametitle{Monopolio}
			Fiscalía Nacional Económica (FNE):
			\begin{itemize}
				\item Servicio público dirigido por el Fiscal Nacional Económico
				\item Supervisado por el Presidente de la República a través del Ministerio de Economía
			\end{itemize}
		\end{frame}	

		\begin{frame}
			\frametitle{Monopolio}
			Funciones de la FNE:
			\begin{itemize}
				\item Realizar investigaciones vinculadas a situaciones que contravengan la legislación de defensa de la libre competencia.
				\item Tiene facultad para presentar requerimientos (demandas ante el TDLC) y proponer sanciones o medidas correctivas que procedan.
				\item Velar por el cumplimiento de las medidas que establezca el TDLC en sus fallos/sentencias/resoluciones.
			\end{itemize}
		\end{frame}	

		\begin{frame}
			\frametitle{Monopolio}
			Tribunal de Defensa de la Libre Competencia (TDLC):
			\begin{itemize}
				\item Órgano jurisdiccional especial e independiente
				\item Sujeto a la superintendencia directiva, correccional y económica de la Corte Suprema
				\item Integrado por 5 Ministros, tres de los cuales son abogados y dos son economistas.  Los acuerdos o decisiones se adoptan por mayoría y, en caso de empate, dirime su Presidente.
			\end{itemize}
		\end{frame}	

		\begin{frame}
			\frametitle{Monopolio}
			Funciones del TDLC:
			\begin{itemize}
				\item Prevenir, corregir y sancionar los atentados a la libre competencia aplicando las normas contenidas en el Decreto Ley Nº 211.
				\item Conocer y resolver las demandas o requerimientos presentados por el Fiscal Nacional Económico.
				\item Conocer y resolver consultas presentadas por particulares o la FNE respceto de asuntos que puedan atenten contra la libre competencia.
			\end{itemize}
		\end{frame}	

		\begin{frame}
			\frametitle{Monopolio}
			\begin{itemize}
				\item Dictar instrucciones de carácter general
				\item Proponer al Presidente de la República la modificación o derogación de preceptos legales y reglamentarios que estime contrarios a la libre competencia.
			\end{itemize}
		\end{frame}	

		\begin{frame}
			\frametitle{Monopolio}
			Conductas que son materia de investigación por parte de la FNE:
			\begin{itemize}
				\item Colusiones o acuerdos horizontales
				\begin{itemize}
					\item Acuerdos de precios
					\item Reparto de mercado
					\item Reparto de cuotas de producción
					\item Licitaciones colusivas
				\end{itemize}
			\end{itemize}
		\end{frame}	

		\begin{frame}
			\frametitle{Monopolio}
			\begin{itemize}
				\item Abuso de posición dominante
				\begin{itemize}
					\item Precios excesivos
					\item Discriminación arbitraria de precios
					\item Precios predatorios
					\item Negativa de venta o compra de bienes y servicios
					\item Fijación de precios
					\item Restricciones verticales
				\end{itemize}
			\end{itemize}
		\end{frame}

		\begin{frame}
			\frametitle{Monopolio}
			\begin{itemize}
				\item Competencia desleal (realizada con el objetivo de alcanzar, mantener o incrementar una posición dominante)
				\item Operaciones de concentración
				\begin{itemize}
					\item Integración vertical
					\item Integración horizontal
					\item Integración de conglomerados
				\end{itemize}
			\end{itemize}
		\end{frame}

		\begin{frame}
			\frametitle{Monopolio}
			Las leyes antimonopolio tienen tanto costos como beneficios
			\begin{itemize}
				\item A veces las prácticas que se persiguen tienen motivos de eficiencia.
				\item Por ejemplo, a veces las empresas se fusionan para reducir costos.
			\end{itemize}
		\end{frame}

		\begin{frame}
			\frametitle{Monopolio}
			\textbf{Regulación:}
			\begin{itemize}
				\item Enfoque común en el caso de los monopolios naturales.
				\item Ejemplo: fijación de tarifas en distribución de agua potable y electricidad por parte de organismos gubernamentales.
			\end{itemize}
		\end{frame}

		\begin{frame}
			\frametitle{Monopolio}
			Problema: no es obvio qué precio establecer
			\begin{itemize}
				\item ¿$p=CMg$? A primera vista parece razonable, pero...
				\begin{itemize}
					\item Economías de escala $\implies CMg<CMeT$
					\item Luego, $p=CMg\implies\pi<0$.
					\item El monopolio abandonaría el mercado con esta regulación.
				\end{itemize}
			\end{itemize}
		\end{frame}

		\begin{frame}
			\frametitle{Monopolio}
			\centering
			\begin{tikzpicture}[scale=1]
				\draw [fill,opacity=.2,green] (0,4/3)--(3,4/3)--(3,1)--(0,1);
				\draw [dashed,help lines] (0,4/3)--(3,4/3)--(3,0);
				\draw [dashed,help lines] (0,1)--(3,1);
				\draw [ultra thick,teal] plot [domain=0:5] (\x,{1});
				\draw [ultra thick,teal] plot [domain=0:4] (\x,{4-\x});
				\draw [ultra thick,purple] plot [domain=.25:5] (\x,{1/\x+1});
				\draw [<->,ultra thick] (5,0)--(0,0)--(0,5);
				\draw [fill,blue] (3,4/3) circle [radius=.08];
				\draw [fill,blue] (3,1) circle [radius=.08];
				\node [below] at (3,0) {\tiny $q_\peq{0}$};
				\node [left] at (0,4/3) {\tiny $CMeT_\peq{0}$};
				\node [left] at (0,1) {\tiny $p_\peq{0}$};
				\node [right,teal] at (5,1) {\tiny $CMg\rp{q}$};
				\node [right,purple] at (5,1.2) {\tiny $CMeT\rp{q}$};
				\node [above] at (1.5,.9) {$\mathbf{\pi_\peq{0}}<0$};
				\node [above right,teal] at (4,0) {\tiny $D$};
				\node [left] at (0,5) {$p$};
				\node [below] at (5,0) {$q$};
				\node [below] at (0,0) {\tiny 0};
			\end{tikzpicture}
		\end{frame}	

		\begin{frame}
			\frametitle{Monopolio}
			\begin{itemize}
				\item El gobierno puede subsidiar al monopolio, pero para gastar requiere recaudar impuestos.
				\item Otra alternativa es fijar $p=CMeT\implies\pi=0$. En este caso el monopolio no abandona el mercado.
				\item Pero ya no es cierto que $p=CMg\implies$ hay pérdida social (aunque menor que sin regulación).
			\end{itemize}
		\end{frame}

		\begin{frame}
			\frametitle{Monopolio}
			\begin{itemize}
				\item En ambos casos el monopolio no tiene incentivos a invertir en reducciones de costos porque no disfrutará de los beneficios.
				\item En la práctica el regualdor resuelve permitiendo que el monopolio se quede con parte de los beneficios.
			\end{itemize}
		\end{frame}

		\begin{frame}
			\frametitle{Monopolio}
			\textbf{Propiedad pública:}
			\begin{itemize}
				\item El gobierno administra directamente el monopolio.
				\item Enfoque común en varios países europeos.
				\item En general los economistas prefieren la propiedad privada sobre la pública (incentivos a reducir costos).
			\end{itemize}
		\end{frame}

		\begin{frame}
			\frametitle{Monopolio}
			\textbf{No hacer nada:}
			\textbf{Propiedad pública:}
			\begin{itemize}
				\item A veces el costo de la solución puede ser mayor que el problema que intenta resolver.
				\item Según Stigler, la ``falla del mercado'' es mucho menor que la ``falla política''.
			\end{itemize}
		\end{frame}
	
		\begin{frame}
			\frametitle{Competencia Monopolística}
			\begin{mydef}
				\textbf{Competencia monopolística:} Estructura de mercado en la cual muchas empresas venden productos similares, pero \textbf{no} idénticos (productos diferenciados).
			\end{mydef}
		\end{frame}		
	
		\begin{frame}
			\frametitle{Competencia Monopolística}
			Modelaremos una situación con las siguientes características:
			\begin{itemize}
				\item Grupo grande de vendedores compite por el mismo grupo de clientes
				\item Productos diferenciados $\implies$ cada empresa enfrenta demanda con pendiente negativa
				\item Libre entrada y salida $\implies\pi=0$ en el equilibrio de largo plazo
			\end{itemize}
		\end{frame}				


		\begin{frame}
			\frametitle{Competencia Monopolística}
			Decisión de la firma:
			\begin{itemize}
				\item Cada empresa es en esencia un monopolio local sobre su variedad de producto
				\item La decisión de producción está dada por la regla $IMg\rp{q}=CMg\rp{q}$.
			\end{itemize}
		\end{frame}	

		\begin{frame}
			\frametitle{Competencia Monopolística}
			\begin{figure}[htbp!]
				\begin{subfigure}[b]{0.49\textwidth}
					\centering
					\begin{tikzpicture}[scale=0.009]
						\draw [fill,opacity=.3,green] (0,132.08081)--(129.63,132.08081)--(129.63,85.185)--(0,85.185);
						\draw [dashed,help lines] (0,132.08081)--(129.63,132.08081)--(129.63,0);
						\draw [dashed,help lines] (0,85.185)--(129.63,85.185);
						%\draw [dashed,help lines] (0,56.4269378)--(171.7865311,56.4269378);
						%\draw [dashed,help lines] (230.19,0)--(230.19,169.81)--(0,169.81);
						\draw [ultra thick,teal] plot [domain=0:300] (\x,{150-.5*\x}); % D
						\draw [ultra thick,purple] plot [domain=0:190] (\x,{150-\x}); % IMg
						\draw [ultra thick,teal] plot [domain=50:290] (\x,{(.0072*\x-1.2)*\x+(.06*\x-10)^2+50}); % CMg
						\draw [ultra thick,blue] plot [domain=50:400] (\x,{(.06*\x-10)^2+50+10000/\x}); %CMeT
						\draw [<->,ultra thick] (0,450) -- (0,0) -- (450,0);
						\node [below] at (450,0) {$q$};
						\node [below] at (0,0) {\tiny 0};
						\node [right,purple] at (190,-40) {$IMg$};
						\node [left] at (0,450) {$p$};
						\node [right,teal] at (290,362.28) {$CMg$};
						\node [right,blue] at (400,271) {$CMeT$};
						\draw [fill,blue] (129.63,20.37) circle [radius=8];
						\draw [fill,blue] (129.63,85.185) circle [radius=8];
						\draw [fill,blue] (129.63,132.08081) circle [radius=8];
						%\draw [fill,blue] (171.7865311,228.2134689) circle [radius=8];
						%\draw [fill,blue] (171.7865311,108.3061483) circle [radius=8];
						%\draw [fill,blue] (230.19,169.81) circle [radius=8];
						\node [below] at (129.63,0) {\tiny $q^\peq{*}_\peq{i}$};
						%\node [below] at (230.19,0) {\tiny $q^\peq{*}_\peq{C}$};
						\node [left] at (0,85.185) {\tiny $p^\peq{*}_\peq{i}$};
						\node [left] at (0,132.08081) {\tiny $CMeT^\peq{*}_\peq{i}$};
						%\node [left] at (0,56.4269378) {\tiny $CMg^\peq{*}_\peq{M}$};
						\node [above right,teal] at (300,0) {$D$};
						\node at (225,430) {\underline{$\mathbf{\pi_\peq{i}<0}$}};
					\end{tikzpicture}
				\end{subfigure}	
				\begin{subfigure}[b]{0.49\textwidth}
					\centering
					\begin{tikzpicture}[scale=0.009]
						\draw [fill,opacity=.3,green] (0,228.2134689)--(171.7865311,228.2134689)--(171.7865311,108.3061483)--(0,108.3061483);
						\draw [dashed,help lines] (0,228.2134689)--(171.7865311,228.2134689)--(171.7865311,0);
						\draw [dashed,help lines] (0,108.3061483)--(171.7865311,108.3061483);
						\draw [ultra thick,teal] plot [domain=0:400] (\x,{400-\x}); % D
						\draw [ultra thick,purple] plot [domain=0:220] (\x,{400-2*\x}); % IMg
						\draw [ultra thick,teal] plot [domain=50:290] (\x,{(.0072*\x-1.2)*\x+(.06*\x-10)^2+50}); % CMg
						\draw [ultra thick,blue] plot [domain=50:400] (\x,{(.06*\x-10)^2+50+10000/\x}); %CMeT
						\draw [<->,ultra thick] (0,450) -- (0,0) -- (450,0);
						\node [below] at (450,0) {$q$};
						\node [below] at (0,0) {\tiny 0};
						\node [right,purple] at (220,-40) {$IMg$};
						\node [left] at (0,450) {$p$};
						\node [right,teal] at (290,362.28) {$CMg$};
						\node [right,blue] at (400,271) {$CMeT$};
						\draw [fill,blue] (171.7865311,56.4269378) circle [radius=8];
						\draw [fill,blue] (171.7865311,228.2134689) circle [radius=8];
						\draw [fill,blue] (171.7865311,108.3061483) circle [radius=8];
						\node [below] at (171.7865311,0) {\tiny $q^\peq{*}_\peq{i}$};
						\node [left] at (0,228.2134689) {\tiny $p^\peq{*}_\peq{i}$};
						\node [left] at (0,108.3061483) {\tiny $CMeT^\peq{*}_\peq{i}$};
						\node [above right,teal] at (400,0) {$D$};
						\node at (225,430) {\underline{$\mathbf{\pi_\peq{i}>0}$}};
					\end{tikzpicture}
				\end{subfigure}
			\end{figure}	
		\end{frame}		

		\begin{frame}
			\frametitle{Competencia Monopolística}
			Pero
			\begin{itemize}
				\item $\pi>0\implies$ incentivo a entrar $\implies$ más sustitutos $\implies\Delta^\peq{-}$Demanda de cada firma$\implies\Delta^\peq{-}\pi$ 
				\item $\pi<0\implies$ incentivo a salir $\implies$ menos sustitutos $\implies\Delta^\peq{+}$Demanda de cada firma$\implies\Delta^\peq{+}\pi$
				\item En equilibrio la cantidad de empresas es tal que $\pi=0\implies$ tangencia entre curvas de demanda y $CMeT$
			\end{itemize}
		\end{frame}	

		\begin{frame}
			\frametitle{Competencia Monopolística}
			\begin{figure}[htbp!]
				\centering
				\begin{tikzpicture}[scale=0.0125]
					\draw [dashed,help lines] (0,134.6506)--(126.729,134.6506)--(126.729,0);
					%\draw [dashed,help lines] (0,108.3061483)--(171.7865311,108.3061483);
					\draw [ultra thick,teal] plot [domain=0:274.6634] (\x,{250-.910205*\x}); % D
					\draw [ultra thick,purple] plot [domain=0:160] (\x,{250-2*.910205*\x}); % IMg
					\draw [ultra thick,teal] plot [domain=50:290] (\x,{(.0072*\x-1.2)*\x+(.06*\x-10)^2+50}); % CMg
					\draw [ultra thick,blue] plot [domain=50:400] (\x,{(.06*\x-10)^2+50+10000/\x}); % CMeT
					\draw [<->,ultra thick] (0,450) -- (0,0) -- (450,0);
					\node [below] at (450,0) {$q$};
					\node [below] at (0,0) {\tiny 0};
					\node [right,purple] at (160,-41.2656) {$IMg$};
					\node [left] at (0,450) {$p$};
					\node [right,teal] at (290,362.28) {$CMg$};
					\node [right,blue] at (400,271) {$CMeT$};
					\draw [fill,blue] (126.729,19.300986) circle [radius=8];
					\draw [fill,blue] (126.729,134.6506) circle [radius=8];
					\node [below] at (126.729,0) {\tiny $q^\peq{*}_\peq{i}$};
					\node [left] at (0,134.6506) {\tiny $p^\peq{*}_\peq{i}=CMeT^\peq{*}_\peq{i}$};
					\node [above right,teal] at (274.6634,0) {$D$};
				\end{tikzpicture}
			\end{figure}	
		\end{frame}		

		\begin{frame}
			\frametitle{Competencia Monopolística}
			Conclusiones:
			\begin{itemize}
				\item Al igual que en monopolio, $p^\peq{*}>CMg$
				\item Al igual que en competencia perfecta, $p^\peq{*}=CMeT$ (es decir $\pi=0$)
			\end{itemize}
		\end{frame}	

		\begin{frame}
			\frametitle{Competencia Monopolística}
			Competencia monopolística versus competencia perfecta
			\begin{figure}[htbp!]
				\begin{subfigure}[b]{0.49\textwidth}
					\centering
					\begin{tikzpicture}[scale=0.008]
						\draw [dashed,help lines] (0,134.6506)--(126.729,134.6506)--(126.729,0);
						%\draw [dashed,help lines] (0,108.3061483)--(171.7865311,108.3061483);
						\draw [ultra thick,teal] plot [domain=0:274.6634] (\x,{250-.910205*\x}); % D
						\draw [ultra thick,purple] plot [domain=0:160] (\x,{250-2*.910205*\x}); % IMg
						\draw [ultra thick,teal] plot [domain=50:290] (\x,{(.0072*\x-1.2)*\x+(.06*\x-10)^2+50}); % CMg
						\draw [ultra thick,blue] plot [domain=50:400] (\x,{(.06*\x-10)^2+50+10000/\x}); % CMeT
						\draw [<->,ultra thick] (0,450) -- (0,0) -- (450,0);
						\node [below] at (450,0) {$q$};
						\node [below] at (0,0) {\tiny 0};
						\node [right,purple] at (160,-41.2656) {$IMg$};
						\node [left] at (0,450) {$p$};
						\node [right,teal] at (290,362.28) {$CMg$};
						\node [right,blue] at (400,271) {$CMeT$};
						\draw [fill,blue] (126.729,19.300986) circle [radius=8];
						\draw [fill,blue] (126.729,134.6506) circle [radius=8];
						\node [below] at (126.729,0) {\tiny $q^\peq{*}_\peq{CM}$};
						\node [left] at (0,134.6506) {\tiny $p^\peq{*}_\peq{CM}=CMeT^\peq{*}_\peq{CM}$};
						\node [above right,teal] at (274.6634,0) {$D$};
						\node at (225,430) {\underline{Empresa en CM}};
					\end{tikzpicture}
				\end{subfigure}	
				\begin{subfigure}[b]{0.49\textwidth}
					\centering
					\begin{tikzpicture}[scale=0.008]
						\draw [dashed,help lines] (201.033,103.99484)--(201.033,0);
						\draw [ultra thick,white] plot [domain=0:220] (\x,{400-2*\x}); % IMg
						\draw [ultra thick,teal] plot [domain=50:290] (\x,{(.0072*\x-1.2)*\x+(.06*\x-10)^2+50}); % CMg
						\draw [ultra thick,purple] plot [domain=0:450] (\x,{103.99484}); % P
						\draw [ultra thick,blue] plot [domain=50:400] (\x,{(.06*\x-10)^2+50+10000/\x}); %CMeT
						\draw [<->,ultra thick] (0,450) -- (0,0) -- (450,0);
						\node [below] at (450,0) {$q$};
						\node [below] at (0,0) {\tiny 0};
						\node [right,white] at (220,-40) {$IMg$};
						\node [left] at (0,450) {$p$};
						\node [right,teal] at (290,362.28) {$CMg$};
						\node [right,blue] at (400,271) {$CMeT$};
						\draw [fill,blue] (201.033,103.99484) circle [radius=8];
						\node [below] at (201.033,0) {\tiny $q^\peq{*}_\peq{CP}$};
						\node [left] at (0,103.99484) {\tiny $p^\peq{*}_\peq{CP}=CMeT^\peq{*}_\peq{CP}$};
						\node at (225,430) {\underline{Empresa en CP}};
					\end{tikzpicture}
				\end{subfigure}
			\end{figure}	
		\end{frame}	
		
		\begin{frame}
			\frametitle{Competencia Monopolística}
			Diferencias con competencia perfecta:
			\begin{itemize}
				\item \textbf{Exceso de capacidad:} En el largo plazo las empresas en CP producen a la escala eficiente. Empresas en CM producen a una escala menor. Podrían reducir su $CMeT$ aumentando la escala, pero no aprovechan esta oportunidad porque tendrían que cobrar más barato ($\Delta^\peq{-}p$).
				\item \textbf{Margen sobre costo marginal:} Empresa en CP está indiferente entre vender una unidad adicional o no al precio de equilibrio. Empresa en CM querría vender más al precio de equilibrio porque $p^\peq{*}>CMg\rp{q^\peq{*}}$.
			\end{itemize}
		\end{frame}
	
		\begin{frame}
			\frametitle{Competencia Monopolística}
			Pérdida social:
			\begin{itemize}
				\item $p>CMg\implies$ hay compradores con disposición a pagar mayor que CMg que no están comprando (como en el caso del monopolio).
				\item La cantidad de empresas en equilibrio podría no ser eficiente ya que la entrada produce externalidades:
					\begin{itemize}
						\item Externalidad positiva: mayor variedad de productos
						\item Externalidad negativa: menores ganancias para empresas incumbentes (``business stealing'')
					\end{itemize}
			\end{itemize}
		\end{frame}		
	
		\begin{frame}
			\frametitle{Competencia Monopolística}
			Pero también es importante el escenario contrafactual:
			\begin{itemize}
				\item La variedad de productos tiene un valor intrínseco al ampliar los conjuntos de elección de los consumidores. 
				\item La variedad también proporciona un beneficio externo.
				\item Si la alternativa es un mundo sin variedad de productos, no es claro qué es mejor.
			\end{itemize}
		\end{frame}		

		\begin{frame}
			\frametitle{Competencia Monopolística}
			Publicidad:
			\begin{itemize}
				\item $p>CMg$ genera incentivo a buscar formas de atraer a nuevos clientes.
				\item Debate:
					\begin{itemize}
						\item ¿La sociedad malgasta sus recursos en publicidad?
						\item ¿La publicidad sirve para algún propósito útil?
					\end{itemize}
			\end{itemize}
		\end{frame}		

		\begin{frame}
			\frametitle{Competencia Monopolística}
			Crítica:
			\begin{itemize}
				\item Sirve para manipular los gustos de los consumidores.
					\begin{itemize}
						\item Gran parte de la publicidad es más psicológica que informativa.
						\item Genera un deseo que no existiría en ausencia de publicidad.
					\end{itemize}
				\item Reduce la competencia.
					\begin{itemize}
						\item Convence a los consumidores de que los productos son más diferentes de lo que son en realidad.
						\item Los consumidores se fijan menos en las diferencias de precios.
						\item Una demanda menos elástica permite un mayor margen sobre el costo marginal.
					\end{itemize}
			\end{itemize}
		\end{frame}		
				
		\begin{frame}
			\frametitle{Competencia Monopolística}
			Defensa:
			\begin{itemize}
				\item Proporciona información a los consumidores.
					\begin{itemize}
						\item Precios, aparición de nuevos productos, localización de las tiendas.
						\item Facilita la toma de decisiones.
					\end{itemize}
				\item Estimula la competencia.
					\begin{itemize}
						\item Consumidores más informados aprovechan mejor las diferencias de precios.
						\item Facilita la entrada de nuevas empresas.
					\end{itemize}
			\end{itemize}
		\end{frame}		

		\begin{frame}
			\frametitle{Competencia Monopolística}
			Un argumento más sutil a favor:
			\begin{itemize}
				\item Entrega más información de lo que se observa a simple vista.
					\begin{itemize}
						\item Disposición de la empresa a gastar en publicidad puede ser una \textbf{señal} de la calidad del producto para los consumidores.
					\end{itemize}
			\end{itemize}
		\end{frame}	
		
		\begin{frame}
			\frametitle{Teoría de Juegos}
			Un juego (en forma normal o estratégica) está compuesto por 3 elementos:
			\begin{itemize}
				\item Un conjunto de jugadores. Supondremos 2: $$J_\peq{i}, \ i\in\{1,2\}$$
			\end{itemize}
		\end{frame}		

		\begin{frame}
			\frametitle{Teoría de Juegos}
			\begin{itemize}
				\item Un conjunto de acciones disponibles para cada jugador. Supondremos 2 por jugador: $$J_\peq{1}\text{ escoge } a\in\{a_\peq{1},a_\peq{2}\}$$ $$J_\peq{2}\text{ escoge } b\in\{b_\peq{1},b_\peq{2}\}$$
			\end{itemize}
		\end{frame}		

		\begin{frame}
			\frametitle{Teoría de Juegos}
			\begin{itemize}
				\item Una función de pagos para cada jugador que depende de las acciones de ambos: $$u_\peq{i}\rp{a,b}, \ i\in\{1,2\}$$
			\end{itemize}
		\end{frame}	

		\begin{frame}
			\frametitle{Teoría de Juegos}
			Podemos representar el juego a través de la siguiente matriz de pagos
			\begin{table}[htbp!]
				\centering
				%\resizebox{10cm}{!}{
					\begin{tabular}{|l|c|c|}\hline
						$J_\peq{1}$/$J_\peq{2}$&$b_\peq{1}$&$b_\peq{2}$ \\ [1ex] \hline
						$a_\peq{1}$						 &$\rp{u_\peq{1}\rp{a_\peq{1},b_\peq{1}},u_\peq{2}\rp{a_\peq{1},b_\peq{1}}}$&$\rp{u_\peq{1}\rp{a_\peq{1},b_\peq{2}},u_\peq{2}\rp{a_\peq{1},b_\peq{2}}}$ \\ [1ex] \hline 
						$a_\peq{2}$						 &$\rp{u_\peq{1}\rp{a_\peq{2},b_\peq{1}},u_\peq{2}\rp{a_\peq{2},b_\peq{1}}}$&$\rp{u_\peq{1}\rp{a_\peq{2},b_\peq{2}},u_\peq{2}\rp{a_\peq{2},b_\peq{2}}}$ \\ [1ex] \hline 
					\end{tabular}%}
			\end{table}
		\end{frame}	

		\begin{frame}
			\frametitle{Teoría de Juegos}
			\begin{itemize}
				\item El objetivo de cada jugador es maximizar su función de pago, pero debe considerar las reacciones del otro jugador.
				\item El jugador debe decidir qué acción maximizaría su pago ante (condicional en) cada acción del rival.
			\end{itemize}
		\end{frame}	

		\begin{frame}
			\frametitle{Teoría de Juegos}
			\begin{mydef}
				\textbf{Mejor respuesta:} Función que indica la acción que maximiza el pago del jugador $i$ para cada acción posible del jugador $j$ ($j,i\in\{1,2\},j\neq i$). $$r_\peq{1}\rp{b}\in\{a_\peq{1},a_\peq{2}\}$$ $$r_\peq{2}\rp{a}\in\{b_\peq{1},b_\peq{2}\}$$
			\end{mydef}
		\end{frame}	

		\begin{frame}
			\frametitle{Teoría de Juegos}
			Aterrizando conceptos: dilema del prisionero
			\begin{itemize}
				\item La policía detiene a 2 delincuentes pero no tiene evidencia suficiente para condenarlos y necesita que confiesen el uno en contra del otro.
				\item Los encierran en celdas separadas para prevenir que se comuniquen y le dicen a cada uno que si testifica contra el otro, saldrá libre y recibirá un premio, siempre y cuando el otro no testifique en su contra.
			\end{itemize}
		\end{frame}	

		\begin{frame}
			\frametitle{Teoría de Juegos}
			Escenarios posibles:
			\begin{itemize}
				\item Si ninguno testifica, ambos salen libres y sin premio.
				\item Si uno testifica, el otro va a prisión.
				\item Si ambos testifican, ambos van a prisión pero con premio por confesar.
			\end{itemize}
		\end{frame}	

		\begin{frame}
			\frametitle{Teoría de Juegos}
			El juego:
			\begin{itemize}
				\item Ambos jugadores deciden simultáneamente sus acciones: testificar ($T$) o no ($N$).
				\item Si ninguno testifica (cooperan entre ellos), cada uno recibe un pago de 1. $$u_\peq{1}\rp{N,N}=u_\peq{2}\rp{N,N}=1$$
				\item Si ambos testifican (no cooperan), obtienen 0 cada uno. $$u_\peq{1}\rp{T,T}=u_\peq{2}\rp{T,T}=0$$
			\end{itemize}
		\end{frame}	

		\begin{frame}
			\frametitle{Teoría de Juegos}
			\begin{itemize}
				\item Si uno testifica y el otro no, el que testifica queda con pago 2 y el otro con pago -1 $$u_\peq{1}\rp{N,T}=u_\peq{2}\rp{T,N}=-1$$ $$u_\peq{1}\rp{T,N}=u_\peq{2}\rp{N,T}=2$$
			\end{itemize}
		\end{frame}	

		\begin{frame}
			\frametitle{Teoría de Juegos}
			\begin{table}[htbp!]
				\centering
				\resizebox{9cm}{!}{
					\begin{tabular}{|l|c|c|}\hline
						$J_\peq{1}$/$J_\peq{2}$&$N$&$T$ \\ [1ex] \hline
						$N$						 &$\rp{1,1}$&$\rp{-1,2}$ \\ [1ex] \hline 
						$T$						 &$\rp{2,-1}$&$\rp{0,0}$ \\ [1ex] \hline 
					\end{tabular}}
			\end{table}
		\end{frame}	

		\begin{frame}
			\frametitle{Teoría de Juegos}
			\begin{table}[htbp!]
				\centering
				\resizebox{9cm}{!}{
					\begin{tabular}{|l|c|c|}\hline
						$J_\peq{1}$/$J_\peq{2}$&$N$&$T$ \\ [1ex] \hline
						$N$						 &$\rp{1,1}$&$\rp{-1,2}$ \\ [1ex] \hline 
						$T$						 &$\rp{\color{blue}\underline{{\color{black}2}}\color{black},-1}$&$\rp{0,0}$ \\ [1ex] \hline 
					\end{tabular}}
			\end{table}
		\end{frame}	

		\begin{frame}
			\frametitle{Teoría de Juegos}
			\begin{table}[htbp!]
				\centering
				\resizebox{9cm}{!}{
					\begin{tabular}{|l|c|c|}\hline
						$J_\peq{1}$/$J_\peq{2}$&$N$&$T$ \\ [1ex] \hline
						$N$						 &$\rp{1,1}$&$\rp{-1,2}$ \\ [1ex] \hline 
						$T$						 &$\rp{\color{blue}\underline{{\color{black}2}}\color{black},-1}$&$\rp{\color{blue}\underline{{\color{black}0}}\color{black},0}$ \\ [1ex] \hline 
					\end{tabular}}
			\end{table}
		\end{frame}	

		\begin{frame}
			\frametitle{Teoría de Juegos}
			\begin{table}[htbp!]
				\centering
				\resizebox{9cm}{!}{
					\begin{tabular}{|l|c|c|}\hline
						$J_\peq{1}$/$J_\peq{2}$&$N$&$T$ \\ [1ex] \hline
						$N$						 &$\rp{1,1}$&$\rp{-1,\color{green}\underline{{\color{black}2}}\color{black}}$ \\ [1ex] \hline 
						$T$						 &$\rp{\color{blue}\underline{{\color{black}2}}\color{black},-1}$&$\rp{\color{blue}\underline{{\color{black}0}}\color{black},0}$ \\ [1ex] \hline 
					\end{tabular}}
			\end{table}
		\end{frame}	

		\begin{frame}
			\frametitle{Teoría de Juegos}
			\begin{table}[htbp!]
				\centering
				\resizebox{9cm}{!}{
					\begin{tabular}{|l|c|c|}\hline
						$J_\peq{1}$/$J_\peq{2}$&$N$&$T$ \\ [1ex] \hline
						$N$						 &$\rp{1,1}$&$\rp{-1,\color{green}\underline{{\color{black}2}}\color{black}}$ \\ [1ex] \hline 
						$T$						 &$\rp{\color{blue}\underline{{\color{black}2}}\color{black},-1}$&$\rp{\color{blue}\underline{{\color{black}0}}\color{black},\color{green}\underline{{\color{black}0}}\color{black}}$ \\ [1ex] \hline 
					\end{tabular}}
			\end{table}
		\end{frame}	

		\begin{frame}
			\frametitle{Teoría de Juegos}
			Mejores respuestas:
			\begin{table}[htbp!]
				\centering
				%\resizebox{9cm}{!}{
					\begin{tabular}{|c|c|}\hline
						$J_\peq{1}$				 &$J_\peq{2}$\\ [1ex] \hline
						$r_\peq{1}\rp{N}=T$&$r_\peq{2}\rp{N}=T$ \\ [1ex] 
						$r_\peq{1}\rp{T}=T$&$r_\peq{2}\rp{T}=T$ \\ [1ex] \hline 
					\end{tabular}%}
			\end{table}
		\end{frame}	

		\begin{frame}
			\frametitle{Teoría de Juegos}
			\begin{mydef}
				\textbf{Estrategia dominante:} Estrategia que es la mejor respuesta de un jugador para todas las acciones del otro jugador. Escogerá esta acción independientemente de lo que haga el otro.
			\end{mydef}
			En este ejemplo testificar (no cooperar) es una estrategia dominante para ambos jugadores.
		\end{frame}	

		\begin{frame}
			\frametitle{Teoría de Juegos}
			\begin{mydef}
				\textbf{Equilibrio de Nash:} Situación en que ningún jugador tiene incentivos a cambiar de estrategia ya que cada uno está jugando su mejor respuesta a la estrategia del otro (las estrategias de equilibrio son mutuas mejores respuestas).
			\end{mydef}
		\end{frame}	

		\begin{frame}
			\frametitle{Teoría de Juegos}
			\begin{itemize}
				\item En este ejemplo el equilibrio de Nash es $\rp{T,T}$ con pagos $\rp{0,0}$.
				\item Si ambos jugadores cooperaran (perfil de estrategias $\rp{N,N}$) podrían obtener un mejor resultado para ambos (perfil de pagos $\rp{1,1}$), pero sus incentivos individuales impiden la cooperación en equilibrio.
			\end{itemize}
		\end{frame}	
		
		\begin{frame}
			\frametitle{Oligopolio}
			\begin{mydef}
				\textbf{Oligopolio:} Estructura de mercado en la cual pocos vendedores ofrecen productos similares o idénticos.
			\end{mydef}
		\end{frame}		
	
		\begin{frame}
			\frametitle{Oligopolio}
			Al haber pocas empresas se genera:
			\begin{itemize}
				\item Interdependencia: Las acciones de cada empresa influyen en las ganancias de las demas (interacción estratégica).
				\item Tentación de cooperar: El oligopolio podría aumentar las ganancias de cada una de las empresas que lo componen si se coordinaran y actuaran como un monopolio (colusión).
			\end{itemize}
		\end{frame}	
	
		\begin{frame}
			\frametitle{Oligopolio}
			\begin{mydef}
				\begin{itemize}
					\item \textbf{Colusión:} Acuerdo entre las empresas de un mercado sobre las cantidades que producirán o los precios que cobrarán.
					\item \textbf{Cartel:} Grupo de empresas que se coluden.
				\end{itemize}
			\end{mydef}
		\end{frame}			

		\begin{frame}
			\frametitle{Oligopolio}
			Ejemplo: juego de colusión en un duopolio (estático)
			\begin{itemize}
				\item 2 empresas, $J_\peq{1}$ y $J_\peq{2}$ con funciones de costo típicas e idénticas (ver pizarra).
				\item Curva de demanda de mercado con pendiente negativa (ver pizarra).
			\end{itemize}
		\end{frame}			

		\begin{frame}
			\frametitle{Oligopolio}
			\begin{itemize}
				\item El bien producido por una empresa es sustituto perfecto del bien producido por la otra.
				\item El precio de mercado es el mismo para los productos de ambas empresas y la cantidad demandada depende de este precio.
			\end{itemize}
		\end{frame}			

		\begin{frame}
			\frametitle{Oligopolio}
			Las condiciones de costos y demanda son las siguientes
			\begin{figure}[htbp!]
				\begin{subfigure}[b]{0.49\textwidth}
					\centering
					\begin{tikzpicture}[scale=0.008]
						\draw [dashed,help lines] (0,103.99484)--(201.033,103.99484)--(201.033,0);
						%\draw [dashed,help lines] (0,108.3061483)--(171.7865311,108.3061483);
						%\draw [ultra thick,teal] plot [domain=0:274.6634] (\x,{250-.910205*\x}); % D
						%\draw [ultra thick,purple] plot [domain=0:160] (\x,{250-2*.910205*\x}); % IMg
						\draw [ultra thick,teal] plot [domain=50:290] (\x,{(.0072*\x-1.2)*\x+(.06*\x-10)^2+50}); % CMg
						\draw [ultra thick,purple] plot [domain=50:400] (\x,{(.06*\x-10)^2+50+10000/\x}); % CMeT
						\draw [<->,ultra thick] (0,450) -- (0,0) -- (450,0);
						\node [below] at (450,0) {$q$};
						\node [below] at (0,0) {\tiny 0};
						%\node [right,purple] at (160,-41.2656) {$IMg$};
						\node [left] at (0,450) {$p$};
						\node [right,teal] at (290,362.28) {\tiny $CMg$};
						\node [right,purple] at (400,271) {\tiny $CMeT$};
						\draw [fill,blue] (201.033,103.99484) circle [radius=8];
						\node [below] at (201.033,0) {\tiny 3.000};
						\node [left] at (0,103.99484) {\tiny 6.000};
						%\node [above right,teal] at (274.6634,0) {$D$};
						\node at (230,430) {\scriptsize \underline{Costos de cada firma}};
					\end{tikzpicture}
				\end{subfigure}	
				\begin{subfigure}[b]{0.49\textwidth}
					\centering
					\begin{tikzpicture}[scale=0.008]
						\draw [dashed,help lines] (0,103.99484)--(201.033,103.99484)--(201.033,0);
						%\draw [dashed,help lines] (0,1.3*128.45104)--(4/6*201.033,1.3*128.45104)--(4/6*201.033,0);
						\draw [ultra thick,white] plot [domain=0:220] (\x,{400-2*\x}); 
						\draw [ultra thick,teal] plot [domain=0:311.664] (\x,{292.96937-.94001746*\x}); % D
						\draw [<->,ultra thick] (0,450) -- (0,0) -- (450,0);
						\node [below] at (450,0) {$q$};
						\node [below] at (0,0) {\tiny 0};
						\node [right,white] at (220,-40) {$IMg$};
						\node [left] at (0,450) {$p$};
						%\draw [fill,blue] (4/6*201.033,1.3*128.45104) circle [radius=8];
						\draw [fill,blue] (201.033,103.99484) circle [radius=8];
						\node [below] at (201.033,0) {\tiny 6.000};
						%\node [below] at (4/6*201.033,0) {\tiny 4.000};
						%\node [left] at (0,1.3*128.45104) {\tiny 9.000};
						\node [left] at (0,103.99484) {\tiny 6.000};
						\node [above right,teal] at (311.664,0) {\tiny $D$};
						\node at (230,430) {\scriptsize \underline{Demanda de mercado}};
					\end{tikzpicture}
				\end{subfigure}
			\end{figure}	
		\end{frame}	

		\begin{frame}
			\frametitle{Oligopolio}
			\begin{itemize}
				\item Las empresas participan en un acuerdo de colusión.
				\item Cada una tienen dos estrategias disponibles: cumplir ($C$) o hacer trampa ($T$).
			\end{itemize}
		\end{frame}			
	
		\begin{frame}
			\frametitle{Oligopolio}
			\begin{itemize}
				\item Si ambas cumplen, actuarán como un monopolio y se repartirán la ganancia.
				\item Producirán $4.000$ unidades semanales y cobrarán $\$9.000$ por unidad.
			\end{itemize}
		\end{frame}					

		\begin{frame}
			\frametitle{Oligopolio}
			\begin{itemize}
				\item Supongamos que se reparten la producción en partes iguales, $2.000$ unidades cada una.
				\item $CMeT\rp{2.000}=\$8.000/\text{u}\implies\pi_\peq{1}=\pi_\peq{2}=\$2.000.000$
				\item En la notación del juego, $$u_\peq{1}\rp{C,C}=u_\peq{2}\rp{C,C}=2$$
			\end{itemize}
		\end{frame}	

		\begin{frame}
			\frametitle{Oligopolio}
			\begin{figure}[htbp!]
				\begin{subfigure}[b]{0.49\textwidth}
					\centering
					\begin{tikzpicture}[scale=0.008]
					\draw [fill,opacity=.2,green] (0,1.3*128.45104)--(4/6*201.033,1.3*128.45104)--(4/6*201.033,128.45104)--(0,128.45104);
						\draw [dashed,help lines] (0,128.45104)--(2/3*201.033,128.45104)--(2/3*201.033,0);
						\draw [dashed,help lines] (0,1.3*128.45104)--(450,1.3*128.45104);
						\draw [dashed,help lines] (2/3*201.033,128.45104)--(2/3*201.033,1.3*128.45104);
						%\draw [dashed,help lines] (0,108.3061483)--(171.7865311,108.3061483);
						%\draw [ultra thick,teal] plot [domain=0:274.6634] (\x,{250-.910205*\x}); % D
						%\draw [ultra thick,purple] plot [domain=0:160] (\x,{250-2*.910205*\x}); % IMg
						\draw [ultra thick,teal] plot [domain=50:290] (\x,{(.0072*\x-1.2)*\x+(.06*\x-10)^2+50}); % CMg
						\draw [ultra thick,purple] plot [domain=50:400] (\x,{(.06*\x-10)^2+50+10000/\x}); % CMeT
						\draw [<->,ultra thick] (0,450) -- (0,0) -- (450,0);
						\node [below] at (450,0) {$q$};
						\node [below] at (0,0) {\tiny 0};
						%\node [right,purple] at (160,-41.2656) {$IMg$};
						\node [left] at (0,450) {$p$};
						\node [right,teal] at (290,362.28) {\tiny $CMg$};
						\node [right,purple] at (400,271) {\tiny $CMeT$};
						\draw [fill,blue] (2/3*201.033,128.45104) circle [radius=8];
						\draw [fill,blue] (2/3*201.033,1.3*128.45104) circle [radius=8];
						\node [below] at (2/3*201.033,0) {\tiny 2.000};
						\node [left] at (0,1.3*128.45104) {\tiny 9.000};
						\node [left] at (0,128.45104) {\tiny 8.000};
						%\node [above right,teal] at (274.6634,0) {$D$};
						\node at (230,430) {\scriptsize \underline{Ambas cumplen}};
					\end{tikzpicture}
				\end{subfigure}	
				\begin{subfigure}[b]{0.49\textwidth}
					\centering
					\begin{tikzpicture}[scale=0.008]
						\draw [dashed,help lines] (0,1.3*128.45104)--(4/6*201.033,1.3*128.45104)--(4/6*201.033,0);
						\draw [ultra thick,white] plot [domain=0:220] (\x,{400-2*\x}); 
						\draw [ultra thick,teal] plot [domain=0:311.664] (\x,{292.96937-.94001746*\x}); % D
						\draw [ultra thick,purple] plot [domain=0:155.832] (\x,{292.96937-2*.94001746*\x}); % IMg
						\draw [ultra thick,teal] plot [domain=80:240] (\x,{((27/1250)*\x-24/5)*\x + 300}); % CMgT
						\draw [<->,ultra thick] (0,450) -- (0,0) -- (450,0);
						\node [below] at (450,0) {$q$};
						\node [below] at (0,0) {\tiny 0};
						\node [right,white] at (220,-40) {$IMg$};
						\node [right,teal] at (240,392.16) {\tiny $\sum{CMg}$};
						\node [above,teal] at (311.664,0) {\tiny $D$};
						\node [above right,purple] at (155.832,0) {\tiny $IMg$};
						\node [left] at (0,450) {$p$};
						\draw [fill,blue] (4/6*201.033,1.3*128.45104) circle [radius=8];
						\draw [fill,blue] (132.731,43.429597) circle [radius=8];
						%\draw [fill,blue] (201.033,103.99484) circle [radius=8];
						%\node [below] at (201.033,0) {\tiny 6.000};
						\node [below] at (4/6*201.033,0) {\tiny 4.000};
						\node [left] at (0,1.3*128.45104) {\tiny 9.000};
						%\node [left] at (0,103.99484) {\tiny 6.000};
						\node at (230,430) {\scriptsize \underline{Mercado}};
					\end{tikzpicture}
				\end{subfigure}
			\end{figure}	
		\end{frame}	

		\begin{frame}
			\frametitle{Oligopolio}
			\begin{itemize}
				\item Pero, como $p>CMg$, cada empresa podría aumentar sus ganancias produciendo más que lo acordado.
				\item Supongamos que una hace trampa y produce $3.000$ unidades mientras la otra cumple el acuerdo y produce $2.000$.
				\item La cantidad ofrecida es $5.000$ y el precio dado por la curva de demanda es $\$7.500/\text{u}$
			\end{itemize}
		\end{frame}	

		\begin{frame}
			\frametitle{Oligopolio}
			\begin{itemize}
				\item La empresa que hace trampa produce a $CMeT\rp{3.000}=\$6.000/\text{u}\implies\pi_\peq{i}=\$4.500.000$.
				\item La empresa que cumple sigue produciendo a $CMeT\rp{2.000}=\$8.000/\text{u}\implies\pi_\peq{j}=\$-1.000.000$.
				\item En la notación del juego, $$u_\peq{1}\rp{T,C}=u_\peq{2}\rp{C,T}=4.5$$ $$u_\peq{1}\rp{C,T}=u_\peq{2}\rp{T,C}=-1$$
			\end{itemize}
		\end{frame}	

		\begin{frame}
			\frametitle{Oligopolio}
			\begin{figure}[htbp!]
				\centering
				%\begin{subfigure}[b]{0.32\textwidth}
					%\resizebox{\linewidth}{!}{
						\begin{tikzpicture}[scale=0.0125]
							\draw [fill,opacity=.2,green] (0,116.22294)--(4/6*201.033,116.22294)--(4/6*201.033,128.45104)--(0,128.45104);
							\draw [dashed,help lines] (0,128.45104)--(2/3*201.033,128.45104)--(2/3*201.033,0);
							\draw [dashed,help lines] (0,116.22294)--(450,116.22294);
							\draw [dashed,help lines] (2/3*201.033,128.45104)--(2/3*201.033,116.22294);
							%\draw [dashed,help lines] (0,108.3061483)--(171.7865311,108.3061483);
							%\draw [ultra thick,teal] plot [domain=0:274.6634] (\x,{250-.910205*\x}); % D
							%\draw [ultra thick,purple] plot [domain=0:160] (\x,{250-2*.910205*\x}); % IMg
							\draw [ultra thick,teal] plot [domain=50:290] (\x,{(.0072*\x-1.2)*\x+(.06*\x-10)^2+50}); % CMg
							\draw [ultra thick,purple] plot [domain=50:400] (\x,{(.06*\x-10)^2+50+10000/\x}); % CMeT
							\draw [<->,ultra thick] (0,450) -- (0,0) -- (450,0);
							\node [below] at (450,0) {$q$};
							\node [below] at (0,0) {\tiny 0};
							%\node [right,purple] at (160,-41.2656) {$IMg$};
							\node [left] at (0,450) {$p$};
							\node [right,teal] at (290,362.28) {\tiny $CMg$};
							\node [right,purple] at (400,271) {\tiny $CMeT$};
							\draw [fill,blue] (2/3*201.033,128.45104) circle [radius=4];
							\draw [fill,blue] (2/3*201.033,116.22294) circle [radius=4];
							\node [below] at (2/3*201.033,0) {\tiny 2.000};
							\node [left] at (0,116.22294) {\tiny 7.500};
							\node [left] at (0,128.45104) {\tiny 8.000};
							%\node [above right,teal] at (274.6634,0) {$D$};
							\node at (230,430) {\scriptsize \underline{Empresa que cumple}};
						\end{tikzpicture}
					%}
				%\end{subfigure}
			\end{figure}	
		\end{frame}	

		\begin{frame}
			\frametitle{Oligopolio}
			\begin{figure}[htbp!]
				\centering
				%\begin{subfigure}[b]{0.32\textwidth}
					%\resizebox{\linewidth}{!}{
						\begin{tikzpicture}[scale=0.0125]
							\draw [fill,opacity=.2,green] (0,116.22294)--(201.033,116.22294)--(201.033,103.99484)--(0,103.99484);
							\draw [dashed,help lines] (0,103.99484)--(201.033,103.99484)--(201.033,0);
							\draw [dashed,help lines] (0,116.22294)--(450,116.22294);
							\draw [dashed,help lines] (201.033,103.99484)--(201.033,116.22294);
							%\draw [dashed,help lines] (0,108.3061483)--(171.7865311,108.3061483);
							%\draw [ultra thick,teal] plot [domain=0:274.6634] (\x,{250-.910205*\x}); % D
							%\draw [ultra thick,purple] plot [domain=0:160] (\x,{250-2*.910205*\x}); % IMg
							\draw [ultra thick,teal] plot [domain=50:290] (\x,{(.0072*\x-1.2)*\x+(.06*\x-10)^2+50}); % CMg
							\draw [ultra thick,purple] plot [domain=50:400] (\x,{(.06*\x-10)^2+50+10000/\x}); % CMeT
							\draw [<->,ultra thick] (0,450) -- (0,0) -- (450,0);
							\node [below] at (450,0) {$q$};
							\node [below] at (0,0) {\tiny 0};
							%\node [right,purple] at (160,-41.2656) {$IMg$};
							\node [left] at (0,450) {$p$};
							\node [right,teal] at (290,362.28) {\tiny $CMg$};
							\node [right,purple] at (400,271) {\tiny $CMeT$};
							\draw [fill,blue] (201.033,103.99484) circle [radius=4];
							\draw [fill,blue] (201.033,116.22294) circle [radius=4];
							\node [below] at (201.033,0) {\tiny 3.000};
							\node [left] at (0,116.22294) {\tiny 7.500};
							\node [left] at (0,103.99484) {\tiny 6.000};
							%\node [above right,teal] at (274.6634,0) {$D$};
							\node at (230,430) {\scriptsize \underline{Empresa que hace trampa}};
						\end{tikzpicture}
					%}
				%\end{subfigure}
			\end{figure}	
		\end{frame}	

		\begin{frame}
			\frametitle{Oligopolio}
			\begin{figure}[htbp!]
				\centering
				%\begin{subfigure}[b]{0.32\textwidth}
					%\resizebox{\linewidth}{!}{
						\begin{tikzpicture}[scale=0.0125]
						\draw [dashed,help lines] (0,124.52458)--(175.91409,124.52458)--(175.91409,0);
						\draw [ultra thick,white] plot [domain=0:220] (\x,{400-2*\x}); 
						\draw [ultra thick,teal] plot [domain=0:311.664] (\x,{292.96937-.94001746*\x}); % D
						\draw [<->,ultra thick] (0,450) -- (0,0) -- (450,0);
						\node [below] at (450,0) {$q$};
						\node [below] at (0,0) {\tiny 0};
						\node [right,white] at (220,-40) {$IMg$};
						\node [above,teal] at (311.664,0) {\tiny $D$};
						\node [left] at (0,450) {$p$};
						\draw [fill,blue] (175.91409,124.52458) circle [radius=8];
						%\node [below] at (201.033,0) {\tiny 6.000};
						\node [below] at (5/6*201.033,0) {\tiny 5.000};
						\node [left] at (0,135.49059) {\tiny 7.500};
						%\node [left] at (0,103.99484) {\tiny 6.000};
						\node at (230,430) {\scriptsize \underline{Mercado}};
						\end{tikzpicture}
					%}
				%\end{subfigure}
			\end{figure}	
		\end{frame}	

		\begin{frame}
			\frametitle{Oligopolio}
			\begin{itemize}
				\item Por último, si ambas hacen trampa y producen $3.000$ unidades cada una, el precio será $6.000$.
				\item $CMeT\rp{3.000}=\$6.000/\text{u}\implies\pi_\peq{1}=\pi_\peq{2}=\$0.$
				\item En la notación del juego, $$u_\peq{1}\rp{T,T}=u_\peq{2}\rp{T,T}=0$$
			\end{itemize}
		\end{frame}	

		\begin{frame}
			\frametitle{Oligopolio}
			\begin{figure}[htbp!]
				\begin{subfigure}[b]{0.49\textwidth}
					\centering
					\begin{tikzpicture}[scale=0.008]
						\draw [dashed,help lines] (0,103.99484)--(201.033,103.99484)--(201.033,0);
						%\draw [dashed,help lines] (0,108.3061483)--(171.7865311,108.3061483);
						%\draw [ultra thick,teal] plot [domain=0:274.6634] (\x,{250-.910205*\x}); % D
						%\draw [ultra thick,purple] plot [domain=0:160] (\x,{250-2*.910205*\x}); % IMg
						\draw [ultra thick,teal] plot [domain=50:290] (\x,{(.0072*\x-1.2)*\x+(.06*\x-10)^2+50}); % CMg
						\draw [ultra thick,purple] plot [domain=50:400] (\x,{(.06*\x-10)^2+50+10000/\x}); % CMeT
						\draw [<->,ultra thick] (0,450) -- (0,0) -- (450,0);
						\node [below] at (450,0) {$q$};
						\node [below] at (0,0) {\tiny 0};
						%\node [right,purple] at (160,-41.2656) {$IMg$};
						\node [left] at (0,450) {$p$};
						\node [right,teal] at (290,362.28) {\tiny $CMg$};
						\node [right,purple] at (400,271) {\tiny $CMeT$};
						\draw [fill,blue] (201.033,103.99484) circle [radius=8];
						\node [below] at (201.033,0) {\tiny 3.000};
						\node [left] at (0,103.99484) {\tiny 6.000};
						%\node [above right,teal] at (274.6634,0) {$D$};
						\node at (230,430) {\scriptsize \underline{Ambas hacen trampa}};
					\end{tikzpicture}
				\end{subfigure}	
				\begin{subfigure}[b]{0.49\textwidth}
					\centering
					\begin{tikzpicture}[scale=0.008]
						\draw [dashed,help lines] (0,103.99484)--(201.033,103.99484)--(201.033,0);
						%\draw [dashed,help lines] (0,1.3*128.45104)--(4/6*201.033,1.3*128.45104)--(4/6*201.033,0);
						\draw [ultra thick,white] plot [domain=0:220] (\x,{400-2*\x}); 
						\draw [ultra thick,teal] plot [domain=180:240] (\x,{((27/1250)*\x-24/5)*\x + 300-100}); % CMgT
						\draw [ultra thick,teal] plot [domain=0:311.664] (\x,{292.96937-.94001746*\x}); % D
						\draw [<->,ultra thick] (0,450) -- (0,0) -- (450,0);
						\node [below] at (450,0) {$q$};
						\node [below] at (0,0) {\tiny 0};
						\node [right,white] at (220,-40) {$IMg$};
						\node [left] at (0,450) {$p$};
						%\draw [fill,blue] (4/6*201.033,1.3*128.45104) circle [radius=8];
						\draw [fill,blue] (201.033,103.99484) circle [radius=8];
						\node [below] at (201.033,0) {\tiny 6.000};
						%\node [below] at (4/6*201.033,0) {\tiny 4.000};
						%\node [left] at (0,1.3*128.45104) {\tiny 9.000};
						\node [left] at (0,103.99484) {\tiny 6.000};
						\node [above right,teal] at (311.664,0) {\tiny $D$};
						\node [right,teal] at (240,292.16) {\tiny $\sum{CMg}$};
						\node at (230,430) {\scriptsize \underline{Mercado}};
					\end{tikzpicture}
				\end{subfigure}
			\end{figure}	
		\end{frame}	

		\begin{frame}
			\frametitle{Oligopolio}
			\begin{table}[htbp!]
				\centering
				\resizebox{9cm}{!}{
					\begin{tabular}{|l|c|c|}\hline
						$J_\peq{1}$/$J_\peq{2}$&$T$&$C$ \\ [1ex] \hline
						$T$						 &$\rp{0,0}$&$\rp{4.5,-1}$ \\ [1ex] \hline 
						$C$						 &$\rp{-1,4.5}$&$\rp{2,2}$ \\ [1ex] \hline 
					\end{tabular}}
			\end{table}
		\end{frame}	

		\begin{frame}
			\frametitle{Oligopolio}
			\begin{table}[htbp!]
				\centering
				\resizebox{9cm}{!}{
					\begin{tabular}{|l|c|c|}\hline
						$J_\peq{1}$/$J_\peq{2}$&$T$&$C$ \\ [1ex] \hline
						$T$						 &$\rp{\color{blue}\underline{{\color{black}0}}\color{black},0}$&$\rp{4.5,-1}$ \\ [1ex] \hline 
						$C$						 &$\rp{-1,4.5}$&$\rp{2,2}$ \\ [1ex] \hline 
					\end{tabular}}
			\end{table}
		\end{frame}	

		\begin{frame}
			\frametitle{Oligopolio}
			\begin{table}[htbp!]
				\centering
				\resizebox{9cm}{!}{
					\begin{tabular}{|l|c|c|}\hline
						$J_\peq{1}$/$J_\peq{2}$&$T$&$C$ \\ [1ex] \hline
						$T$						 &$\rp{\color{blue}\underline{{\color{black}0}}\color{black},0}$&$\rp{\color{blue}\underline{{\color{black}4.5}}\color{black},-1}$ \\ [1ex] \hline 
						$C$						 &$\rp{-1,4.5}$&$\rp{2,2}$ \\ [1ex] \hline 
					\end{tabular}}
			\end{table}
		\end{frame}	

		\begin{frame}
			\frametitle{Oligopolio}
			\begin{table}[htbp!]
				\centering
				\resizebox{9cm}{!}{
					\begin{tabular}{|l|c|c|}\hline
						$J_\peq{1}$/$J_\peq{2}$&$T$&$C$ \\ [1ex] \hline
						$T$						 &$\rp{\color{blue}\underline{{\color{black}0}}\color{black},\color{green}\underline{{\color{black}0}}\color{black}}$&$\rp{\color{blue}\underline{{\color{black}4.5}}\color{black},-1}$ \\ [1ex] \hline 
						$C$						 &$\rp{-1,4.5}$&$\rp{2,2}$ \\ [1ex] \hline 
					\end{tabular}}
			\end{table}
		\end{frame}	

		\begin{frame}
			\frametitle{Oligopolio}
			\begin{table}[htbp!]
				\centering
				\resizebox{9cm}{!}{
					\begin{tabular}{|l|c|c|}\hline
						$J_\peq{1}$/$J_\peq{2}$&$T$&$C$ \\ [1ex] \hline
						$T$						 &$\rp{\color{blue}\underline{{\color{black}0}}\color{black},\color{green}\underline{{\color{black}0}}\color{black}}$&$\rp{\color{blue}\underline{{\color{black}4.5}}\color{black},-1}$ \\ [1ex] \hline 
						$C$						 &$\rp{-1,\color{green}\underline{{\color{black}4.5}}\color{black}}$&$\rp{2,2}$ \\ [1ex] \hline 
					\end{tabular}}
			\end{table}
		\end{frame}	

		\begin{frame}
			\frametitle{Oligopolio}
			Mejores respuestas:
			\begin{table}[htbp!]
				\centering
				%\resizebox{9cm}{!}{
					\begin{tabular}{|c|c|}\hline
						$J_\peq{1}$				 &$J_\peq{2}$\\ [1ex] \hline
						$r_\peq{1}\rp{T}=T$&$r_\peq{2}\rp{T}=T$ \\ [1ex] 
						$r_\peq{1}\rp{C}=T$&$r_\peq{2}\rp{C}=T$ \\ [1ex] \hline 
					\end{tabular}%}
			\end{table}
		\end{frame}	

		\begin{frame}
			\frametitle{Oligopolio}
			\begin{itemize}
				\item En equlibrio ambas empresas hacen trampa y el acuerdo de colusión no se sostiene.
				\item Se trata de un dilema del prisionero: si cooperaran obtendrían un mejor resultado pero sus incentivos individuales inhiben la cooperación.
			\end{itemize}
		\end{frame}	
	
		\begin{frame}
			\frametitle{Oligopolio}
			Algunas consideraciones:
			\begin{itemize}
				\item La colusión es sostenible en equilibrio en un contexto de juegos repetidos (bajo ciertas condiciones).
				\item Cuando no hay equilibrio colusivo (modelo estático), la cantidad de empresas afecta el resultado. $$\Delta^\peq{+}n\implies\text{mercado más parecido a competencia perfecta}$$
			\end{itemize}
		\end{frame}			

		\begin{frame}
			\frametitle{Oligopolio}
			En su decisión marginal de producción el oligopolista enfrenta dos incentivos:
			\begin{itemize}
				\item \textbf{Efecto producción:} $p>CMg\implies\Delta^\peq{+}\pi_\peq{i}$ al vender la unidad adicional.
				\item \textbf{Efecto precio:} $\Delta^\peq{+}q_\peq{i}\implies\Delta^\peq{+}Q\implies\Delta^\peq{-}p\implies\Delta^\peq{-}\pi_peq{i}$.
				\item A medida que aumenta la cantidad de empresas, el efecto precio se va haciendo menos relevante.
			\end{itemize}
		\end{frame}		

		\begin{frame}
			\frametitle{Oligopolio}
			Política pública:
			\begin{itemize}
				\item La cooperación entre oligopolistas genera una mayor pérdida social (equilibrio monopólico).
				\item Un objetivo de política pública relevante es inducir a los oligopolios a competir en lugar de cooperar.
				\item Una herramienta disponible son las leyes antimonopolio.
			\end{itemize}
		\end{frame}		

		\begin{frame}
			\frametitle{Oligopolio}
			Leyes antimonopolio:
			\begin{itemize}
				\item Suelen prohibir los acuerdos entre empresas sobre cuotas de producción y fijación de precios.
				\item También se usan para impedir fusiones que causarían que una sola empresa tuviera excesivo poder en el mercado.
			\end{itemize}
		\end{frame}		

		\begin{frame}
			\frametitle{Oligopolio}
			Controversia:
			\begin{itemize}
				\item ¿Qué tipo de conducta se debe prohibir?
				\item Se ha prohibido algunas prácticas de negocio cuyo efecto no es evidente:
					\begin{itemize}
						\item Precios de reventa (exigir un precio de venta a público a los minoristas).
						\item Precios depredatorios.
						\item Venta atada.
					\end{itemize}
			\end{itemize}
		\end{frame}	
		
	\section{Externalidades}
	
		\begin{frame}
			\frametitle{Externalidades}
			\begin{mydef}
				\textbf{Externalidad:} efecto no compensado de las acciones de una persona o empresa sobre el bienestar de un tercero
			\end{mydef}
		\end{frame}

		\begin{frame}
			\frametitle{Externalidades}
			Hasta aquí habíamos supesto que:
				\begin{itemize}
					\item Nadie más que los participantes del mercado obtiene beneficios $$\implies BMg \text{ privado} = BMg \text{ social}$$
					\item No existen costos adicionales a los que deben asumir las empresas que participan en el mercado $$\implies CMg \text{ privado} = CMg \text{ social}$$
				\end{itemize}
		\end{frame}

		\begin{frame}
			\frametitle{Externalidades}
				\begin{itemize}
					\item Bajo estos supuestos, el bienestar de la sociedad puede ser medido como la suma del excedente del productor y el excedente del consumidor. En presencia de externalidades, esto ya no es cierto.
					\item La existencia de efectos secundarios ligados al consumo o la producción implica que el equilibrio del mercado es ineficiente desde el punto de vista de la sociedad.
				\end{itemize}
		\end{frame}

		\begin{frame}
			\frametitle{Externalidades}
			Ejemplos:
			\begin{table}[htbp!]
				\centering
				\resizebox{11.5cm}{!}{
					\begin{tabular}{|l|c|c|}\hline
																 &Externalidad en la producción&Externalidad en el consumo\\ [1ex] \hline
						Externalidad positiva&\parbox{4cm}{\begin{itemize}\item Difusión tecnológica\item Agricultores y apicultores\end{itemize}}&\parbox{4cm}{\begin{itemize}\item Educación\item Vacunas\end{itemize}}\\ [1ex] \hline
						Externalidad negativa&\parbox{4cm}{\begin{itemize}\item Contaminación ambiental\item Construcción de edificios\end{itemize}}&\parbox{4cm}{\begin{itemize}\item Música a alto volumen\item Cigarro\end{itemize}} \\ [1ex] \hline
					\end{tabular}}
			\end{table}
		\end{frame}	

		\begin{frame}
			\frametitle{Externalidades}
			Ejemplo: Contaminación
			
			\centering
			\begin{tikzpicture}[scale=0.0125]
				\draw [fill=black,opacity=0.2] (100,100) -- (100,200) -- (50,150);
				\draw [dashed,help lines] (0,100) -- (100,100) -- (100,0);
				\draw [dashed,help lines] (100,100) -- (100,200);
				\draw [dashed,help lines] (0,150) -- (50,150) -- (50,0);
				\draw [ultra thick,teal] plot [domain=0:200](\x,{200-\x});
				\draw [ultra thick,purple] plot [domain=0:200](\x,{\x+100});
				\draw [ultra thick,teal] plot [domain=0:300](\x,{\x});
				\draw [<->,ultra thick] (0,350) -- (0,0) -- (350,0);
				\node [below] at (350,0) {$q$};
				\node [left] at (0,350) {$p$};
				\node [right,teal] at (300,300) {\tiny $S$};
				\node [above right,teal] at (200,0) {\tiny $D$};
				\node [above right,purple] at (200,300) {\tiny $CMg_\peq{\text{social}}$};
				\node [left] at (0,100) {\tiny $p^{*}_\peq{M}$};
				\node [below] at (100,0) {\tiny $q^{*}_\peq{M}$};
				\node [left] at (0,150) {\tiny $p^{*}_\peq{S}$};
				\node [below] at (50,0) {\tiny $q^{*}_\peq{S}$};
			\end{tikzpicture}
		\end{frame}	
		
		\begin{frame}
			\frametitle{Externalidades}
			Ejemplo: Educación
			
			\centering
			\begin{tikzpicture}[scale=0.0125]
				\draw [dashed,help lines] (0,100) -- (100,100) -- (100,0);
				\draw [dashed,help lines] (100,100) -- (100,200);
				\draw [dashed,help lines] (0,150) -- (150,150) -- (150,0);
				\draw [fill=black,opacity=0.2] (100,100) -- (100,200) -- (150,150);
				\draw [ultra thick,teal] plot [domain=0:200](\x,{200-\x});
				\draw [ultra thick,purple] plot [domain=0:300](\x,{300-\x});
				\draw [ultra thick,teal] plot [domain=0:300](\x,{\x});
				\draw [<->,ultra thick] (0,350) -- (0,0) -- (350,0);
				\node [below] at (350,0) {$q$};
				\node [left] at (0,350) {$p$};
				\node [right,teal] at (300,300) {\tiny $S$};
				\node [above right,teal] at (200,0) {\tiny $D$};
				\node [above right,purple] at (300,0) {\tiny $BMg_\peq{\text{social}}$};
				\node [left] at (0,100) {\tiny $p^{*}_\peq{M}$};
				\node [below] at (100,0) {\tiny $q^{*}_\peq{M}$};
				\node [left] at (0,150) {\tiny $p^{*}_\peq{S}$};
				\node [below] at (150,0) {\tiny $q^{*}_\peq{S}$};
			\end{tikzpicture}
		\end{frame}			
	
		\begin{frame}
			\frametitle{Externalidades}
			Modelo:
			\begin{align}
				\text{Demanda: } p^\peq{d}&=a-bq^\peq{d}& \\
				\text{Oferta: }  p^\peq{s}&=c+dq^\peq{s}& \\
				BMg \text{ social: } p^\peq{d}&=a-bq^\peq{d}+E\\
				\text{No-impuesto: } p^\peq{d}&=p^\peq{s}=p& \\
				\text{Equilibrio: } q^\peq{d}&=q^\peq{s}=q^\peq{*}&
			\end{align}
		\end{frame}	

		\begin{frame}
			\frametitle{Externalidades}
			donde
			\begin{itemize}
				\item $E>0\implies$ externalidad positiva
				\item $E<0\implies$ externalidad negativa
			\end{itemize}
		\end{frame}	
	
		\begin{frame}
			\frametitle{Externalidades}
			Resolviendo:
			\begin{align*}
				\text{Equilibrio de mercado: } q^\peq{*}_\peq{M}&=\frac{a-c}{b+d}& \\
				\text{Óptimo social: }  q^\peq{*}_\peq{S}&=\frac{a-c}{b+d}+\frac{E}{b+d}& \\
			\end{align*}
		\end{frame}	

		\begin{frame}
			\frametitle{Externalidades}
			Estudiaremos dos tipos de solución al problema de las externalidades:
				\begin{itemize}
					\item Soluciones privadas
					\item Soluciones de política pública
				\end{itemize}
		\end{frame}	
		
		\begin{frame}
			\frametitle{Externalidades}
			Soluciones privadas:
				\begin{itemize}
					\item Códigos y sanciones morales
					\item Instituciones de beneficencia
					\item Integración de negocios
					\item Contratos
				\end{itemize}
		\end{frame}			

		\begin{frame}
			\frametitle{Externalidades}
				\textbf{Teorema de Coase:} 
				
				\vspace{.3cm}
				Si 
					\begin{itemize}
						\item las partes involucradas pueden negociar sin costos de transacción y
						\item los derechos de propiedad están plenamente asignados.
					\end{itemize}
					Entonces,
					\begin{itemize}
						\item las partes (el mercado) resolverán el problema asignando eficientemente los recursos, independientemente de la asignación de los derechos de propiedad.
					\end{itemize}
		\end{frame}			

		\begin{frame}
			\frametitle{Externalidades}
			Hay dos enfoques de política pública:
				\begin{itemize}
					\item Políticas de orden y control: regulación (exigir o prohibir actividades)
					\item Políticas basadas en el mercado: impuestos/subsidios correctivos, permisos transables
				\end{itemize}
		\end{frame}	

		\begin{frame}
			\frametitle{Externalidades}
			Objeciones a las políticas de orden y control:
				\begin{itemize}
					\item Requieren información detallada
					\item No consideran heterogeneidad (por ejemplo en costos de reducir la contaminación)
					\item No son contingentes a los costos de las empresas (que varían en el tiempo)
					\item Generalmente ineficientes
				\end{itemize}
		\end{frame}	

		\begin{frame}
			\frametitle{Externalidades}
			Impuestos y subsidios:
				\begin{itemize}
					\item Si $q^\peq{*}_\peq{M}>q^\peq{*}_\peq{S}$ se puede implementar un impuesto para reducir la cantidad.
					\item Si $q^\peq{*}_\peq{M}<q^\peq{*}_\peq{S}$ se puede implementar un subsidio para incrementar la cantidad.
					\item El monto del impuesto/subsidio ideal (que elimina la ineficiencia) es el costo/beneficio externo de la actividad.
				\end{itemize}
		\end{frame}	
	
		\begin{frame}
			\frametitle{Externalidades}
			Permisos transables:
				\begin{itemize}
					\item Una forma de reducir la contaminación hasta una cantidad objetivo es emitir permisos para contaminar.
					\item Ejemplo: si se desea reducir la contaminación a un nivel $\overline{c}$, basta con emitir $\overline{c}$ permisos para emitir una unidad de contaminación y repartirlos entre las empresas.
				\end{itemize}
		\end{frame}		

		\begin{frame}
			\frametitle{Externalidades}
				\begin{itemize}
					\item Si las empresas pueden negociar libremente los permisos entre ellas, aquellas con menores costos de reducción de contaminación venderán a aquellas con mayores costos.
					\item En equilibrio se logra la cantidad deseada de contaminación, pero en una forma eficiente.
				\end{itemize}
		\end{frame}	
	
	\section{Bienes Públicos y Recursos Comunes}

		\begin{frame}
			\frametitle{Bienes Públicos}
			Para entender el problema de los bienes públicos y los recursos comunes, nos concentraremos en 2 características de los bienes que nos permitirán clasificarlos en 4 categorías.
		\end{frame}

		\begin{frame}
			\frametitle{Bienes Públicos}
			\begin{itemize}
				\item \textbf{Excluibilidad:} Es posible impedir que una persona lo consuma.
					\begin{itemize}
						\item Ejemplos de bienes excluibles: un chocolate, ideas con patente, carreteras con peaje, bienes físicos en general.
						\item Ejemplos de bienes no excluibles: ideas sin patente, defensa nacional, peces en el mar, espectáculo de fuegos artificiales.
					\end{itemize}
			\end{itemize}
		\end{frame}

		\begin{frame}
			\frametitle{Bienes Públicos}
			\begin{itemize}
				\item \textbf{Rivalidad en el consumo:} El consumo de una por parte de una persona reduce el consumo potencial de otras (se gasta).
					\begin{itemize}
						\item Ejemplos de bienes rivales: un chocolate, bienes físicos en general, carreteras congestionadas.
						\item Ejemplos de bienes no rivales: ideas en general, espectáculo de fuegos artificiales, sistemas de advertencia de catástrofe.
					\end{itemize}
			\end{itemize}
		\end{frame}

		\begin{frame}
			\frametitle{Bienes Públicos}
			Clasificación:
			\begin{table}[htbp!]
				\centering
				\resizebox{11.5cm}{!}{
					\begin{tabular}{|l|c|c|}\hline
											&Excluible				& No excluible		 \\ [1ex] \hline
						Rival			&\parbox{5cm}{Bienes privados\tiny\begin{itemize}\item Vaso de helado \item Autopista congestionada con peaje\end{itemize}}&\parbox{5cm}{Recursos comunes\tiny\begin{itemize}\item Peces en el mar \item Autopista congestionada libre de peaje\end{itemize}} \\ [1ex] \hline
						No rival	&\parbox{5cm}{Bienes reservados\tiny\begin{itemize}\item TV por cable\item Autopista no congestionada con peaje\end{itemize}}&\parbox{5cm}{Bienes públicos\tiny\begin{itemize}\item Defensa nacional\item Autopista no congestionada y libre de peaje\end{itemize}}	 \\ [1ex] \hline
					\end{tabular}}
			\end{table}
		\end{frame}	

		\begin{frame}
			\frametitle{Bienes Públicos}
			\begin{mydef}
				\textbf{Bien público:} Bien que no es excluible ni rival en el consumo.
			\end{mydef}
		\end{frame}

		\begin{frame}
			\frametitle{Bienes Públicos}
			Ejemplo: provisión de alumbrado público
			\begin{itemize}
				\item 2 vecinos, $J_\peq{1}$ y $J_\peq{2}$
				\item Cada uno valora el alumbrado público en $\$3$
				\item Costo de provisión de alumbrado público es $\$4$
				\item Notar que el costo es menor que la suma de las valoraciones
			\end{itemize}
		\end{frame}	

		\begin{frame}
			\frametitle{Bienes Públicos}
			\begin{itemize}
				\item Basta que un vecino contribuya para que se provea el alumbrado público
				\item Si ambos contribuyen, el costo se reparte en partes iguales, $\$2$ c/u
				\item Si sólo uno contribuye, paga $\$4$
				\item Costo de provisión de alumbrado público es $\$4$
				\item Si ninguno contribuye, no se provee el servicio
			\end{itemize}
		\end{frame}	

		\begin{frame}
			\frametitle{Bienes Públicos}
			\begin{itemize}
				\item Ambos vecinos deciden simultáneamente entre contribuir ($C$) y no contribuir ($N$) 
				\item Si ambos contribuyen, cada uno paga $\$2$ por un bien que valora en $\$3$, obteniendo un excedente de $\$1$ $$u_\peq{1}\rp{C,C}=u_\peq{2}\rp{C,C}=1$$
			\end{itemize}
		\end{frame}	

		\begin{frame}
			\frametitle{Bienes Públicos}
			\begin{itemize}
				\item Si uno contribuye y el otro no, el que contribuye paga $\$4$ por un bien que valora en $\$3$, obteniendo un excedente de $\$-1$, mientras el que no lo hace obtiene un beneficio de $\$3$ sin pagar $$u_\peq{1}\rp{N,C}=u_\peq{2}\rp{C,N}=3$$ $$u_\peq{1}\rp{C,N}=u_\peq{2}\rp{N,C}=-1$$ 
			\end{itemize}
		\end{frame}	

		\begin{frame}
			\frametitle{Bienes Públicos}
			\begin{itemize}
				\item Por último, si ninguno contribuye no se provee el servicio y ambos tienen excedente de $\$0$ $$u_\peq{1}\rp{N,N}=u_\peq{2}\rp{N,N}=0$$ 
			\end{itemize}
		\end{frame}	

		\begin{frame}
			\frametitle{Bienes Públicos}
			\begin{table}[htbp!]
				\centering
				\resizebox{9cm}{!}{
					\begin{tabular}{|l|c|c|}\hline
						$J_\peq{1}$/$J_\peq{2}$&$N$&$C$ \\ [1ex] \hline
						$N$						 &$\rp{0,0}$&$\rp{3,-1}$ \\ [1ex] \hline 
						$C$						 &$\rp{-1,3}$&$\rp{1,1}$ \\ [1ex] \hline 
					\end{tabular}}
			\end{table}
		\end{frame}	

		\begin{frame}
			\frametitle{Bienes Públicos}
			\begin{table}[htbp!]
				\centering
				\resizebox{9cm}{!}{
					\begin{tabular}{|l|c|c|}\hline
						$J_\peq{1}$/$J_\peq{2}$&$N$&$C$ \\ [1ex] \hline
						$N$						 &$\rp{\color{blue}\underline{{\color{black}0}}\color{black},0}$&$\rp{3,-1}$ \\ [1ex] \hline 
						$C$						 &$\rp{-1,3}$&$\rp{1,1}$ \\ [1ex] \hline 
					\end{tabular}}
			\end{table}
		\end{frame}	

		\begin{frame}
			\frametitle{Bienes Públicos}
			\begin{table}[htbp!]
				\centering
				\resizebox{9cm}{!}{
					\begin{tabular}{|l|c|c|}\hline
						$J_\peq{1}$/$J_\peq{2}$&$N$&$C$ \\ [1ex] \hline
						$N$						 &$\rp{\color{blue}\underline{{\color{black}0}}\color{black},0}$&$\rp{\color{blue}\underline{{\color{black}3}}\color{black},-1}$ \\ [1ex] \hline 
						$C$						 &$\rp{-1,3}$&$\rp{1,1}$ \\ [1ex] \hline 
					\end{tabular}}
			\end{table}
		\end{frame}	

		\begin{frame}
			\frametitle{Bienes Públicos}
			\begin{table}[htbp!]
				\centering
				\resizebox{9cm}{!}{
					\begin{tabular}{|l|c|c|}\hline
						$J_\peq{1}$/$J_\peq{2}$&$N$&$C$ \\ [1ex] \hline
						$N$						 &$\rp{\color{blue}\underline{{\color{black}0}}\color{black},\color{green}\underline{{\color{black}0}}\color{black}}$&$\rp{\color{blue}\underline{{\color{black}3}}\color{black},-1}$ \\ [1ex] \hline 
						$C$						 &$\rp{-1,3}$&$\rp{1,1}$ \\ [1ex] \hline 
					\end{tabular}}
			\end{table}
		\end{frame}	

		\begin{frame}
			\frametitle{Bienes Públicos}
			\begin{table}[htbp!]
				\centering
				\resizebox{9cm}{!}{
					\begin{tabular}{|l|c|c|}\hline
						$J_\peq{1}$/$J_\peq{2}$&$N$&$C$ \\ [1ex] \hline
						$N$						 &$\rp{\color{blue}\underline{{\color{black}0}}\color{black},\color{green}\underline{{\color{black}0}}\color{black}}$&$\rp{\color{blue}\underline{{\color{black}3}}\color{black},-1}$ \\ [1ex] \hline 
						$C$						 &$\rp{-1,\color{green}\underline{{\color{black}3}}\color{black}}$&$\rp{1,1}$ \\ [1ex] \hline 
					\end{tabular}}
			\end{table}
		\end{frame}	

		\begin{frame}
			\frametitle{Bienes Públicos}
			Mejores respuestas:
			\begin{table}[htbp!]
				\centering
				%\resizebox{9cm}{!}{
					\begin{tabular}{|c|c|}\hline
						$J_\peq{1}$				 &$J_\peq{2}$\\ [1ex] \hline
						$r_\peq{1}\rp{N}=N$&$r_\peq{2}\rp{N}=N$ \\ [1ex] 
						$r_\peq{1}\rp{C}=N$&$r_\peq{2}\rp{C}=N$ \\ [1ex] \hline 
					\end{tabular}%}
			\end{table}
		\end{frame}	

		\begin{frame}
			\frametitle{Bienes Públicos}
			\begin{itemize}
				\item En equlibrio ningún vecino contribuye y no se provee alumbrado público.
				\item Se trata de un dilema del prisionero: si cooperaran obtendrían un mejor resultado pero sus incentivos individuales inhiben la cooperación. Ambos tienen incentivos a actuar como parásitos o \emph{free-riders}
			\end{itemize}
		\end{frame}	

		\begin{frame}
			\frametitle{Bienes Públicos}
			\begin{mydef}
				\textbf{Parásito o \emph{free-rider}:} Persona que recibe el beneficio de un bien evadiendo el pago
			\end{mydef}
		\end{frame}	

		\begin{frame}
			\frametitle{Bienes Públicos}
			``¿Para qué pagar por el bien si otro lo hace por mí?''
			\begin{itemize}
				\item Al tratarse de un bien no-excluible y no-rival en el consumo, se genera una \textbf{externalidad positiva}: cuando un individuo paga, otros pueden consumir el bien sin pagar.
				\item Conclusion: el problema del \emph{free-rider} impide que el mercado provea los bienes públicos en forma eficiente.
			\end{itemize}
		\end{frame}

		\begin{frame}
			\frametitle{Bienes Públicos}
			Política pública:
			\begin{itemize}
				\item El gobierno puede proveer directamente el bien público o financiar la provisión a través de privados.
				\item Pero no es tan sencillo: debe determinar qué tipo de bienes y en qué cantidad financiar.
				\item Dificultad: no es fácil estimar el valor del bien para la sociedad. Individuos no tienen incentivos a revelar su verdadera valoración.
			\end{itemize}
		\end{frame}

		\begin{frame}
			\frametitle{Recursos Comunes}
			\begin{mydef}
				\textbf{Recurso Común:} Bien que no es excluible, pero sí rival en el consumo.
			\end{mydef}
		\end{frame}

		\begin{frame}
			\frametitle{Recursos Comunes}
			Ejemplo: terreno compartido para pastar
			\begin{itemize}
				\item 2 pastores, $J_\peq{1}$ y $J_\peq{2}$
				\item Deciden simultáneamente cuánto pastarán sus ovejas
				\item Dos alternativas: media jornada (M) o jornada completa (C)
				\item Si una oveja pasta más, se puede vender a mejor precio pero perjudica la reproducción del pasto
			\end{itemize}
		\end{frame}	

		\begin{frame}
			\frametitle{Recursos Comunes}
			\begin{table}[htbp!]
				\centering
				\resizebox{9cm}{!}{
					\begin{tabular}{|l|c|c|}\hline
						$J_\peq{1}$/$J_\peq{2}$&$M$&$C$ \\ [1ex] \hline
						$M$						 &$\rp{5,5}$&$\rp{1,6}$ \\ [1ex] \hline 
						$C$						 &$\rp{6,1}$&$\rp{3,3}$ \\ [1ex] \hline 
					\end{tabular}}
			\end{table}
		\end{frame}	

		\begin{frame}
			\frametitle{Recursos Comunes}
			\begin{table}[htbp!]
				\centering
				\resizebox{9cm}{!}{
					\begin{tabular}{|l|c|c|}\hline
						$J_\peq{1}$/$J_\peq{2}$&$M$&$C$ \\ [1ex] \hline
						$M$						 &$\rp{5,5}$&$\rp{1,6}$ \\ [1ex] \hline 
						$C$						 &$\rp{\color{blue}\underline{{\color{black}6}}\color{black},1}$&$\rp{3,3}$ \\ [1ex] \hline 
					\end{tabular}}
			\end{table}
		\end{frame}	

		\begin{frame}
			\frametitle{Recursos Comunes}
			\begin{table}[htbp!]
				\centering
				\resizebox{9cm}{!}{
					\begin{tabular}{|l|c|c|}\hline
						$J_\peq{1}$/$J_\peq{2}$&$M$&$C$ \\ [1ex] \hline
						$M$						 &$\rp{5,5}$&$\rp{1,6}$ \\ [1ex] \hline 
						$C$						 &$\rp{\color{blue}\underline{{\color{black}6}}\color{black},1}$&$\rp{\color{blue}\underline{{\color{black}3}}\color{black},3}$ \\ [1ex] \hline 
					\end{tabular}}
			\end{table}
		\end{frame}	

		\begin{frame}
			\frametitle{Recursos Comunes}
			\begin{table}[htbp!]
				\centering
				\resizebox{9cm}{!}{
					\begin{tabular}{|l|c|c|}\hline
						$J_\peq{1}$/$J_\peq{2}$&$M$&$C$ \\ [1ex] \hline
						$M$						 &$\rp{5,5}$&$\rp{1,\color{green}\underline{{\color{black}6}}\color{black}}$ \\ [1ex] \hline 
						$C$						 &$\rp{\color{blue}\underline{{\color{black}6}}\color{black},1}$&$\rp{\color{blue}\underline{{\color{black}3}}\color{black},3}$ \\ [1ex] \hline 
					\end{tabular}}
			\end{table}
		\end{frame}	

		\begin{frame}
			\frametitle{Recursos Comunes}
			\begin{table}[htbp!]
				\centering
				\resizebox{9cm}{!}{
					\begin{tabular}{|l|c|c|}\hline
						$J_\peq{1}$/$J_\peq{2}$&$M$&$C$ \\ [1ex] \hline
						$M$						 &$\rp{5,5}$&$\rp{1,\color{green}\underline{{\color{black}6}}\color{black}}$ \\ [1ex] \hline 
						$C$						 &$\rp{\color{blue}\underline{{\color{black}6}}\color{black},1}$&$\rp{\color{blue}\underline{{\color{black}3}}\color{black},\color{green}\underline{{\color{black}3}}\color{black}}$ \\ [1ex] \hline 
					\end{tabular}}
			\end{table}
		\end{frame}	

		\begin{frame}
			\frametitle{Recursos Comunes}
			Mejores respuestas:
			\begin{table}[htbp!]
				\centering
				%\resizebox{9cm}{!}{
					\begin{tabular}{|c|c|}\hline
						$J_\peq{1}$				 &$J_\peq{2}$\\ [1ex] \hline
						$r_\peq{1}\rp{M}=C$&$r_\peq{2}\rp{M}=C$ \\ [1ex] 
						$r_\peq{1}\rp{C}=C$&$r_\peq{2}\rp{C}=C$ \\ [1ex] \hline 
					\end{tabular}%}
			\end{table}
		\end{frame}	

		\begin{frame}
			\frametitle{Recursos Comunes}
			\begin{itemize}
				\item En equlibrio ambos pastores explotan el recurso en jornada completa.
				\item Si cooperaran obtendrían un mejor resultado pero sus incentivos individuales inhiben la cooperación. 
			\end{itemize}
		\end{frame}	

		\begin{frame}
			\frametitle{Recursos Comunes}
			\begin{itemize}
				\item Al tratarse de un bien no-excluible, pero rival en el consumo, se genera una \textbf{externalidad negativa}: cuando un individuo consume, hay menos recurso disponible para el consumo de los demás.
				\item Conclusion: la externalidad negativa hace que el recurso sea sobreexplotado en equilibrio.
			\end{itemize}
		\end{frame}

		\begin{frame}
			\frametitle{Recursos Comunes}
			Política pública:
			\begin{itemize}
				\item El gobierno puede resolver mediante regulación o impuestos.
				\item También puede asignar derechos de propiedad, transformando el recurso común en un bien privado.
			\end{itemize}
		\end{frame}

\end{document}