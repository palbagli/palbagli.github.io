
\documentclass[dvipsnames,table,leqno]{beamer}
%\documentclass{beamer}
%\usepackage{beamerthemesplit} 
%\usetheme{Berkeley}
%\usecolortheme{dolphin}
\usetheme{Szeged}

\usepackage{amsfonts}
\usepackage{txfonts}
\usepackage[spanish]{babel}
%\usepackage[latin1]{inputenc}
\usepackage[utf8]{inputenc}
%\usepackage[dvips]{graphicx}
\usepackage{cancel}
%\usepackage{bm}
\usepackage{ae,aecompl,amsmath,amsbsy}
\setbeamertemplate{navigation symbols}{}

\beamertemplateballitem

\usepackage{tikz}
\usepackage{pbox}
%\usepackage{subfigure}
\usepackage{subcaption}
\usepackage{centernot}
\usepackage{etoolbox}
\AtBeginEnvironment{align}{\setcounter{equation}{0}}

\usetikzlibrary{babel,decorations.pathreplacing,decorations.markings}

\decimalpoint

\newtheorem{mydef}{Definición}
\newcommand{\peq}[1]{{\scriptscriptstyle{#1}}} 
\newcommand{\rp}[1]{\left(#1\right)}
\newcommand{\sqp}[1]{\left[#1\right]}

\title{EAE105A \\ Introducción a la Economía}
\subtitle{II. Microeconomía: Equilibrio de Mercado y Política Económica} 
\author{Pinjas Albagli}
\institute{Instituto de Economía \\ Pontificia Universidad Católica de Chile}
\date{Primer Semestre de 2018}

\begin{document}

	\maketitle 
	
	\section{Equilibrio de Mercado}
	
		\begin{frame}
			\frametitle{Equilibrio de Mercado}
			\begin{mydef}
				\begin{itemize}
					\item \textbf{Equilibrio:} Situación en la que ningún agente tiene incentivos a cambiar su comportamiento.
					\item \textbf{Equilibrio del mercado perfectamente competitivo:} Situación en la que la cantidad demandada por todos los demandantes del mercado en conjunto es igual a la cantidad ofrecida en conjunto por todos los oferentes del mercado.
				\end{itemize}
			\end{mydef}
		\end{frame}

		\begin{frame}
			\frametitle{Equilibrio de Mercado}
			\begin{mydef}
				\begin{itemize}
					\item \textbf{Precio de equilibrio ($\mathbf{p^\peq{*}}$):} Precio que balancea la cantidad ofrecida con la cantidad demandada. $$q^\peq{d}\rp{p^\peq{*}}=q^\peq{s}\rp{p^\peq{*}}$$
					\item \textbf{Cantidad de equilibrio ($\mathbf{q^\peq{*}}$):} Cantidad ofrecida y demandada al precio de equilibrio. $$q^\peq{*}=q^\peq{d}\rp{p^\peq{*}}=q^\peq{s}\rp{p^\peq{*}}$$
				\end{itemize}
			\end{mydef}
		\end{frame}

		\begin{frame}
			\frametitle{Equilibrio de Mercado}
			\centering
			\begin{tikzpicture}[scale=1.25]
				\draw [dashed,help lines] (0,2)--(2,2)--(2,0);
				\draw [ultra thick,teal] plot [domain=0:4] (\x,{4-\x});
				\node [above right,teal] at (4,0) {$D$};
				\draw [ultra thick,teal] plot [domain=0:4] (\x,{\x});
				\draw [<->,ultra thick] (0,5)--(0,0)--(5,0);
				\node [right,teal] at (4,4) {$S$};
				\node [left] at (0,5) {$p$};
				\node [below] at (5,0) {$q$};
				\draw [fill,blue] (2,2) circle [radius=.1]; 
				\node [left] at (0,2) {\tiny $p^\peq{*}$};
				\node [below] at (2,0) {\tiny $q^\peq{d}\rp{p^\peq{*}}=q^\peq{*}=q^\peq{s}\rp{p^\peq{*}}$};
				\node [below] at (0,0) {\tiny 0};
			\end{tikzpicture}
		\end{frame}

		\begin{frame}
			\frametitle{Equilibrio de Mercado}
			\begin{mydef}
				\textbf{Ley de la oferta y la demanda:} El precio de un bien se ajusta para llevar al equilibrio la cantidad ofrecida y la cantidad demandada.
			\end{mydef}
		\end{frame}

		\begin{frame}
			\frametitle{Equilibrio de Mercado}
			¿Qué pasa si el mercado no está en equilibrio?
				\begin{itemize}
					\item Por definición: $q^\peq{d}\rp{p^\peq{*}}=q^\peq{s}\rp{p^\peq{*}}=q^\peq{*}$
					\item Ley de la demanda: $\Delta^\peq{+}p\implies\Delta^\peq{-}q^\peq{d}$
					\item Ley de la oferta: $\Delta^\peq{+}p\implies\Delta^\peq{+}q^\peq{s}$
				\end{itemize}
		\end{frame}

		\begin{frame}
			\frametitle{Equilibrio de Mercado}
			Si el precio de mercado es $p_\peq{0}<p^\peq{*}$,
				\begin{itemize}
					\item $q^\peq{d}\rp{p_\peq{0}}>q^\peq{d}\rp{p^\peq{*}}=q^\peq{*}$
					\item $q^\peq{s}\rp{p_\peq{0}}<q^\peq{s}\rp{p^\peq{*}}=q^\peq{*}$
					\item Por transitividad, $q^\peq{d}\rp{p_\peq{0}}>q^\peq{s}\rp{p_\peq{0}}$
				\end{itemize}
			Se produce un \textbf{exceso de demanda}: $q^\peq{d}\rp{p_\peq{0}}-q^\peq{s}\rp{p_\peq{0}}>0$
		\end{frame}
	
		\begin{frame}
			\frametitle{Equilibrio de Mercado}
			\centering
			\begin{tikzpicture}[scale=1.1]
				\draw [ultra thick,teal] plot [domain=0:4] (\x,{4-\x});
				\node [above right,teal] at (4,0) {$D$};
				\draw [ultra thick,teal] plot [domain=0:4] (\x,{\x});
				\draw [thick,blue] (0,2)--(2,2)--(2,0);
				\draw [thick,red] (0,1)--(3,1)--(3,0);
				\draw [thick,red] (1,1)--(1,0);
				\draw [<->,ultra thick] (0,5)--(0,0)--(5,0);
				\node [right,teal] at (4,4) {$S$};
				\node [left] at (0,5) {$p$};
				\node [below] at (5,0) {$q$};
				\node [left] at (0,2) {\tiny $p^\peq{*}$};
				\node [left] at (0,1) {\tiny $p_\peq{0}$};
				\node [below] at (2,0) {\tiny $q^\peq{*}$};
				\node [below] at (3,0) {\tiny $q^\peq{d}\rp{p_\peq{0}}$};
				\node [below] at (1,0) {\tiny $q^\peq{s}\rp{p_\peq{0}}$};
				\node [below] at (0,0) {\tiny 0};
				\draw [decorate,decoration={brace,amplitude=6pt,mirror},xshift=0.2pt,yshift=-0.2pt,red](1,-.4) -- (3,-.4) node[red,midway,yshift=-0.5cm] {\tiny \pbox{\textwidth}{Exceso de \\ demanda}};
			\end{tikzpicture}
		\end{frame}	

		\begin{frame}
			\frametitle{Equilibrio de Mercado}
				\begin{itemize}
					\item Demandantes no pueden comprar todo lo que quisieran a $p_\peq{0}$.
					\item Oferentes observan que pueden incrementar el precio sin perder ventas y así aumentar sus ganancias.
					\item Precio comienza a subir, cayendo $q^\peq{d}$ y aumentando $q^\peq{s}$, lo que reduce el exceso de demanda. 
					\item Esto ocurre hasta que $p=p^\peq{*}$ y $q^\peq{d}\rp{p^\peq{*}}-q^\peq{s}\rp{p^\peq{*}}=0$, desapareciendo el incentivo individual a cobrar más caro.
				\end{itemize}
		\end{frame}

		\begin{frame}
			\frametitle{Equilibrio de Mercado}
			\centering
			\begin{tikzpicture}[scale=1.1]
				\draw [ultra thick,teal] plot [domain=0:4] (\x,{4-\x});
				\node [above right,teal] at (4,0) {$D$};
				\draw [ultra thick,teal] plot [domain=0:4] (\x,{\x});
				\draw [thick,blue] (0,2)--(2,2)--(2,0);
				\draw [thick,red] (0,1)--(3,1)--(3,0);
				\draw [thick,red] (1,1)--(1,0);
				\draw [<->,ultra thick] (0,5)--(0,0)--(5,0);
				\node [right,teal] at (4,4) {$S$};
				\node [left] at (0,5) {$p$};
				\node [below] at (5,0) {$q$};
				\node [left] at (0,2) {\tiny $p^\peq{*}$};
				\node [left] at (0,1) {\tiny $p_\peq{0}$};
				\node [below] at (2,0) {\tiny $q^\peq{*}$};
				\node [below] at (3,0) {\tiny $q^\peq{d}\rp{p_\peq{0}}$};
				\node [below] at (1,0) {\tiny $q^\peq{s}\rp{p_\peq{0}}$};
				\node [below] at (0,0) {\tiny 0};
				\draw [decorate,decoration={brace,amplitude=6pt,mirror},xshift=0.2pt,yshift=-0.2pt,red](1,-.4) -- (3,-.4) node[red,midway,yshift=-0.5cm] {\tiny \pbox{\textwidth}{Exceso de \\ demanda}};
				\begin{scope}[ultra thick,decoration={markings,mark=at position 0.5 with {\arrow{>}}}] 
						\draw[postaction={decorate},green] (0,1)--(0,2);
						\draw[postaction={decorate},green] (1,1)--(2,2);
						\draw[postaction={decorate},green] (3,1)--(2,2);
						\draw[postaction={decorate},green] (1,0)--(2,0);
						\draw[postaction={decorate},green] (3,0)--(2,0);
				\end{scope}
			\end{tikzpicture}
		\end{frame}	
	
		\begin{frame}
			\frametitle{Equilibrio de Mercado}
			Si el precio de mercado es $p_\peq{1}>p^\peq{*}$,
				\begin{itemize}
					\item $q^\peq{d}\rp{p_\peq{1}}<q^\peq{d}\rp{p^\peq{*}}=q^\peq{*}$
					\item $q^\peq{s}\rp{p_\peq{1}}>q^\peq{s}\rp{p^\peq{*}}=q^\peq{*}$
					\item Por transitividad, $q^\peq{s}\rp{p_\peq{1}}>q^\peq{d}\rp{p_\peq{1}}$
				\end{itemize}
			Se produce un \textbf{exceso de oferta}: $q^\peq{s}\rp{p_\peq{1}}-q^\peq{d}\rp{p_\peq{1}}>0$
		\end{frame}
	
		\begin{frame}
			\frametitle{Equilibrio de Mercado}
			\centering
			\begin{tikzpicture}[scale=1.1]
				\draw [ultra thick,teal] plot [domain=0:4] (\x,{4-\x});
				\node [above right,teal] at (4,0) {$D$};
				\draw [ultra thick,teal] plot [domain=0:4] (\x,{\x});
				\draw [thick,blue] (0,2)--(2,2)--(2,0);
				\draw [thick,red] (0,3)--(1,3)--(1,0);
				\draw [thick,red] (1,3)--(3,3)--(3,0);
				\draw [<->,ultra thick] (0,5)--(0,0)--(5,0);
				\node [right,teal] at (4,4) {$S$};
				\node [left] at (0,5) {$p$};
				\node [below] at (5,0) {$q$};
				\node [left] at (0,2) {\tiny $p^\peq{*}$};
				\node [left] at (0,3) {\tiny $p_\peq{1}$};
				\node [below] at (2,0) {\tiny $q^\peq{*}$};
				\node [below] at (3,0) {\tiny $q^\peq{s}\rp{p_\peq{1}}$};
				\node [below] at (1,0) {\tiny $q^\peq{d}\rp{p_\peq{1}}$};
				\node [below] at (0,0) {\tiny 0};
				\draw [decorate,decoration={brace,amplitude=6pt,mirror},xshift=0.2pt,yshift=-0.2pt,red](1,-.4) -- (3,-.4) node[red,midway,yshift=-0.5cm] {\tiny \pbox{\textwidth}{Exceso \\ de oferta}};
			\end{tikzpicture}
		\end{frame}	

		\begin{frame}
			\frametitle{Equilibrio de Mercado}
				\begin{itemize}
					\item Oferentes no pueden vender todo lo que quisieran a $p_\peq{1}$.
					\item Oferentes observan que pueden aumentar sus ganancias reduciendo el precio.
					\item Precio comienza a caer, aumentando $q^\peq{d}$ y reduciendo $q^\peq{s}$, lo que reduce el exceso de oferta. 
					\item Esto ocurre hasta que $p=p^\peq{*}$ y $q^\peq{s}\rp{p^\peq{*}}-q^\peq{d}\rp{p^\peq{*}}=0$, desapareciendo el incentivo individual a cobrar más barato.
				\end{itemize}
		\end{frame}

		\begin{frame}
			\frametitle{Equilibrio de Mercado}
			\centering
			\begin{tikzpicture}[scale=1.1]
				\draw [ultra thick,teal] plot [domain=0:4] (\x,{4-\x});
				\node [above right,teal] at (4,0) {$D$};
				\draw [ultra thick,teal] plot [domain=0:4] (\x,{\x});
				\draw [thick,blue] (0,2)--(2,2)--(2,0);
				\draw [thick,red] (0,3)--(1,3)--(1,0);
				\draw [thick,red] (1,3)--(3,3)--(3,0);
				\draw [<->,ultra thick] (0,5)--(0,0)--(5,0);
				\node [right,teal] at (4,4) {$S$};
				\node [left] at (0,5) {$p$};
				\node [below] at (5,0) {$q$};
				\node [left] at (0,2) {\tiny $p^\peq{*}$};
				\node [left] at (0,3) {\tiny $p_\peq{1}$};
				\node [below] at (2,0) {\tiny $q^\peq{*}$};
				\node [below] at (3,0) {\tiny $q^\peq{s}\rp{p_\peq{1}}$};
				\node [below] at (1,0) {\tiny $q^\peq{d}\rp{p_\peq{1}}$};
				\node [below] at (0,0) {\tiny 0};
				\draw [decorate,decoration={brace,amplitude=6pt,mirror},xshift=0.2pt,yshift=-0.2pt,red](1,-.4) -- (3,-.4) node[red,midway,yshift=-0.5cm] {\tiny \pbox{\textwidth}{Exceso \\ de oferta}};
				\begin{scope}[ultra thick,decoration={markings,mark=at position 0.5 with {\arrow{>}}}] 
						\draw[postaction={decorate},green] (0,3)--(0,2);
						\draw[postaction={decorate},green] (1,3)--(2,2);
						\draw[postaction={decorate},green] (3,3)--(2,2);
						\draw[postaction={decorate},green] (1,0)--(2,0);
						\draw[postaction={decorate},green] (3,0)--(2,0);
				\end{scope}
			\end{tikzpicture}
		\end{frame}	

		\begin{frame}
			\frametitle{Equilibrio de Mercado}
			También es posible entender el equilibrio a través del ajuste de las cantidades. Si se transa una cantidad $q_\peq{0}<q^\peq{*}$,
				\begin{itemize}
					\item $p^\peq{d}\rp{q_\peq{0}}>p^\peq{d}\rp{q^\peq{*}}=p^\peq{*}$
					\item $p^\peq{s}\rp{q_\peq{0}}<p^\peq{s}\rp{q^\peq{*}}=p^\peq{*}$
					\item Por transitividad, $p^\peq{d}\rp{q_\peq{0}}>p^\peq{s}\rp{q_\peq{0}}$
				\end{itemize}
		\end{frame}

		\begin{frame}
			\frametitle{Equilibrio de Mercado}
			\centering
			\begin{tikzpicture}[scale=1.1]
				\draw [ultra thick,teal] plot [domain=0:4] (\x,{4-\x});
				\node [above right,teal] at (4,0) {$D$};
				\draw [ultra thick,teal] plot [domain=0:4] (\x,{\x});
				\draw [thick,blue] (0,2)--(2,2)--(2,0);
				\draw [thick,red] (1,0)--(1,3)--(0,3);
				\draw [thick,red] (1,1)--(0,1);
				\draw [<->,ultra thick] (0,5)--(0,0)--(5,0);
				\node [right,teal] at (4,4) {$S$};
				\node [left] at (0,5) {$p$};
				\node [below] at (5,0) {$q$};
				\node [left] at (0,2) {\tiny $p^\peq{*}$};
				\node [left] at (0,1) {\tiny $p^\peq{s}\rp{q_\peq{0}}$};
				\node [left] at (0,3) {\tiny $p^\peq{d}\rp{q_\peq{0}}$};
				\node [below] at (2,0) {\tiny $q^\peq{*}$};
				\node [below] at (1,0) {\tiny $q_\peq{0}$};
				\node [below] at (0,0) {\tiny 0};
			\end{tikzpicture}
		\end{frame}	
		
		\begin{frame}
			\frametitle{Equilibrio de Mercado}
				\begin{itemize}
					\item La disposición (marginal) a pagar de los consumidores es mayor que el costo marginal de producción.
					\item Los oferentes pueden incrementar sus ganancias aumentando la cantidad ofrecida .
					\item Mientras $p^\peq{s}\rp{q}\leq p^\peq{d}\rp{q}$, los consumidores estarán dispuestos a transar la unidad marginal.
					\item La cantidad comienza a aumentar, reduciendo $p^\peq{d}$ y aumentando $p^\peq{s}$, hasta que $q=q^\peq{*}$ y $p^\peq{d}\rp{q_\peq{0}}-p^\peq{s}\rp{q_\peq{0}}=0$, desapareciendo el incentivo a transar más.
				\end{itemize}
		\end{frame}

				\begin{frame}
			\frametitle{Equilibrio de Mercado}
			\centering
			\begin{tikzpicture}[scale=1.1]
				\draw [ultra thick,teal] plot [domain=0:4] (\x,{4-\x});
				\node [above right,teal] at (4,0) {$D$};
				\draw [ultra thick,teal] plot [domain=0:4] (\x,{\x});
				\draw [thick,blue] (0,2)--(2,2)--(2,0);
				\draw [thick,red] (1,0)--(1,3)--(0,3);
				\draw [thick,red] (1,1)--(0,1);
				\draw [<->,ultra thick] (0,5)--(0,0)--(5,0);
				\node [right,teal] at (4,4) {$S$};
				\node [left] at (0,5) {$p$};
				\node [below] at (5,0) {$q$};
				\node [left] at (0,2) {\tiny $p^\peq{*}$};
				\node [left] at (0,1) {\tiny $p^\peq{s}\rp{q_\peq{0}}$};
				\node [left] at (0,3) {\tiny $p^\peq{d}\rp{q_\peq{0}}$};
				\node [below] at (2,0) {\tiny $q^\peq{*}$};
				\node [below] at (1,0) {\tiny $q_\peq{0}$};
				\node [below] at (0,0) {\tiny 0};
				\begin{scope}[ultra thick,decoration={markings,mark=at position 0.5 with {\arrow{>}}}] 
						\draw[postaction={decorate},green] (1,0)--(2,0);
						\draw[postaction={decorate},green] (0,1)--(0,2);
						\draw[postaction={decorate},green] (0,3)--(0,2);
						\draw[postaction={decorate},green] (1,3)--(2,2);
						\draw[postaction={decorate},green] (1,1)--(2,2);
				\end{scope}
			\end{tikzpicture}
		\end{frame}	
	
		\begin{frame}
			\frametitle{Equilibrio de Mercado}
			Si se transa una cantidad $q_\peq{1}>q^\peq{*}$,
				\begin{itemize}
					\item $p^\peq{d}\rp{q_\peq{1}}<p^\peq{d}\rp{q^\peq{*}}=p^\peq{*}$
					\item $p^\peq{s}\rp{q_\peq{1}}>p^\peq{s}\rp{q^\peq{*}}=p^\peq{*}$
					\item Por transitividad, $p^\peq{s}\rp{q_\peq{1}}>p^\peq{d}\rp{q_\peq{1}}$
				\end{itemize}
		\end{frame}

		\begin{frame}
			\frametitle{Equilibrio de Mercado}
			\centering
			\begin{tikzpicture}[scale=1.1]
				\draw [ultra thick,teal] plot [domain=0:4] (\x,{4-\x});
				\node [above right,teal] at (4,0) {$D$};
				\draw [ultra thick,teal] plot [domain=0:4] (\x,{\x});
				\draw [thick,blue] (0,2)--(2,2)--(2,0);
				\draw [thick,red] (3,0)--(3,3)--(0,3);
				\draw [thick,red] (3,1)--(0,1);
				\draw [<->,ultra thick] (0,5)--(0,0)--(5,0);
				\node [right,teal] at (4,4) {$S$};
				\node [left] at (0,5) {$p$};
				\node [below] at (5,0) {$q$};
				\node [left] at (0,2) {\tiny $p^\peq{*}$};
				\node [left] at (0,1) {\tiny $p^\peq{d}\rp{q_\peq{1}}$};
				\node [left] at (0,3) {\tiny $p^\peq{s}\rp{q_\peq{1}}$};
				\node [below] at (2,0) {\tiny $q^\peq{*}$};
				\node [below] at (3,0) {\tiny $q_\peq{1}$};
				\node [below] at (0,0) {\tiny 0};
			\end{tikzpicture}
		\end{frame}	
		
		\begin{frame}
			\frametitle{Equilibrio de Mercado}
				\begin{itemize}
					\item El costo marginal de producción es mayor que la disposición a pagar de los consumidores.
					\item Los oferentes pueden incrementar sus ganancias reduciendo la cantidad ofrecida.
					\item De hecho, mientras $p^\peq{s}\rp{q}> p^\peq{d}\rp{q}$, los consumidores no estarán dispuestos a comprar la unidad marginal.
					\item La cantidad comienza a caer, aumentando $p^\peq{d}$ y reduciendo $p^\peq{s}$, hasta que $q=q^\peq{*}$ y $p^\peq{s}\rp{q_\peq{0}}-p^\peq{d}\rp{q_\peq{0}}=0$, desapareciendo el incentivo a transar menos.
				\end{itemize}
		\end{frame}

		\begin{frame}
			\frametitle{Equilibrio de Mercado}
			\centering
			\begin{tikzpicture}[scale=1.1]
				\draw [ultra thick,teal] plot [domain=0:4] (\x,{4-\x});
				\node [above right,teal] at (4,0) {$D$};
				\draw [ultra thick,teal] plot [domain=0:4] (\x,{\x});
				\draw [thick,blue] (0,2)--(2,2)--(2,0);
				\draw [thick,red] (3,0)--(3,3)--(0,3);
				\draw [thick,red] (3,1)--(0,1);
				\draw [<->,ultra thick] (0,5)--(0,0)--(5,0);
				\node [right,teal] at (4,4) {$S$};
				\node [left] at (0,5) {$p$};
				\node [below] at (5,0) {$q$};
				\node [left] at (0,2) {\tiny $p^\peq{*}$};
				\node [left] at (0,1) {\tiny $p^\peq{d}\rp{q_\peq{1}}$};
				\node [left] at (0,3) {\tiny $p^\peq{s}\rp{q_\peq{1}}$};
				\node [below] at (2,0) {\tiny $q^\peq{*}$};
				\node [below] at (3,0) {\tiny $q_\peq{1}$};
				\node [below] at (0,0) {\tiny 0};
				\begin{scope}[ultra thick,decoration={markings,mark=at position 0.5 with {\arrow{>}}}] 
						\draw[postaction={decorate},green] (3,0)--(2,0);
						\draw[postaction={decorate},green] (0,1)--(0,2);
						\draw[postaction={decorate},green] (0,3)--(0,2);
						\draw[postaction={decorate},green] (3,3)--(2,2);
						\draw[postaction={decorate},green] (3,1)--(2,2);
				\end{scope}
			\end{tikzpicture}
		\end{frame}	

		\begin{frame}
			\frametitle{Equilibrio de Mercado}
			Tenemos el siguiente modelo del equilibrio del mercado:
				\begin{align}
					\text{Demanda: } q^\peq{d}&=f\rp{p^\peq{d}}& \\
					\text{Oferta: }  q^\peq{s}&=g\rp{p^\peq{s}}& \\
					\text{No-distorsión: } p^\peq{d}&=p^\peq{s}=p& \\
					\text{Equilibrio: } q^\peq{d}&=q^\peq{s}=q^\peq{*}&
				\end{align}
		\end{frame}

		\begin{frame}
			\frametitle{Equilibrio de Mercado}
			Ejemplo: oferta y demanda lineales
				\begin{align}
					\text{Demanda: } p^\peq{d}&=a-bq^\peq{d}& \\
					\text{Oferta: }  p^\peq{s}&=c+dq^\peq{s}& \\
					\text{No-distorsión: } p^\peq{d}&=p^\peq{s}=p& \\
					\text{Equilibrio: } q^\peq{d}&=q^\peq{s}=q^\peq{*}&
				\end{align}
		\end{frame}
	
		\begin{frame}
			\frametitle{Equilibrio de Mercado}
			Resolviendo:
				\begin{align*}
					&p^\peq{*}=a-bq^\peq{*} \\
					&p^\peq{*}=c+dq^\peq{*} \\
					\implies &q^\peq{*}=\frac{a-c}{b+d} \\
					\implies &p^\peq{*}=\frac{ad+bc}{b+d} \\
				\end{align*}
		\end{frame}

		\begin{frame}
			\frametitle{Equilibrio de Mercado}
			Un ejemplo más ``aterrizado'': 
				\begin{align}
					\text{Demanda: } p^\peq{d}&=10.000-q^\peq{d}& \\
					\text{Oferta: }  p^\peq{s}&=500+q^\peq{s}& \\
					\text{No-distorsión: } p^\peq{d}&=p^\peq{s}=p& \\
					\text{Equilibrio: } q^\peq{d}&=q^\peq{s}=q^\peq{*}&
				\end{align}
		\end{frame}
	
		\begin{frame}
			\frametitle{Equilibrio de Mercado}
			Resolviendo:
				\begin{align*}
					&p^\peq{*}=10.000-q^\peq{*} \\
					&p^\peq{*}=500+q^\peq{*} \\
					\implies  &q^\peq{*}=4.750 \\
					\implies  &p^\peq{*}=5.250 \\
				\end{align*}
		\end{frame}

		\begin{frame}
			\frametitle{Equilibrio de Mercado}
			\centering
			\begin{tikzpicture}[scale=1.25]
				\draw [dashed,help lines] (0,2.1)--(1.9,2.1)--(1.9,0);
				\draw [ultra thick,teal] plot [domain=0:4] (\x,{4-\x});
				\node [above right,teal] at (4,0) {$D$};
				\draw [ultra thick,teal] plot [domain=0:4] (\x,{.2+\x});
				\draw [<->,ultra thick] (0,5)--(0,0)--(5,0);
				\node [right,teal] at (4,4.2) {$S$};
				\node [left] at (0,5) {$p$};
				\node [left] at (0,4) {\tiny 10.000};
				\node [below] at (5,0) {$q$};
				\node [below] at (4,0) {\tiny 10.000};
				\draw [fill,blue] (1.9,2.1) circle [radius=.1]; 
				\node [left] at (0,2.1) {\tiny 5.250};
				\node [left] at (0,.2) {\tiny 500};
				\node [below] at (1.9,0) {\tiny 4.750};
				\node [below] at (0,0) {\tiny 0};
			\end{tikzpicture}
		\end{frame}
	
		\begin{frame}
			\frametitle{Equilibrio de Mercado}
			Estática comparativa:
				\begin{itemize}
					\item Comparación entre dos situaciones estáticas (equilibrios).
					\item Queremos saber cómo cambios en la oferta y/o demanda afectan al equilibrio del mercado.
					\item ¿Cómo cambian $p^\peq{*}$ y $p^\peq{*}$ cuando un acontecimiento afecta a la oferta o la demanda?
				\end{itemize}
		\end{frame}	

		\begin{frame}
			\frametitle{Equilibrio de Mercado}
			Procedemos en 3 pasos:
				\begin{enumerate}
					\item Determinar si el acontecimiento desplaza
					\begin{enumerate}
						\item la curva de oferta
						\item la curva de demanda
					\end{enumerate}
					\item Determinar las direcciones de los desplazamientos
					\item Graficar
				\end{enumerate}
		\end{frame}	

		\begin{frame}
			\frametitle{Equilibrio de Mercado}
			Ejemplo 1: $\Delta^\peq{+} D$
				\begin{itemize}
					\item Mercado: helado
					\item Acontecimiento: verano muy caluroso
				\end{itemize}
		\end{frame}	
	
		\begin{frame}
			\frametitle{Equilibrio de Mercado}
			Análisis en 3 pasos:
				\begin{enumerate}
					\item El acontecimiento desplaza la curva de demanda (preferencias) y no afecta a la curva de oferta
					\item La demanda aumenta
					\item ...
				\end{enumerate}
		\end{frame}	

		\begin{frame}
			\frametitle{Equilibrio de Mercado}
			\centering
			\begin{tikzpicture}[scale=.9]
				\draw [dashed,help lines] (0,2)--(2,2)--(2,0);
				\draw [dashed,help lines] (0,3)--(3,3)--(3,0);
				\draw [ultra thick,teal] plot [domain=0:4] (\x,{4-\x});
				\node [above right,teal] at (4,0) {$D_\peq{0}$};
				\draw [ultra thick,purple] plot [domain=0:6] (\x,{6-\x});
				\node [above right,purple] at (6,0) {$D_\peq{1}$};
				\draw [ultra thick,teal] plot [domain=0:6] (\x,{\x});
				\draw [<->,ultra thick] (0,7)--(0,0)--(7,0);
				\node [right,teal] at (6,6) {$S$};
				\node [left] at (0,7) {$p$};
				\node [below] at (7,0) {$q$};
				\draw [fill,blue] (2,2) circle [radius=.1];
				\draw [fill,blue] (3,3) circle [radius=.1]; 
				\node [left] at (0,2) {\tiny $p^\peq{*}_\peq{0}$};
				\node [below] at (2,0) {\tiny $q^\peq{*}_\peq{0}$};
				\node [left] at (0,3) {\tiny $p^\peq{*}_\peq{1}$};
				\node [below] at (3,0) {\tiny $q^\peq{*}_\peq{1}$};
				\node [below] at (0,0) {\tiny 0};
			\end{tikzpicture}
		\end{frame}

		\begin{frame}
			\frametitle{Equilibrio de Mercado}
			Ejemplo 2: $\Delta^\peq{-} S$
				\begin{itemize}
					\item Mercado: helado
					\item Acontecimiento: terremoto destruye fábricas de helado
				\end{itemize}
		\end{frame}	
	
		\begin{frame}
			\frametitle{Equilibrio de Mercado}
			Análisis en 3 pasos:
				\begin{enumerate}
					\item El acontecimiento desplaza la curva de oferta (costo marginal) y no afecta a la curva de demanda
					\item La oferta disminuye
					\item ...
				\end{enumerate}
		\end{frame}	

		\begin{frame}
			\frametitle{Equilibrio de Mercado}
			\centering
			\begin{tikzpicture}[scale=.9]
				\draw [dashed,help lines] (0,2)--(2,2)--(2,0);
				\draw [dashed,help lines] (0,3)--(1,3)--(1,0);
				\draw [ultra thick,teal] plot [domain=0:4] (\x,{4-\x});
				\node [above right,teal] at (4,0) {$D$};
				%\draw [ultra thick,teal] plot [domain=0:6] (\x,{6-\x});
				%\node [above right,teal] at (6,0) {$D_\peq{1}$};
				\draw [ultra thick,teal] plot [domain=0:6] (\x,{\x});
				\draw [<->,ultra thick] (0,7)--(0,0)--(7,0);
				\draw [ultra thick,purple] plot [domain=0:5] (\x,{2+\x});
				\draw [<->,ultra thick] (0,7)--(0,0)--(7,0);
				\node [right,teal] at (6,6) {$S_\peq{0}$};
				\node [right,purple] at (5,7) {$S_\peq{1}$};
				\node [left] at (0,7) {$p$};
				\node [below] at (7,0) {$q$};
				\draw [fill,blue] (2,2) circle [radius=.1];
				\draw [fill,blue] (1,3) circle [radius=.1]; 
				\node [left] at (0,2) {\tiny $p^\peq{*}_\peq{0}$};
				\node [below] at (2,0) {\tiny $q^\peq{*}_\peq{0}$};
				\node [left] at (0,3) {\tiny $p^\peq{*}_\peq{1}$};
				\node [below] at (1,0) {\tiny $q^\peq{*}_\peq{1}$};
				\node [below] at (0,0) {\tiny 0};
			\end{tikzpicture}
		\end{frame}

		\begin{frame}
			\frametitle{Equilibrio de Mercado}
			Ejemplo 3: $\Delta^\peq{+} D$ y $\Delta^\peq{-} S$
				\begin{itemize}
					\item Mercado: helado
					\item Acontecimiento: verano caluroso y terremoto destruye fábricas de helado
				\end{itemize}
		\end{frame}	
	
		\begin{frame}
			\frametitle{Equilibrio de Mercado}
			Análisis en 3 pasos:
				\begin{enumerate}
					\item El acontecimiento desplaza la curva de oferta y la curva de demanda
					\item La oferta disminuye y la demanda aumenta
					\item Ahora tenemos dos escenarios posibles...
				\end{enumerate}
		\end{frame}	

		\begin{frame}
			\frametitle{Equilibrio de Mercado}
			\begin{figure}[hbtp!]
				\centering
				\begin{subfigure}[b]{0.49\textwidth}
					\begin{tikzpicture}[scale=.7]
						\draw [dashed,help lines] (0,2)--(2,2)--(2,0);
						\draw [dashed,help lines] (0,3.5)--(1.5,3.5)--(1.5,0);
						\draw [ultra thick,teal] plot [domain=0:4] (\x,{4-\x});
						\draw [ultra thick,purple] plot [domain=0:5] (\x,{5-\x});
						\node [above,teal] at (4,0) {$D_\peq{0}$};
						\node [above,purple] at (5,0) {$D_\peq{1}$};
						\draw [ultra thick,teal] plot [domain=0:6] (\x,{\x});
						\draw [<->,ultra thick] (0,7)--(0,0)--(7,0);
						\draw [ultra thick,purple] plot [domain=0:5] (\x,{2+\x});
						\draw [<->,ultra thick] (0,7)--(0,0)--(7,0);
						\node [right,teal] at (6,6) {$S_\peq{0}$};
						\node [right,purple] at (5,7) {$S_\peq{1}$};
						\node [left] at (0,7) {$p$};
						\node [below] at (7,0) {$q$};
						\draw [fill,blue] (2,2) circle [radius=.1];
						\draw [fill,blue] (1.5,3.5) circle [radius=.1]; 
						\node [left] at (0,2) {\tiny $p^\peq{*}_\peq{0}$};
						\node [below] at (2,0) {\tiny $q^\peq{*}_\peq{0}$};
						\node [left] at (0,3.5) {\tiny $p^\peq{*}_\peq{1}$};
						\node [below] at (1.5,0) {\tiny $q^\peq{*}_\peq{1}$};
						\node [below] at (0,0) {\tiny 0};
					\end{tikzpicture}
				\end{subfigure}
				\begin{subfigure}[b]{0.49\textwidth}
					\begin{tikzpicture}[scale=.7]
						\draw [dashed,help lines] (0,2)--(2,2)--(2,0);
						\draw [dashed,help lines] (0,3.5)--(2.5,3.5)--(2.5,0);
						\draw [ultra thick,teal] plot [domain=0:4] (\x,{4-\x});
						\draw [ultra thick,purple] plot [domain=0:6] (\x,{6-\x});
						\node [above,teal] at (4,0) {$D_\peq{0}$};
						\node [above,purple] at (6,0) {$D_\peq{1}$};
						\draw [ultra thick,teal] plot [domain=0:6] (\x,{\x});
						\draw [<->,ultra thick] (0,7)--(0,0)--(7,0);
						\draw [ultra thick,purple] plot [domain=0:6] (\x,{1+\x});
						\draw [<->,ultra thick] (0,7)--(0,0)--(7,0);
						\node [right,teal] at (6,6) {$S_\peq{0}$};
						\node [right,purple] at (6,7) {$S_\peq{1}$};
						\node [left] at (0,7) {$p$};
						\node [below] at (7,0) {$q$};
						\draw [fill,blue] (2,2) circle [radius=.1];
						\draw [fill,blue] (2.5,3.5) circle [radius=.1]; 
						\node [left] at (0,2) {\tiny $p^\peq{*}_\peq{0}$};
						\node [below] at (2,0) {\tiny $q^\peq{*}_\peq{0}$};
						\node [left] at (0,3.5) {\tiny $p^\peq{*}_\peq{1}$};
						\node [below] at (2.5,0) {\tiny $q^\peq{*}_\peq{1}$};
						\node [below] at (0,0) {\tiny 0};
					\end{tikzpicture}
				\end{subfigure}
			\end{figure}
		\end{frame}

		\begin{frame}
			\frametitle{Equilibrio de Mercado}
			En general:
			\begin{table}[htbp!]
				\centering
				\resizebox{10cm}{!}{
					\begin{tabular}{|l|c|c|c|}\hline
												&$\centernot\Delta S$&$\Delta^\peq{+} S$&$\Delta^\peq{-} S$ \\  \hline 
	 $\centernot\Delta D$ &\pbox{\textwidth}{$\centernot\Delta p^\peq{*}$\\$\centernot\Delta q^\peq{*}$}&\pbox{\textwidth}{$\Delta^\peq{-}p^\peq{*}$\\$\Delta^\peq{+}q^\peq{*}$}&\pbox{\textwidth}{$\Delta^\peq{+}p^\peq{*}$\\$\Delta^\peq{-}q^\peq{*}$}\\  \hline
		 $\Delta^\peq{+} D$	&\pbox{\textwidth}{$\Delta^\peq{+}p^\peq{*}$ \\ $\Delta^\peq{+}q^\peq{*}$}&\pbox{\textwidth}{$\Delta^\peq{?}p^\peq{*}$\\$\Delta^\peq{+}q^\peq{*}$}&\pbox{\textwidth}{$\Delta^\peq{+}p^\peq{*}$\\$\Delta^\peq{?}q^\peq{*}$}\\  \hline 
		 $\Delta^\peq{-} D$	&\pbox{\textwidth}{$\Delta^\peq{-}p^\peq{*}$ \\ $\Delta^\peq{-}q^\peq{*}$}&\pbox{\textwidth}{$\Delta^\peq{-}p^\peq{*}$\\$\Delta^\peq{?}q^\peq{*}$}&\pbox{\textwidth}{$\Delta^\peq{?}p^\peq{*}$\\$\Delta^\peq{-}q^\peq{*}$}\\  \hline
					\end{tabular}}
			\end{table}
		\end{frame}	

		\begin{frame}
			\frametitle{Equilibrio de Mercado}
			Ejemplo 4:
				\begin{itemize}
					\item Se observa que año tras año aumentan tanto el precio como la cantidad de automóviles comprados en el año.
					\item ¿Significa que no se cumple la ley de la demanda? (¿demanda con pendiente positiva?)
				\end{itemize}
		\end{frame}	

		\begin{frame}
			\frametitle{Equilibrio de Mercado}
				\begin{itemize}
					\item \textbf{No}. Los pares $\rp{q_\peq{t},p_\peq{t}}$ observados no son necesariamente puntos en una misma curva de demanda.
					\item Son el resultado de la interacción entre la oferta y la demanda (o sea el equilibrio del mercado) en cada año $t$.
				\end{itemize}
		\end{frame}	

		\begin{frame}
			\frametitle{Equilibrio de Mercado}
			\begin{figure}[htbp!]
				\centering
				\begin{subfigure}[b]{0.32\textwidth}
					%\resizebox{\linewidth}{!}{
						\begin{tikzpicture}[scale=.2]
							\draw [dashed,help lines] (0,2) -- (2,2) -- (2,0);
							\draw [dashed,help lines] (0,3.5) -- (3.5,3.5) -- (3.5,0);
							\draw [dashed,help lines] (0,5) -- (5,5) -- (5,0);
							\draw [ultra thick,teal] plot [domain=0:4] (\x,{4-\x});
							\draw [ultra thick,purple] plot [domain=0:7] (\x,{7-\x});
							\draw [ultra thick,brown] plot [domain=0:10] (\x,{10-\x});
							\draw [ultra thick,teal] plot [domain=0:10] (\x,{\x});
							\draw [ultra thick,dashed,green] (2,2)--(3.5,3.5)--(5,5);
							\draw [<->,thick] (0,12) -- (0,0) -- (12,0);
							\node [above,teal] at (4,0) {\scriptsize $D_{\scriptscriptstyle{0}}$};
							\node [above,purple] at (7,0) {\scriptsize $D_{\scriptscriptstyle{1}}$};
							\node [above,brown] at (10,0) {\scriptsize $D_{\scriptscriptstyle{2}}$};
							\node [right,teal] at (10,10) {\scriptsize $S$};
							\node [below] at (12,0) {$q$};
							\node [below] at (2,0) {\tiny $q^\peq{*}_{\scriptscriptstyle{0}}$};
							\node [below] at (3.5,0) {\tiny $q^\peq{*}_{\scriptscriptstyle{1}}$};
							\node [below] at (5,0) {\tiny $q^\peq{*}_{\scriptscriptstyle{2}}$};
							\node [left] at (0,12) {$p$};
							\node [left] at (0,2) {\tiny $p^\peq{*}_{\scriptscriptstyle{0}}$};
							\node [left] at (0,3.5) {\tiny $p^\peq{*}_{\scriptscriptstyle{1}}$};
							\node [left] at (0,5) {\tiny $p^\peq{*}_{\scriptscriptstyle{2}}$};
							\draw [fill,blue] (2,2) circle [radius=.3];
							\draw [fill,blue] (3.5,3.5) circle [radius=.3];
							\draw [fill,blue] (5,5) circle [radius=.3];
						\end{tikzpicture}
					%}
				\end{subfigure}
				\begin{subfigure}[b]{0.32\textwidth}
					%\resizebox{\linewidth}{!}{
						\begin{tikzpicture}[scale=.2]
							\draw [dashed,help lines] (0,3) -- (1,3) -- (1,0);
							\draw [dashed,help lines] (0,3.75) -- (3.75,3.75) -- (3.75,0);
							\draw [dashed,help lines] (0,4.5) -- (6.5,4.5) -- (6.5,0);
							\draw [ultra thick,teal] plot [domain=0:4] (\x,{4-\x});
							\draw [ultra thick,purple] plot [domain=0:7.5] (\x,{7.5-\x});
							\draw [ultra thick,brown] plot [domain=0:11] (\x,{11-\x});
							\draw [ultra thick,teal] plot [domain=0:10] (\x,{\x+2});
							\draw [ultra thick,purple] plot [domain=0:10] (\x,{\x});
							\draw [ultra thick,brown] plot [domain=2:10] (\x,{\x-2});
							\draw [ultra thick,dashed,green] (1,3)--(3.75,3.75)--(6.5,4.5);
							\draw [<->,thick] (0,12) -- (0,0) -- (12,0);
							\node [above,teal] at (4.5,0) {\scriptsize $D_{\scriptscriptstyle{0}}$};
							\node [above,purple] at (7.75,0) {\scriptsize $D_{\scriptscriptstyle{1}}$};
							\node [above,brown] at (11,0) {\scriptsize $D_{\scriptscriptstyle{2}}$};
							\node [right,teal] at (10,12) {\scriptsize $S_{\scriptscriptstyle{0}}$};
							\node [right,purple] at (10,10) {\scriptsize $S_{\scriptscriptstyle{1}}$};
							\node [right,brown] at (10,8) {\scriptsize $S_{\scriptscriptstyle{2}}$};
							\node [below] at (12,0) {$q$};
							\node [below] at (1,0) {\tiny $q^\peq{*}_{\scriptscriptstyle{0}}$};
							\node [below] at (3.75,0) {\tiny $q^\peq{*}_{\scriptscriptstyle{1}}$};
							\node [below] at (6.5,0) {\tiny $q^\peq{*}_{\scriptscriptstyle{2}}$};
							\node [left] at (0,12) {$p$};
							\node [left] at (0,2.75) {\tiny $p^\peq{*}_{\scriptscriptstyle{0}}$};
							\node [left] at (0,3.75) {\tiny $p^\peq{*}_{\scriptscriptstyle{1}}$};
							\node [left] at (0,4.75) {\tiny $p^\peq{*}_{\scriptscriptstyle{2}}$};
							\draw [fill,blue] (1,3) circle [radius=.3];
							\draw [fill,blue] (3.75,3.75) circle [radius=.3];
							\draw [fill,blue] (6.5,4.5) circle [radius=.3];
						\end{tikzpicture}
					%}
				\end{subfigure}
				\begin{subfigure}[b]{0.32\textwidth}
					%\resizebox{\linewidth}{!}{
						\begin{tikzpicture}[scale=.2]
							\draw [dashed,help lines] (0,1) -- (3,1) -- (3,0);
							\draw [dashed,help lines] (0,3.75) -- (3.75,3.75) -- (3.75,0);
							\draw [dashed,help lines] (0,6.5) -- (4.5,6.5) -- (4.5,0);
							\draw [ultra thick,teal] plot [domain=0:4] (\x,{4-\x});
							\draw [ultra thick,purple] plot [domain=0:7.5] (\x,{7.5-\x});
							\draw [ultra thick,brown] plot [domain=0:11] (\x,{11-\x});
							\draw [ultra thick,brown] plot [domain=0:10] (\x,{\x+2});
							\draw [ultra thick,purple] plot [domain=0:10] (\x,{\x});
							\draw [ultra thick,teal] plot [domain=2:10] (\x,{\x-2});
							\draw [ultra thick,dashed,green] (3,1)--(3.75,3.75)--(4.5,6.5);
							\draw [<->,thick] (0,12) -- (0,0) -- (12,0);
							\node [above,teal] at (4.5,0) {\scriptsize $D_{\scriptscriptstyle{0}}$};
							\node [above,purple] at (7.75,0) {\scriptsize $D_{\scriptscriptstyle{1}}$};
							\node [above,brown] at (11,0) {\scriptsize $D_{\scriptscriptstyle{2}}$};
							\node [right,brown] at (10,12) {\scriptsize $S_{\scriptscriptstyle{2}}$};
							\node [right,purple] at (10,10) {\scriptsize $S_{\scriptscriptstyle{1}}$};
							\node [right,teal] at (10,8) {\scriptsize $S_{\scriptscriptstyle{0}}$};
							\node [below] at (12,0) {$q$};
							\node [below] at (2.75,0) {\tiny $q^\peq{*}_{\scriptscriptstyle{0}}$};
							\node [below] at (3.75,0) {\tiny $q^\peq{*}_{\scriptscriptstyle{1}}$};
							\node [below] at (4.75,0) {\tiny $q^\peq{*}_{\scriptscriptstyle{2}}$};
							\node [left] at (0,12) {$p$};
							\node [left] at (0,1) {\tiny $p^\peq{*}_{\scriptscriptstyle{0}}$};
							\node [left] at (0,3.75) {\tiny $p^\peq{*}_{\scriptscriptstyle{1}}$};
							\node [left] at (0,6.5) {\tiny $p^\peq{*}_{\scriptscriptstyle{2}}$};
							\draw [fill,blue] (3,1) circle [radius=.3];
							\draw [fill,blue] (3.75,3.75) circle [radius=.3];
							\draw [fill,blue] (4.5,6.5) circle [radius=.3];
						\end{tikzpicture}
					%}
				\end{subfigure}
			\end{figure}	
		\end{frame}	

		\begin{frame}
			\frametitle{Equilibrio en el Largo Plazo}
			Bajo los siguientes supuestos la oferta de mercado en el largo plazo es perfectamente elástica.
				\begin{itemize}
					\item Firmas tomadoras de precios.
					\item Libre entrada y salida.
					\item Firmas homogéneas (misma función de costos).
				\end{itemize}
		\end{frame}

		\begin{frame}
			\frametitle{Equilibrio en el Largo Plazo}
			Ejemplo:
				\begin{itemize}
					\item $t=0$: Mercado en equilibrio de corto y largo plazo. $n^\peq{*}_\peq{0}$ empresas produciendo $q^\peq{*}_\peq{0}$ cada una, con $Q^\peq{*}_\peq{0}=n^\peq{*}_\peq{0}\cdot q^\peq{*}_\peq{0}$ y $\pi^\peq{*}_\peq{0}=0$.
				\end{itemize}
		\end{frame}

		\begin{frame}
			\frametitle{Equilibrio en el Largo Plazo}
			\begin{figure}[hbtp!]
				\centering
				\begin{subfigure}[b]{0.49\textwidth}
					\begin{tikzpicture}[scale=.7]
						\draw [dashed,help lines] (2,2)--(4,2)--(4,0);
						%\draw [dashed,help lines] (0,3.5)--(1.5,3.5)--(1.5,0);
						%\draw [ultra thick,red] plot [domain=3.5:7.8] (\x,{2-.003*\x-.2*\x^2+.035*\x^3});
						\draw [ultra thick,purple] plot [domain=2.5:6] (\x,{(-.5*\x+2)^2+2});
						%\draw [ultra thick,teal] plot [domain=0:4] (\x,{4-\x});
						%\draw [ultra thick,purple] plot [domain=0:5] (\x,{5-\x});
						%\node [above,teal] at (4,0) {$D_\peq{0}$};
						%\node [above,purple] at (5,0) {$D_\peq{1}$};
						%\draw [ultra thick,teal] plot [domain=0:6] (\x,{\x});
						%\draw [<->,ultra thick] (0,7)--(0,0)--(7,0);
						\draw [ultra thick,teal] plot [domain=3.2:6] (\x,{2*\x-6});
						\draw [<->,ultra thick] (2,7)--(2,0)--(9,0);
						%\node [right,teal] at (6,6) {$S_\peq{0}$};
						%\node [right,purple] at (5,7) {$S_\peq{1}$};
						\node [left] at (2,7) {$p$};
						\node [below] at (9,0) {$q$};
						\node [right,teal] at (6,6) {\scriptsize $CMg\rp{q}$};
						\node [right,purple] at (6,3) {\scriptsize $CMeT\rp{q}$};
						\draw [fill,blue] (4,2) circle [radius=.1];
						%\draw [fill,blue] (1.5,3.5) circle [radius=.1]; 
						\node [left] at (2,2) {\tiny $p^\peq{*}_\peq{0}$};
						\node [below] at (4,0) {\tiny $q^\peq{*}_\peq{0}$};
						%\node [left] at (0,3.5) {\tiny $p^\peq{*}_\peq{1}$};
						%\node [below] at (1.5,0) {\tiny $q^\peq{*}_\peq{1}$};
						\node [below] at (2,0) {\tiny 0};
						\node at (2,8) {$t=0$};
						\node at (5,7) {\scriptsize \underline{Firma}};
					\end{tikzpicture}
				\end{subfigure}
				\begin{subfigure}[b]{0.49\textwidth}
					\begin{tikzpicture}[scale=.7]
						\draw [dashed,help lines] (0,2)--(2,2)--(2,0);
						%\draw [dashed,help lines] (0,3.5)--(2.5,3.5)--(2.5,0);
						\draw [ultra thick,teal] plot [domain=0:4] (\x,{4-\x});
						%\draw [ultra thick,purple] plot [domain=0:6] (\x,{6-\x});
						\node [above,teal] at (4,0) {$D_\peq{0}$};
						%\node [above,purple] at (6,0) {$D_\peq{1}$};
						\draw [ultra thick,teal] plot [domain=0:6] (\x,{\x});
						\draw [<->,ultra thick] (0,7)--(0,0)--(7,0);
						%\draw [ultra thick,purple] plot [domain=0:6] (\x,{1+\x});
						\draw [<->,ultra thick] (0,7)--(0,0)--(7,0);
						\node [right,teal] at (6,6) {$S^\peq{CP}_\peq{0}$};
						%\node [right,purple] at (6,7) {$S_\peq{1}$};
						\node [left] at (0,7) {$p$};
						\node [below] at (7,0) {$Q$};
						\draw [fill,blue] (2,2) circle [radius=.1];
						%\draw [fill,blue] (2.5,3.5) circle [radius=.1]; 
						\node [left] at (0,2) {\tiny $p^\peq{*}_\peq{0}$};
						\node [below] at (2,0) {\tiny $Q^\peq{*}_\peq{0}$};
						%\node [left] at (0,3.5) {\tiny $p^\peq{*}_\peq{1}$};
						%\node [below] at (2.5,0) {\tiny $q^\peq{*}_\peq{1}$};
						\node [below] at (0,0) {\tiny 0};
						\node at (3,7) {\scriptsize \underline{Mercado}};
					\end{tikzpicture}
				\end{subfigure}
			\end{figure}
		\end{frame}

		\begin{frame}
			\frametitle{Equilibrio en el Largo Plazo}
				\begin{itemize}
					\item $t=1$: Aumenta demanda de mercado. Nuevo equilibrio de corto plazo con las mismas $n^\peq{*}_\peq{0}$ empresas produciendo $q^\peq{*}_\peq{1}>q^\peq{*}_\peq{0}$ cada una, $Q^\peq{*}_\peq{1}=n^\peq{*}_\peq{0}\cdot q^\peq{*}_\peq{1}>Q^\peq{*}_\peq{0}$, $p^\peq{*}_\peq{1}>p^\peq{*}_\peq{0}$ y $\pi^\peq{*}_\peq{1}>0$.
				\end{itemize}
		\end{frame}

		\begin{frame}
			\frametitle{Equilibrio en el Largo Plazo}
			\begin{figure}[hbtp!]
				\centering
				\begin{subfigure}[b]{0.49\textwidth}
					\begin{tikzpicture}[scale=.7]
						\draw [fill,green,opacity=.5] (2,4.5)--(5.25,4.5)--(5.25,2.390625)--(2,2.390625);
						\draw [dashed,help lines] (2,2)--(4,2)--(4,0);
						\draw [dashed,help lines] (2,4.5)--(5.25,4.5)--(5.25,0);
						\draw [dashed,help lines] (2,2.390625)--(5.25,2.390625);
						%\draw [dashed,help lines] (0,3.5)--(1.5,3.5)--(1.5,0);
						%\draw [ultra thick,red] plot [domain=3.5:7.8] (\x,{2-.003*\x-.2*\x^2+.035*\x^3});
						\draw [ultra thick,purple] plot [domain=2.5:6] (\x,{(-.5*\x+2)^2+2});
						%\draw [ultra thick,teal] plot [domain=0:4] (\x,{4-\x});
						%\draw [ultra thick,purple] plot [domain=0:5] (\x,{5-\x});
						%\node [above,teal] at (4,0) {$D_\peq{0}$};
						%\node [above,purple] at (5,0) {$D_\peq{1}$};
						%\draw [ultra thick,teal] plot [domain=0:6] (\x,{\x});
						%\draw [<->,ultra thick] (0,7)--(0,0)--(7,0);
						\draw [ultra thick,teal] plot [domain=3.2:6] (\x,{2*\x-6});
						\draw [<->,ultra thick] (2,7)--(2,0)--(8,0);
						%\node [right,teal] at (6,6) {$S_\peq{0}$};
						%\node [right,purple] at (5,7) {$S_\peq{1}$};
						\node [left] at (2,7) {$p$};
						\node [below] at (8,0) {$q$};
						\node [right,teal] at (6,6) {\scriptsize $CMg\rp{q}$};
						\node [right,purple] at (6,3) {\scriptsize $CMeT\rp{q}$};
						\draw [fill,blue] (4,2) circle [radius=.1];
						\draw [fill,blue] (5.25,4.5) circle [radius=.1];
						\draw [fill,blue] (5.25,2.390625) circle [radius=.1];
						\node [left] at (2,2) {\tiny $p^\peq{*}_\peq{0}$};
						\node [below] at (4,0) {\tiny $q^\peq{*}_\peq{0}$};
						\node [left] at (2,4.5) {\tiny $p^\peq{*}_\peq{1}$};
						\node [left] at (2,2.390625) {\tiny $CMeT_\peq{1}$};
						\node [below] at (5.25,0) {\tiny $q^\peq{*}_\peq{1}$};
						\node [below] at (2,0) {\tiny 0};
						\node at (2,8) {$t=1$};
						\node at (5,7) {\scriptsize \underline{Firma}};
						\node at (3.5,3.5) {\scriptsize $\pi^\peq{*}_\peq{1}$};
					\end{tikzpicture}
				\end{subfigure}
				\begin{subfigure}[b]{0.49\textwidth}
					\begin{tikzpicture}[scale=.7]
						\draw [dashed,help lines] (0,2)--(2,2)--(2,0);
						\draw [dashed,help lines] (0,4.5)--(4.5,4.5)--(4.5,0);
						\draw [ultra thick,teal] plot [domain=0:4] (\x,{4-\x});
						\draw [ultra thick,purple] plot [domain=2.5:8] (\x,{9-\x});
						\node [above,teal] at (4,0) {$D_\peq{0}$};
						\node [above,purple] at (8,1) {$D_\peq{1}$};
						\draw [ultra thick,teal] plot [domain=0:6] (\x,{\x});
						%\draw [ultra thick,purple] plot [domain=5:8] (\x,{\x-5});
						\draw [<->,ultra thick] (0,7)--(0,0)--(8,0);
						%\draw [ultra thick,purple] plot [domain=0:6] (\x,{1+\x});
						\node [right,teal] at (6,6) {$S^\peq{CP}_\peq{0}$};
						%\node [right,purple] at (6,7) {$S^\peq{CP}_\peq{1}$};
						\node [left] at (0,7) {$p$};
						\node [below] at (8,0) {$Q$};
						\draw [fill,blue] (2,2) circle [radius=.1];
						\draw [fill,blue] (4.5,4.5) circle [radius=.1]; 
						\node [left] at (0,2) {\tiny $p^\peq{*}_\peq{0}$};
						\node [below] at (2,0) {\tiny $Q^\peq{*}_\peq{0}$};
						\node [left] at (0,4.5) {\tiny $p^\peq{*}_\peq{1}$};
						\node [below] at (4.5,0) {\tiny $Q^\peq{*}_\peq{1}$};
						\node [below] at (0,0) {\tiny 0};
						\node at (4,7) {\scriptsize \underline{Mercado}};
					\end{tikzpicture}
				\end{subfigure}
			\end{figure}
		\end{frame}

		\begin{frame}
			\frametitle{Equilibrio en el Largo Plazo}
				\begin{itemize}
					\item $t=2$: $\pi>0$ incentiva entrada de nuevas empresas, desplazando la oferta de corto plazo hacia la izquierda. Esto ocurre hasta que desaparece el incentivo cuando $\pi^\peq{*}_\peq{2}=0$. Nuevo equilibrio de largo plazo con $n^\peq{*}_\peq{2}>n^\peq{*}_\peq{0}$ empresas produciendo $q^\peq{*}_\peq{2}=q^\peq{*}_\peq{0}$ cada una, $Q^\peq{*}_\peq{2}=n^\peq{*}_\peq{2}\cdot q^\peq{*}_\peq{0}>Q^\peq{*}_\peq{1}$ y $p^\peq{*}_\peq{2}>p^\peq{*}_\peq{0}$.
				\end{itemize}
		\end{frame}

		\begin{frame}
			\frametitle{Equilibrio en el Largo Plazo}
			\begin{figure}[hbtp!]
				\centering
				\begin{subfigure}[b]{0.49\textwidth}
					\begin{tikzpicture}[scale=.7]
						%\draw [fill,green,opacity=.5] (2,4.5)--(5.25,4.5)--(5.25,2.390625)--(2,2.390625);
						\draw [dashed,help lines] (2,2)--(4,2)--(4,0);
						\draw [dashed,help lines] (2,4.5)--(5.25,4.5)--(5.25,0);
						%\draw [dashed,help lines] (2,2.390625)--(5.25,2.390625);
						%\draw [dashed,help lines] (0,3.5)--(1.5,3.5)--(1.5,0);
						%\draw [ultra thick,red] plot [domain=3.5:7.8] (\x,{2-.003*\x-.2*\x^2+.035*\x^3});
						\draw [ultra thick,purple] plot [domain=2.5:6] (\x,{(-.5*\x+2)^2+2});
						%\draw [ultra thick,teal] plot [domain=0:4] (\x,{4-\x});
						%\draw [ultra thick,purple] plot [domain=0:5] (\x,{5-\x});
						%\node [above,teal] at (4,0) {$D_\peq{0}$};
						%\node [above,purple] at (5,0) {$D_\peq{1}$};
						%\draw [ultra thick,teal] plot [domain=0:6] (\x,{\x});
						%\draw [<->,ultra thick] (0,7)--(0,0)--(7,0);
						\draw [ultra thick,teal] plot [domain=3.2:6] (\x,{2*\x-6});
						\draw [<->,ultra thick] (2,7)--(2,0)--(8,0);
						%\node [right,teal] at (6,6) {$S_\peq{0}$};
						%\node [right,purple] at (5,7) {$S_\peq{1}$};
						\node [left] at (2,7) {$p$};
						\node [below] at (8,0) {$q$};
						\node [right,teal] at (6,6) {\scriptsize $CMg\rp{q}$};
						\node [right,purple] at (6,3) {\scriptsize $CMeT\rp{q}$};
						\draw [fill,blue] (4,2) circle [radius=.1];
						\draw [fill,blue] (5.25,4.5) circle [radius=.1];
						%\draw [fill,blue] (5.25,2.390625) circle [radius=.1];
						\node [left] at (2,2) {\tiny $p^\peq{*}_\peq{2}$};
						\node [below] at (4,0) {\tiny $q^\peq{*}_\peq{2}$};
						\node [left] at (2,4.5) {\tiny $p^\peq{*}_\peq{1}$};
						%\node [left] at (2,2.390625) {\tiny $CMeT_\peq{1}$};
						\node [below] at (5.25,0) {\tiny $q^\peq{*}_\peq{1}$};
						\node [below] at (2,0) {\tiny 0};
						\node at (2,8) {$t=2$};
						\node at (5,7) {\scriptsize \underline{Firma}};
						%\node at (3.5,3.5) {\scriptsize $\pi^\peq{*}_\peq{1}$};
					\end{tikzpicture}
				\end{subfigure}
				\begin{subfigure}[b]{0.49\textwidth}
					\begin{tikzpicture}[scale=.7]
						\draw [dashed,help lines] (0,2)--(2,2)--(2,0);
						\draw [dashed,help lines] (0,4.5)--(4.5,4.5)--(4.5,0);
						\draw [dashed,help lines] (2,2)--(7,2)--(7,0);
						\draw [ultra thick,teal] plot [domain=0:4] (\x,{4-\x});
						\draw [ultra thick,purple] plot [domain=2.5:8] (\x,{9-\x});
						\node [above,teal] at (4,0) {$D_\peq{0}$};
						\node [above,purple] at (8,1) {$D_\peq{1}$};
						\draw [ultra thick,teal] plot [domain=0:6] (\x,{\x});
						\draw [ultra thick,purple] plot [domain=5:8] (\x,{\x-5});
						\draw [<->,ultra thick] (0,7)--(0,0)--(8,0);
						%\draw [ultra thick,purple] plot [domain=0:6] (\x,{1+\x});
						\node [right,teal] at (6,6) {$S^\peq{CP}_\peq{0}$};
						\node [above,purple] at (8,3) {$S^\peq{CP}_\peq{2}$};
						%\node [right,purple] at (6,7) {$S_\peq{1}$};
						\node [left] at (0,7) {$p$};
						\node [below] at (8,0) {$Q$};
						\draw [fill,blue] (2,2) circle [radius=.1];
						\draw [fill,blue] (4.5,4.5) circle [radius=.1];
						\draw [fill,blue] (7,2) circle [radius=.1];
						\node [left] at (0,2) {\tiny $p^\peq{*}_\peq{2}$};
						\node [below] at (2,0) {\tiny $Q^\peq{*}_\peq{0}$};
						\node [left] at (0,4.5) {\tiny $p^\peq{*}_\peq{1}$};
						\node [below] at (4.5,0) {\tiny $Q^\peq{*}_\peq{1}$};
						\node [below] at (7,0) {\tiny $Q^\peq{*}_\peq{2}$};
						\node [below] at (0,0) {\tiny 0};
						\node at (4,7) {\scriptsize \underline{Mercado}};
					\end{tikzpicture}
				\end{subfigure}
			\end{figure}
		\end{frame}

		\begin{frame}
			\frametitle{Equilibrio de Mercado}
			\begin{figure}[hbtp!]
				\centering
				\begin{subfigure}[b]{0.49\textwidth}
					\begin{tikzpicture}[scale=.7]
						%\draw [fill,green,opacity=.5] (2,4.5)--(5.25,4.5)--(5.25,2.390625)--(2,2.390625);
						\draw [dashed,help lines] (2,2)--(4,2)--(4,0);
						%\draw [dashed,help lines] (2,4.5)--(5.25,4.5)--(5.25,0);
						%\draw [dashed,help lines] (2,2.390625)--(5.25,2.390625);
						%\draw [dashed,help lines] (0,3.5)--(1.5,3.5)--(1.5,0);
						%\draw [ultra thick,red] plot [domain=3.5:7.8] (\x,{2-.003*\x-.2*\x^2+.035*\x^3});
						\draw [ultra thick,purple] plot [domain=2.5:6] (\x,{(-.5*\x+2)^2+2});
						%\draw [ultra thick,teal] plot [domain=0:4] (\x,{4-\x});
						%\draw [ultra thick,purple] plot [domain=0:5] (\x,{5-\x});
						%\node [above,teal] at (4,0) {$D_\peq{0}$};
						%\node [above,purple] at (5,0) {$D_\peq{1}$};
						%\draw [ultra thick,teal] plot [domain=0:6] (\x,{\x});
						%\draw [<->,ultra thick] (0,7)--(0,0)--(7,0);
						\draw [ultra thick,teal] plot [domain=3.2:6] (\x,{2*\x-6});
						\draw [<->,ultra thick] (2,7)--(2,0)--(8,0);
						%\node [right,teal] at (6,6) {$S_\peq{0}$};
						%\node [right,purple] at (5,7) {$S_\peq{1}$};
						\node [left] at (2,7) {$p$};
						\node [below] at (8,0) {$q$};
						\node [right,teal] at (6,6) {\scriptsize $CMg\rp{q}$};
						\node [right,purple] at (6,3) {\scriptsize $CMeT\rp{q}$};
						\draw [fill,blue] (4,2) circle [radius=.1];
						%\draw [fill,blue] (5.25,4.5) circle [radius=.1];
						%\draw [fill,blue] (5.25,2.390625) circle [radius=.1];
						\node [left] at (2,2) {\tiny $p^\peq{*}$};
						\node [below] at (4,0) {\tiny $q^\peq{*}$};
						%\node [left] at (2,4.5) {\tiny $p^\peq{*}_\peq{1}$};
						%\node [left] at (2,2.390625) {\tiny $CMeT_\peq{1}$};
						\node [below] at (2,0) {\tiny 0};
						\node at (5,7) {\scriptsize \underline{Firma}};
						%\node at (3.5,3.5) {\scriptsize $\pi^\peq{*}_\peq{1}$};
`						\node at (2,8) {Largo plazo};
					\end{tikzpicture}
				\end{subfigure}
				\begin{subfigure}[b]{0.49\textwidth}
					\begin{tikzpicture}[scale=.7]
						\draw [dashed,help lines] (0,2)--(2,2)--(2,0);
						\draw [dashed,help lines] (7,2)--(7,0);
						\draw [ultra thick,teal] (0,2)--(8,2);
						\draw [ultra thick,teal] plot [domain=0:4] (\x,{4-\x});
						\draw [ultra thick,purple] plot [domain=2.5:8] (\x,{9-\x});
						\node [above,teal] at (4,0) {$D_\peq{0}$};
						\node [above,purple] at (8,1) {$D_\peq{1}$};
						%\draw [ultra thick,teal] plot [domain=0:6] (\x,{\x});
						%\draw [ultra thick,purple] plot [domain=5:8] (\x,{\x-5});
						\draw [<->,ultra thick] (0,7)--(0,0)--(8,0);
						%\draw [ultra thick,purple] plot [domain=0:6] (\x,{1+\x});
						%\node [right,teal] at (6,6) {$S^\peq{CP}_\peq{0}$};
						\node [above,teal] at (8,2) {$S^\peq{LP}$};
						%\node [above,purple] at (8,3) {$S^\peq{CP}_\peq{1}$};
						%\node [right,purple] at (6,7) {$S_\peq{1}$};
						\node [left] at (0,7) {$p$};
						\node [below] at (8,0) {$Q$};
						\draw [fill,blue] (2,2) circle [radius=.1];
						%\draw [fill,blue] (4.5,4.5) circle [radius=.1];
						\draw [fill,blue] (7,2) circle [radius=.1];
						\node [left] at (0,2) {\tiny $p^\peq{*}$};
						\node [below] at (2,0) {\tiny $Q^\peq{*}_\peq{0}$};
						\node [below] at (7,0) {\tiny $Q^\peq{*}_\peq{2}$};
						\node [below] at (0,0) {\tiny 0};
						\node at (4,7) {\scriptsize \underline{Mercado}};
					\end{tikzpicture}
				\end{subfigure}
			\end{figure}
		\end{frame}

	\section{Análisis de Bienestar}

		\begin{frame}
			\frametitle{Eficiencia del Equilibrio de Mercado}
			\begin{mydef}
				\begin{itemize}
					\item \textbf{Economía del bienestar:} Estudio de cómo la asignación de recursos afecta el bienestar económico.
					\item \textbf{Excedente total:} Suma del excedente del productor y el excedente del consumidor. $$ET=EP+EC$$
				\end{itemize}
			\end{mydef}
		\end{frame}

		\begin{frame}
			\frametitle{Eficiencia del Equilibrio de Mercado}
			Notar que el excedente total que se genera al transar una unidad adicional es $$\sqp{v\rp{q}-p}+\sqp{p-c\rp{q}}=v\rp{q}-c\rp{q}$$ donde $v\rp{q}$ representa la disposición de los consumidores a pagar por una unidad adicional y $c\rp{q}$ es el costo de producir una unidad adicional para los productores.
		\end{frame}
	
		\begin{frame}
			\frametitle{Eficiencia del Equilibrio de Mercado}
			\centering
			\begin{tikzpicture}
				\draw [ultra thick,teal] plot [domain=0:4] (\x,{4-\x});
				\draw [ultra thick,teal] plot [domain=0:4] (\x,{\x});
				\draw [dashed] (1,0)--(1,3)--(0,3);
				\draw [dashed] (1,1)--(0,1);
				\draw [dashed] (1.5,0)--(1.5,2.5)--(0,2.5);
				\draw [dashed] (1.5,1.5)--(0,1.5);
				\draw [<->,ultra thick] (0,6)--(0,0)--(6,0);
				\draw [fill,blue] (1,3) circle [radius=.1];
				\draw [fill,blue] (1,1) circle [radius=.1];
				\draw [fill,blue] (1.5,2.5) circle [radius=.1];
				\draw [fill,blue] (1.5,1.5) circle [radius=.1];
				\node [below] at (6,0) {$q$};
				\node [left] at (0,6) {$p$};
				\node [below] at (1,0) {\tiny $q_\peq{0}$};
				\node [below] at (1.5,0) {\tiny $q_\peq{1}$};
				\node [left] at (0,1.5) {\tiny $c\rp{q_\peq{1}}$};
				\node [left] at (0,2.5) {\tiny $v\rp{q_\peq{1}}$};
				\node [left] at (0,1) {\tiny $c\rp{q_\peq{0}}$};
				\node [left] at (0,3) {\tiny $v\rp{q_\peq{0}}$};
				\node [above right,teal] at (4,0) {$D$};
				\node [right,teal] at (4,4) {$S$};
			\end{tikzpicture}
		\end{frame}	

		\begin{frame}
			\frametitle{Eficiencia del Equilibrio de Mercado}
			\centering
			\begin{tikzpicture}
				\draw [fill,opacity=.2,color=blue] (0,0)--(1,1)--(1,3)--(0,4);
				\draw [ultra thick,teal] plot [domain=0:4] (\x,{4-\x});
				\draw [ultra thick,teal] plot [domain=0:4] (\x,{\x});
				\draw [dashed] (1,0)--(1,3);
				%\draw [dashed] (1.5,0)--(1.5,2.5);
				\draw [<->,ultra thick] (0,6)--(0,0)--(6,0);
				\node [below] at (6,0) {$q$};
				\node [left] at (0,6) {$p$};
				\node [below] at (1,0) {\tiny $q_\peq{0}$};
				%\node [below] at (1.5,0) {\tiny $q_\peq{1}$};
				\node at (.5,2) {$ET_\peq{0}$};
				\node [above right,teal] at (4,0) {$D$};
				\node [right,teal] at (4,4) {$S$};
			\end{tikzpicture}
		\end{frame}	

		\begin{frame}
			\frametitle{Eficiencia del Equilibrio de Mercado}
			\centering
			\begin{tikzpicture}
				\draw [fill,opacity=.2,color=blue] (0,0)--(1.5,1.5)--(1.5,2.5)--(0,4);
				\draw [ultra thick,teal] plot [domain=0:4] (\x,{4-\x});
				\draw [ultra thick,teal] plot [domain=0:4] (\x,{\x});
				%\draw [dashed] (1,0)--(1,3);
				\draw [dashed] (1.5,0)--(1.5,2.5);
				\draw [<->,ultra thick] (0,6)--(0,0)--(6,0);
				\node [below] at (6,0) {$q$};
				\node [left] at (0,6) {$p$};
				%\node [below] at (1,0) {\tiny $q_\peq{0}$};
				\node [below] at (1.5,0) {\tiny $q_\peq{1}$};
				\node at (.8,2) {$ET_\peq{1}$};
				\node [above right,teal] at (4,0) {$D$};
				\node [right,teal] at (4,4) {$S$};
			\end{tikzpicture}
		\end{frame}	

		\begin{frame}
			\frametitle{Eficiencia del Equilibrio de Mercado}
			\begin{mydef}
				\begin{itemize}
					\item \textbf{Asignación eficiente:} Asignación que maximiza el excedente total.
					\item \textbf{Planificador social benevolente:} Personaje hipotético que actúa como un dictador todo poderoso, omnisapiente y bien intencionado.
				\end{itemize}
			\end{mydef}
			Notar que el planificador social benevolente escogería una asignación eficiente.
		\end{frame}

		\begin{frame}
			\frametitle{Eficiencia del Equilibrio de Mercado}
			¿Podría el planificador social mejorar el bienestar social escogiendo una asignación distinta a la asignación de equilibrio del mercado?
		\end{frame}

		\begin{frame}
			\frametitle{Eficiencia del Equilibrio de Mercado}
			Las fuerzas del mercado asignan la oferta del bien a los compradores que lo valoran más:
				\begin{itemize}
					\item En equilibrio, quienes compran son los consumidores que tienen una valoración superior al precio de equilibrio. 
						\begin{itemize}
							\item $\implies$ Los que no compran lo valoran menos.
						\end{itemize}
				\end{itemize}
		\end{frame}

		\begin{frame}
			\frametitle{Eficiencia del Equilibrio de Mercado}
			Las fuerzas del mercado asignan la demanda del bien a los vendedores que pueden producirlo al costo más bajo:
				\begin{itemize}
					\item En equilibrio, quienes producen son los vendedores que tienen un costo marginal inferior al precio de equilibrio. 
						\begin{itemize}
							\item $\implies$ Los que no venden tienen costos mayores.
						\end{itemize}
				\end{itemize}
		\end{frame}
	
		\begin{frame}
			\frametitle{Eficiencia del Equilibrio de Mercado}
			\textbf{Conclusión:} Dada la cantidad transada en el equilibrio de mercado ($q^\peq{*}$), el planificador social no puede aumentar el excedente total reasignando el consumo o la producción.
		\end{frame}

		\begin{frame}
			\frametitle{Eficiencia del Equilibrio de Mercado}
			Pero, ¿puede aumentar el excedente escogiendo una cantidad distinta a $q^\peq{*}$?
		\end{frame}

		\begin{frame}
			\frametitle{Eficiencia del Equilibrio de Mercado}
			\centering
			\begin{tikzpicture}
				\draw [fill,opacity=.2,color=blue] (0,0)--(1,1)--(1,3)--(0,4);
				\draw [ultra thick,teal] plot [domain=0:4] (\x,{4-\x});
				\draw [ultra thick,teal] plot [domain=0:4] (\x,{\x});
				\draw [dashed] (1,0)--(1,3);
				\draw [dashed] (2,0)--(2,2);
				%\draw [dashed] (1,1)--(0,1);
				\draw [dashed] (1.5,0)--(1.5,2.5)--(0,2.5);
				\draw [dashed] (1.5,1.5)--(0,1.5);
				\draw [<->,ultra thick] (0,6)--(0,0)--(6,0);
				\draw [ultra thick, green] (1.5,1.5)--(1.5,2.5);
				\draw [fill,blue] (1.5,2.5) circle [radius=.1];
				\draw [fill,blue] (1.5,1.5) circle [radius=.1];
				\draw [fill,blue] (2,2) circle [radius=.1];
				\node [below] at (6,0) {$q$};
				\node [left] at (0,6) {$p$};
				\node [below] at (2,0) {\tiny $q^\peq{*}$};
				\node [below] at (1,0) {\tiny $q_\peq{0}$};
				\node [below] at (1.5,0) {\tiny $q_\peq{1}$};
				\node [left] at (0,1.5) {\tiny $c\rp{q_\peq{1}}$};
				\node [left] at (0,2.5) {\tiny $v\rp{q_\peq{1}}$};
				%\node [left] at (0,1) {\tiny $c\rp{q_\peq{0}}$};
				%\node [left] at (0,3) {\tiny $v\rp{q_\peq{0}}$};
				\node [above right,teal] at (4,0) {$D$};
				\node [right,teal] at (4,4) {$S$};
				\node at (.5,2) {$ET_\peq{0}$};
			\end{tikzpicture}
		\end{frame}	

		\begin{frame}
			\frametitle{Eficiencia del Equilibrio de Mercado}
			\centering
			\begin{tikzpicture}
				\draw [fill,opacity=.2,color=blue] (0,0)--(2,2)--(0,4);
				\draw [fill,opacity=.2,color=purple] (2,2)--(3.5,3.5)--(3.5,.5);
				\draw [ultra thick,teal] plot [domain=0:4] (\x,{4-\x});
				\draw [ultra thick,teal] plot [domain=0:4] (\x,{\x});
				\draw [dashed] (2,0)--(2,2);
				%\draw [dashed] (1,1)--(0,1);
				\draw [dashed] (2.5,0)--(2.5,2.5)--(0,2.5);
				\draw [dashed] (2.5,1.5)--(0,1.5);
				\draw [dashed] (3.5,0)--(3.5,3.5);
				\draw [<->,ultra thick] (0,6)--(0,0)--(6,0);
				\draw [ultra thick,red] (2.5,1.5)--(2.5,2.5);
				\draw [fill,blue] (2.5,2.5) circle [radius=.1];
				\draw [fill,blue] (2.5,1.5) circle [radius=.1];
				\draw [fill,blue] (2,2) circle [radius=.1];
				\node [below] at (6,0) {$q$};
				\node [left] at (0,6) {$p$};
				\node [below] at (2,0) {\tiny $q^\peq{*}$};
				\node [below] at (3.5,0) {\tiny $q_\peq{0}$};
				\node [below] at (2.5,0) {\tiny $q_\peq{1}$};
				\node [left] at (0,1.5) {\tiny $v\rp{q_\peq{1}}$};
				\node [left] at (0,2.5) {\tiny $c\rp{q_\peq{1}}$};
				%\node [left] at (0,1) {\tiny $c\rp{q_\peq{0}}$};
				%\node [left] at (0,3) {\tiny $v\rp{q_\peq{0}}$};
				\node [above right,teal] at (4,0) {$D$};
				\node [right,teal] at (4,4) {$S$};
				\node at (.5,2) {$ET^\peq{*}$};
				\node at (3,2) {$-ET_\peq{0}$};
			\end{tikzpicture}
		\end{frame}	
	
\begin{frame}
			\frametitle{Eficiencia del Equilibrio de Mercado}
			\textbf{Conclusión:} La cantidad transada en el equilibrio de mercado ($q^\peq{*}$) maximiza el excedente total. El planificador social no puede aumentar el excedente total escogiendo una cantidad distinta.
		\end{frame}	
	
	\section{Política Económica}

		\begin{frame}
			\frametitle{Política Económica en Mercados Competitivos}
			Estudiaremos los efectos de distintas políticas sobre la asignación de recursos y el excedente total:
			\begin{itemize}
				\item Controles de precios
				\item Impuestos y subsidios
				\item Apertura comercial
				\item Aranceles y cuotas de importación
			\end{itemize}
		\end{frame}

		\begin{frame}
			\frametitle{Política Económica en Mercados Competitivos}
			Notar que
			\begin{itemize}
				\item Ya vimos que el excedente total se maximiza en la asignación de equilibrio del mercado.
					\begin{itemize}
						\item $\implies$ Cualquier política que produzca una asignación diferente resultará en un excedente total inferior. 
					\end{itemize}	
			\end{itemize}
		\end{frame}

		\begin{frame}
			\frametitle{Controles de precios}
			\begin{mydef}
				\begin{itemize}
					\item \textbf{Precio máximo ($\mathbf{\overline{p}}$):} Precio más alto al que legalmente se puede transar un bien.
					\item \textbf{Precio mínimo ($\mathbf{\underline{p}}$):} Precio más bajo al que legalmente se puede transar un bien.
				\end{itemize}
			\end{mydef}
		\end{frame}
		
		\begin{frame}
			\frametitle{Precio Máximo}
			Caso 1: $\overline{p}>p^\peq{*}$
			
			\vspace{.1cm}
			
			\centering
			\begin{tikzpicture}[scale=1]
				\draw [ultra thick,teal] plot [domain=0:4] (\x,{4-\x});
				\node [above right,teal] at (4,0) {$D$};
				\draw [ultra thick,teal] plot [domain=0:4] (\x,{\x});
				\draw [thick,blue] (0,2)--(2,2)--(2,0);
				\draw [thick,red] (0,3)--(5,3);
				\draw [<->,ultra thick] (0,5)--(0,0)--(5,0);
				\node [right,teal] at (4,4) {$S$};
				\node [left] at (0,5) {$p$};
				\node [below] at (5,0) {$q$};
				\node [left] at (0,2) {\tiny $p^\peq{*}$};
				\node [left] at (0,3) {\tiny $\overline{p}$};
				\node [below] at (2,0) {\tiny $q^\peq{*}$};
				%\node [below] at (3,0) {\tiny $q^\peq{s}\rp{p_\peq{1}}$};
				%\node [below] at (1,0) {\tiny $q^\peq{d}\rp{p_\peq{1}}$};
				\node [below] at (0,0) {\tiny 0};
				\node [below,align=center] at (3.5,2) {\scriptsize \textbf{No tiene} \\ \scriptsize \textbf{efecto}};
				%\draw [decorate,decoration={brace,amplitude=6pt,mirror},xshift=0.2pt,yshift=-0.2pt,red](1,-.4) -- (3,-.4) node[red,midway,yshift=-0.5cm] {\tiny \pbox{\textwidth}{Exceso \\ de oferta}};
			\end{tikzpicture}
		\end{frame}	
		
		\begin{frame}
			\frametitle{Precio Máximo}
			Caso 2: $\overline{p}<p^\peq{*}$
			
			\vspace{.1cm}
			
			\centering
			\begin{tikzpicture}[scale=1]
				\draw [ultra thick,teal] plot [domain=0:4] (\x,{4-\x});
				\node [above right,teal] at (4,0) {$D$};
				\draw [ultra thick,teal] plot [domain=0:4] (\x,{\x});
				\draw [thick,blue] (0,2)--(2,2)--(2,0);
				\draw [thick,red] (0,1)--(5,1);
				\draw [thick,red] (1,1)--(1,0);
				\draw [thick,red] (3,1)--(3,0);
				\draw [<->,ultra thick] (0,5)--(0,0)--(5,0);
				\node [right,teal] at (4,4) {$S$};
				\node [left] at (0,5) {$p$};
				\node [below] at (5,0) {$q$};
				\node [left] at (0,2) {\tiny $p^\peq{*}$};
				\node [left] at (0,1) {\tiny $\overline{p}$};
				\node [below] at (2,0) {\tiny $q^\peq{*}$};
				\node [below] at (3,0) {\tiny $q^\peq{d}\rp{\overline{p}}$};
				\node [below] at (1,0) {\tiny $q^\peq{s}\rp{\overline{p}}$};
				\node [below] at (0,0) {\tiny 0};
				\draw [decorate,decoration={brace,amplitude=6pt,mirror},xshift=0.2pt,yshift=-0.2pt,red](1,-.4) -- (3,-.4) node[red,midway,yshift=-0.5cm] {\tiny \pbox{\textwidth}{Exceso de \\ demanda}};
			\end{tikzpicture}
		\end{frame}	

		\begin{frame}
			\frametitle{Precio Máximo}
			Caso 2: $\overline{p}<p^\peq{*}$
			
			\vspace{.1cm}
			
			\centering
			\begin{tikzpicture}[scale=1]
				\draw [ultra thick,teal] plot [domain=0:4] (\x,{4-\x});
				\node [above right,teal] at (4,0) {$D$};
				\draw [ultra thick,teal] plot [domain=0:4] (\x,{\x});
				\draw [thick,blue] (0,2)--(2,2)--(2,0);
				\draw [thick,red] (0,1)--(5,1);
				\draw [thick,red] (1,1)--(1,0);
				\draw [thick,red] (3,1)--(3,0);
				\draw [thick,red] (1,1)--(1,5);
				\draw [<->,ultra thick] (0,5)--(0,0)--(5,0);
				\node [right,teal] at (4,4) {$S$};
				\node [left] at (0,5) {$p$};
				\node [below] at (5,0) {$q$};
				\node [left] at (0,2) {\tiny $p^\peq{*}$};
				\node [left] at (0,1) {\tiny $\overline{p}$};
				\node [below] at (2,0) {\tiny $q^\peq{*}$};
				\node [below] at (3,0) {\tiny $q^\peq{d}\rp{\overline{p}}$};
				\node [below] at (1,0) {\tiny $q^\peq{s}\rp{\overline{p}}$};
				\node [below] at (0,0) {\tiny 0};
				\node [above] at (.3,2.2) {\textbf{A}};
				\node [above] at (1.2,2.2) {\textbf{B}};
				\node [below] at (.3,1.8) {\textbf{C}};
				\node [below] at (1.2,1.8) {\textbf{D}};
				\node [below] at (.3,1) {\textbf{E}};
				\draw [decorate,decoration={brace,amplitude=6pt,mirror},xshift=0.2pt,yshift=-0.2pt,red](1,-.4) -- (3,-.4) node[red,midway,yshift=-0.5cm] {\tiny \pbox{\textwidth}{Exceso de \\ demanda}};
			\end{tikzpicture}
		\end{frame}	

		\begin{frame}
			\frametitle{Precio Máximo}
			Excedente del consumidor
			\begin{figure}[hbtp!]
				\centering
				\begin{subfigure}[b]{0.49\textwidth}
					\begin{tikzpicture}[scale=1]
						\draw [fill,opacity=.3,blue] (0,4)--(2,2)--(0,2);
						\draw [ultra thick,teal] plot [domain=0:4] (\x,{4-\x});
						\node [above right,teal] at (4,0) {$D$};
						\draw [ultra thick,teal] plot [domain=0:4] (\x,{\x});
						\draw [thick,blue] (0,2)--(2,2)--(2,0);
						\draw [thick,red] (0,1)--(5,1);
						\draw [thick,red] (1,1)--(1,0);
						\draw [thick,red] (3,1)--(3,0);
						\draw [thick,red] (1,1)--(1,5);
						\draw [<->,ultra thick] (0,5)--(0,0)--(5,0);
						\node [right,teal] at (4,4) {$S$};
						\node [left] at (0,5) {$p$};
						\node [below] at (5,0) {$q$};
						\node [left] at (0,2) {\tiny $p^\peq{*}$};
						\node [left] at (0,1) {\tiny $\overline{p}$};
						\node [below] at (2,0) {\tiny $q^\peq{*}$};
						\node [below] at (3,0) {\tiny $q^\peq{d}\rp{\overline{p}}$};
						\node [below] at (1,0) {\tiny $q^\peq{s}\rp{\overline{p}}$};
						\node [below] at (0,0) {\tiny 0};
						\node [above] at (.3,2.2) {\textbf{A}};
						\node [above] at (1.2,2.2) {\textbf{B}};
						\node [below] at (.3,1.8) {\textbf{C}};
						\node [below] at (1.2,1.8) {\textbf{D}};
						\node [below] at (.3,1) {\textbf{E}};
						\node at (2.5,5) {\underline{Sin política}};
					\end{tikzpicture}
				\end{subfigure}
				\begin{subfigure}[b]{0.49\textwidth}
					\begin{tikzpicture}[scale=1]
						\draw [fill,opacity=.3,blue] (0,4)--(1,3)--(1,1)--(0,1);
						\draw [ultra thick,teal] plot [domain=0:4] (\x,{4-\x});
						\node [above right,teal] at (4,0) {$D$};
						\draw [ultra thick,teal] plot [domain=0:4] (\x,{\x});
						\draw [thick,blue] (0,2)--(2,2)--(2,0);
						\draw [thick,red] (0,1)--(5,1);
						\draw [thick,red] (1,1)--(1,0);
						\draw [thick,red] (3,1)--(3,0);
						\draw [thick,red] (1,1)--(1,5);
						\draw [<->,ultra thick] (0,5)--(0,0)--(5,0);
						\node [right,teal] at (4,4) {$S$};
						\node [left] at (0,5) {$p$};
						\node [below] at (5,0) {$q$};
						\node [left] at (0,2) {\tiny $p^\peq{*}$};
						\node [left] at (0,1) {\tiny $\overline{p}$};
						\node [below] at (2,0) {\tiny $q^\peq{*}$};
						\node [below] at (3,0) {\tiny $q^\peq{d}\rp{\overline{p}}$};
						\node [below] at (1,0) {\tiny $q^\peq{s}\rp{\overline{p}}$};
						\node [below] at (0,0) {\tiny 0};
						\node [above] at (.3,2.2) {\textbf{A}};
						\node [above] at (1.2,2.2) {\textbf{B}};
						\node [below] at (.3,1.8) {\textbf{C}};
						\node [below] at (1.2,1.8) {\textbf{D}};
						\node [below] at (.3,1) {\textbf{E}};
						\node at (2.5,5) {\underline{Con política}};
					\end{tikzpicture}
				\end{subfigure}
			\end{figure}
		\end{frame}

		\begin{frame}
			\frametitle{Precio Máximo}
			Excedente del productor
			\begin{figure}[hbtp!]
				\centering
				\begin{subfigure}[b]{0.49\textwidth}
					\begin{tikzpicture}[scale=1]
						\draw [fill,opacity=.3,green] (0,0)--(2,2)--(0,2);
						\draw [ultra thick,teal] plot [domain=0:4] (\x,{4-\x});
						\node [above right,teal] at (4,0) {$D$};
						\draw [ultra thick,teal] plot [domain=0:4] (\x,{\x});
						\draw [thick,blue] (0,2)--(2,2)--(2,0);
						\draw [thick,red] (0,1)--(5,1);
						\draw [thick,red] (1,1)--(1,0);
						\draw [thick,red] (3,1)--(3,0);
						\draw [thick,red] (1,1)--(1,5);
						\draw [<->,ultra thick] (0,5)--(0,0)--(5,0);
						\node [right,teal] at (4,4) {$S$};
						\node [left] at (0,5) {$p$};
						\node [below] at (5,0) {$q$};
						\node [left] at (0,2) {\tiny $p^\peq{*}$};
						\node [left] at (0,1) {\tiny $\overline{p}$};
						\node [below] at (2,0) {\tiny $q^\peq{*}$};
						\node [below] at (3,0) {\tiny $q^\peq{d}\rp{\overline{p}}$};
						\node [below] at (1,0) {\tiny $q^\peq{s}\rp{\overline{p}}$};
						\node [below] at (0,0) {\tiny 0};
						\node [above] at (.3,2.2) {\textbf{A}};
						\node [above] at (1.2,2.2) {\textbf{B}};
						\node [below] at (.3,1.8) {\textbf{C}};
						\node [below] at (1.2,1.8) {\textbf{D}};
						\node [below] at (.3,1) {\textbf{E}};
						\node at (2.5,5) {\underline{Sin política}};
					\end{tikzpicture}
				\end{subfigure}
				\begin{subfigure}[b]{0.49\textwidth}
					\begin{tikzpicture}[scale=1]
						\draw [fill,opacity=.3,green] (0,0)--(1,1)--(0,1);
						\draw [ultra thick,teal] plot [domain=0:4] (\x,{4-\x});
						\node [above right,teal] at (4,0) {$D$};
						\draw [ultra thick,teal] plot [domain=0:4] (\x,{\x});
						\draw [thick,blue] (0,2)--(2,2)--(2,0);
						\draw [thick,red] (0,1)--(5,1);
						\draw [thick,red] (1,1)--(1,0);
						\draw [thick,red] (3,1)--(3,0);
						\draw [thick,red] (1,1)--(1,5);
						\draw [<->,ultra thick] (0,5)--(0,0)--(5,0);
						\node [right,teal] at (4,4) {$S$};
						\node [left] at (0,5) {$p$};
						\node [below] at (5,0) {$q$};
						\node [left] at (0,2) {\tiny $p^\peq{*}$};
						\node [left] at (0,1) {\tiny $\overline{p}$};
						\node [below] at (2,0) {\tiny $q^\peq{*}$};
						\node [below] at (3,0) {\tiny $q^\peq{d}\rp{\overline{p}}$};
						\node [below] at (1,0) {\tiny $q^\peq{s}\rp{\overline{p}}$};
						\node [below] at (0,0) {\tiny 0};
						\node [above] at (.3,2.2) {\textbf{A}};
						\node [above] at (1.2,2.2) {\textbf{B}};
						\node [below] at (.3,1.8) {\textbf{C}};
						\node [below] at (1.2,1.8) {\textbf{D}};
						\node [below] at (.3,1) {\textbf{E}};
						\node at (2.5,5) {\underline{Con política}};
					\end{tikzpicture}
				\end{subfigure}
			\end{figure}
		\end{frame}

		\begin{frame}
			\frametitle{Precio Máximo}
			Excedente total
			\begin{figure}[hbtp!]
				\centering
				\begin{subfigure}[b]{0.49\textwidth}
					\begin{tikzpicture}[scale=1]
						\draw [fill,opacity=.3,purple] (0,0)--(2,2)--(0,4);
						\draw [ultra thick,teal] plot [domain=0:4] (\x,{4-\x});
						\node [above right,teal] at (4,0) {$D$};
						\draw [ultra thick,teal] plot [domain=0:4] (\x,{\x});
						\draw [thick,blue] (0,2)--(2,2)--(2,0);
						\draw [thick,red] (0,1)--(5,1);
						\draw [thick,red] (1,1)--(1,0);
						\draw [thick,red] (3,1)--(3,0);
						\draw [thick,red] (1,1)--(1,5);
						\draw [<->,ultra thick] (0,5)--(0,0)--(5,0);
						\node [right,teal] at (4,4) {$S$};
						\node [left] at (0,5) {$p$};
						\node [below] at (5,0) {$q$};
						\node [left] at (0,2) {\tiny $p^\peq{*}$};
						\node [left] at (0,1) {\tiny $\overline{p}$};
						\node [below] at (2,0) {\tiny $q^\peq{*}$};
						\node [below] at (3,0) {\tiny $q^\peq{d}\rp{\overline{p}}$};
						\node [below] at (1,0) {\tiny $q^\peq{s}\rp{\overline{p}}$};
						\node [below] at (0,0) {\tiny 0};
						\node [above] at (.3,2.2) {\textbf{A}};
						\node [above] at (1.2,2.2) {\textbf{B}};
						\node [below] at (.3,1.8) {\textbf{C}};
						\node [below] at (1.2,1.8) {\textbf{D}};
						\node [below] at (.3,1) {\textbf{E}};
						\node at (2.5,5) {\underline{Sin política}};
					\end{tikzpicture}
				\end{subfigure}
				\begin{subfigure}[b]{0.49\textwidth}
					\begin{tikzpicture}[scale=1]
						\draw [fill,opacity=.3,purple]  (0,4)--(1,3)--(1,1)--(0,0);
						\draw [ultra thick,teal] plot [domain=0:4] (\x,{4-\x});
						\node [above right,teal] at (4,0) {$D$};
						\draw [ultra thick,teal] plot [domain=0:4] (\x,{\x});
						\draw [thick,blue] (0,2)--(2,2)--(2,0);
						\draw [thick,red] (0,1)--(5,1);
						\draw [thick,red] (1,1)--(1,0);
						\draw [thick,red] (3,1)--(3,0);
						\draw [thick,red] (1,1)--(1,5);
						\draw [<->,ultra thick] (0,5)--(0,0)--(5,0);
						\node [right,teal] at (4,4) {$S$};
						\node [left] at (0,5) {$p$};
						\node [below] at (5,0) {$q$};
						\node [left] at (0,2) {\tiny $p^\peq{*}$};
						\node [left] at (0,1) {\tiny $\overline{p}$};
						\node [below] at (2,0) {\tiny $q^\peq{*}$};
						\node [below] at (3,0) {\tiny $q^\peq{d}\rp{\overline{p}}$};
						\node [below] at (1,0) {\tiny $q^\peq{s}\rp{\overline{p}}$};
						\node [below] at (0,0) {\tiny 0};
						\node [above] at (.3,2.2) {\textbf{A}};
						\node [above] at (1.2,2.2) {\textbf{B}};
						\node [below] at (.3,1.8) {\textbf{C}};
						\node [below] at (1.2,1.8) {\textbf{D}};
						\node [below] at (.3,1) {\textbf{E}};
						\node at (2.5,5) {\underline{Con política}};
					\end{tikzpicture}
				\end{subfigure}
			\end{figure}
		\end{frame}

		\begin{frame}
			\frametitle{Precio Máximo}
			Pérdida social
			
			\vspace{.1cm}
			
			\centering
			\begin{tikzpicture}[scale=1]
				\draw [fill,opacity=.3,black] (1,1)--(2,2)--(1,3);
				\draw [ultra thick,teal] plot [domain=0:4] (\x,{4-\x});
				\node [above right,teal] at (4,0) {$D$};
				\draw [ultra thick,teal] plot [domain=0:4] (\x,{\x});
				\draw [thick,blue] (0,2)--(2,2)--(2,0);
				\draw [thick,red] (0,1)--(5,1);
				\draw [thick,red] (1,1)--(1,0);
				\draw [thick,red] (3,1)--(3,0);
				\draw [thick,red] (1,1)--(1,5);
				\draw [<->,ultra thick] (0,5)--(0,0)--(5,0);
				\node [right,teal] at (4,4) {$S$};
				\node [left] at (0,5) {$p$};
				\node [below] at (5,0) {$q$};
				\node [left] at (0,2) {\tiny $p^\peq{*}$};
				\node [left] at (0,1) {\tiny $\overline{p}$};
				\node [below] at (2,0) {\tiny $q^\peq{*}$};
				\node [below] at (3,0) {\tiny $q^\peq{d}\rp{\overline{p}}$};
				\node [below] at (1,0) {\tiny $q^\peq{s}\rp{\overline{p}}$};
				\node [below] at (0,0) {\tiny 0};
				\node [above] at (.3,2.2) {\textbf{A}};
				\node [above] at (1.2,2.2) {\textbf{B}};
				\node [below] at (.3,1.8) {\textbf{C}};
				\node [below] at (1.2,1.8) {\textbf{D}};
				\node [below] at (.3,1) {\textbf{E}};
			\end{tikzpicture}
		\end{frame}	

		\begin{frame}
			\frametitle{Precio Máximo}
			\begin{table}[htbp!]
				\centering
				\resizebox{11cm}{!}{
					\begin{tabular}{l c c c}\hline
												&	Sin política	&	Con política	&	Cambio	\\  \hline 
									 $EC$ &			$A+B$			&		$A+C$				&	$-B+C$	\\
									 $EP$ &			$C+D+E$		&			$E$				&	$-(C+D)$\\ \hline
									 $ET$ &	$A+B+C+D+E$		&		$A+C+E$			&	$-(B+D)$\\ \hline
					\end{tabular}}
			\end{table}
		\end{frame}

		\begin{frame}
			\frametitle{Precio Máximo}
			Ejemplo 1: Filas de espera en gasolineras
				\begin{itemize}
					\item OPEP aumentó precio del petróleo crudo en 1973.
					\item Se trata del principal insumo en la producción de gasolina. \\
				 $\implies$ Se redujo la oferta de gasolina.
				\end{itemize}
		\end{frame}

		\begin{frame}
			\frametitle{Precio Máximo}
				\begin{itemize}
					\item Se produjeron largas filas de espera en gasolineras..
					\item Opinión pública culpó a la OPEP.
				\end{itemize}
		\end{frame}

		\begin{frame}
			\frametitle{Precio Máximo}
			Pero, ¿qué rol tuvo la política de control de precios de EEUU?
		\end{frame}
		
		\begin{frame}
			\frametitle{Precio Máximo}
			\centering
			\begin{tikzpicture}[scale=0.5]
				\draw  [teal,ultra thick](0,8) -- (8,0);
				%\draw  [purple,ultra thick] (0,2) -- (12,10);
				\draw  [teal,ultra thick] (6,0) -- (12,4);
				%\draw  [blue,thick] (3.6,0) -- (3.6,4.4) -- (0,4.4);
				\draw  [blue,thick] (0,0.8) -- (7.2,0.8) -- (7.2,0);
				\draw  [red,thick] (0,3) -- (12,3);
				%\draw  [red,thick] (5,3) -- (12,3);
				%\draw  [red,thick] (1.5,3) -- (1.5,0);
				\draw [<->,ultra thick] (0,12) -- (0,0) -- (12,0);
				\node [above right,teal] at (8,0) {\scriptsize $D$};
				%\node [above right,purple] at (12,10) {\scriptsize $S_{\scriptscriptstyle{1}}$};
				\node [above right,teal] at (12,4) {\scriptsize $S_{\scriptscriptstyle{0}}$};
				\node [below] at (12,0) {$q$};
				%\node [below] at (3.6,0) {\tiny $q^\peq{*}_{\scriptscriptstyle{1}}$};
				\node [below] at (7.2,0) {\tiny $q^\peq{*}_{\scriptscriptstyle{0}}$};
				\node [left] at (0,12) {$p$};
				%\node [left] at (0,4.4) {\tiny $p^\peq{*}_{\scriptscriptstyle{1}}$};
				\node [left] at (0,0.8) {\tiny $p^\peq{*}_{\scriptscriptstyle{0}}$};
				\node [left] at (0,3) {\tiny $\overline{p}$};
				%\node [below] at (5,0) {\tiny $q^\peq{d}_{\scriptscriptstyle{1}}\rp{\overline{p}}$};
				%\node [below] at (1.5,0) {\tiny $q^\peq{s}_{\scriptscriptstyle{1}}\rp{\overline{p}}$};
				%\draw [thick, red,decorate,decoration={brace,amplitude=10pt,mirror},xshift=0.4pt,yshift=-0.4pt](1.5,-1.2) -- (5,-1.2) node[black,midway,yshift=-0.6cm] {\footnotesize \textcolor{red}{\scriptsize exceso de demanda}};
			\end{tikzpicture}
		\end{frame}

		\begin{frame}
			\frametitle{Precio Máximo}
			\centering
			\begin{tikzpicture}[scale=0.5]
				\draw  [teal,ultra thick](0,8) -- (8,0);
				%\draw  [purple,ultra thick] (0,2) -- (12,10);
				\draw  [teal,ultra thick] (6,0) -- (12,4);
				%\draw  [blue,thick] (3.6,0) -- (3.6,4.4) -- (0,4.4);
				\draw  [blue,thick] (0,0.8) -- (7.2,0.8) -- (7.2,0);
				%\draw  [red,thick] (0,3) -- (12,3);
				%\draw  [red,thick] (5,3) -- (12,3);
				%\draw  [red,thick] (1.5,3) -- (1.5,0);
				\draw [<->,ultra thick] (0,12) -- (0,0) -- (12,0);
				\node [above right,teal] at (8,0) {\scriptsize $D$};
				%\node [above right,purple] at (12,10) {\scriptsize $S_{\scriptscriptstyle{1}}$};
				\node [above right,teal] at (12,4) {\scriptsize $S_{\scriptscriptstyle{0}}$};
				\node [below] at (12,0) {$q$};
				%\node [below] at (3.6,0) {\tiny $q^\peq{*}_{\scriptscriptstyle{1}}$};
				\node [below] at (7.2,0) {\tiny $q^\peq{*}_{\scriptscriptstyle{0}}$};
				\node [left] at (0,12) {$p$};
				%\node [left] at (0,4.4) {\tiny $p^\peq{*}_{\scriptscriptstyle{1}}$};
				\node [left] at (0,0.8) {\tiny $p^\peq{*}_{\scriptscriptstyle{0}}$};
				%\node [left] at (0,3) {\tiny $\overline{p}$};
				%\node [below] at (5,0) {\tiny $q^\peq{d}_{\scriptscriptstyle{1}}\rp{\overline{p}}$};
				%\node [below] at (1.5,0) {\tiny $q^\peq{s}_{\scriptscriptstyle{1}}\rp{\overline{p}}$};
				%\draw [thick, red,decorate,decoration={brace,amplitude=10pt,mirror},xshift=0.4pt,yshift=-0.4pt](1.5,-1.2) -- (5,-1.2) node[black,midway,yshift=-0.6cm] {\footnotesize \textcolor{red}{\scriptsize exceso de demanda}};
			\end{tikzpicture}
		\end{frame}

		\begin{frame}
			\frametitle{Precio Máximo}
			\centering
			\begin{tikzpicture}[scale=0.5]
				\draw  [teal,ultra thick](0,8) -- (8,0);
				\draw  [purple,ultra thick] (0,2) -- (12,10);
				\draw  [teal,ultra thick] (6,0) -- (12,4);
				%\draw  [blue,thick] (3.6,0) -- (3.6,4.4) -- (0,4.4);
				\draw  [blue,thick] (0,0.8) -- (7.2,0.8) -- (7.2,0);
				%\draw  [red,thick] (0,3) -- (12,3);
				%\draw  [red,thick] (5,3) -- (12,3);
				%\draw  [red,thick] (1.5,3) -- (1.5,0);
				\draw [<->,ultra thick] (0,12) -- (0,0) -- (12,0);
				\node [above right,teal] at (8,0) {\scriptsize $D$};
				\node [above right,purple] at (12,10) {\scriptsize $S_{\scriptscriptstyle{1}}$};
				\node [above right,teal] at (12,4) {\scriptsize $S_{\scriptscriptstyle{0}}$};
				\node [below] at (12,0) {$q$};
				%\node [below] at (3.6,0) {\tiny $q^\peq{*}_{\scriptscriptstyle{1}}$};
				\node [below] at (7.2,0) {\tiny $q^\peq{*}_{\scriptscriptstyle{0}}$};
				\node [left] at (0,12) {$p$};
				%\node [left] at (0,4.4) {\tiny $p^\peq{*}_{\scriptscriptstyle{1}}$};
				\node [left] at (0,0.8) {\tiny $p^\peq{*}_{\scriptscriptstyle{0}}$};
				%\node [left] at (0,3) {\tiny $\overline{p}$};
				%\node [below] at (5,0) {\tiny $q^\peq{d}_{\scriptscriptstyle{1}}\rp{\overline{p}}$};
				%\node [below] at (1.5,0) {\tiny $q^\peq{s}_{\scriptscriptstyle{1}}\rp{\overline{p}}$};
				%\draw [thick, red,decorate,decoration={brace,amplitude=10pt,mirror},xshift=0.4pt,yshift=-0.4pt](1.5,-1.2) -- (5,-1.2) node[black,midway,yshift=-0.6cm] {\footnotesize \textcolor{red}{\scriptsize exceso de demanda}};
			\end{tikzpicture}
		\end{frame}

		\begin{frame}
			\frametitle{Precio Máximo}
			\centering
			\begin{tikzpicture}[scale=0.5]
				\draw  [teal,ultra thick](0,8) -- (8,0);
				\draw  [purple,ultra thick] (0,2) -- (12,10);
				\draw  [teal,ultra thick] (6,0) -- (12,4);
				\draw  [blue,thick] (3.6,0) -- (3.6,4.4) -- (0,4.4);
				\draw  [blue,thick] (0,0.8) -- (7.2,0.8) -- (7.2,0);
				%\draw  [red,thick] (0,3) -- (12,3);
				%\draw  [red,thick] (5,3) -- (12,3);
				%\draw  [red,thick] (1.5,3) -- (1.5,0);
				\draw [<->,ultra thick] (0,12) -- (0,0) -- (12,0);
				\node [above right,teal] at (8,0) {\scriptsize $D$};
				\node [above right,purple] at (12,10) {\scriptsize $S_{\scriptscriptstyle{1}}$};
				\node [above right,teal] at (12,4) {\scriptsize $S_{\scriptscriptstyle{0}}$};
				\node [below] at (12,0) {$q$};
				\node [below] at (3.6,0) {\tiny $q^\peq{*}_{\scriptscriptstyle{1}}$};
				\node [below] at (7.2,0) {\tiny $q^\peq{*}_{\scriptscriptstyle{0}}$};
				\node [left] at (0,12) {$p$};
				\node [left] at (0,4.4) {\tiny $p^\peq{*}_{\scriptscriptstyle{1}}$};
				\node [left] at (0,0.8) {\tiny $p^\peq{*}_{\scriptscriptstyle{0}}$};
				%\node [left] at (0,3) {\tiny $\overline{p}$};
				%\node [below] at (5,0) {\tiny $q^\peq{d}_{\scriptscriptstyle{1}}\rp{\overline{p}}$};
				%\node [below] at (1.5,0) {\tiny $q^\peq{s}_{\scriptscriptstyle{1}}\rp{\overline{p}}$};
				%\draw [thick, red,decorate,decoration={brace,amplitude=10pt,mirror},xshift=0.4pt,yshift=-0.4pt](1.5,-1.2) -- (5,-1.2) node[black,midway,yshift=-0.6cm] {\footnotesize \textcolor{red}{\scriptsize exceso de demanda}};
			\end{tikzpicture}
		\end{frame}

		\begin{frame}
			\frametitle{Precio Máximo}
			\centering
			\begin{tikzpicture}[scale=0.5]
				\draw  [teal,ultra thick](0,8) -- (8,0);
				\draw  [purple,ultra thick] (0,2) -- (12,10);
				\draw  [teal,ultra thick] (6,0) -- (12,4);
				\draw  [blue,thick] (3.6,0) -- (3.6,4.4) -- (0,4.4);
				\draw  [blue,thick] (0,0.8) -- (7.2,0.8) -- (7.2,0);
				\draw  [red,thick] (0,3) -- (5,3) -- (5,0);
				\draw  [red,thick] (5,3) -- (12,3);
				\draw  [red,thick] (1.5,3) -- (1.5,0);
				\draw [<->,ultra thick] (0,12) -- (0,0) -- (12,0);
				\node [above right,teal] at (8,0) {\scriptsize $D$};
				\node [above right,purple] at (12,10) {\scriptsize $S_{\scriptscriptstyle{1}}$};
				\node [above right,teal] at (12,4) {\scriptsize $S_{\scriptscriptstyle{0}}$};
				\node [below] at (12,0) {$q$};
				\node [below] at (3.6,0) {\tiny $q^\peq{*}_{\scriptscriptstyle{1}}$};
				\node [below] at (7.2,0) {\tiny $q^\peq{*}_{\scriptscriptstyle{0}}$};
				\node [left] at (0,12) {$p$};
				\node [left] at (0,4.4) {\tiny $p^\peq{*}_{\scriptscriptstyle{1}}$};
				\node [left] at (0,0.8) {\tiny $p^\peq{*}_{\scriptscriptstyle{0}}$};
				\node [left] at (0,3) {\tiny $\overline{p}$};
				\node [below] at (5,0) {\tiny $q^\peq{d}_{\scriptscriptstyle{1}}\rp{\overline{p}}$};
				\node [below] at (1.5,0) {\tiny $q^\peq{s}_{\scriptscriptstyle{1}}\rp{\overline{p}}$};
			\end{tikzpicture}
		\end{frame}
		
		\begin{frame}
			\frametitle{Precio Máximo}
			Ejemplo 2: Control del alquiler a corto y largo plazo
				\begin{itemize}
					\item En muchas ciudades de EEUU el gobierno local impone un máximo al precio que se puede cobrar por el arriendo de departamentos.
				\end{itemize}
		\end{frame}		
		
				\begin{frame}
			\frametitle{Precio Máximo}
				\begin{itemize}
					\item El efecto en el corto plazo puede ser una reducción significativa del precio con un exceso de demanda pequeño.
					\item Pero en el largo plazo el exceso de demanda puede ser enorme.
				\end{itemize}
		\end{frame}	

		\begin{frame}
			\frametitle{Precio Máximo}
			\begin{figure}[hbtp!]
				\centering
				\begin{subfigure}[b]{0.49\textwidth}
					\begin{tikzpicture}[scale=1]
						\draw [ultra thick,teal] plot [domain=.5:3.5] (\x,{4-\x});
						\draw [ultra thick,teal] (2,0)--(2,4);
						\node [above right,teal] at (3.5,.5) {$D$};
						\draw [thick,red] (0,1)--(5,1);
						\draw [thick,red] (3,1)--(3,0);
						\draw [thick,blue] (0,2)--(2,2);
						\draw [<->,ultra thick] (0,5)--(0,0)--(5,0);
						\node [right,teal] at (2,4) {$S$};
						\node [left] at (0,5) {$p$};
						\node [below] at (5,0) {$q$};
						\node [left] at (0,2) {\tiny $p^\peq{*}$};
						\node [left] at (0,1) {\tiny $\overline{p}$};
						\node [below] at (2,0) {\tiny $q^\peq{*}=q^\peq{s}\rp{\overline{p}}$};
						\node [below] at (3,0) {\tiny $q^\peq{d}\rp{\overline{p}}$};
						\node [below] at (0,0) {\tiny 0};
						\node at (2.5,5) {\underline{Corto plazo}};
						\draw [thick, red,decorate,decoration={brace,amplitude=5pt,mirror},xshift=0.4pt,yshift=-0.4pt](2,-.5) -- (3,-.5) node[black,midway,yshift=-0.6cm] {\footnotesize \textcolor{red}{\scriptsize exceso de demanda}};
					\end{tikzpicture}
				\end{subfigure}
				\begin{subfigure}[b]{0.49\textwidth}
					\begin{tikzpicture}[scale=1]
						\draw [ultra thick,teal] plot [domain=.5:4.5] (\x,{3-.5*\x});
						\node [above right,teal] at (4.5,.5) {$D$};
						\draw [ultra thick,teal] plot [domain=.5:3.5] (\x,{\x});
						\draw [thick,red] (0,1)--(5,1);
						\draw [thick,red] (4,1)--(4,0);
						\draw [thick,red] (1,1)--(1,0);
						\draw [thick,blue] (0,2)--(2,2)--(2,0);
						\draw [<->,ultra thick] (0,5)--(0,0)--(5,0);
						\node [right,teal] at (3.5,3.5) {$S$};
						\node [left] at (0,5) {$p$};
						\node [below] at (5,0) {$q$};
						\node [left] at (0,2) {\tiny $p^\peq{*}$};
						\node [below] at (2,0) {\tiny $q^\peq{*}$};
						\node [below] at (0,0) {\tiny 0};
						\node [below] at (4,0) {\tiny $q^\peq{d}\rp{\overline{p}}$};
						\node [below] at (1,0) {\tiny $q^\peq{s}\rp{\overline{p}}$};
						\node [left] at (0,1) {\tiny $\overline{p}$};
						\node at (2.5,5) {\underline{Largo plazo}};
						\draw [thick, red,decorate,decoration={brace,amplitude=10pt,mirror},xshift=0.4pt,yshift=-0.4pt](1,-.5) -- (4,-.5) node[black,midway,yshift=-0.6cm] {\footnotesize \textcolor{red}{\scriptsize exceso de demanda}};
					\end{tikzpicture}
				\end{subfigure}
			\end{figure}
		\end{frame}
		
		\begin{frame}
			\frametitle{Precio Mínimo}
			Caso 1: $\underline{p}<p^\peq{*}$
			
			\vspace{.1cm}
			
			\centering
			\begin{tikzpicture}[scale=1]
				\draw [ultra thick,teal] plot [domain=0:4] (\x,{4-\x});
				\node [above right,teal] at (4,0) {$D$};
				\draw [ultra thick,teal] plot [domain=0:4] (\x,{\x});
				\draw [thick,blue] (0,2)--(2,2)--(2,0);
				\draw [thick,red] (0,1)--(5,1);
				\draw [<->,ultra thick] (0,5)--(0,0)--(5,0);
				\node [right,teal] at (4,4) {$S$};
				\node [left] at (0,5) {$p$};
				\node [below] at (5,0) {$q$};
				\node [left] at (0,2) {\tiny $p^\peq{*}$};
				\node [left] at (0,1) {\tiny $\underline{p}$};
				\node [below] at (2,0) {\tiny $q^\peq{*}$};
				%\node [below] at (3,0) {\tiny $q^\peq{s}\rp{p_\peq{1}}$};
				%\node [below] at (1,0) {\tiny $q^\peq{d}\rp{p_\peq{1}}$};
				\node [below] at (0,0) {\tiny 0};
				\node [below,align=center] at (3.5,2) {\scriptsize \textbf{No tiene} \\ \scriptsize \textbf{efecto}};
				%\draw [decorate,decoration={brace,amplitude=6pt,mirror},xshift=0.2pt,yshift=-0.2pt,red](1,-.4) -- (3,-.4) node[red,midway,yshift=-0.5cm] {\tiny \pbox{\textwidth}{Exceso \\ de oferta}};
			\end{tikzpicture}
		\end{frame}	
		
		\begin{frame}
			\frametitle{Precio Mínimo}
			Caso 2: $\underline{p}>p^\peq{*}$
			
			\vspace{.1cm}
			
			\centering
			\begin{tikzpicture}[scale=1]
				\draw [ultra thick,teal] plot [domain=0:4] (\x,{4-\x});
				\node [above right,teal] at (4,0) {$D$};
				\draw [ultra thick,teal] plot [domain=0:4] (\x,{\x});
				\draw [thick,blue] (0,2)--(2,2)--(2,0);
				\draw [thick,red] (0,3)--(5,3);
				\draw [thick,red] (1,3)--(1,0);
				\draw [thick,red] (3,3)--(3,0);
				\draw [<->,ultra thick] (0,5)--(0,0)--(5,0);
				\node [right,teal] at (4,4) {$S$};
				\node [left] at (0,5) {$p$};
				\node [below] at (5,0) {$q$};
				\node [left] at (0,2) {\tiny $p^\peq{*}$};
				\node [left] at (0,3) {\tiny $\underline{p}$};
				\node [below] at (2,0) {\tiny $q^\peq{*}$};
				\node [below] at (3,0) {\tiny $q^\peq{s}\rp{\underline{p}}$};
				\node [below] at (1,0) {\tiny $q^\peq{d}\rp{\underline{p}}$};
				\node [below] at (0,0) {\tiny 0};
				\draw [decorate,decoration={brace,amplitude=6pt,mirror},xshift=0.2pt,yshift=-0.2pt,red](1,-.4) -- (3,-.4) node[red,midway,yshift=-0.5cm] {\tiny \pbox{\textwidth}{Exceso\\de oferta}};
			\end{tikzpicture}
		\end{frame}	

		\begin{frame}
			\frametitle{Precio Mínimo}
			Caso 2: $\underline{p}>p^\peq{*}$
			
			\vspace{.1cm}
			
			\centering
			\begin{tikzpicture}[scale=1]
				\draw [ultra thick,teal] plot [domain=0:4] (\x,{4-\x});
				\node [above right,teal] at (4,0) {$D$};
				\draw [ultra thick,teal] plot [domain=0:4] (\x,{\x});
				\draw [thick,blue] (0,2)--(2,2)--(2,0);
				\draw [thick,red] (0,3)--(5,3);
				\draw [thick,red] (1,3)--(1,0);
				\draw [thick,red] (3,3)--(3,0);
				\draw [thick,red] (1,3)--(1,5);
				\draw [<->,ultra thick] (0,5)--(0,0)--(5,0);
				\node [right,teal] at (4,4) {$S$};
				\node [left] at (0,5) {$p$};
				\node [below] at (5,0) {$q$};
				\node [left] at (0,2) {\tiny $p^\peq{*}$};
				\node [left] at (0,3) {\tiny $\underline{p}$};
				\node [below] at (2,0) {\tiny $q^\peq{*}$};
				\node [below] at (3,0) {\tiny $q^\peq{s}\rp{\underline{p}}$};
				\node [below] at (1,0) {\tiny $q^\peq{d}\rp{\underline{p}}$};
				\node [below] at (0,0) {\tiny 0};
				\node [above] at (.3,3) {\textbf{A}};
				\node [above] at (.3,2.2) {\textbf{B}};
				\node [above] at (1.2,2.2) {\textbf{C}};
				\node [below] at (.3,1.8) {\textbf{D}};
				\node [below] at (1.2,1.8) {\textbf{E}};
				\draw [decorate,decoration={brace,amplitude=6pt,mirror},xshift=0.2pt,yshift=-0.2pt,red](1,-.4) -- (3,-.4) node[red,midway,yshift=-0.5cm] {\tiny \pbox{\textwidth}{Exceso\\de oferta}};
			\end{tikzpicture}
		\end{frame}	

		\begin{frame}
			\frametitle{Precio Mínimo}
			Excedente del consumidor
			\begin{figure}[hbtp!]
				\centering
				\begin{subfigure}[b]{0.49\textwidth}
					\begin{tikzpicture}[scale=1]
						\draw [fill,opacity=.3,blue] (0,4)--(2,2)--(0,2);
						\draw [ultra thick,teal] plot [domain=0:4] (\x,{4-\x});
						\node [above right,teal] at (4,0) {$D$};
						\draw [ultra thick,teal] plot [domain=0:4] (\x,{\x});
						\draw [thick,blue] (0,2)--(2,2)--(2,0);
						\draw [thick,red] (0,3)--(5,3);
						\draw [thick,red] (1,3)--(1,0);
						\draw [thick,red] (3,3)--(3,0);
						\draw [thick,red] (1,3)--(1,5);
						\draw [<->,ultra thick] (0,5)--(0,0)--(5,0);
						\node [right,teal] at (4,4) {$S$};
						\node [left] at (0,5) {$p$};
						\node [below] at (5,0) {$q$};
						\node [left] at (0,2) {\tiny $p^\peq{*}$};
						\node [left] at (0,3) {\tiny $\underline{p}$};
						\node [below] at (2,0) {\tiny $q^\peq{*}$};
						\node [below] at (3,0) {\tiny $q^\peq{s}\rp{\underline{p}}$};
						\node [below] at (1,0) {\tiny $q^\peq{d}\rp{\underline{p}}$};
						\node [below] at (0,0) {\tiny 0};
						\node [above] at (.3,3) {\textbf{A}};
						\node [above] at (.3,2.2) {\textbf{B}};
						\node [above] at (1.2,2.2) {\textbf{C}};
						\node [below] at (.3,1.8) {\textbf{D}};
						\node [below] at (1.2,1.8) {\textbf{E}};
						\node at (2.5,5) {\underline{Sin política}};
					\end{tikzpicture}
				\end{subfigure}
				\begin{subfigure}[b]{0.49\textwidth}
					\begin{tikzpicture}[scale=1]
						\draw [fill,opacity=.3,blue] (0,4)--(1,3)--(0,3);
						\draw [ultra thick,teal] plot [domain=0:4] (\x,{4-\x});
						\node [above right,teal] at (4,0) {$D$};
						\draw [ultra thick,teal] plot [domain=0:4] (\x,{\x});
						\draw [thick,blue] (0,2)--(2,2)--(2,0);
						\draw [thick,red] (0,3)--(5,3);
						\draw [thick,red] (1,3)--(1,0);
						\draw [thick,red] (3,3)--(3,0);
						\draw [thick,red] (1,3)--(1,5);
						\draw [<->,ultra thick] (0,5)--(0,0)--(5,0);
						\node [right,teal] at (4,4) {$S$};
						\node [left] at (0,5) {$p$};
						\node [below] at (5,0) {$q$};
						\node [left] at (0,2) {\tiny $p^\peq{*}$};
						\node [left] at (0,3) {\tiny $\underline{p}$};
						\node [below] at (2,0) {\tiny $q^\peq{*}$};
						\node [below] at (3,0) {\tiny $q^\peq{s}\rp{\underline{p}}$};
						\node [below] at (1,0) {\tiny $q^\peq{d}\rp{\underline{p}}$};
						\node [below] at (0,0) {\tiny 0};
						\node [above] at (.3,3) {\textbf{A}};
						\node [above] at (.3,2.2) {\textbf{B}};
						\node [above] at (1.2,2.2) {\textbf{C}};
						\node [below] at (.3,1.8) {\textbf{D}};
						\node [below] at (1.2,1.8) {\textbf{E}};	
						\node at (2.5,5) {\underline{Con política}};
					\end{tikzpicture}
				\end{subfigure}
			\end{figure}
		\end{frame}

		\begin{frame}
			\frametitle{Precio Mínimo}
			Excedente del productor
			\begin{figure}[hbtp!]
				\centering
				\begin{subfigure}[b]{0.49\textwidth}
					\begin{tikzpicture}[scale=1]
						\draw [fill,opacity=.3,green] (0,0)--(2,2)--(0,2);
						\draw [ultra thick,teal] plot [domain=0:4] (\x,{4-\x});
						\node [above right,teal] at (4,0) {$D$};
						\draw [ultra thick,teal] plot [domain=0:4] (\x,{\x});
						\draw [thick,blue] (0,2)--(2,2)--(2,0);
						\draw [thick,red] (0,3)--(5,3);
						\draw [thick,red] (1,3)--(1,0);
						\draw [thick,red] (3,3)--(3,0);
						\draw [thick,red] (1,3)--(1,5);
						\draw [<->,ultra thick] (0,5)--(0,0)--(5,0);
						\node [right,teal] at (4,4) {$S$};
						\node [left] at (0,5) {$p$};
						\node [below] at (5,0) {$q$};
						\node [left] at (0,2) {\tiny $p^\peq{*}$};
						\node [left] at (0,3) {\tiny $\underline{p}$};
						\node [below] at (2,0) {\tiny $q^\peq{*}$};
						\node [below] at (3,0) {\tiny $q^\peq{s}\rp{\underline{p}}$};
						\node [below] at (1,0) {\tiny $q^\peq{d}\rp{\underline{p}}$};
						\node [below] at (0,0) {\tiny 0};
						\node [above] at (.3,3) {\textbf{A}};
						\node [above] at (.3,2.2) {\textbf{B}};
						\node [above] at (1.2,2.2) {\textbf{C}};
						\node [below] at (.3,1.8) {\textbf{D}};
						\node [below] at (1.2,1.8) {\textbf{E}};
						\node at (2.5,5) {\underline{Sin política}};
					\end{tikzpicture}
				\end{subfigure}
				\begin{subfigure}[b]{0.49\textwidth}
					\begin{tikzpicture}[scale=1]
						\draw [fill,opacity=.3,green] (0,0)--(1,1)--(1,3)--(0,3);
						\draw [ultra thick,teal] plot [domain=0:4] (\x,{4-\x});
						\node [above right,teal] at (4,0) {$D$};
						\draw [ultra thick,teal] plot [domain=0:4] (\x,{\x});
						\draw [thick,blue] (0,2)--(2,2)--(2,0);
						\draw [thick,red] (0,3)--(5,3);
						\draw [thick,red] (1,3)--(1,0);
						\draw [thick,red] (3,3)--(3,0);
						\draw [thick,red] (1,3)--(1,5);
						\draw [<->,ultra thick] (0,5)--(0,0)--(5,0);
						\node [right,teal] at (4,4) {$S$};
						\node [left] at (0,5) {$p$};
						\node [below] at (5,0) {$q$};
						\node [left] at (0,2) {\tiny $p^\peq{*}$};
						\node [left] at (0,3) {\tiny $\underline{p}$};
						\node [below] at (2,0) {\tiny $q^\peq{*}$};
						\node [below] at (3,0) {\tiny $q^\peq{s}\rp{\underline{p}}$};
						\node [below] at (1,0) {\tiny $q^\peq{d}\rp{\underline{p}}$};
						\node [below] at (0,0) {\tiny 0};
						\node [above] at (.3,3) {\textbf{A}};
						\node [above] at (.3,2.2) {\textbf{B}};
						\node [above] at (1.2,2.2) {\textbf{C}};
						\node [below] at (.3,1.8) {\textbf{D}};
						\node [below] at (1.2,1.8) {\textbf{E}};
						\node at (2.5,5) {\underline{Con política}};
					\end{tikzpicture}
				\end{subfigure}
			\end{figure}
		\end{frame}

		\begin{frame}
			\frametitle{Precio Mínimo}
			Excedente total
			\begin{figure}[hbtp!]
				\centering
				\begin{subfigure}[b]{0.49\textwidth}
					\begin{tikzpicture}[scale=1]
						\draw [fill,opacity=.3,purple] (0,0)--(2,2)--(0,4);
						\draw [ultra thick,teal] plot [domain=0:4] (\x,{4-\x});
						\node [above right,teal] at (4,0) {$D$};
						\draw [ultra thick,teal] plot [domain=0:4] (\x,{\x});
						\draw [thick,blue] (0,2)--(2,2)--(2,0);
						\draw [thick,red] (0,3)--(5,3);
						\draw [thick,red] (1,3)--(1,0);
						\draw [thick,red] (3,3)--(3,0);
						\draw [thick,red] (1,3)--(1,5);
						\draw [<->,ultra thick] (0,5)--(0,0)--(5,0);
						\node [right,teal] at (4,4) {$S$};
						\node [left] at (0,5) {$p$};
						\node [below] at (5,0) {$q$};
						\node [left] at (0,2) {\tiny $p^\peq{*}$};
						\node [left] at (0,3) {\tiny $\underline{p}$};
						\node [below] at (2,0) {\tiny $q^\peq{*}$};
						\node [below] at (3,0) {\tiny $q^\peq{s}\rp{\underline{p}}$};
						\node [below] at (1,0) {\tiny $q^\peq{d}\rp{\underline{p}}$};
						\node [below] at (0,0) {\tiny 0};
						\node [above] at (.3,3) {\textbf{A}};
						\node [above] at (.3,2.2) {\textbf{B}};
						\node [above] at (1.2,2.2) {\textbf{C}};
						\node [below] at (.3,1.8) {\textbf{D}};
						\node [below] at (1.2,1.8) {\textbf{E}};
						\node at (2.5,5) {\underline{Sin política}};
					\end{tikzpicture}
				\end{subfigure}
				\begin{subfigure}[b]{0.49\textwidth}
					\begin{tikzpicture}[scale=1]
						\draw [fill,opacity=.3,purple]  (0,4)--(1,3)--(1,1)--(0,0);
						\draw [ultra thick,teal] plot [domain=0:4] (\x,{4-\x});
						\node [above right,teal] at (4,0) {$D$};
						\draw [ultra thick,teal] plot [domain=0:4] (\x,{\x});
						\draw [thick,blue] (0,2)--(2,2)--(2,0);
						\draw [thick,red] (0,3)--(5,3);
						\draw [thick,red] (1,3)--(1,0);
						\draw [thick,red] (3,3)--(3,0);
						\draw [thick,red] (1,3)--(1,5);
						\draw [<->,ultra thick] (0,5)--(0,0)--(5,0);
						\node [right,teal] at (4,4) {$S$};
						\node [left] at (0,5) {$p$};
						\node [below] at (5,0) {$q$};
						\node [left] at (0,2) {\tiny $p^\peq{*}$};
						\node [left] at (0,3) {\tiny $\underline{p}$};
						\node [below] at (2,0) {\tiny $q^\peq{*}$};
						\node [below] at (3,0) {\tiny $q^\peq{s}\rp{\underline{p}}$};
						\node [below] at (1,0) {\tiny $q^\peq{d}\rp{\underline{p}}$};
						\node [below] at (0,0) {\tiny 0};
						\node [above] at (.3,3) {\textbf{A}};
						\node [above] at (.3,2.2) {\textbf{B}};
						\node [above] at (1.2,2.2) {\textbf{C}};
						\node [below] at (.3,1.8) {\textbf{D}};
						\node [below] at (1.2,1.8) {\textbf{E}};
						\node at (2.5,5) {\underline{Con política}};
					\end{tikzpicture}
				\end{subfigure}
			\end{figure}
		\end{frame}

		\begin{frame}
			\frametitle{Precio Mínimo}
			Pérdida social
			
			\vspace{.1cm}
			
			\centering
			\begin{tikzpicture}[scale=1]
				\draw [fill,opacity=.3,black] (1,1)--(2,2)--(1,3);
				\draw [ultra thick,teal] plot [domain=0:4] (\x,{4-\x});
				\node [above right,teal] at (4,0) {$D$};
				\draw [ultra thick,teal] plot [domain=0:4] (\x,{\x});
				\draw [thick,blue] (0,2)--(2,2)--(2,0);
				\draw [thick,red] (0,3)--(5,3);
				\draw [thick,red] (1,3)--(1,0);
				\draw [thick,red] (3,3)--(3,0);
				\draw [thick,red] (1,3)--(1,5);
				\draw [<->,ultra thick] (0,5)--(0,0)--(5,0);
				\node [right,teal] at (4,4) {$S$};
				\node [left] at (0,5) {$p$};
				\node [below] at (5,0) {$q$};
				\node [left] at (0,2) {\tiny $p^\peq{*}$};
				\node [left] at (0,3) {\tiny $\underline{p}$};
				\node [below] at (2,0) {\tiny $q^\peq{*}$};
				\node [below] at (3,0) {\tiny $q^\peq{s}\rp{\underline{p}}$};
				\node [below] at (1,0) {\tiny $q^\peq{d}\rp{\underline{p}}$};
				\node [below] at (0,0) {\tiny 0};
				\node [above] at (.3,3) {\textbf{A}};
				\node [above] at (.3,2.2) {\textbf{B}};
				\node [above] at (1.2,2.2) {\textbf{C}};
				\node [below] at (.3,1.8) {\textbf{D}};
				\node [below] at (1.2,1.8) {\textbf{E}};
			\end{tikzpicture}
		\end{frame}	

		\begin{frame}
			\frametitle{Precio Mínimo}
			\begin{table}[htbp!]
				\centering
				\resizebox{11cm}{!}{
					\begin{tabular}{l c c c}\hline
												&	Sin política	&	Con política	&	Cambio	\\  \hline 
									 $EC$ &		$A+B+C$			&			$A$				&	$-(B+C)$\\
									 $EP$ &			$D+E$			&			$D+B$			&	$-E+B$	\\ \hline
									 $ET$ &	$A+B+C+D+E$		&		$A+B+D$			&	$-(C+E)$\\ \hline
					\end{tabular}}
			\end{table}
		\end{frame}

		\begin{frame}
			\frametitle{Precio Mínimo}
			Ejemplo: Salario mínimo
			
			\vspace{.1cm}
			
				\centering
				\begin{tikzpicture}[scale=1]
					\draw [ultra thick,teal] plot [domain=0:4] (\x,{4-\x});
					\node [above right,teal] at (4,0) {$D$};
					\draw [ultra thick,teal] plot [domain=0:4] (\x,{\x});
					\draw [thick,blue] (0,2)--(2,2)--(2,0);
					\draw [thick,red] (0,3)--(5,3);
					\draw [thick,red] (1,3)--(1,0);
					\draw [thick,red] (3,3)--(3,0);
					\draw [<->,ultra thick] (0,5)--(0,0)--(5,0);
					\node [right,teal] at (4,4) {$S$};
					\node [left] at (0,5) {$w$};
					\node [below] at (5,0) {$L$};
					\node [left] at (0,2) {\tiny $w^\peq{*}$};
					\node [left] at (0,3) {\tiny $\underline{w}$};
					\node [below] at (2,0) {\tiny $L^\peq{*}$};
					\node [below] at (3,0) {\tiny $L^\peq{s}\rp{\underline{w}}$};
					\node [below] at (1,0) {\tiny $L^\peq{d}\rp{\underline{w}}$};
					\node [below] at (0,0) {\tiny 0};
					\draw [decorate,decoration={brace,amplitude=6pt,mirror},xshift=0.2pt,yshift=-0.2pt,red](1,-.4) -- (3,-.4) node[red,midway,yshift=-0.5cm] {\tiny \pbox{\textwidth}{Desempleo}};
				\end{tikzpicture}
		\end{frame}
		
		\begin{frame}
			\frametitle{Precio Mínimo}
				\centering
				\begin{tikzpicture}[scale=1]
					\draw [ultra thick,teal] plot [domain=0:4] (\x,{4-\x});
					\node [above right,teal] at (4,0) {$D$};
					\draw [ultra thick,teal] plot [domain=0:4] (\x,{\x});
					\draw [thick,brown] (0,3.5)--(5,3.5);
					\draw [thick,brown] (.5,3.5)--(.5,0);
					\draw [thick,brown] (3.5,3.5)--(3.5,0);
					\draw [thick,red] (0,3)--(5,3);
					\draw [thick,red] (1,3)--(1,0);
					\draw [thick,red] (3,3)--(3,0);
					\draw [<->,ultra thick] (0,5)--(0,0)--(5,0);
					\node [right,teal] at (4,4) {$S$};
					\node [left] at (0,5) {$w$};
					\node [below] at (5,0) {$L$};
					\node [left] at (0,3) {\tiny $\underline{w}_\peq{0}$};
					\node [left] at (0,3.5) {\tiny $\underline{w}_\peq{1}$};
					\node [below] at (3,0) {\tiny $L^\peq{s}_\peq{0}$};
					\node [below] at (.5,0) {\tiny $L^\peq{d}_\peq{0}$};
					\node [below] at (3.5,0) {\tiny $L^\peq{s}_\peq{1}$};
					\node [below] at (1,0) {\tiny $L^\peq{d}_\peq{1}$};
					\node [below] at (0,0) {\tiny 0};
					\draw [decorate,decoration={brace,amplitude=4pt,mirror},xshift=0.2pt,yshift=-0.2pt,red](1,-.4) -- (3,-.4) node[red,midway,yshift=-8] {\tiny $U_\peq{0}$};
					\draw [decorate,decoration={brace,amplitude=10pt,mirror},xshift=0.2pt,yshift=-0.2pt,brown](.5,-.5) -- (3.5,-.5) node[brown,midway,yshift=-15] {\tiny $U_\peq{1}$};
				\end{tikzpicture}
		\end{frame}		

		\begin{frame}
			\frametitle{Precio Mínimo}
			Algunas consideraciones respecto del salario mínimo:
			\begin{enumerate}
				\item Margen cuantitativo: magnitud del efecto depende de $\eta_\peq{L,w}$ y $\varepsilon_\peq{L,w}$.
				\item ¿A quiénes afecta?
				\item Ingreso laboral vs ingreso no laboral \href{http://www.ingresoetico.gob.cl/que-es-ief/}{(¿Ingreso Ético Familiar?)}
				\item Estructura del mercado: en caso de monopsonio u \href{http://webcurso.uc.cl/access/content/group/eae105a-9-22-2017/Bhaskar\%2C\%20Manning\%20_\%20To\%20_2002_\%20-\%20Oligopsony\%20and\%20Monopsonistic\%20Competition\%20in\%20Labor\%20Markets.pdf}{oligopsonio} podría incluso aumentar el empleo. 
			\end{enumerate}
		\end{frame}

		\begin{frame}
			\frametitle{Impuestos}
			Cuando el gobierno grava un bien con un impuesto, ¿quién soporta la carga tributaria?
			\begin{mydef}
				\textbf{Incidencia fiscal:} Forma en que los participantes de un mercado comparten la carga de un impuesto.
			\end{mydef}
		\end{frame}

		\begin{frame}
			\frametitle{Impuestos}
			Caso 1: Impuesto a los vendedores.
			\begin{itemize}
				\item Gobierno aplica impuesto de monto $\$t$ por unidad a los vendedores.
				\item Si los consumidores pagan $p^\peq{d}$ por unidad, el precio efectivo que perciben los productores es $p^\peq{s}=p^\peq{d}-t$.
				\item Alternativamente, para inducir a un productor a vender una unidad cuyo costo marginal es $p^\peq{s}$, hay que pagarle al menos $p^\peq{s}+t$.
			\end{itemize}
		\end{frame}

		\begin{frame}
			\frametitle{Impuestos}		
			\centering
			\begin{tikzpicture}[scale=.9]
				\draw [dashed,help lines] (0,2)--(2,2)--(2,0);
				\draw [dashed,help lines] (0,3)--(1,3)--(1,0);
				\draw [dashed,help lines] (0,1)--(1,1);
				\draw [ultra thick,teal] plot [domain=0:4] (\x,{4-\x});
				\node [above right,teal] at (4,0) {$D$};
				%\draw [ultra thick,teal] plot [domain=0:6] (\x,{6-\x});
				%\node [above right,teal] at (6,0) {$D_\peq{1}$};
				\draw [ultra thick,teal] plot [domain=0:6] (\x,{\x});
				\draw [<->,ultra thick] (0,7)--(0,0)--(7,0);
				\draw [ultra thick,purple] plot [domain=0:5] (\x,{2+\x});
				\draw [<->,ultra thick] (0,7)--(0,0)--(7,0);
				\node [right,teal] at (6,6) {$S_\peq{0}$};
				\node [right,purple] at (5,7) {$S_\peq{1}$};
				\node [left] at (0,7) {$p^\peq{d}$};
				\node [below] at (7,0) {$q$};
				\draw [fill,blue] (2,2) circle [radius=.1];
				\draw [fill,blue] (1,3) circle [radius=.1]; 
				\draw [fill,blue] (1,1) circle [radius=.1]; 
				\node [left] at (0,2) {\tiny $p^\peq{*}_\peq{0}$};
				\node [below] at (2,0) {\tiny $q^\peq{*}_\peq{0}$};
				\node [left] at (0,3) {\tiny $p^\peq{d}_\peq{1}$};
				\node [left] at (0,1) {\tiny $p^\peq{s}_\peq{1}$};
				\node [below] at (1,0) {\tiny $q^\peq{*}_\peq{1}$};
				\node [below] at (0,0) {\tiny 0};
				\draw [decorate,decoration={brace,amplitude=10pt},xshift=0.2pt,yshift=-0.2pt,blue,thick](4,6) -- (4,4) node[blue,midway,xshift=15] {\scriptsize $t$};
			\end{tikzpicture}
		\end{frame}

		\begin{frame}
			\frametitle{Impuestos}
			Ejemplo:
				\begin{align}
					\text{Demanda: } q^\peq{d}&=20-2p^\peq{d}& \\
					\text{Oferta: }  q^\peq{s}&=2+4p^\peq{s}& \\
					\text{Impuesto: } p^\peq{s}&=p^\peq{d}-2& %\\
					%\text{Equilibrio: } q^\peq{d}&=q^\peq{s}=q^\peq{*}&
				\end{align}
		\end{frame}

		\begin{frame}
			\frametitle{Impuestos}
			Equilibrio sin impuesto:
				\begin{align}
					\text{Demanda: } q^\peq{d}&=20-2p^\peq{d}& \\
					\text{Oferta: }  q^\peq{s}&=2+4p^\peq{s}& \\
					\text{No-distorsión: } p^\peq{s}&=p^\peq{d}& \\
					\text{Equilibrio: } q^\peq{s}&=q^\peq{d}&
				\end{align}
		\end{frame}

		\begin{frame}
			\frametitle{Impuestos}
			Resolviendo...
				\begin{align*}
					p^\peq{*}_\peq{0}&=3 \\
					q^\peq{*}_\peq{0}&=14
				\end{align*}
		\end{frame}
	
				\begin{frame}
			\frametitle{Impuestos}		
			\centering
			\begin{tikzpicture}[scale=.4]
				\draw [dashed,help lines] (0,3)--(14,3)--(14,0);
				%\draw [dashed,help lines] (0,3)--(1,3)--(1,0);
				%\draw [dashed,help lines] (0,1)--(1,1);
				\draw [ultra thick,teal] plot [domain=0:20] (\x,{10-.5*\x});
				\node [above,teal] at (20,0) {$D$};
				%\draw [ultra thick,teal] plot [domain=0:6] (\x,{6-\x});
				%\node [above right,teal] at (6,0) {$D_\peq{1}$};
				\draw [ultra thick,teal] plot [domain=2:20] (\x,{-.5+.25*\x});
				%\draw [ultra thick,purple] plot [domain=0:5] (\x,{2+\x});
				\draw [<->,ultra thick] (0,15)--(0,0)--(22,0);
				\node [right,teal] at (20,4.5) {$S_\peq{0}$};
				%\node [right,purple] at (5,7) {$S_\peq{1}$};
				\node [left] at (0,15) {$p^\peq{d}$};
				\node [below] at (22,0) {$q$};
				\draw [fill,blue] (14,3) circle [radius=.25];
				%\draw [fill,blue] (1,3) circle [radius=.1]; 
				%\draw [fill,blue] (1,1) circle [radius=.1]; 
				\node [left] at (0,3) {\tiny 3};
				\node [below] at (14,0) {\tiny 14};
				%\node [left] at (0,3) {\tiny $p^\peq{d}_\peq{1}$};
				%\node [left] at (0,1) {\tiny $p^\peq{s}_\peq{1}$};
				%\node [below] at (1,0) {\tiny $q^\peq{*}_\peq{1}$};
				%\node [below] at (0,0) {\tiny 0};
				%\draw [decorate,decoration={brace,amplitude=10pt},xshift=0.2pt,yshift=-0.2pt,blue,thick](4,6) -- (4,4) node[blue,midway,xshift=15] {\scriptsize $t$};
			\end{tikzpicture}
		\end{frame}

		\begin{frame}
			\frametitle{Impuestos}
			Equilibrio con impuesto:
				\begin{align}
					\text{Demanda: } q^\peq{d}&=20-2p^\peq{d}& \\
					\text{Oferta: }  q^\peq{s}&=2+4p^\peq{s}& \\
					\text{Impuesto: } p^\peq{s}&=p^\peq{d}-2& \\
					\text{Equilibrio: } q^\peq{s}&=q^\peq{d}&
				\end{align}
		\end{frame}

		\begin{frame}
			\frametitle{Impuestos}
			Reescribimos la curva de oferta en función del precio que paga el consumidor 
				\begin{align*}
					q^\peq{s}&=2+4p^\peq{s}\\
					q^\peq{s}&=2+4\rp{p^\peq{d}-2}\\
					q^\peq{s}&=-6+4p^\peq{d}
				\end{align*}
		\end{frame}
	
		\begin{frame}
			\frametitle{Impuestos}		
			\centering
			\begin{tikzpicture}[scale=.4]
				\draw [dashed,help lines] (0,3)--(14,3)--(14,0);
				%\draw [dashed,help lines] (0,3)--(1,3)--(1,0);
				%\draw [dashed,help lines] (0,1)--(1,1);
				\draw [ultra thick,teal] plot [domain=0:20] (\x,{10-.5*\x});
				\node [above,teal] at (20,0) {$D$};
				%\draw [ultra thick,teal] plot [domain=0:6] (\x,{6-\x});
				%\node [above right,teal] at (6,0) {$D_\peq{1}$};
				\draw [ultra thick,teal] plot [domain=2:20] (\x,{-.5+.25*\x});
				\draw [ultra thick,purple] plot [domain=0:20] (\x,{1.5+.25*\x});
				\draw [<->,ultra thick] (0,15)--(0,0)--(22,0);
				\node [right,teal] at (20,4.5) {$S_\peq{0}$};
				\node [right,purple] at (20,6.5) {$S_\peq{1}$};
				\node [left] at (0,15) {$p$};
				\node [below] at (22,0) {$q$};
				\draw [fill,blue] (14,3) circle [radius=.25];
				%\draw [fill,blue] (1,3) circle [radius=.1]; 
				%\draw [fill,blue] (1,1) circle [radius=.1]; 
				\node [left] at (0,3) {\tiny 3};
				\node [below] at (14,0) {\tiny 14};
				%\node [left] at (0,3) {\tiny $p^\peq{d}_\peq{1}$};
				%\node [left] at (0,1) {\tiny $p^\peq{s}_\peq{1}$};
				%\node [below] at (1,0) {\tiny $q^\peq{*}_\peq{1}$};
				%\node [below] at (0,0) {\tiny 0};
				%\draw [decorate,decoration={brace,amplitude=10pt},xshift=0.2pt,yshift=-0.2pt,blue,thick](4,6) -- (4,4) node[blue,midway,xshift=15] {\scriptsize $t$};
			\end{tikzpicture}
		\end{frame}

		\begin{frame}
			\frametitle{Impuestos}
			Resolviendo...
				\begin{align*}
					q^\peq{s}\rp{p^\peq{d}}&=q^\peq{d}\rp{p^\peq{d}}\text{...}\\
					p^\peq{d}_\peq{1}&=4.\bar{3}\\
					p^\peq{s}_\peq{1}&=2.\bar{3}\\
					q^\peq{*}_\peq{1}&=11.\bar{3}\\
				\end{align*}
		\end{frame}
	
		\begin{frame}
			\frametitle{Impuestos}		
			\centering
			\begin{tikzpicture}[scale=.4]
				\draw [dashed,help lines] (0,3)--(14,3)--(14,0);
				\draw [dashed,help lines] (0,4.333)--(11.333,4.333)--(11.333,0);
				\draw [dashed,help lines] (0,2.333)--(11.333,2.333);
				\draw [ultra thick,teal] plot [domain=0:20] (\x,{10-.5*\x});
				\node [above,teal] at (20,0) {$D$};
				%\draw [ultra thick,teal] plot [domain=0:6] (\x,{6-\x});
				%\node [above right,teal] at (6,0) {$D_\peq{1}$};
				\draw [ultra thick,teal] plot [domain=2:20] (\x,{-.5+.25*\x});
				\draw [ultra thick,purple] plot [domain=0:20] (\x,{1.5+.25*\x});
				\draw [<->,ultra thick] (0,15)--(0,0)--(22,0);
				\node [right,teal] at (20,4.5) {$S_\peq{0}$};
				\node [right,purple] at (20,6.5) {$S_\peq{1}$};
				\node [left] at (0,15) {$p$};
				\node [below] at (22,0) {$q$};
				\draw [fill,blue] (14,3) circle [radius=.25];
				\draw [fill,blue] (11.333,4.333) circle [radius=.25]; 
				\draw [fill,blue] (11.333,2.333) circle [radius=.25]; 
				\node [left] at (0,3) {\tiny 3};
				\node [left] at (0,2.333) {\tiny $2.\bar{3}$};
				\node [left] at (0,4.333) {\tiny $4.\bar{3}$};
				\node [below] at (11.333,0) {\tiny $11.\bar{3}$};
				\node [below] at (14,0) {\tiny 14};
				%\node [left] at (0,3) {\tiny $p^\peq{d}_\peq{1}$};
				%\node [left] at (0,1) {\tiny $p^\peq{s}_\peq{1}$};
				%\node [below] at (1,0) {\tiny $q^\peq{*}_\peq{1}$};
				%\node [below] at (0,0) {\tiny 0};
				%\draw [decorate,decoration={brace,amplitude=10pt},xshift=0.2pt,yshift=-0.2pt,blue,thick](4,6) -- (4,4) node[blue,midway,xshift=15] {\scriptsize $t$};
			\end{tikzpicture}
		\end{frame}

		\begin{frame}
			\frametitle{Impuestos}
			Caso 2: Impuesto a los compradores.
			\begin{itemize}
				\item Gobierno aplica impuesto de monto $\$t$ por unidad a los compradores.
				\item Si los vendedores cobran $p^\peq{s}$ por unidad, el precio efectivo que pagan los consumidores es $p^\peq{d}=p^\peq{s}+t$.
				\item Alternativamente, para inducir a un consumidor a comprar una unidad que valora en $p^\peq{d}$, hay que cobrarle a lo más $p^\peq{d}-t$.
			\end{itemize}
		\end{frame}

		\begin{frame}
			\frametitle{Impuestos}		
			\centering
			\begin{tikzpicture}[scale=.9]
				\draw [dashed,help lines] (0,3)--(3,3)--(3,0);
				\draw [dashed,help lines] (0,4)--(2,4)--(2,0);
				\draw [dashed,help lines] (0,2)--(2,2);
				\draw [ultra thick,purple] plot [domain=0:4] (\x,{4-\x});
				\node [above right,purple] at (4,0) {$D_\peq{1}$};
				\draw [ultra thick,teal] plot [domain=0:6] (\x,{6-\x});
				\node [above right,teal] at (6,0) {$D_\peq{0}$};
				\draw [ultra thick,teal] plot [domain=0:6] (\x,{\x});
				\draw [<->,ultra thick] (0,7)--(0,0)--(7,0);
				%\draw [ultra thick,purple] plot [domain=0:5] (\x,{2+\x});
				\draw [<->,ultra thick] (0,7)--(0,0)--(7,0);
				\node [right,teal] at (6,6) {$S$};
				%\node [right,purple] at (5,7) {$S_\peq{1}$};
				\node [left] at (0,7) {$p^\peq{s}$};
				\node [below] at (7,0) {$q$};
				\draw [fill,blue] (2,2) circle [radius=.1];
				\draw [fill,blue] (2,4) circle [radius=.1]; 
				\draw [fill,blue] (3,3) circle [radius=.1]; 
				\node [left] at (0,3) {\tiny $p^\peq{*}_\peq{0}$};
				\node [below] at (2,0) {\tiny $q^\peq{*}_\peq{1}$};
				\node [left] at (0,4) {\tiny $p^\peq{d}_\peq{1}$};
				\node [left] at (0,2) {\tiny $p^\peq{s}_\peq{1}$};
				\node [below] at (4,0) {\tiny $q^\peq{*}_\peq{0}$};
				\node [below] at (0,0) {\tiny 0};
				\draw [decorate,decoration={brace,amplitude=10pt},xshift=0.2pt,yshift=-0.2pt,blue,thick](.5,5.5) -- (.5,3.5) node[blue,midway,xshift=15] {\scriptsize $t$};
			\end{tikzpicture}
		\end{frame}

		\begin{frame}
			\frametitle{Impuestos}
			Volviendo al ejemplo:
				\begin{align}
					\text{Demanda: } q^\peq{d}&=20-2p^\peq{d}& \\
					\text{Oferta: }  q^\peq{s}&=2+4p^\peq{s}& \\
					\text{Impuesto: } p^\peq{d}&=p^\peq{s}+2& %\\
					%\text{Equilibrio: } q^\peq{d}&=q^\peq{s}=q^\peq{*}&
				\end{align}
		\end{frame}

		\begin{frame}
			\frametitle{Impuestos}
			Equilibrio sin impuesto:
				\begin{align}
					\text{Demanda: } q^\peq{d}&=20-2p^\peq{d}& \\
					\text{Oferta: }  q^\peq{s}&=2+4p^\peq{s}& \\
					\text{No-distorsión: } p^\peq{s}&=p^\peq{d}& \\
					\text{Equilibrio: } q^\peq{s}&=q^\peq{d}&
				\end{align}
		\end{frame}

		\begin{frame}
			\frametitle{Impuestos}
			Resolviendo...
				\begin{align*}
					p^\peq{*}_\peq{0}&=3 \\
					q^\peq{*}_\peq{0}&=14
				\end{align*}
		\end{frame}
	
				\begin{frame}
			\frametitle{Impuestos}		
			\centering
			\begin{tikzpicture}[scale=.4]
				\draw [dashed,help lines] (0,3)--(14,3)--(14,0);
				%\draw [dashed,help lines] (0,3)--(1,3)--(1,0);
				%\draw [dashed,help lines] (0,1)--(1,1);
				\draw [ultra thick,teal] plot [domain=0:20] (\x,{10-.5*\x});
				\node [above,teal] at (20,0) {$D$};
				%\draw [ultra thick,teal] plot [domain=0:6] (\x,{6-\x});
				%\node [above right,teal] at (6,0) {$D_\peq{1}$};
				\draw [ultra thick,teal] plot [domain=2:20] (\x,{-.5+.25*\x});
				%\draw [ultra thick,purple] plot [domain=0:5] (\x,{2+\x});
				\draw [<->,ultra thick] (0,15)--(0,0)--(22,0);
				\node [right,teal] at (20,4.5) {$S_\peq{0}$};
				%\node [right,purple] at (5,7) {$S_\peq{1}$};
				\node [left] at (0,15) {$p^\peq{d}$};
				\node [below] at (22,0) {$q$};
				\draw [fill,blue] (14,3) circle [radius=.25];
				%\draw [fill,blue] (1,3) circle [radius=.1]; 
				%\draw [fill,blue] (1,1) circle [radius=.1]; 
				\node [left] at (0,3) {\tiny 3};
				\node [below] at (14,0) {\tiny 14};
				%\node [left] at (0,3) {\tiny $p^\peq{d}_\peq{1}$};
				%\node [left] at (0,1) {\tiny $p^\peq{s}_\peq{1}$};
				%\node [below] at (1,0) {\tiny $q^\peq{*}_\peq{1}$};
				%\node [below] at (0,0) {\tiny 0};
				%\draw [decorate,decoration={brace,amplitude=10pt},xshift=0.2pt,yshift=-0.2pt,blue,thick](4,6) -- (4,4) node[blue,midway,xshift=15] {\scriptsize $t$};
			\end{tikzpicture}
		\end{frame}

		\begin{frame}
			\frametitle{Impuestos}
			Equilibrio con impuesto:
				\begin{align}
					\text{Demanda: } q^\peq{d}&=20-2p^\peq{d}& \\
					\text{Oferta: }  q^\peq{s}&=2+4p^\peq{s}& \\
					\text{Impuesto: } p^\peq{d}&=p^\peq{s}+2& \\
					\text{Equilibrio: } q^\peq{s}&=q^\peq{d}&
				\end{align}
		\end{frame}

		\begin{frame}
			\frametitle{Impuestos}
			Reescribimos la curva de demanda en función del precio que cobra el productor 
				\begin{align*}
					q^\peq{d}&=20-2p^\peq{d}\\
					q^\peq{s}&=20-2\rp{p^\peq{s}+2}\\
					q^\peq{s}&=16-2p^\peq{s}
				\end{align*}
		\end{frame}
	
		\begin{frame}
			\frametitle{Impuestos}		
			\centering
			\begin{tikzpicture}[scale=.4]
				\draw [dashed,help lines] (0,3)--(14,3)--(14,0);
				%\draw [dashed,help lines] (0,3)--(1,3)--(1,0);
				%\draw [dashed,help lines] (0,1)--(1,1);
				\draw [ultra thick,teal] plot [domain=0:20] (\x,{10-.5*\x});
				\node [above,teal] at (20,0) {$D_\peq{0}$};
				\draw [ultra thick,purple] plot [domain=0:16] (\x,{8-.5*\x});
				\node [above right,purple] at (16,0) {$D_\peq{1}$};
				\draw [ultra thick,teal] plot [domain=2:20] (\x,{-.5+.25*\x});
				%\draw [ultra thick,purple] plot [domain=0:20] (\x,{1.5+.25*\x});
				\draw [<->,ultra thick] (0,15)--(0,0)--(22,0);
				\node [right,teal] at (20,4.5) {$S$};
				%\node [right,purple] at (20,6.5) {$S_\peq{1}$};
				\node [left] at (0,15) {$p$};
				\node [below] at (22,0) {$q$};
				\draw [fill,blue] (14,3) circle [radius=.25];
				%\draw [fill,blue] (1,3) circle [radius=.1]; 
				%\draw [fill,blue] (1,1) circle [radius=.1]; 
				\node [left] at (0,3) {\tiny 3};
				\node [below] at (14,0) {\tiny 14};
				%\node [left] at (0,3) {\tiny $p^\peq{d}_\peq{1}$};
				%\node [left] at (0,1) {\tiny $p^\peq{s}_\peq{1}$};
				%\node [below] at (1,0) {\tiny $q^\peq{*}_\peq{1}$};
				%\node [below] at (0,0) {\tiny 0};
				%\draw [decorate,decoration={brace,amplitude=10pt},xshift=0.2pt,yshift=-0.2pt,blue,thick](4,6) -- (4,4) node[blue,midway,xshift=15] {\scriptsize $t$};
			\end{tikzpicture}
		\end{frame}

		\begin{frame}
			\frametitle{Impuestos}
			Resolviendo... 
				\begin{align*}
					q^\peq{s}\rp{p^\peq{s}}&=q^\peq{d}\rp{p^\peq{s}}\text{...}\\
					p^\peq{s}_\peq{1}&=2.\bar{3}\\
					p^\peq{d}_\peq{1}&=4.\bar{3}\\
					q^\peq{*}_\peq{1}&=11.\bar{3}\\
				\end{align*}
		\end{frame}
	
		\begin{frame}
			\frametitle{Impuestos}		
			\centering
			\begin{tikzpicture}[scale=.4]
				\draw [dashed,help lines] (0,3)--(14,3)--(14,0);
				\draw [dashed,help lines] (0,4.333)--(11.333,4.333)--(11.333,0);
				\draw [dashed,help lines] (0,2.333)--(11.333,2.333);
				\draw [ultra thick,teal] plot [domain=0:20] (\x,{10-.5*\x});
				\node [above,teal] at (20,0) {$D$};
				\draw [ultra thick,purple] plot [domain=0:16] (\x,{8-.5*\x});
				\node [above right,purple] at (16,0) {$D_\peq{1}$};
				\draw [ultra thick,teal] plot [domain=2:20] (\x,{-.5+.25*\x});
				%\draw [ultra thick,purple] plot [domain=0:20] (\x,{1.5+.25*\x});
				\draw [<->,ultra thick] (0,15)--(0,0)--(22,0);
				\node [right,teal] at (20,4.5) {$S$};
				%\node [right,purple] at (20,6.5) {$S_\peq{1}$};
				\node [left] at (0,15) {$p$};
				\node [below] at (22,0) {$q$};
				\draw [fill,blue] (14,3) circle [radius=.25];
				\draw [fill,blue] (11.333,4.333) circle [radius=.25]; 
				\draw [fill,blue] (11.333,2.333) circle [radius=.25]; 
				\node [left] at (0,3) {\tiny 3};
				\node [left] at (0,2.333) {\tiny $2.\bar{3}$};
				\node [left] at (0,4.333) {\tiny $4.\bar{3}$};
				\node [below] at (11.333,0) {\tiny $11.\bar{3}$};
				\node [below] at (14,0) {\tiny 14};
				%\node [left] at (0,3) {\tiny $p^\peq{d}_\peq{1}$};
				%\node [left] at (0,1) {\tiny $p^\peq{s}_\peq{1}$};
				%\node [below] at (1,0) {\tiny $q^\peq{*}_\peq{1}$};
				%\node [below] at (0,0) {\tiny 0};
				%\draw [decorate,decoration={brace,amplitude=10pt},xshift=0.2pt,yshift=-0.2pt,blue,thick](4,6) -- (4,4) node[blue,midway,xshift=15] {\scriptsize $t$};
			\end{tikzpicture}
		\end{frame}

		\begin{frame}
			\frametitle{Impuestos}
			Notar que el impuesto a los compradores es equivalente al impuesto a los vendedores. Las ecuaciones
				\begin{align}
					p^\peq{s}&=p^\peq{d}-t\\
					p^\peq{d}&=p^\peq{s}+t
				\end{align}
			son equivalentes. El impuesto introduce una brecha entre el precio pagado por el consumidor y el precio percibido por el productor: $$p^\peq{d}-p^\peq{s}=t$$
		\end{frame}

		\begin{frame}
			\frametitle{Impuestos}		
			\centering
			\begin{tikzpicture}[scale=.9]
				\draw [thick,blue] (0,2)--(2,2)--(2,0);
				\draw [thick,purple] (0,3)--(1,3);
				\draw [thick,purple] (0,1)--(1,1)--(1,0);
				\draw [ultra thick,purple] (1,3)--(1,1);
				\draw [ultra thick,purple] (1,3)--(1,1);
				\draw [ultra thick,teal] plot [domain=0:4] (\x,{4-\x});
				\node [above right,teal] at (4,0) {$D$};
				%\draw [ultra thick,teal] plot [domain=0:6] (\x,{6-\x});
				%\node [above right,teal] at (6,0) {$D_\peq{1}$};
				\draw [ultra thick,teal] plot [domain=0:6] (\x,{\x});
				\draw [<->,ultra thick] (0,7)--(0,0)--(7,0);
				%\draw [ultra thick,purple] plot [domain=0:5] (\x,{2+\x});
				\draw [<->,ultra thick] (0,7)--(0,0)--(7,0);
				\node [right,teal] at (6,6) {$S$};
				%\node [right,purple] at (5,7) {$S_\peq{1}$};
				\node [left] at (0,7) {$p$};
				\node [below] at (7,0) {$q$};
				\draw [fill,blue] (2,2) circle [radius=.1];
				\draw [fill,blue] (1,3) circle [radius=.1]; 
				\draw [fill,blue] (1,1) circle [radius=.1]; 
				\node [left] at (0,2) {\tiny $p^\peq{*}_\peq{0}$};
				\node [below] at (2,0) {\tiny $q^\peq{*}_\peq{0}$};
				\node [left] at (0,3) {\tiny $p^\peq{d}_\peq{1}$};
				\node [left] at (0,1) {\tiny $p^\peq{s}_\peq{1}$};
				\node [below] at (1,0) {\tiny $q^\peq{*}_\peq{1}$};
				\node [below] at (0,0) {\tiny 0};
				\draw [decorate,decoration={brace,amplitude=6pt,mirror},xshift=0.2pt,yshift=-0.2pt,purple,thick](-.5,3) -- (-.5,1) node[purple,midway,xshift=-10] {\scriptsize $t$};
			\end{tikzpicture}
		\end{frame}

		\begin{frame}
			\frametitle{Impuestos}
			Conclusiones:
			\begin{itemize}
				\item Los impuestos desincentivan la actividad económica: en el equilibrio con impuesto se transa menos.
				\item Indiferencia del sujeto gravado: impuesto a consumidores es equivalente a impuesto a vendedores.
				\item Compradores y vendedores comparten la carga tributaria: $\Delta^\peq{+}p^\peq{d}$ y $\Delta^\peq{-}p^\peq{s}$.
			\end{itemize}
		\end{frame}

		\begin{frame}
			\frametitle{Impuestos}
			Elasticidades e incidencia fiscal:
			\begin{itemize}
				\item El lado del mercado (oferta o demanda) que es más inelástico soporta una mayor proporción de la carga tributaria.
				\item Se trata de quienes están menos dispuestos a abandonar el mercado.
			\end{itemize}
		\end{frame}

		\begin{frame}
			\frametitle{Impuestos}
			\begin{figure}[hbtp!]
				\centering
				\begin{subfigure}[b]{0.49\textwidth}
					\begin{tikzpicture}[scale=.65]
						\draw [thick,blue] (0,2)--(2,2)--(2,0);
						\draw [thick,purple] (0,10/3)--(4/3,10/3);
						\draw [thick,purple] (0,4/3)--(4/3,4/3)--(4/3,0);
						\draw [ultra thick,purple] (4/3,10/3)--(4/3,4/3);
						\draw [ultra thick,teal] plot [domain=0:3] (\x,{6-2*\x});
						\node [above right,teal] at (3,0) {$D$};
						\draw [ultra thick,teal] plot [domain=0:6] (\x,{\x});
						\draw [<->,ultra thick] (0,7)--(0,0)--(7,0);
						\node [right,teal] at (6,6) {$S$};
						\node [left] at (0,7) {$p$};
						\node [below] at (7,0) {$q$};
						\draw [fill,blue] (2,2) circle [radius=.1];
						\draw [fill,blue] (4/3,10/3) circle [radius=.1]; 
						\draw [fill,blue] (4/3,4/3) circle [radius=.1]; 
						\node [left] at (0,2) {\tiny $p^\peq{*}_\peq{0}$};
						\node [below] at (2,0) {\tiny $q^\peq{*}_\peq{0}$};
						\node [left] at (0,10/3) {\tiny $p^\peq{d}_\peq{1}$};
						\node [left] at (0,4/3) {\tiny $p^\peq{s}_\peq{1}$};
						\node [below] at (4/3,0) {\tiny $q^\peq{*}_\peq{1}$};
						\node [below] at (0,0) {\tiny 0};
						\draw [decorate,decoration={brace,amplitude=3pt,mirror},xshift=0.2pt,yshift=-0.2pt,purple,thick](-.8,10/3) -- (-.8,2) node[purple,midway,xshift=-13] {\scriptsize $\Delta^\peq{+}p^\peq{d}$};
						\draw [decorate,decoration={brace,amplitude=2pt,mirror},xshift=0.2pt,yshift=-0.2pt,purple,thick](-.8,2) -- (-.8,4/3) node[purple,midway,xshift=-13] {\scriptsize $\Delta^\peq{-}p^\peq{s}$};
					\end{tikzpicture}
				\end{subfigure}
				\begin{subfigure}[b]{0.49\textwidth}
					\begin{tikzpicture}[scale=.65]
						\draw [thick,blue] (0,2)--(2,2)--(2,0);
						\draw [thick,purple] (0,8/3)--(4/3,8/3);
						\draw [thick,purple] (0,2/3)--(4/3,2/3)--(4/3,0);
						\draw [ultra thick,purple] (4/3,8/3)--(4/3,2/3);
						\draw [ultra thick,teal] plot [domain=0:4] (\x,{4-\x});
						\node [above right,teal] at (4,0) {$D$};
						\draw [ultra thick,teal] plot [domain=1:4] (\x,{-2+2*\x});
						\draw [<->,ultra thick] (0,7)--(0,0)--(7,0);
						\node [right,teal] at (4,6) {$S$};
						\node [left] at (0,7) {$p$};
						\node [below] at (7,0) {$q$};
						\draw [fill,blue] (2,2) circle [radius=.1];
						\draw [fill,blue] (4/3,8/3) circle [radius=.1]; 
						\draw [fill,blue] (4/3,2/3) circle [radius=.1]; 
						\node [left] at (0,2) {\tiny $p^\peq{*}_\peq{0}$};
						\node [below] at (2,0) {\tiny $q^\peq{*}_\peq{0}$};
						\node [left] at (0,8/3) {\tiny $p^\peq{d}_\peq{1}$};
						\node [left] at (0,2/3) {\tiny $p^\peq{s}_\peq{1}$};
						\node [below] at (4/3,0) {\tiny $q^\peq{*}_\peq{1}$};
						\node [below] at (0,0) {\tiny 0};
						\draw [decorate,decoration={brace,amplitude=2pt,mirror},xshift=0.2pt,yshift=-0.2pt,purple,thick](-.8,8/3) -- (-.8,2) node[purple,midway,xshift=-13] {\scriptsize $\Delta^\peq{+}p^\peq{d}$};
						\draw [decorate,decoration={brace,amplitude=3pt,mirror},xshift=0.2pt,yshift=-0.2pt,purple,thick](-.8,2) -- (-.8,2/3) node[purple,midway,xshift=-13] {\scriptsize $\Delta^\peq{-}p^\peq{s}$};
					\end{tikzpicture}
				\end{subfigure}
			\end{figure}
		\end{frame}

		\begin{frame}
			\frametitle{Impuestos}
			Excedente del consumidor
			\begin{figure}[hbtp!]
				\centering
				\begin{subfigure}[b]{0.49\textwidth}
					\begin{tikzpicture}[scale=1]
						\draw [fill,opacity=.3,blue] (0,4)--(2,2)--(0,2);
						\draw [ultra thick,teal] plot [domain=0:4] (\x,{4-\x});
						\node [above right,teal] at (4,0) {$D$};
						\draw [ultra thick,teal] plot [domain=0:4] (\x,{\x});
						\draw [thick,blue] (0,2)--(2,2)--(2,0);
						\draw [thick,purple] (0,3)--(1,3);
						\draw [thick,purple] (0,1)--(1,1)--(1,0);
						\draw [thick,purple] (1,3)--(1,1);
						\draw [<->,ultra thick] (0,5)--(0,0)--(5,0);
						\node [right,teal] at (4,4) {$S$};
						\node [left] at (0,5) {$p$};
						\node [below] at (5,0) {$q$};
						\node [left] at (0,2) {\tiny $p^\peq{*}$};
						\node [left] at (0,3) {\tiny $p^\peq{d}_\peq{1}$};
						\node [left] at (0,1) {\tiny $p^\peq{s}_\peq{1}$};
						\node [below] at (2,0) {\tiny $q^\peq{*}_\peq{0}$};
						\node [below] at (1,0) {\tiny $q^\peq{*}_\peq{1}$};
						\node [below] at (0,0) {\tiny 0};
						\node [above] at (.3,3) {\textbf{A}};
						\node [above] at (.3,2.2) {\textbf{B}};
						\node [above] at (1.2,2.2) {\textbf{C}};
						\node [below] at (.3,1.8) {\textbf{D}};
						\node [below] at (1.2,1.8) {\textbf{E}};
						\node [below] at (.3,1) {\textbf{F}};
						\node at (2.5,5) {\underline{Sin política}};
					\end{tikzpicture}
				\end{subfigure}
				\begin{subfigure}[b]{0.49\textwidth}
					\begin{tikzpicture}[scale=1]
						\draw [fill,opacity=.3,blue] (0,4)--(1,3)--(0,3);
						\draw [ultra thick,teal] plot [domain=0:4] (\x,{4-\x});
						\node [above right,teal] at (4,0) {$D$};
						\draw [ultra thick,teal] plot [domain=0:4] (\x,{\x});
						\draw [thick,blue] (0,2)--(2,2)--(2,0);
						\draw [thick,purple] (0,3)--(1,3);
						\draw [thick,purple] (0,1)--(1,1)--(1,0);
						\draw [thick,purple] (1,3)--(1,1);
						\draw [<->,ultra thick] (0,5)--(0,0)--(5,0);
						\node [right,teal] at (4,4) {$S$};
						\node [left] at (0,5) {$p$};
						\node [below] at (5,0) {$q$};
						\node [left] at (0,2) {\tiny $p^\peq{*}$};
						\node [left] at (0,3) {\tiny $p^\peq{d}_\peq{1}$};
						\node [left] at (0,1) {\tiny $p^\peq{s}_\peq{1}$};
						\node [below] at (2,0) {\tiny $q^\peq{*}_\peq{0}$};
						\node [below] at (1,0) {\tiny $q^\peq{*}_\peq{1}$};
						\node [below] at (0,0) {\tiny 0};
						\node [above] at (.3,3) {\textbf{A}};
						\node [above] at (.3,2.2) {\textbf{B}};
						\node [above] at (1.2,2.2) {\textbf{C}};
						\node [below] at (.3,1.8) {\textbf{D}};
						\node [below] at (1.2,1.8) {\textbf{E}};
						\node [below] at (.3,1) {\textbf{F}};	
						\node at (2.5,5) {\underline{Con política}};
					\end{tikzpicture}
				\end{subfigure}
			\end{figure}
		\end{frame}

		\begin{frame}
			\frametitle{Impuestos}
			Excedente del productor
			\begin{figure}[hbtp!]
				\centering
				\begin{subfigure}[b]{0.49\textwidth}
					\begin{tikzpicture}[scale=1]
						\draw [fill,opacity=.3,green] (0,0)--(2,2)--(0,2);
						\draw [ultra thick,teal] plot [domain=0:4] (\x,{4-\x});
						\node [above right,teal] at (4,0) {$D$};
						\draw [ultra thick,teal] plot [domain=0:4] (\x,{\x});
						\draw [thick,blue] (0,2)--(2,2)--(2,0);
						\draw [thick,purple] (0,3)--(1,3);
						\draw [thick,purple] (0,1)--(1,1)--(1,0);
						\draw [thick,purple] (1,3)--(1,1);
						\draw [<->,ultra thick] (0,5)--(0,0)--(5,0);
						\node [right,teal] at (4,4) {$S$};
						\node [left] at (0,5) {$p$};
						\node [below] at (5,0) {$q$};
						\node [left] at (0,2) {\tiny $p^\peq{*}$};
						\node [left] at (0,3) {\tiny $p^\peq{d}_\peq{1}$};
						\node [left] at (0,1) {\tiny $p^\peq{s}_\peq{1}$};
						\node [below] at (2,0) {\tiny $q^\peq{*}_\peq{0}$};
						\node [below] at (1,0) {\tiny $q^\peq{*}_\peq{1}$};
						\node [below] at (0,0) {\tiny 0};
						\node [above] at (.3,3) {\textbf{A}};
						\node [above] at (.3,2.2) {\textbf{B}};
						\node [above] at (1.2,2.2) {\textbf{C}};
						\node [below] at (.3,1.8) {\textbf{D}};
						\node [below] at (1.2,1.8) {\textbf{E}};
						\node [below] at (.3,1) {\textbf{F}};
						\node at (2.5,5) {\underline{Sin política}};
					\end{tikzpicture}
				\end{subfigure}
				\begin{subfigure}[b]{0.49\textwidth}
					\begin{tikzpicture}[scale=1]
						\draw [fill,opacity=.3,green] (0,0)--(1,1)--(0,1);
						\draw [ultra thick,teal] plot [domain=0:4] (\x,{4-\x});
						\node [above right,teal] at (4,0) {$D$};
						\draw [ultra thick,teal] plot [domain=0:4] (\x,{\x});
						\draw [thick,blue] (0,2)--(2,2)--(2,0);
						\draw [thick,purple] (0,3)--(1,3);
						\draw [thick,purple] (0,1)--(1,1)--(1,0);
						\draw [thick,purple] (1,3)--(1,1);
						\draw [<->,ultra thick] (0,5)--(0,0)--(5,0);
						\node [right,teal] at (4,4) {$S$};
						\node [left] at (0,5) {$p$};
						\node [below] at (5,0) {$q$};
						\node [left] at (0,2) {\tiny $p^\peq{*}$};
						\node [left] at (0,3) {\tiny $p^\peq{d}_\peq{1}$};
						\node [left] at (0,1) {\tiny $p^\peq{s}_\peq{1}$};
						\node [below] at (2,0) {\tiny $q^\peq{*}_\peq{0}$};
						\node [below] at (1,0) {\tiny $q^\peq{*}_\peq{1}$};
						\node [below] at (0,0) {\tiny 0};
						\node [above] at (.3,3) {\textbf{A}};
						\node [above] at (.3,2.2) {\textbf{B}};
						\node [above] at (1.2,2.2) {\textbf{C}};
						\node [below] at (.3,1.8) {\textbf{D}};
						\node [below] at (1.2,1.8) {\textbf{E}};
						\node [below] at (.3,1) {\textbf{F}};
						\node at (2.5,5) {\underline{Con política}};
					\end{tikzpicture}
				\end{subfigure}
			\end{figure}
		\end{frame}

		\begin{frame}
			\frametitle{Impuestos}
			Recaudación fiscal
			\begin{figure}[hbtp!]
				\centering
				\begin{subfigure}[b]{0.49\textwidth}
					\begin{tikzpicture}[scale=1]
						\draw [ultra thick,teal] plot [domain=0:4] (\x,{4-\x});
						\node [above right,teal] at (4,0) {$D$};
						\draw [ultra thick,teal] plot [domain=0:4] (\x,{\x});
						\draw [thick,blue] (0,2)--(2,2)--(2,0);
						\draw [thick,purple] (0,3)--(1,3);
						\draw [thick,purple] (0,1)--(1,1)--(1,0);
						\draw [thick,purple] (1,3)--(1,1);
						\draw [<->,ultra thick] (0,5)--(0,0)--(5,0);
						\node [right,teal] at (4,4) {$S$};
						\node [left] at (0,5) {$p$};
						\node [below] at (5,0) {$q$};
						\node [left] at (0,2) {\tiny $p^\peq{*}$};
						\node [left] at (0,3) {\tiny $p^\peq{d}_\peq{1}$};
						\node [left] at (0,1) {\tiny $p^\peq{s}_\peq{1}$};
						\node [below] at (2,0) {\tiny $q^\peq{*}_\peq{0}$};
						\node [below] at (1,0) {\tiny $q^\peq{*}_\peq{1}$};
						\node [below] at (0,0) {\tiny 0};
						\node [above] at (.3,3) {\textbf{A}};
						\node [above] at (.3,2.2) {\textbf{B}};
						\node [above] at (1.2,2.2) {\textbf{C}};
						\node [below] at (.3,1.8) {\textbf{D}};
						\node [below] at (1.2,1.8) {\textbf{E}};
						\node [below] at (.3,1) {\textbf{F}};
						\node at (2.5,5) {\underline{Sin política}};
					\end{tikzpicture}
				\end{subfigure}
				\begin{subfigure}[b]{0.49\textwidth}
					\begin{tikzpicture}[scale=1]
						\draw [fill,opacity=.4,yellow] (0,1)--(1,1)--(1,3)--(0,3);
						\draw [ultra thick,teal] plot [domain=0:4] (\x,{4-\x});
						\node [above right,teal] at (4,0) {$D$};
						\draw [ultra thick,teal] plot [domain=0:4] (\x,{\x});
						\draw [thick,blue] (0,2)--(2,2)--(2,0);
						\draw [thick,purple] (0,3)--(1,3);
						\draw [thick,purple] (0,1)--(1,1)--(1,0);
						\draw [thick,purple] (1,3)--(1,1);
						\draw [<->,ultra thick] (0,5)--(0,0)--(5,0);
						\node [right,teal] at (4,4) {$S$};
						\node [left] at (0,5) {$p$};
						\node [below] at (5,0) {$q$};
						\node [left] at (0,2) {\tiny $p^\peq{*}$};
						\node [left] at (0,3) {\tiny $p^\peq{d}_\peq{1}$};
						\node [left] at (0,1) {\tiny $p^\peq{s}_\peq{1}$};
						\node [below] at (2,0) {\tiny $q^\peq{*}_\peq{0}$};
						\node [below] at (1,0) {\tiny $q^\peq{*}_\peq{1}$};
						\node [below] at (0,0) {\tiny 0};
						\node [above] at (.3,3) {\textbf{A}};
						\node [above] at (.3,2.2) {\textbf{B}};
						\node [above] at (1.2,2.2) {\textbf{C}};
						\node [below] at (.3,1.8) {\textbf{D}};
						\node [below] at (1.2,1.8) {\textbf{E}};
						\node [below] at (.3,1) {\textbf{F}};
						\node at (2.5,5) {\underline{Con política}};
					\end{tikzpicture}
				\end{subfigure}
			\end{figure}
		\end{frame}

		\begin{frame}
			\frametitle{Impuestos}
			Excedente total
			\begin{figure}[hbtp!]
				\centering
				\begin{subfigure}[b]{0.49\textwidth}
					\begin{tikzpicture}[scale=1]
						\draw [fill,opacity=.3,purple] (0,0)--(2,2)--(0,4);
						\draw [ultra thick,teal] plot [domain=0:4] (\x,{4-\x});
						\node [above right,teal] at (4,0) {$D$};
						\draw [ultra thick,teal] plot [domain=0:4] (\x,{\x});
						\draw [thick,blue] (0,2)--(2,2)--(2,0);
						\draw [thick,purple] (0,3)--(1,3);
						\draw [thick,purple] (0,1)--(1,1)--(1,0);
						\draw [thick,purple] (1,3)--(1,1);
						\draw [<->,ultra thick] (0,5)--(0,0)--(5,0);
						\node [right,teal] at (4,4) {$S$};
						\node [left] at (0,5) {$p$};
						\node [below] at (5,0) {$q$};
						\node [left] at (0,2) {\tiny $p^\peq{*}$};
						\node [left] at (0,3) {\tiny $p^\peq{d}_\peq{1}$};
						\node [left] at (0,1) {\tiny $p^\peq{s}_\peq{1}$};
						\node [below] at (2,0) {\tiny $q^\peq{*}_\peq{0}$};
						\node [below] at (1,0) {\tiny $q^\peq{*}_\peq{1}$};
						\node [below] at (0,0) {\tiny 0};
						\node [above] at (.3,3) {\textbf{A}};
						\node [above] at (.3,2.2) {\textbf{B}};
						\node [above] at (1.2,2.2) {\textbf{C}};
						\node [below] at (.3,1.8) {\textbf{D}};
						\node [below] at (1.2,1.8) {\textbf{E}};
						\node [below] at (.3,1) {\textbf{F}};
						\node at (2.5,5) {\underline{Sin política}};
					\end{tikzpicture}
				\end{subfigure}
				\begin{subfigure}[b]{0.49\textwidth}
					\begin{tikzpicture}[scale=1]
						\draw [fill,opacity=.3,purple]  (0,4)--(1,3)--(1,1)--(0,0);
						\draw [ultra thick,teal] plot [domain=0:4] (\x,{4-\x});
						\node [above right,teal] at (4,0) {$D$};
						\draw [ultra thick,teal] plot [domain=0:4] (\x,{\x});
						\draw [thick,blue] (0,2)--(2,2)--(2,0);
						\draw [thick,purple] (0,3)--(1,3);
						\draw [thick,purple] (0,1)--(1,1)--(1,0);
						\draw [thick,purple] (1,3)--(1,1);
						\draw [<->,ultra thick] (0,5)--(0,0)--(5,0);
						\node [right,teal] at (4,4) {$S$};
						\node [left] at (0,5) {$p$};
						\node [below] at (5,0) {$q$};
						\node [left] at (0,2) {\tiny $p^\peq{*}$};
						\node [left] at (0,3) {\tiny $p^\peq{d}_\peq{1}$};
						\node [left] at (0,1) {\tiny $p^\peq{s}_\peq{1}$};
						\node [below] at (2,0) {\tiny $q^\peq{*}_\peq{0}$};
						\node [below] at (1,0) {\tiny $q^\peq{*}_\peq{1}$};
						\node [below] at (0,0) {\tiny 0};
						\node [above] at (.3,3) {\textbf{A}};
						\node [above] at (.3,2.2) {\textbf{B}};
						\node [above] at (1.2,2.2) {\textbf{C}};
						\node [below] at (.3,1.8) {\textbf{D}};
						\node [below] at (1.2,1.8) {\textbf{E}};
						\node [below] at (.3,1) {\textbf{F}};
						\node at (2.5,5) {\underline{Con política}};
					\end{tikzpicture}
				\end{subfigure}
			\end{figure}
		\end{frame}

		\begin{frame}
			\frametitle{Impuestos}
			Pérdida social
			
			\vspace{.1cm}
			
			\centering
			\begin{tikzpicture}[scale=1]
				\draw [fill,opacity=.3,black] (1,1)--(2,2)--(1,3);
						\draw [ultra thick,teal] plot [domain=0:4] (\x,{4-\x});
						\node [above right,teal] at (4,0) {$D$};
						\draw [ultra thick,teal] plot [domain=0:4] (\x,{\x});
						\draw [thick,blue] (0,2)--(2,2)--(2,0);
						\draw [thick,purple] (0,3)--(1,3);
						\draw [thick,purple] (0,1)--(1,1)--(1,0);
						\draw [thick,purple] (1,3)--(1,1);
						\draw [<->,ultra thick] (0,5)--(0,0)--(5,0);
						\node [right,teal] at (4,4) {$S$};
						\node [left] at (0,5) {$p$};
						\node [below] at (5,0) {$q$};
						\node [left] at (0,2) {\tiny $p^\peq{*}$};
						\node [left] at (0,3) {\tiny $p^\peq{d}_\peq{1}$};
						\node [left] at (0,1) {\tiny $p^\peq{s}_\peq{1}$};
						\node [below] at (2,0) {\tiny $q^\peq{*}_\peq{0}$};
						\node [below] at (1,0) {\tiny $q^\peq{*}_\peq{1}$};
						\node [below] at (0,0) {\tiny 0};
						\node [above] at (.3,3) {\textbf{A}};
						\node [above] at (.3,2.2) {\textbf{B}};
						\node [above] at (1.2,2.2) {\textbf{C}};
						\node [below] at (.3,1.8) {\textbf{D}};
						\node [below] at (1.2,1.8) {\textbf{E}};
						\node [below] at (.3,1) {\textbf{F}};
			\end{tikzpicture}
		\end{frame}	

		\begin{frame}
			\frametitle{Impuestos}
			\begin{table}[htbp!]
				\centering
				\resizebox{11cm}{!}{
					\begin{tabular}{l c c c}\hline
												&	Sin política	&	Con política	&	Cambio	\\  \hline 
									 $EC$ &		$A+B+C$			&			$A$				&	$-(B+C)$\\
									 $EP$ &		$D+E+F$			&			$F$				&	$-(D+E)$\\ 
									 $RF$ &				--			&			$B+D$			&	$+(B+D)$\\ \hline
									 $ET$ &	$A+B+C+D+E+F$	&		$A+B+D+F$		&	$-(C+E)$\\ \hline
					\end{tabular}}
			\end{table}
		\end{frame}

		\begin{frame}
			\frametitle{Impuestos}
			Elasticidades y pérdida social
			\begin{figure}[hbtp!]
				\centering
				\begin{subfigure}[b]{0.49\textwidth}
					\begin{tikzpicture}[scale=.7]
					%º\draw [fill,opacity=.3,black] (3/2,1)--(2,2)--(3/2,3);
						\draw [thick,blue] (0,2)--(2,2)--(2,0);
						\draw [thick,purple] (0,3)--(3/2,3);
						\draw [thick,purple] (0,1)--(3/2,1)--(3/2,0);
						\draw [ultra thick,purple] (3/2,3)--(3/2,1);
						\draw [ultra thick,teal] plot [domain=0:3] (\x,{6-2*\x});
						\node [above right,teal] at (3,0) {$D$};
						\draw [ultra thick,teal] plot [domain=1:4] (\x,{-2+2*\x});
						\draw [<->,ultra thick] (0,7)--(0,0)--(7,0);
						\node [right,teal] at (4,6) {$S$};
						\node [left] at (0,7) {$p$};
						\node [below] at (7,0) {$q$};
						\draw [fill,blue] (2,2) circle [radius=.1];
						\draw [fill,blue] (3/2,3) circle [radius=.1]; 
						\draw [fill,blue] (3/2,1) circle [radius=.1]; 
						\node [left] at (0,2) {\tiny $p^\peq{*}_\peq{0}$};
						\node [below] at (2,0) {\tiny $q^\peq{*}_\peq{0}$};
						\node [left] at (0,3) {\tiny $p^\peq{d}_\peq{1}$};
						\node [left] at (0,1) {\tiny $p^\peq{s}_\peq{1}$};
						\node [below] at (4/3,0) {\tiny $q^\peq{*}_\peq{1}$};
						\node [below] at (0,0) {\tiny 0};
					\end{tikzpicture}
				\end{subfigure}
				\begin{subfigure}[b]{0.49\textwidth}
					\begin{tikzpicture}[scale=.7]
						\draw [fill,opacity=.3,black] (2,1)--(4,2)--(2,3);
						\draw [thick,blue] (0,2)--(4,2)--(4,0);
						\draw [thick,purple] (0,3)--(2,3);
						\draw [thick,purple] (0,1)--(2,1)--(2,0);
						\draw [ultra thick,purple] (2,3)--(2,1);
						\draw [ultra thick,teal] plot [domain=0:7] (\x,{4-.5*\x});
						\node [above,teal] at (7,.5) {$D$};
						\draw [ultra thick,teal] plot [domain=0:7] (\x,{.5*\x});
						\draw [<->,ultra thick] (0,7)--(0,0)--(7,0);
						\node [right,teal] at (7,3.5) {$S$};
						\node [left] at (0,7) {$p$};
						\node [below] at (7,0) {$q$};
						\draw [fill,blue] (4,2) circle [radius=.1];
						\draw [fill,blue] (2,3) circle [radius=.1]; 
						\draw [fill,blue] (2,1) circle [radius=.1]; 
						\node [left] at (0,2) {\tiny $p^\peq{*}_\peq{0}$};
						\node [below] at (4,0) {\tiny $q^\peq{*}_\peq{0}$};
						\node [left] at (0,3) {\tiny $p^\peq{d}_\peq{1}$};
						\node [left] at (0,1) {\tiny $p^\peq{s}_\peq{1}$};
						\node [below] at (2,0) {\tiny $q^\peq{*}_\peq{1}$};
						\node [below] at (0,0) {\tiny 0};
					\end{tikzpicture}
				\end{subfigure}
			\end{figure}
		\end{frame}

		\begin{frame}
			\frametitle{Impuestos}
			Monto del impuesto y recaudación fiscal
			\begin{figure}[htbp!]
				\centering
				\begin{subfigure}[b]{0.32\textwidth}
					%\resizebox{\linewidth}{!}{
						\begin{tikzpicture}[scale=.6]
							\draw [fill,opacity=.3,yellow] (0,1.75)--(1.75,1.75)--(1.75,2.25)--(0,2.25);
							\draw [ultra thick,teal] plot [domain=0:4] (\x,{4-\x});
							\node [above right,teal] at (4,0) {$D$};
							\draw [ultra thick,teal] plot [domain=0:4] (\x,{\x});
							\draw [ultra thick,purple] (0,2.25)--(1.75,2.25)--(1.75,1.75)--(0,1.75);
							\draw [<->,ultra thick] (0,5)--(0,0)--(5,0);
							\node [right,teal] at (4,4) {$S$};
							\node [left] at (0,5) {$p$};
							\node [below] at (5,0) {$q$};
							\node [below] at (0,0) {\tiny 0};
						\end{tikzpicture}
					%}
				\end{subfigure}
				\begin{subfigure}[b]{0.32\textwidth}
					%\resizebox{\linewidth}{!}{
						\begin{tikzpicture}[scale=.6]
							\draw [fill,opacity=.3,yellow] (0,2.5)--(1.5,2.5)--(1.5,1.5)--(0,1.5);
							\draw [ultra thick,teal] plot [domain=0:4] (\x,{4-\x});
							\node [above right,teal] at (4,0) {$D$};
							\draw [ultra thick,teal] plot [domain=0:4] (\x,{\x});
							\draw [ultra thick,purple] (0,2.5)--(1.5,2.5)--(1.5,1.5)--(0,1.5);
							\draw [<->,ultra thick] (0,5)--(0,0)--(5,0);
							\node [right,teal] at (4,4) {$S$};
							\node [left] at (0,5) {$p$};
							\node [below] at (5,0) {$q$};
							\node [below] at (0,0) {\tiny 0};
						\end{tikzpicture}
					%}
				\end{subfigure}
				\begin{subfigure}[b]{0.32\textwidth}
					%\resizebox{\linewidth}{!}{
						\begin{tikzpicture}[scale=.6]
							\draw [fill,opacity=.3,yellow] (0,3.5)--(.5,3.5)--(.5,.5)--(0,.5);
							\draw [ultra thick,teal] plot [domain=0:4] (\x,{4-\x});
							\node [above right,teal] at (4,0) {$D$};
							\draw [ultra thick,teal] plot [domain=0:4] (\x,{\x});
							\draw [ultra thick,purple] (0,3.5)--(.5,3.5)--(.5,.5)--(0,.5);
							\draw [<->,ultra thick] (0,5)--(0,0)--(5,0);
							\node [right,teal] at (4,4) {$S$};
							\node [left] at (0,5) {$p$};
							\node [below] at (5,0) {$q$};
							\node [below] at (0,0) {\tiny 0};
						\end{tikzpicture}
					%}
				\end{subfigure}
			\end{figure}	
		\end{frame}	

		\begin{frame}
			\frametitle{Impuestos}
			Curva de Laffer
			
			\vspace{.1cm}
			
			\centering
			\begin{tikzpicture}[scale=1.5]
				\draw [ultra thick,teal] plot [domain=0:4] (\x,{2*\x-.5*\x^2});
				\draw [<->,ultra thick] (0,3)--(0,0)--(4.5,0);
				\node [left] at (0,3) {$RF$};
				\node [below] at (4.5,0) {$t$};
				\node [below] at (0,0) {\tiny 0};
			\end{tikzpicture}
		\end{frame}

		\begin{frame}
			\frametitle{Impuestos: apéndice matemático}
			Modelo:
				\begin{align}
					\text{Demanda: } p^\peq{d}&=a-bq^\peq{d}& \\
					\text{Oferta: }  p^\peq{s}&=c+dp^\peq{s}& \\
					\text{Impuesto: } p^\peq{d}-p^\peq{s}&=t& \\
					\text{Equilibrio: } q^\peq{s}&=q^\peq{d}&
				\end{align}
		\end{frame}

		\begin{frame}
			\frametitle{Impuestos: apéndice matemático}
			Equilibrio sin impuesto:
				\begin{align*}
					q^\peq{*}_\peq{0}&=\frac{a-c}{b+d}\\
					p^\peq{*}_\peq{0}&=\frac{ad+cb}{b+d}
				\end{align*}
		\end{frame}

		\begin{frame}
			\frametitle{Impuestos: apéndice matemático}
			Equilibrio con impuesto:
				\begin{align*}
					q^\peq{*}_\peq{1}&=\frac{a-c}{b+d}-\frac{t}{b+d}\\
					p^\peq{d}_\peq{1}&=\frac{ad+cb}{b+d}+\frac{bt}{b+d}\\
					p^\peq{s}_\peq{1}&=\frac{ad+cb}{b+d}-\frac{dt}{b+d}\\
				\end{align*}
		\end{frame}

		\begin{frame}
			\frametitle{Impuestos: apéndice matemático}
			Conclusiones:
			\begin{itemize}
				\item Tamaño del mercado se reduce: $$q^\peq{*}_\peq{1}=q^\peq{*}_\peq{0}-\rp{\frac{1}{b+d}}t\implies q^\peq{*}_\peq{1}<q^\peq{*}_\peq{0}$$
			\end{itemize}
		\end{frame}

		\begin{frame}
			\frametitle{Impuestos: apéndice matemático}
			\begin{itemize}
				\item Consumidores pagan más: $$p^\peq{d}_\peq{1}=p^\peq{*}_\peq{0}+\rp{\frac{b}{b+d}}t\implies p^\peq{d}_\peq{1}>p^\peq{*}_\peq{0}$$
				\item Productores reciben menos: $$p^\peq{s}_\peq{1}=p^\peq{*}_\peq{0}-\rp{\frac{d}{b+d}}t\implies p^\peq{s}_\peq{1}<p^\peq{*}_\peq{0}$$
			\end{itemize}
		\end{frame}

		\begin{frame}
			\frametitle{Impuestos: apéndice matemático}
			\begin{itemize}
				\item Aumento del precio pagado por el consumidor: $$\Delta p^\peq{d}=p^\peq{d}_\peq{1}-p^\peq{*}_\peq{0}=\rp{\frac{b}{b+d}}t$$
				\item Reducción del precio percibido por el productor: $$\Delta p^\peq{s}=p^\peq{s}_\peq{1}-p^\peq{*}_\peq{0}=-\rp{\frac{d}{b+d}}t$$
			\end{itemize}
		\end{frame}

		\begin{frame}
			\frametitle{Impuestos: apéndice matemático}
			\begin{itemize}
				\item Carga tributaria se reparte entre productores y consumidores: $$\Delta p^\peq{d}+\left|\Delta p^\peq{s}\right|=t$$
			\end{itemize}
		\end{frame}

		\begin{frame}
			\frametitle{Impuestos: apéndice matemático}
			\begin{itemize}
				\item Elasticidad precio de la demanda en equilibrio sin impuesto: $$\left|\eta^\peq{*}\right|=\rp{\frac{1}{b}}\rp{\frac{p^\peq{*}_\peq{0}}{q^\peq{*}_\peq{0}}}=\rp{\frac{1}{b}}\rp{\frac{ad+bc}{a-c}}$$
				\item Elasticidad precio de la oferta en equilibrio sin impuesto: $$\varepsilon^\peq{*}=\rp{\frac{1}{d}}\rp{\frac{p^\peq{*}_\peq{0}}{q^\peq{*}_\peq{0}}}=\rp{\frac{1}{d}}\rp{\frac{ad+bc}{a-c}}$$
			\end{itemize}
		\end{frame}

		\begin{frame}
			\frametitle{Impuestos: apéndice matemático}
			\begin{itemize}
				\item En términos relativos: $$\frac{\left|\eta^\peq{*}\right|}{\varepsilon^\peq{*}}=\frac{d}{b}$$
				\item Notar que 
					\begin{itemize}
						\item $\frac{\left|\eta^\peq{*}\right|}{\varepsilon^\peq{*}}>1\implies$ la oferta es más inelástica que la demanda
						\item $\frac{\left|\eta^\peq{*}\right|}{\varepsilon^\peq{*}}<1\implies$ la demanda es más inelástica que la oferta
					\end{itemize}
			\end{itemize}
		\end{frame}

		\begin{frame}
			\frametitle{Impuestos: apéndice matemático}
				\begin{itemize}
					\item El lado más inelástico del mercado (oferta o demanda) soporta una mayor propoción de la carga tributaria:
						\begin{itemize}
							\item $b>d \implies\frac{\left|\eta^\peq{*}\right|}{\varepsilon^\peq{*}}>1\wedge\frac{b}{b+d}>\frac{d}{b+d}$
							\item $b<d \implies\frac{\left|\eta^\peq{*}\right|}{\varepsilon^\peq{*}}<1\wedge\frac{b}{b+d}<\frac{d}{b+d}$
						\end{itemize}
				\end{itemize}
		\end{frame}

		\begin{frame}
			\frametitle{Impuestos: apéndice matemático}
				\begin{itemize}
					\item Pérdida social crece con monto del impuesto y elasticidades precio: $$PS=\frac{1}{2}t\rp{q^\peq{*}_\peq{0}-q^\peq{*}_\peq{1}}=\frac{1}{2}\frac{t^2}{b+d}$$
				\end{itemize}
		\end{frame}

		\begin{frame}
			\frametitle{Impuestos: apéndice matemático}
				\begin{itemize}
					\item Curva de Laffer: $$RF=t\cdot q^\peq{*}_\peq{1}=t\rp{\frac{a-c}{b+d}-\frac{t}{b+d}}=\frac{\rp{a-c}t}{b+d}-\frac{t^2}{b+d}$$
					\item Recaudación fiscal es cero cuando $t=0$ o $t=a-c$ y alcanza su máximo cuando $t=\frac{a-c}{2}$
				\end{itemize}
		\end{frame}

		\begin{frame}
			\frametitle{Subsidios}
			Gobierno subsidia un bien pagando $\$s$ por unidad
				\begin{align}
					\text{Subsidio a los vendedores: } p^\peq{s}&=p^\peq{d}+s\\
					\text{Subsidio a los compradores: } p^\peq{d}&=p^\peq{s}-s
				\end{align}
			Son equivalentes. El subsidio introduce una brecha entre el precio que cobra el productor y el precio que paga el consumidor: $$p^\peq{s}-p^\peq{d}=s$$
		\end{frame}

		\begin{frame}
			\frametitle{Subsidios}
			El subsidio es equivalente a un impuesto de monto negativo
				\begin{align*}
					p^\peq{s}-p^\peq{d}&=s\\
					p^\peq{d}-p^\peq{s}&=-s\\
					p^\peq{d}-p^\peq{s}&=t\text{, con } t=-s
				\end{align*}
		\end{frame}

		\begin{frame}
			\frametitle{Subsidios}
			\centering
			\begin{tikzpicture}[scale=1.25]
				\draw [ultra thick,teal] plot [domain=0:4] (\x,{4-\x});
				\node [above right,teal] at (4,0) {$D$};
				\draw [ultra thick,teal] plot [domain=0:4] (\x,{\x});
				\draw [thick,purple] (0,3)--(3,3);
				\draw [thick,purple] (0,1)--(3,1)--(3,0);
				\draw [ultra thick,purple] (3,3)--(3,1);
				\draw [thick,blue] (0,2)--(2,2)--(2,0);
				\draw [<->,ultra thick] (0,5)--(0,0)--(5,0);
				\node [right,teal] at (4,4) {$S$};
				\node [left] at (0,5) {$p$};
				\node [below] at (5,0) {$q$};
				\node [below] at (2,0) {\tiny $q^\peq{*}_\peq{0}$};
				\node [below] at (3,0) {\tiny $q^\peq{*}_\peq{1}$};
				\node [left] at (0,2) {\tiny $p^\peq{*}_\peq{0}$};
				\node [left] at (0,3) {\tiny $p^\peq{s}_\peq{1}$};
				\node [left] at (0,1) {\tiny $p^\peq{d}_\peq{1}$};
				\node [below] at (0,0) {\tiny 0};
				\draw [fill,blue] (2,2) circle [radius=.06];
				\draw [fill,blue] (3,3) circle [radius=.06];
				\draw [fill,blue] (3,1) circle [radius=.06];
				\draw [decorate,decoration={brace,amplitude=6pt,mirror},xshift=0.2pt,yshift=-0.2pt,purple,thick](-.4,3) -- (-.4,1) node[purple,midway,xshift=-10] {\scriptsize $s$};
			\end{tikzpicture}
		\end{frame}

		\begin{frame}
			\frametitle{Subsidios}
			Excedente del consumidor
			\begin{figure}[hbtp!]
				\centering
				\begin{subfigure}[b]{0.49\textwidth}
					\begin{tikzpicture}[scale=1]
						\draw [fill,opacity=.3,blue] (0,4)--(2,2)--(0,2);
						\draw [ultra thick,teal] plot [domain=0:4] (\x,{4-\x});
						\node [above right,teal] at (4,0) {$D$};
						\draw [ultra thick,teal] plot [domain=0:4] (\x,{\x});
						\draw [thick,purple] (0,3)--(3,3);
						\draw [thick,purple] (0,1)--(3,1)--(3,0);
						\draw [thick,purple] (3,3)--(3,1);
						\draw [thick,blue] (0,2)--(2,2)--(2,0);
						\draw [<->,ultra thick] (0,5)--(0,0)--(5,0);
						\node [right,teal] at (4,4) {$S$};
						\node [left] at (0,5) {$p$};
						\node [below] at (5,0) {$q$};
						\node [below] at (2,0) {\tiny $q^\peq{*}_\peq{0}$};
						\node [below] at (3,0) {\tiny $q^\peq{*}_\peq{1}$};
						\node [left] at (0,2) {\tiny $p^\peq{*}_\peq{0}$};
						\node [left] at (0,3) {\tiny $p^\peq{s}_\peq{1}$};
						\node [left] at (0,1) {\tiny $p^\peq{d}_\peq{1}$};
						\node [below] at (0,0) {\tiny 0};
						\node [above] at (.3,3) {\textbf{A}};
						\node [above] at (.8,2.2) {\textbf{B}};
						\node [below] at (.8,1.8) {\textbf{C}};
						\node [below] at (.3,1) {\textbf{D}};
						\node [above] at (2,2.2) {\textbf{E}};
						\node at (2.5,2) {\textbf{F}};
						\node [below] at (1.6,1.5) {\textbf{G}};
						\node [below] at (2.3,1.5) {\textbf{H}};
						\node at (2.5,5) {\underline{Sin política}};
					\end{tikzpicture}
				\end{subfigure}
				\begin{subfigure}[b]{0.49\textwidth}
					\begin{tikzpicture}[scale=1]
						\draw [fill,opacity=.3,blue] (0,4)--(3,1)--(0,1);
						\draw [ultra thick,teal] plot [domain=0:4] (\x,{4-\x});
						\node [above right,teal] at (4,0) {$D$};
						\draw [ultra thick,teal] plot [domain=0:4] (\x,{\x});
						\draw [thick,purple] (0,3)--(3,3);
						\draw [thick,purple] (0,1)--(3,1)--(3,0);
						\draw [thick,purple] (3,3)--(3,1);
						\draw [thick,blue] (0,2)--(2,2)--(2,0);
						\draw [<->,ultra thick] (0,5)--(0,0)--(5,0);
						\node [right,teal] at (4,4) {$S$};
						\node [left] at (0,5) {$p$};
						\node [below] at (5,0) {$q$};
						\node [below] at (2,0) {\tiny $q^\peq{*}_\peq{0}$};
						\node [below] at (3,0) {\tiny $q^\peq{*}_\peq{1}$};
						\node [left] at (0,2) {\tiny $p^\peq{*}_\peq{0}$};
						\node [left] at (0,3) {\tiny $p^\peq{s}_\peq{1}$};
						\node [left] at (0,1) {\tiny $p^\peq{d}_\peq{1}$};
						\node [below] at (0,0) {\tiny 0};
						\node [above] at (.3,3) {\textbf{A}};
						\node [above] at (.8,2.2) {\textbf{B}};
						\node [below] at (.8,1.8) {\textbf{C}};
						\node [below] at (.3,1) {\textbf{D}};
						\node [above] at (2,2.2) {\textbf{E}};
						\node at (2.5,2) {\textbf{F}};
						\node [below] at (1.6,1.5) {\textbf{G}};
						\node [below] at (2.3,1.5) {\textbf{H}};
						\node at (2.5,5) {\underline{Con política}};
					\end{tikzpicture}
				\end{subfigure}
			\end{figure}
		\end{frame}

		\begin{frame}
			\frametitle{Subsidios}
			Excedente del productor
			\begin{figure}[hbtp!]
				\centering
				\begin{subfigure}[b]{0.49\textwidth}
					\begin{tikzpicture}[scale=1]
						\draw [fill,opacity=.3,green] (0,0)--(2,2)--(0,2);
						\draw [ultra thick,teal] plot [domain=0:4] (\x,{4-\x});
						\node [above right,teal] at (4,0) {$D$};
						\draw [ultra thick,teal] plot [domain=0:4] (\x,{\x});
						\draw [thick,purple] (0,3)--(3,3);
						\draw [thick,purple] (0,1)--(3,1)--(3,0);
						\draw [thick,purple] (3,3)--(3,1);
						\draw [thick,blue] (0,2)--(2,2)--(2,0);
						\draw [<->,ultra thick] (0,5)--(0,0)--(5,0);
						\node [right,teal] at (4,4) {$S$};
						\node [left] at (0,5) {$p$};
						\node [below] at (5,0) {$q$};
						\node [below] at (2,0) {\tiny $q^\peq{*}_\peq{0}$};
						\node [below] at (3,0) {\tiny $q^\peq{*}_\peq{1}$};
						\node [left] at (0,2) {\tiny $p^\peq{*}_\peq{0}$};
						\node [left] at (0,3) {\tiny $p^\peq{s}_\peq{1}$};
						\node [left] at (0,1) {\tiny $p^\peq{d}_\peq{1}$};
						\node [below] at (0,0) {\tiny 0};
						\node [above] at (.3,3) {\textbf{A}};
						\node [above] at (.8,2.2) {\textbf{B}};
						\node [below] at (.8,1.8) {\textbf{C}};
						\node [below] at (.3,1) {\textbf{D}};
						\node [above] at (2,2.2) {\textbf{E}};
						\node at (2.5,2) {\textbf{F}};
						\node [below] at (1.6,1.5) {\textbf{G}};
						\node [below] at (2.3,1.5) {\textbf{H}};
						\node at (2.5,5) {\underline{Sin política}};
					\end{tikzpicture}
				\end{subfigure}
				\begin{subfigure}[b]{0.49\textwidth}
					\begin{tikzpicture}[scale=1]
						\draw [fill,opacity=.3,green] (0,0)--(3,3)--(0,3);
						\draw [ultra thick,teal] plot [domain=0:4] (\x,{4-\x});
						\node [above right,teal] at (4,0) {$D$};
						\draw [ultra thick,teal] plot [domain=0:4] (\x,{\x});
						\draw [thick,purple] (0,3)--(3,3);
						\draw [thick,purple] (0,1)--(3,1)--(3,0);
						\draw [thick,purple] (3,3)--(3,1);
						\draw [thick,blue] (0,2)--(2,2)--(2,0);
						\draw [<->,ultra thick] (0,5)--(0,0)--(5,0);
						\node [right,teal] at (4,4) {$S$};
						\node [left] at (0,5) {$p$};
						\node [below] at (5,0) {$q$};
						\node [below] at (2,0) {\tiny $q^\peq{*}_\peq{0}$};
						\node [below] at (3,0) {\tiny $q^\peq{*}_\peq{1}$};
						\node [left] at (0,2) {\tiny $p^\peq{*}_\peq{0}$};
						\node [left] at (0,3) {\tiny $p^\peq{s}_\peq{1}$};
						\node [left] at (0,1) {\tiny $p^\peq{d}_\peq{1}$};
						\node [below] at (0,0) {\tiny 0};
						\node [above] at (.3,3) {\textbf{A}};
						\node [above] at (.8,2.2) {\textbf{B}};
						\node [below] at (.8,1.8) {\textbf{C}};
						\node [below] at (.3,1) {\textbf{D}};
						\node [above] at (2,2.2) {\textbf{E}};
						\node at (2.5,2) {\textbf{F}};
						\node [below] at (1.6,1.5) {\textbf{G}};
						\node [below] at (2.3,1.5) {\textbf{H}};
						\node at (2.5,5) {\underline{Con política}};
					\end{tikzpicture}
				\end{subfigure}
			\end{figure}
		\end{frame}

		\begin{frame}
			\frametitle{Subsidios}
			Gasto fiscal
			\begin{figure}[hbtp!]
				\centering
				\begin{subfigure}[b]{0.49\textwidth}
					\begin{tikzpicture}[scale=1]
						%\draw [fill,opacity=.3,green] (0,0)--(2,2)--(4,0);
						\draw [ultra thick,teal] plot [domain=0:4] (\x,{4-\x});
						\node [above right,teal] at (4,0) {$D$};
						\draw [ultra thick,teal] plot [domain=0:4] (\x,{\x});
						\draw [thick,purple] (0,3)--(3,3);
						\draw [thick,purple] (0,1)--(3,1)--(3,0);
						\draw [thick,purple] (3,3)--(3,1);
						\draw [thick,blue] (0,2)--(2,2)--(2,0);
						\draw [<->,ultra thick] (0,5)--(0,0)--(5,0);
						\node [right,teal] at (4,4) {$S$};
						\node [left] at (0,5) {$p$};
						\node [below] at (5,0) {$q$};
						\node [below] at (2,0) {\tiny $q^\peq{*}_\peq{0}$};
						\node [below] at (3,0) {\tiny $q^\peq{*}_\peq{1}$};
						\node [left] at (0,2) {\tiny $p^\peq{*}_\peq{0}$};
						\node [left] at (0,3) {\tiny $p^\peq{s}_\peq{1}$};
						\node [left] at (0,1) {\tiny $p^\peq{d}_\peq{1}$};
						\node [below] at (0,0) {\tiny 0};
						\node [above] at (.3,3) {\textbf{A}};
						\node [above] at (.8,2.2) {\textbf{B}};
						\node [below] at (.8,1.8) {\textbf{C}};
						\node [below] at (.3,1) {\textbf{D}};
						\node [above] at (2,2.2) {\textbf{E}};
						\node at (2.5,2) {\textbf{F}};
						\node [below] at (1.6,1.5) {\textbf{G}};
						\node [below] at (2.3,1.5) {\textbf{H}};
						\node at (2.5,5) {\underline{Sin política}};
					\end{tikzpicture}
				\end{subfigure}
				\begin{subfigure}[b]{0.49\textwidth}
					\begin{tikzpicture}[scale=1]
						\draw [fill,opacity=.4,yellow] (0,3)--(3,3)--(3,1)--(0,1);
						\draw [ultra thick,teal] plot [domain=0:4] (\x,{4-\x});
						\node [above right,teal] at (4,0) {$D$};
						\draw [ultra thick,teal] plot [domain=0:4] (\x,{\x});
						\draw [thick,purple] (0,3)--(3,3);
						\draw [thick,purple] (0,1)--(3,1)--(3,0);
						\draw [thick,purple] (3,3)--(3,1);
						\draw [thick,blue] (0,2)--(2,2)--(2,0);
						\draw [<->,ultra thick] (0,5)--(0,0)--(5,0);
						\node [right,teal] at (4,4) {$S$};
						\node [left] at (0,5) {$p$};
						\node [below] at (5,0) {$q$};
						\node [below] at (2,0) {\tiny $q^\peq{*}_\peq{0}$};
						\node [below] at (3,0) {\tiny $q^\peq{*}_\peq{1}$};
						\node [left] at (0,2) {\tiny $p^\peq{*}_\peq{0}$};
						\node [left] at (0,3) {\tiny $p^\peq{s}_\peq{1}$};
						\node [left] at (0,1) {\tiny $p^\peq{d}_\peq{1}$};
						\node [below] at (0,0) {\tiny 0};
						\node [above] at (.3,3) {\textbf{A}};
						\node [above] at (.8,2.2) {\textbf{B}};
						\node [below] at (.8,1.8) {\textbf{C}};
						\node [below] at (.3,1) {\textbf{D}};
						\node [above] at (2,2.2) {\textbf{E}};
						\node at (2.5,2) {\textbf{F}};
						\node [below] at (1.6,1.5) {\textbf{G}};
						\node [below] at (2.3,1.5) {\textbf{H}};
						\node at (2.5,5) {\underline{Con política}};
					\end{tikzpicture}
				\end{subfigure}
			\end{figure}
		\end{frame}

		\begin{frame}
			\frametitle{Subsidios}
			Excedente total
			\begin{figure}[hbtp!]
				\centering
				\begin{subfigure}[b]{0.49\textwidth}
					\begin{tikzpicture}[scale=1]
						\draw [fill,opacity=.3,purple] (0,0)--(2,2)--(0,4);
						\draw [ultra thick,teal] plot [domain=0:4] (\x,{4-\x});
						\node [above right,teal] at (4,0) {$D$};
						\draw [ultra thick,teal] plot [domain=0:4] (\x,{\x});
						\draw [thick,purple] (0,3)--(3,3);
						\draw [thick,purple] (0,1)--(3,1)--(3,0);
						\draw [thick,purple] (3,3)--(3,1);
						\draw [thick,blue] (0,2)--(2,2)--(2,0);
						\draw [<->,ultra thick] (0,5)--(0,0)--(5,0);
						\node [right,teal] at (4,4) {$S$};
						\node [left] at (0,5) {$p$};
						\node [below] at (5,0) {$q$};
						\node [below] at (2,0) {\tiny $q^\peq{*}_\peq{0}$};
						\node [below] at (3,0) {\tiny $q^\peq{*}_\peq{1}$};
						\node [left] at (0,2) {\tiny $p^\peq{*}_\peq{0}$};
						\node [left] at (0,3) {\tiny $p^\peq{s}_\peq{1}$};
						\node [left] at (0,1) {\tiny $p^\peq{d}_\peq{1}$};
						\node [below] at (0,0) {\tiny 0};
						\node [above] at (.3,3) {\textbf{A}};
						\node [above] at (.8,2.2) {\textbf{B}};
						\node [below] at (.8,1.8) {\textbf{C}};
						\node [below] at (.3,1) {\textbf{D}};
						\node [above] at (2,2.2) {\textbf{E}};
						\node at (2.5,2) {\textbf{F}};
						\node [below] at (1.6,1.5) {\textbf{G}};
						\node [below] at (2.3,1.5) {\textbf{H}};
						\node at (2.5,5) {\underline{Sin política}};
					\end{tikzpicture}
				\end{subfigure}
				\begin{subfigure}[b]{0.49\textwidth}
					\begin{tikzpicture}[scale=1]
						\draw [fill,opacity=.3,purple] (0,0)--(2,2)--(0,4);
						\draw [fill,opacity=.3,black] (2,2)--(3,3)--(3,1);
						\draw [ultra thick,teal] plot [domain=0:4] (\x,{4-\x});
						\node [above right,teal] at (4,0) {$D$};
						\draw [ultra thick,teal] plot [domain=0:4] (\x,{\x});
						\draw [thick,purple] (0,3)--(3,3);
						\draw [thick,purple] (0,1)--(3,1)--(3,0);
						\draw [thick,purple] (3,3)--(3,1);
						\draw [thick,blue] (0,2)--(2,2)--(2,0);
						\draw [<->,ultra thick] (0,5)--(0,0)--(5,0);
						\node [right,teal] at (4,4) {$S$};
						\node [left] at (0,5) {$p$};
						\node [below] at (5,0) {$q$};
						\node [below] at (2,0) {\tiny $q^\peq{*}_\peq{0}$};
						\node [below] at (3,0) {\tiny $q^\peq{*}_\peq{1}$};
						\node [left] at (0,2) {\tiny $p^\peq{*}_\peq{0}$};
						\node [left] at (0,3) {\tiny $p^\peq{s}_\peq{1}$};
						\node [left] at (0,1) {\tiny $p^\peq{d}_\peq{1}$};
						\node [below] at (0,0) {\tiny 0};
						\node [above] at (.3,3) {\textbf{A}};
						\node [above] at (.8,2.2) {\textbf{B}};
						\node [below] at (.8,1.8) {\textbf{C}};
						\node [below] at (.3,1) {\textbf{D}};
						\node [above] at (2,2.2) {\textbf{E}};
						\node at (2.5,2) {\textbf{F}};
						\node [below] at (1.6,1.5) {\textbf{G}};
						\node [below] at (2.3,1.5) {\textbf{H}};
						\node at (2.5,5) {\underline{Con política}};
					\end{tikzpicture}
				\end{subfigure}
			\end{figure}
		\end{frame}

		\begin{frame}
			\frametitle{Subsidios}
			Pérdida social
			
			\vspace{.1cm}
			
			\centering
			\begin{tikzpicture}[scale=1]
				\draw [fill,opacity=.3,black] (2,2)--(3,3)--(3,1);
				\draw [ultra thick,teal] plot [domain=0:4] (\x,{4-\x});
				\node [above right,teal] at (4,0) {$D$};
				\draw [ultra thick,teal] plot [domain=0:4] (\x,{\x});
				\draw [thick,purple] (0,3)--(3,3);
				\draw [thick,purple] (0,1)--(3,1)--(3,0);
				\draw [thick,purple] (3,3)--(3,1);
				\draw [thick,blue] (0,2)--(2,2)--(2,0);
				\draw [<->,ultra thick] (0,5)--(0,0)--(5,0);
				\node [right,teal] at (4,4) {$S$};
				\node [left] at (0,5) {$p$};
				\node [below] at (5,0) {$q$};
				\node [below] at (2,0) {\tiny $q^\peq{*}_\peq{0}$};
				\node [below] at (3,0) {\tiny $q^\peq{*}_\peq{1}$};
				\node [left] at (0,2) {\tiny $p^\peq{*}_\peq{0}$};
				\node [left] at (0,3) {\tiny $p^\peq{s}_\peq{1}$};
				\node [left] at (0,1) {\tiny $p^\peq{d}_\peq{1}$};
				\node [below] at (0,0) {\tiny 0};
				\node [above] at (.3,3) {\textbf{A}};
				\node [above] at (.8,2.2) {\textbf{B}};
				\node [below] at (.8,1.8) {\textbf{C}};
				\node [below] at (.3,1) {\textbf{D}};
				\node [above] at (2,2.2) {\textbf{E}};
				\node at (2.5,2) {\textbf{F}};
				\node [below] at (1.6,1.5) {\textbf{G}};
				\node [below] at (2.3,1.5) {\textbf{H}};
			\end{tikzpicture}
		\end{frame}	

		\begin{frame}
			\frametitle{Subsidios}
			\begin{table}[htbp!]
				\centering
				\resizebox{11cm}{!}{
					\begin{tabular}{l c c c}\hline
												&	Sin política	&	Con política	 &	Cambio					\\  \hline 
									 $EC$ &			$A+B$			&	$A+B+C+G+H$		 &	$+(C+G+H)$			\\
									 $EP$ &			$C+D$			&		$B+C+D+E$		 &	$+(B+E)$				\\ 
									$GF$	&				--			&$-(B+C+E+F+G+H)$&	$-(B+C+E+F+G+H)$\\ \hline
									 $ET$ &		$A+B+C+D$		&	$A+B+C+D-F$		 &	$-F$						\\ \hline
					\end{tabular}}
			\end{table}
		\end{frame}

		\begin{frame}
			\frametitle{Comercio Internacional}
			Hasta ahora hemos estudiado una economía cerrada al comercio internacional. ¿Qué pasa si se abre?
				\begin{itemize}
					\item Supondremos que se trata de una economía tomadora de precios en los mercados internacionales (país pequeño).
					\item Seguiremos llamando $p^\peq{*}$ al precio de equilibrio doméstico (economía cerrada) y llamaremos $p^\peq{m}$ al precio en el mercado internacional.
				\end{itemize}
		\end{frame}

		\begin{frame}
			\frametitle{Comercio Internacional}
			Caso 1: $p^\peq{m}>p^\peq{*}\implies$ el país tiene ventaja comparativa
			\centering
			\begin{tikzpicture}[scale=1]
				\draw [ultra thick,teal] plot [domain=0:4] (\x,{4-\x});
				\node [above right,teal] at (4,0) {$D$};
				\draw [ultra thick,teal] plot [domain=0:4] (\x,{\x});
				\draw [thick,blue] (0,2)--(2,2)--(2,0);
				\draw [thick,purple] (0,3)--(1,3)--(1,0);
				\draw [thick,purple] (1,3)--(3,3)--(3,0);
				\draw [<->,ultra thick] (0,5)--(0,0)--(5,0);
				\node [right,teal] at (4,4) {$S$};
				\node [left] at (0,5) {$p$};
				\node [below] at (5,0) {$q$};
				\node [left] at (0,2) {\tiny $p^\peq{*}$};
				\node [left] at (0,3) {\tiny $p^\peq{m}$};
				\node [below] at (2,0) {\tiny $q^\peq{*}$};
				\node [below] at (3,0) {\tiny $q^\peq{s}\rp{p^\peq{m}}$};
				\node [below] at (1,0) {\tiny $q^\peq{d}\rp{p^\peq{m}}$};
				\node [below] at (0,0) {\tiny 0};
				\draw [decorate,decoration={brace,amplitude=6pt,mirror},xshift=0.2pt,yshift=-0.2pt,purple](1,-.4) -- (3,-.4) node[purple,midway,yshift=-0.5cm] {\tiny \pbox{\textwidth}{$X\rp{p^\peq{m}}$}};
			\end{tikzpicture}
		\end{frame}

		\begin{frame}
			\frametitle{Comercio Internacional}
			Excedente del consumidor
			\begin{figure}[hbtp!]
				\centering
				\begin{subfigure}[b]{0.49\textwidth}
					\begin{tikzpicture}[scale=1]
						\draw [fill,opacity=.3,blue] (0,4)--(2,2)--(0,2);
						\draw [ultra thick,teal] plot [domain=0:4] (\x,{4-\x});
						\node [above right,teal] at (4,0) {$D$};
						\draw [ultra thick,teal] plot [domain=0:4] (\x,{\x});
						\draw [thick,blue] (0,2)--(2,2)--(2,0);
						\draw [thick,purple] (0,3)--(1,3)--(1,0);
						\draw [thick,purple] (1,3)--(3,3)--(3,0);
						\draw [<->,ultra thick] (0,5)--(0,0)--(5,0);
						\node [right,teal] at (4,4) {$S$};
						\node [left] at (0,5) {$p$};
						\node [below] at (5,0) {$q$};
						\node [left] at (0,2) {\tiny $p^\peq{*}$};
						\node [left] at (0,3) {\tiny $p^\peq{m}$};
						\node [below] at (2,0) {\tiny $q^\peq{*}$};
						\node [below] at (3,0) {\tiny $q^\peq{s}\rp{p^\peq{m}}$};
						\node [below] at (1,0) {\tiny $q^\peq{d}\rp{p^\peq{m}}$};
						\node [below] at (0,0) {\tiny 0};
						\node [above] at (.3,3) {\textbf{A}};
						\node [above] at (.3,2.2) {\textbf{B}};
						\node [above] at (1.2,2.2) {\textbf{C}};
						\node [below] at (.3,1.8) {\textbf{D}};
						\node [below] at (1.2,1.8) {\textbf{E}};
						\node at (2,2.5) {\textbf{F}};
						\node at (2.5,5) {\underline{Sin política}};
					\end{tikzpicture}
				\end{subfigure}
				\begin{subfigure}[b]{0.49\textwidth}
					\begin{tikzpicture}[scale=1]
						\draw [fill,opacity=.3,blue] (0,4)--(1,3)--(0,3);
						\draw [ultra thick,teal] plot [domain=0:4] (\x,{4-\x});
						\node [above right,teal] at (4,0) {$D$};
						\draw [ultra thick,teal] plot [domain=0:4] (\x,{\x});
						\draw [thick,blue] (0,2)--(2,2)--(2,0);
						\draw [thick,purple] (0,3)--(1,3)--(1,0);
						\draw [thick,purple] (1,3)--(3,3)--(3,0);
						\draw [<->,ultra thick] (0,5)--(0,0)--(5,0);
						\node [right,teal] at (4,4) {$S$};
						\node [left] at (0,5) {$p$};
						\node [below] at (5,0) {$q$};
						\node [left] at (0,2) {\tiny $p^\peq{*}$};
						\node [left] at (0,3) {\tiny $p^\peq{m}$};
						\node [below] at (2,0) {\tiny $q^\peq{*}$};
						\node [below] at (3,0) {\tiny $q^\peq{s}\rp{p^\peq{m}}$};
						\node [below] at (1,0) {\tiny $q^\peq{d}\rp{p^\peq{m}}$};
						\node [below] at (0,0) {\tiny 0};
						\node [above] at (.3,3) {\textbf{A}};
						\node [above] at (.3,2.2) {\textbf{B}};
						\node [above] at (1.2,2.2) {\textbf{C}};
						\node [below] at (.3,1.8) {\textbf{D}};
						\node [below] at (1.2,1.8) {\textbf{E}};
						\node at (2,2.5) {\textbf{F}};
						\node at (2.5,5) {\underline{Con política}};
					\end{tikzpicture}
				\end{subfigure}
			\end{figure}
		\end{frame}

		\begin{frame}
			\frametitle{Comercio Internacional}
			Excedente del productor
			\begin{figure}[hbtp!]
				\centering
				\begin{subfigure}[b]{0.49\textwidth}
					\begin{tikzpicture}[scale=1]
						\draw [fill,opacity=.3,green] (0,0)--(2,2)--(0,2);
						\draw [ultra thick,teal] plot [domain=0:4] (\x,{4-\x});
						\node [above right,teal] at (4,0) {$D$};
						\draw [ultra thick,teal] plot [domain=0:4] (\x,{\x});
						\draw [thick,blue] (0,2)--(2,2)--(2,0);
						\draw [thick,purple] (0,3)--(1,3)--(1,0);
						\draw [thick,purple] (1,3)--(3,3)--(3,0);
						\draw [<->,ultra thick] (0,5)--(0,0)--(5,0);
						\node [right,teal] at (4,4) {$S$};
						\node [left] at (0,5) {$p$};
						\node [below] at (5,0) {$q$};
						\node [left] at (0,2) {\tiny $p^\peq{*}$};
						\node [left] at (0,3) {\tiny $p^\peq{m}$};
						\node [below] at (2,0) {\tiny $q^\peq{*}$};
						\node [below] at (3,0) {\tiny $q^\peq{s}\rp{p^\peq{m}}$};
						\node [below] at (1,0) {\tiny $q^\peq{d}\rp{p^\peq{m}}$};
						\node [below] at (0,0) {\tiny 0};
						\node [above] at (.3,3) {\textbf{A}};
						\node [above] at (.3,2.2) {\textbf{B}};
						\node [above] at (1.2,2.2) {\textbf{C}};
						\node [below] at (.3,1.8) {\textbf{D}};
						\node [below] at (1.2,1.8) {\textbf{E}};
						\node at (2,2.5) {\textbf{F}};
						\node at (2.5,5) {\underline{Sin política}};
					\end{tikzpicture}
				\end{subfigure}
				\begin{subfigure}[b]{0.49\textwidth}
					\begin{tikzpicture}[scale=1]
						\draw [fill,opacity=.3,green] (0,0)--(3,3)--(0,3);
						\draw [ultra thick,teal] plot [domain=0:4] (\x,{4-\x});
						\node [above right,teal] at (4,0) {$D$};
						\draw [ultra thick,teal] plot [domain=0:4] (\x,{\x});
						\draw [thick,blue] (0,2)--(2,2)--(2,0);
						\draw [thick,purple] (0,3)--(1,3)--(1,0);
						\draw [thick,purple] (1,3)--(3,3)--(3,0);
						\draw [<->,ultra thick] (0,5)--(0,0)--(5,0);
						\node [right,teal] at (4,4) {$S$};
						\node [left] at (0,5) {$p$};
						\node [below] at (5,0) {$q$};
						\node [left] at (0,2) {\tiny $p^\peq{*}$};
						\node [left] at (0,3) {\tiny $p^\peq{m}$};
						\node [below] at (2,0) {\tiny $q^\peq{*}$};
						\node [below] at (3,0) {\tiny $q^\peq{s}\rp{p^\peq{m}}$};
						\node [below] at (1,0) {\tiny $q^\peq{d}\rp{p^\peq{m}}$};
						\node [below] at (0,0) {\tiny 0};
						\node [above] at (.3,3) {\textbf{A}};
						\node [above] at (.3,2.2) {\textbf{B}};
						\node [above] at (1.2,2.2) {\textbf{C}};
						\node [below] at (.3,1.8) {\textbf{D}};
						\node [below] at (1.2,1.8) {\textbf{E}};
						\node at (2,2.5) {\textbf{F}};
						\node at (2.5,5) {\underline{Con política}};
					\end{tikzpicture}
				\end{subfigure}
			\end{figure}
		\end{frame}

		\begin{frame}
			\frametitle{Comercio Internacional}
			Excedente total
			\begin{figure}[hbtp!]
				\centering
				\begin{subfigure}[b]{0.49\textwidth}
					\begin{tikzpicture}[scale=1]
						\draw [fill,opacity=.3,purple] (0,0)--(2,2)--(0,4);
						\draw [ultra thick,teal] plot [domain=0:4] (\x,{4-\x});
						\node [above right,teal] at (4,0) {$D$};
						\draw [ultra thick,teal] plot [domain=0:4] (\x,{\x});
						\draw [thick,blue] (0,2)--(2,2)--(2,0);
						\draw [thick,purple] (0,3)--(1,3)--(1,0);
						\draw [thick,purple] (1,3)--(3,3)--(3,0);
						\draw [<->,ultra thick] (0,5)--(0,0)--(5,0);
						\node [right,teal] at (4,4) {$S$};
						\node [left] at (0,5) {$p$};
						\node [below] at (5,0) {$q$};
						\node [left] at (0,2) {\tiny $p^\peq{*}$};
						\node [left] at (0,3) {\tiny $p^\peq{m}$};
						\node [below] at (2,0) {\tiny $q^\peq{*}$};
						\node [below] at (3,0) {\tiny $q^\peq{s}\rp{p^\peq{m}}$};
						\node [below] at (1,0) {\tiny $q^\peq{d}\rp{p^\peq{m}}$};
						\node [below] at (0,0) {\tiny 0};
						\node [above] at (.3,3) {\textbf{A}};
						\node [above] at (.3,2.2) {\textbf{B}};
						\node [above] at (1.2,2.2) {\textbf{C}};
						\node [below] at (.3,1.8) {\textbf{D}};
						\node [below] at (1.2,1.8) {\textbf{E}};
						\node at (2,2.5) {\textbf{F}};
						\node at (2.5,5) {\underline{Sin política}};
					\end{tikzpicture}
				\end{subfigure}
				\begin{subfigure}[b]{0.49\textwidth}
					\begin{tikzpicture}[scale=1]
						\draw [fill,opacity=.3,purple] (0,0)--(3,3)--(0,3);
						\draw [fill,opacity=.3,purple] (0,4)--(1,3)--(0,3);
						\draw [ultra thick,teal] plot [domain=0:4] (\x,{4-\x});
						\node [above right,teal] at (4,0) {$D$};
						\draw [ultra thick,teal] plot [domain=0:4] (\x,{\x});
						\draw [thick,blue] (0,2)--(2,2)--(2,0);
						\draw [thick,purple] (0,3)--(1,3)--(1,0);
						\draw [thick,purple] (1,3)--(3,3)--(3,0);
						\draw [<->,ultra thick] (0,5)--(0,0)--(5,0);
						\node [right,teal] at (4,4) {$S$};
						\node [left] at (0,5) {$p$};
						\node [below] at (5,0) {$q$};
						\node [left] at (0,2) {\tiny $p^\peq{*}$};
						\node [left] at (0,3) {\tiny $p^\peq{m}$};
						\node [below] at (2,0) {\tiny $q^\peq{*}$};
						\node [below] at (3,0) {\tiny $q^\peq{s}\rp{p^\peq{m}}$};
						\node [below] at (1,0) {\tiny $q^\peq{d}\rp{p^\peq{m}}$};
						\node [below] at (0,0) {\tiny 0};
						\node [above] at (.3,3) {\textbf{A}};
						\node [above] at (.3,2.2) {\textbf{B}};
						\node [above] at (1.2,2.2) {\textbf{C}};
						\node [below] at (.3,1.8) {\textbf{D}};
						\node [below] at (1.2,1.8) {\textbf{E}};
						\node at (2,2.5) {\textbf{F}};
						\node at (2.5,5) {\underline{Con política}};
					\end{tikzpicture}
				\end{subfigure}
			\end{figure}
		\end{frame}

		\begin{frame}
			\frametitle{Comercio Internacional}
			Ganancia del comercio
			
			\vspace{.1cm}
			
			\centering
			\begin{tikzpicture}[scale=1]
				\draw [fill,opacity=.3,teal] (1,3)--(2,2)--(3,3);
				\draw [ultra thick,teal] plot [domain=0:4] (\x,{4-\x});
				\node [above right,teal] at (4,0) {$D$};
				\draw [ultra thick,teal] plot [domain=0:4] (\x,{\x});
				\draw [thick,blue] (0,2)--(2,2)--(2,0);
				\draw [thick,purple] (0,3)--(1,3)--(1,0);
				\draw [thick,purple] (1,3)--(3,3)--(3,0);
				\draw [<->,ultra thick] (0,5)--(0,0)--(5,0);
				\node [right,teal] at (4,4) {$S$};
				\node [left] at (0,5) {$p$};
				\node [below] at (5,0) {$q$};
				\node [left] at (0,2) {\tiny $p^\peq{*}$};
				\node [left] at (0,3) {\tiny $p^\peq{m}$};
				\node [below] at (2,0) {\tiny $q^\peq{*}$};
				\node [below] at (3,0) {\tiny $q^\peq{s}\rp{p^\peq{m}}$};
				\node [below] at (1,0) {\tiny $q^\peq{d}\rp{p^\peq{m}}$};
				\node [below] at (0,0) {\tiny 0};
				\node [above] at (.3,3) {\textbf{A}};
				\node [above] at (.3,2.2) {\textbf{B}};
				\node [above] at (1.2,2.2) {\textbf{C}};
				\node [below] at (.3,1.8) {\textbf{D}};
				\node [below] at (1.2,1.8) {\textbf{E}};
				\node at (2,2.5) {\textbf{F}};
			\end{tikzpicture}
		\end{frame}	

		\begin{frame}
			\frametitle{Comercio Internacional}
			\begin{table}[htbp!]
				\centering
				\resizebox{11cm}{!}{
					\begin{tabular}{l c c c}\hline
												&	Sin política	&	Con política	 &	Cambio					\\  \hline 
									 $EC$ &			$A+B+C$		&			$A$				 &	$-(B+C)$			\\
									 $EP$ &			$D+E$			&		$B+C+D+E+F$	 &	$+(B+C+F)$				\\ \hline
									 $ET$ &		$A+B+C+D+E$	&	$A+B+C+D+E+F$	 &	$+F$						\\ \hline
					\end{tabular}}
			\end{table}
		\end{frame}

		\begin{frame}
			\frametitle{Comercio Internacional}
			Caso 2: $p^\peq{m}<p^\peq{*}\implies$ el resto del mundo tiene ventaja comparativa
			\centering
			\begin{tikzpicture}[scale=1]
				\draw [ultra thick,teal] plot [domain=0:4] (\x,{4-\x});
				\node [above right,teal] at (4,0) {$D$};
				\draw [ultra thick,teal] plot [domain=0:4] (\x,{\x});
				\draw [thick,blue] (0,2)--(2,2)--(2,0);
				\draw [thick,purple] (0,1)--(3,1)--(3,0);
				\draw [thick,purple] (1,1)--(1,0);
				\draw [<->,ultra thick] (0,5)--(0,0)--(5,0);
				\node [right,teal] at (4,4) {$S$};
				\node [left] at (0,5) {$p$};
				\node [below] at (5,0) {$q$};
				\node [left] at (0,2) {\tiny $p^\peq{*}$};
				\node [left] at (0,1) {\tiny $p^\peq{m}$};
				\node [below] at (2,0) {\tiny $q^\peq{*}$};
				\node [below] at (3,0) {\tiny $q^\peq{d}\rp{p^\peq{m}}$};
				\node [below] at (1,0) {\tiny $q^\peq{s}\rp{p^\peq{m}}$};
				\node [below] at (0,0) {\tiny 0};
				\draw [decorate,decoration={brace,amplitude=6pt,mirror},xshift=0.2pt,yshift=-0.2pt,purple](1,-.4) -- (3,-.4) node[purple,midway,yshift=-0.5cm] {\tiny \pbox{\textwidth}{$M\rp{p^\peq{m}}$}};
			\end{tikzpicture}
		\end{frame}

		\begin{frame}
			\frametitle{Comercio Internacional}
			Excedente del consumidor
			\begin{figure}[hbtp!]
				\centering
				\begin{subfigure}[b]{0.49\textwidth}
					\begin{tikzpicture}[scale=1]
						\draw [fill,opacity=.3,blue] (0,4)--(2,2)--(0,2);
						\draw [ultra thick,teal] plot [domain=0:4] (\x,{4-\x});
						\node [above right,teal] at (4,0) {$D$};
						\draw [ultra thick,teal] plot [domain=0:4] (\x,{\x});
						\draw [thick,blue] (0,2)--(2,2)--(2,0);
						\draw [thick,purple] (0,1)--(3,1)--(3,0);
						\draw [thick,purple] (1,1)--(1,0);
						\draw [<->,ultra thick] (0,5)--(0,0)--(5,0);
						\node [right,teal] at (4,4) {$S$};
						\node [left] at (0,5) {$p$};
						\node [below] at (5,0) {$q$};
						\node [left] at (0,2) {\tiny $p^\peq{*}$};
						\node [left] at (0,1) {\tiny $p^\peq{m}$};
						\node [below] at (2,0) {\tiny $q^\peq{*}$};
						\node [below] at (3,0) {\tiny $q^\peq{d}\rp{p^\peq{m}}$};
						\node [below] at (1,0) {\tiny $q^\peq{s}\rp{p^\peq{m}}$};
						\node [below] at (0,0) {\tiny 0};
						\node [above] at (.8,2.2) {\textbf{A}};
						\node [below] at (.8,1.8) {\textbf{B}};
						\node at (.2,.6) {\textbf{C}};
						\node [below] at (1.7,1.6) {\textbf{D}};
						\node [below] at (2.3,1.6) {\textbf{E}};
						\node at (2.5,5) {\underline{Sin política}};
					\end{tikzpicture}
				\end{subfigure}
				\begin{subfigure}[b]{0.49\textwidth}
					\begin{tikzpicture}[scale=1]
						\draw [fill,opacity=.3,blue] (0,4)--(3,1)--(0,1);
						\draw [ultra thick,teal] plot [domain=0:4] (\x,{4-\x});
						\node [above right,teal] at (4,0) {$D$};
						\draw [ultra thick,teal] plot [domain=0:4] (\x,{\x});
						\draw [thick,blue] (0,2)--(2,2)--(2,0);
						\draw [thick,purple] (0,1)--(3,1)--(3,0);
						\draw [thick,purple] (1,1)--(1,0);
						\draw [<->,ultra thick] (0,5)--(0,0)--(5,0);
						\node [right,teal] at (4,4) {$S$};
						\node [left] at (0,5) {$p$};
						\node [below] at (5,0) {$q$};
						\node [left] at (0,2) {\tiny $p^\peq{*}$};
						\node [left] at (0,1) {\tiny $p^\peq{m}$};
						\node [below] at (2,0) {\tiny $q^\peq{*}$};
						\node [below] at (3,0) {\tiny $q^\peq{d}\rp{p^\peq{m}}$};
						\node [below] at (1,0) {\tiny $q^\peq{s}\rp{p^\peq{m}}$};
						\node [below] at (0,0) {\tiny 0};
						\node [above] at (.8,2.2) {\textbf{A}};
						\node [below] at (.8,1.8) {\textbf{B}};
						\node at (.2,.6) {\textbf{C}};
						\node [below] at (1.7,1.6) {\textbf{D}};
						\node [below] at (2.3,1.6) {\textbf{E}};
						\node at (2.5,5) {\underline{Con política}};
					\end{tikzpicture}
				\end{subfigure}
			\end{figure}
		\end{frame}

		\begin{frame}
			\frametitle{Comercio Internacional}
			Excedente del productor
			\begin{figure}[hbtp!]
				\centering
				\begin{subfigure}[b]{0.49\textwidth}
					\begin{tikzpicture}[scale=1]
						\draw [fill,opacity=.3,green] (0,0)--(2,2)--(0,2);
						\draw [ultra thick,teal] plot [domain=0:4] (\x,{4-\x});
						\node [above right,teal] at (4,0) {$D$};
						\draw [ultra thick,teal] plot [domain=0:4] (\x,{\x});
						\draw [thick,blue] (0,2)--(2,2)--(2,0);
						\draw [thick,purple] (0,1)--(3,1)--(3,0);
						\draw [thick,purple] (1,1)--(1,0);
						\draw [<->,ultra thick] (0,5)--(0,0)--(5,0);
						\node [right,teal] at (4,4) {$S$};
						\node [left] at (0,5) {$p$};
						\node [below] at (5,0) {$q$};
						\node [left] at (0,2) {\tiny $p^\peq{*}$};
						\node [left] at (0,1) {\tiny $p^\peq{m}$};
						\node [below] at (2,0) {\tiny $q^\peq{*}$};
						\node [below] at (3,0) {\tiny $q^\peq{d}\rp{p^\peq{m}}$};
						\node [below] at (1,0) {\tiny $q^\peq{s}\rp{p^\peq{m}}$};
						\node [below] at (0,0) {\tiny 0};
						\node [above] at (.8,2.2) {\textbf{A}};
						\node [below] at (.8,1.8) {\textbf{B}};
						\node at (.2,.6) {\textbf{C}};
						\node [below] at (1.7,1.6) {\textbf{D}};
						\node [below] at (2.3,1.6) {\textbf{E}};
						\node at (2.5,5) {\underline{Sin política}};
					\end{tikzpicture}
				\end{subfigure}
				\begin{subfigure}[b]{0.49\textwidth}
					\begin{tikzpicture}[scale=1]
						\draw [fill,opacity=.3,green] (0,0)--(1,1)--(0,1);
						\draw [ultra thick,teal] plot [domain=0:4] (\x,{4-\x});
						\node [above right,teal] at (4,0) {$D$};
						\draw [ultra thick,teal] plot [domain=0:4] (\x,{\x});
						\draw [thick,blue] (0,2)--(2,2)--(2,0);
						\draw [thick,purple] (0,1)--(3,1)--(3,0);
						\draw [thick,purple] (1,1)--(1,0);
						\draw [<->,ultra thick] (0,5)--(0,0)--(5,0);
						\node [right,teal] at (4,4) {$S$};
						\node [left] at (0,5) {$p$};
						\node [below] at (5,0) {$q$};
						\node [left] at (0,2) {\tiny $p^\peq{*}$};
						\node [left] at (0,1) {\tiny $p^\peq{m}$};
						\node [below] at (2,0) {\tiny $q^\peq{*}$};
						\node [below] at (3,0) {\tiny $q^\peq{d}\rp{p^\peq{m}}$};
						\node [below] at (1,0) {\tiny $q^\peq{s}\rp{p^\peq{m}}$};
						\node [below] at (0,0) {\tiny 0};
						\node [above] at (.8,2.2) {\textbf{A}};
						\node [below] at (.8,1.8) {\textbf{B}};
						\node at (.2,.6) {\textbf{C}};
						\node [below] at (1.7,1.6) {\textbf{D}};
						\node [below] at (2.3,1.6) {\textbf{E}};
						\node at (2.5,5) {\underline{Con política}};
					\end{tikzpicture}
				\end{subfigure}
			\end{figure}
		\end{frame}

		\begin{frame}
			\frametitle{Comercio Internacional}
			Excedente total
			\begin{figure}[hbtp!]
				\centering
				\begin{subfigure}[b]{0.49\textwidth}
					\begin{tikzpicture}[scale=1]
						\draw [fill,opacity=.3,purple] (0,0)--(2,2)--(0,4);
						\draw [ultra thick,teal] plot [domain=0:4] (\x,{4-\x});
						\node [above right,teal] at (4,0) {$D$};
						\draw [ultra thick,teal] plot [domain=0:4] (\x,{\x});
						\draw [thick,blue] (0,2)--(2,2)--(2,0);
						\draw [thick,purple] (0,1)--(3,1)--(3,0);
						\draw [thick,purple] (1,1)--(1,0);
						\draw [<->,ultra thick] (0,5)--(0,0)--(5,0);
						\node [right,teal] at (4,4) {$S$};
						\node [left] at (0,5) {$p$};
						\node [below] at (5,0) {$q$};
						\node [left] at (0,2) {\tiny $p^\peq{*}$};
						\node [left] at (0,1) {\tiny $p^\peq{m}$};
						\node [below] at (2,0) {\tiny $q^\peq{*}$};
						\node [below] at (3,0) {\tiny $q^\peq{d}\rp{p^\peq{m}}$};
						\node [below] at (1,0) {\tiny $q^\peq{s}\rp{p^\peq{m}}$};
						\node [below] at (0,0) {\tiny 0};
						\node [above] at (.8,2.2) {\textbf{A}};
						\node [below] at (.8,1.8) {\textbf{B}};
						\node at (.2,.6) {\textbf{C}};
						\node [below] at (1.7,1.6) {\textbf{D}};
						\node [below] at (2.3,1.6) {\textbf{E}};
						\node at (2.5,5) {\underline{Sin política}};
					\end{tikzpicture}
				\end{subfigure}
				\begin{subfigure}[b]{0.49\textwidth}
					\begin{tikzpicture}[scale=1]
						\draw [fill,opacity=.3,purple] (0,0)--(2,2)--(0,4);
						\draw [fill,opacity=.3,purple] (1,1)--(2,2)--(3,1);
						\draw [ultra thick,teal] plot [domain=0:4] (\x,{4-\x});
						\node [above right,teal] at (4,0) {$D$};
						\draw [ultra thick,teal] plot [domain=0:4] (\x,{\x});
						\draw [thick,blue] (0,2)--(2,2)--(2,0);
						\draw [thick,purple] (0,1)--(3,1)--(3,0);
						\draw [thick,purple] (1,1)--(1,0);
						\draw [<->,ultra thick] (0,5)--(0,0)--(5,0);
						\node [right,teal] at (4,4) {$S$};
						\node [left] at (0,5) {$p$};
						\node [below] at (5,0) {$q$};
						\node [left] at (0,2) {\tiny $p^\peq{*}$};
						\node [left] at (0,1) {\tiny $p^\peq{m}$};
						\node [below] at (2,0) {\tiny $q^\peq{*}$};
						\node [below] at (3,0) {\tiny $q^\peq{d}\rp{p^\peq{m}}$};
						\node [below] at (1,0) {\tiny $q^\peq{s}\rp{p^\peq{m}}$};
						\node [below] at (0,0) {\tiny 0};
						\node [above] at (.8,2.2) {\textbf{A}};
						\node [below] at (.8,1.8) {\textbf{B}};
						\node at (.2,.6) {\textbf{C}};
						\node [below] at (1.7,1.6) {\textbf{D}};
						\node [below] at (2.3,1.6) {\textbf{E}};
						\node at (2.5,5) {\underline{Con política}};
					\end{tikzpicture}
				\end{subfigure}
			\end{figure}
		\end{frame}

		\begin{frame}
			\frametitle{Comercio Internacional}
			Ganancia del comercio
			
			\vspace{.1cm}
			
			\centering
			\begin{tikzpicture}[scale=1]
				\draw [fill,opacity=.3,teal] (1,1)--(2,2)--(3,1);
				\draw [ultra thick,teal] plot [domain=0:4] (\x,{4-\x});
				\node [above right,teal] at (4,0) {$D$};
				\draw [ultra thick,teal] plot [domain=0:4] (\x,{\x});
				\draw [thick,blue] (0,2)--(2,2)--(2,0);
				\draw [thick,purple] (0,1)--(3,1)--(3,0);
				\draw [thick,purple] (1,1)--(1,0);
				\draw [<->,ultra thick] (0,5)--(0,0)--(5,0);
				\node [right,teal] at (4,4) {$S$};
				\node [left] at (0,5) {$p$};
				\node [below] at (5,0) {$q$};
				\node [left] at (0,2) {\tiny $p^\peq{*}$};
				\node [left] at (0,1) {\tiny $p^\peq{m}$};
				\node [below] at (2,0) {\tiny $q^\peq{*}$};
				\node [below] at (3,0) {\tiny $q^\peq{d}\rp{p^\peq{m}}$};
				\node [below] at (1,0) {\tiny $q^\peq{s}\rp{p^\peq{m}}$};
				\node [below] at (0,0) {\tiny 0};
				\node [above] at (.8,2.2) {\textbf{A}};
				\node [below] at (.8,1.8) {\textbf{B}};
				\node at (.2,.6) {\textbf{C}};
				\node [below] at (1.7,1.6) {\textbf{D}};
				\node [below] at (2.3,1.6) {\textbf{E}};
			\end{tikzpicture}
		\end{frame}	

		\begin{frame}
			\frametitle{Comercio Internacional}
			\begin{table}[htbp!]
				\centering
				\resizebox{11cm}{!}{
					\begin{tabular}{l c c c}\hline
												&	Sin política	&	Con política	 &	Cambio					\\  \hline 
									 $EC$ &				$A$			&		$A+B+D+E$		 &	$+(B+D+E)$			\\
									 $EP$ &			$B+C$			&				$C$			 &	$-(B)$				\\ \hline
									 $ET$ &			$A+B+C$		&		$A+B+C+D+E$	 &	$+(D+E)$						\\ \hline
					\end{tabular}}
			\end{table}
		\end{frame}

		\begin{frame}
			\frametitle{Comercio Internacional}
			Ahora estudiaremos políticas económicas en una economía pequeña y abierta:
				\begin{itemize}
					\item Aranceles
					\item Cuotas de importación
				\end{itemize}
		\end{frame}

		\begin{frame}
			\frametitle{Aranceles}
				\begin{mydef}
					\textbf{Arancel:} Impuesto sobre los bienes producidos en el extranjero y que se venden en el mercado nacional.
				\end{mydef}
				En otras palabras, un impuesto a las importaciones.
		\end{frame}

		\begin{frame}
			\frametitle{Aranceles}
			Efecto de un arancel de monto $\$t$ por unidad importada
			\centering
			\begin{tikzpicture}[scale=1]
				\draw [ultra thick,teal] plot [domain=0:4] (\x,{4-\x});
				\node [above right,teal] at (4,0) {$D$};
				\draw [ultra thick,teal] plot [domain=0:4] (\x,{\x});
				\draw [thick,purple] (0,1.5)--(2.5,1.5)--(2.5,0);
				\draw [thick,purple] (1.5,1.5)--(1.5,0);
				\draw [thick,blue] (0,1)--(3,1)--(3,0);
				\draw [thick,blue] (1,1)--(1,0);
				\draw [<->,ultra thick] (0,5)--(0,0)--(5,0);
				\node [right,teal] at (4,4) {$S$};
				\node [left] at (0,5) {$p$};
				\node [below] at (5,0) {$q$};
				\node [left] at (0,1) {\tiny $p^\peq{m}$};
				\node [left] at (0,1.5) {\tiny $p^\peq{m}+t$};
				\node [below] at (3,0) {\tiny $q^\peq{d}_\peq{0}$};
				\node [below] at (1,0) {\tiny $q^\peq{s}_\peq{0}$};
				\node [below] at (2.5,0) {\tiny $q^\peq{d}_\peq{1}$};
				\node [below] at (1.5,0) {\tiny $q^\peq{s}_\peq{1}$};
				\node [below] at (0,0) {\tiny 0};
				\draw [decorate,decoration={brace,amplitude=6pt,mirror},xshift=0.2pt,yshift=-0.2pt,blue](1,-.65) -- (3,-.65) node[blue,midway,yshift=-0.4cm] {\tiny \pbox{\textwidth}{$M\rp{p^\peq{m}}$}};
				\draw [decorate,decoration={brace,amplitude=2pt,mirror},xshift=0.2pt,yshift=-0.2pt,purple](1.5,-.4) -- (2.5,-.4) node[purple,midway,yshift=-0.2cm] {\tiny \pbox{\textwidth}{$M\rp{p^\peq{m}+t}$}};
				\draw [decorate,decoration={brace,amplitude=2pt,mirror},xshift=0.2pt,yshift=-0.2pt,purple](-.85,1.5) -- (-.85,1) node[purple,midway,xshift=-0.2cm] {\tiny t};
			\end{tikzpicture}
		\end{frame}

		\begin{frame}
			\frametitle{Aranceles}
			Excedente del consumidor
			\begin{figure}[hbtp!]
				\centering
				\begin{subfigure}[b]{0.49\textwidth}
					\begin{tikzpicture}[scale=1]
						\draw [fill,opacity=.3,blue] (0,4)--(3,1)--(0,1);
						\draw [ultra thick,teal] plot [domain=0:4] (\x,{4-\x});
						\node [above right,teal] at (4,0) {$D$};
						\draw [ultra thick,teal] plot [domain=0:4] (\x,{\x});
						\draw [thick,purple] (0,1.5)--(2.5,1.5)--(2.5,0);
						\draw [thick,purple] (1.5,1.5)--(1.5,0);
						\draw [thick,blue] (0,1)--(3,1)--(3,0);
						\draw [thick,blue] (1,1)--(1,0);
						\draw [<->,ultra thick] (0,5)--(0,0)--(5,0);
						\node [right,teal] at (4,4) {$S$};
						\node [left] at (0,5) {$p$};
						\node [below] at (5,0) {$q$};
						\node [left] at (0,1) {\tiny $p^\peq{m}$};
						\node [left] at (0,1.5) {\tiny $p^\peq{m}+t$};
						\node [below] at (3,0) {\tiny $q^\peq{d}_\peq{0}$};
						\node [below] at (1,0) {\tiny $q^\peq{s}_\peq{0}$};
						\node [below] at (2.5,0) {\tiny $q^\peq{d}_\peq{1}$};
						\node [below] at (1.5,0) {\tiny $q^\peq{s}_\peq{1}$};
						\node [below] at (0,0) {\tiny 0};
						\node [above] at (.8,1.82) {\scriptsize \textbf{A}};
						\node [below] at (2,1.95) {\scriptsize \textbf{B}};
						\node at (.6,1.2) {\scriptsize \textbf{C}};
						\node at (1.35,1.15) {\scriptsize \textbf{D}};
						\node at (2,1.15) {\scriptsize \textbf{E}};
						\node at (2.65,1.15) {\scriptsize \textbf{F}};
						\node at (.2,.6) {\scriptsize \textbf{G}};
						\node at (2.5,5) {\underline{Sin política}};
					\end{tikzpicture}
				\end{subfigure}
				\begin{subfigure}[b]{0.49\textwidth}
					\begin{tikzpicture}[scale=1]
						\draw [fill,opacity=.3,blue] (0,4)--(2.5,1.5)--(0,1.5);
						\draw [ultra thick,teal] plot [domain=0:4] (\x,{4-\x});
						\node [above right,teal] at (4,0) {$D$};
						\draw [ultra thick,teal] plot [domain=0:4] (\x,{\x});
						\draw [thick,purple] (0,1.5)--(2.5,1.5)--(2.5,0);
						\draw [thick,purple] (1.5,1.5)--(1.5,0);
						\draw [thick,blue] (0,1)--(3,1)--(3,0);
						\draw [thick,blue] (1,1)--(1,0);
						\draw [<->,ultra thick] (0,5)--(0,0)--(5,0);
						\node [right,teal] at (4,4) {$S$};
						\node [left] at (0,5) {$p$};
						\node [below] at (5,0) {$q$};
						\node [left] at (0,1) {\tiny $p^\peq{m}$};
						\node [left] at (0,1.5) {\tiny $p^\peq{m}+t$};
						\node [below] at (3,0) {\tiny $q^\peq{d}_\peq{0}$};
						\node [below] at (1,0) {\tiny $q^\peq{s}_\peq{0}$};
						\node [below] at (2.5,0) {\tiny $q^\peq{d}_\peq{1}$};
						\node [below] at (1.5,0) {\tiny $q^\peq{s}_\peq{1}$};
						\node [below] at (0,0) {\tiny 0};
						\node [above] at (.8,1.82) {\scriptsize \textbf{A}};
						\node [below] at (2,1.95) {\scriptsize \textbf{B}};
						\node at (.6,1.2) {\scriptsize \textbf{C}};
						\node at (1.35,1.15) {\scriptsize \textbf{D}};
						\node at (2,1.15) {\scriptsize \textbf{E}};
						\node at (2.65,1.15) {\scriptsize \textbf{F}};
						\node at (.2,.6) {\scriptsize \textbf{G}};
						\node at (2.5,5) {\underline{Con política}};
					\end{tikzpicture}
				\end{subfigure}
			\end{figure}
		\end{frame}

		\begin{frame}
			\frametitle{Aranceles}
			Excedente del productor
			\begin{figure}[hbtp!]
				\centering
				\begin{subfigure}[b]{0.49\textwidth}
					\begin{tikzpicture}[scale=1]
						\draw [fill,opacity=.3,green] (0,0)--(1,1)--(0,1);
						\draw [ultra thick,teal] plot [domain=0:4] (\x,{4-\x});
						\node [above right,teal] at (4,0) {$D$};
						\draw [ultra thick,teal] plot [domain=0:4] (\x,{\x});
						\draw [thick,purple] (0,1.5)--(2.5,1.5)--(2.5,0);
						\draw [thick,purple] (1.5,1.5)--(1.5,0);
						\draw [thick,blue] (0,1)--(3,1)--(3,0);
						\draw [thick,blue] (1,1)--(1,0);
						\draw [<->,ultra thick] (0,5)--(0,0)--(5,0);
						\node [right,teal] at (4,4) {$S$};
						\node [left] at (0,5) {$p$};
						\node [below] at (5,0) {$q$};
						\node [left] at (0,1) {\tiny $p^\peq{m}$};
						\node [left] at (0,1.5) {\tiny $p^\peq{m}+t$};
						\node [below] at (3,0) {\tiny $q^\peq{d}_\peq{0}$};
						\node [below] at (1,0) {\tiny $q^\peq{s}_\peq{0}$};
						\node [below] at (2.5,0) {\tiny $q^\peq{d}_\peq{1}$};
						\node [below] at (1.5,0) {\tiny $q^\peq{s}_\peq{1}$};
						\node [below] at (0,0) {\tiny 0};
						\node [above] at (.8,1.82) {\scriptsize \textbf{A}};
						\node [below] at (2,1.95) {\scriptsize \textbf{B}};
						\node at (.6,1.2) {\scriptsize \textbf{C}};
						\node at (1.35,1.15) {\scriptsize \textbf{D}};
						\node at (2,1.15) {\scriptsize \textbf{E}};
						\node at (2.65,1.15) {\scriptsize \textbf{F}};
						\node at (.2,.6) {\scriptsize \textbf{G}};
						\node at (2.5,5) {\underline{Sin política}};
					\end{tikzpicture}
				\end{subfigure}
				\begin{subfigure}[b]{0.49\textwidth}
					\begin{tikzpicture}[scale=1]
						\draw [fill,opacity=.3,green] (0,0)--(1.5,1.5)--(0,1.5);
						\draw [ultra thick,teal] plot [domain=0:4] (\x,{4-\x});
						\node [above right,teal] at (4,0) {$D$};
						\draw [ultra thick,teal] plot [domain=0:4] (\x,{\x});
						\draw [thick,purple] (0,1.5)--(2.5,1.5)--(2.5,0);
						\draw [thick,purple] (1.5,1.5)--(1.5,0);
						\draw [thick,blue] (0,1)--(3,1)--(3,0);
						\draw [thick,blue] (1,1)--(1,0);
						\draw [<->,ultra thick] (0,5)--(0,0)--(5,0);
						\node [right,teal] at (4,4) {$S$};
						\node [left] at (0,5) {$p$};
						\node [below] at (5,0) {$q$};
						\node [left] at (0,1) {\tiny $p^\peq{m}$};
						\node [left] at (0,1.5) {\tiny $p^\peq{m}+t$};
						\node [below] at (3,0) {\tiny $q^\peq{d}_\peq{0}$};
						\node [below] at (1,0) {\tiny $q^\peq{s}_\peq{0}$};
						\node [below] at (2.5,0) {\tiny $q^\peq{d}_\peq{1}$};
						\node [below] at (1.5,0) {\tiny $q^\peq{s}_\peq{1}$};
						\node [below] at (0,0) {\tiny 0};
						\node [above] at (.8,1.82) {\scriptsize \textbf{A}};
						\node [below] at (2,1.95) {\scriptsize \textbf{B}};
						\node at (.6,1.2) {\scriptsize \textbf{C}};
						\node at (1.35,1.15) {\scriptsize \textbf{D}};
						\node at (2,1.15) {\scriptsize \textbf{E}};
						\node at (2.65,1.15) {\scriptsize \textbf{F}};
						\node at (.2,.6) {\scriptsize \textbf{G}};
						\node at (2.5,5) {\underline{Con política}};
					\end{tikzpicture}
				\end{subfigure}
			\end{figure}
		\end{frame}

		\begin{frame}
			\frametitle{Aranceles}
			Recaudación fiscal
			\begin{figure}[hbtp!]
				\centering
				\begin{subfigure}[b]{0.49\textwidth}
					\begin{tikzpicture}[scale=1]
						\draw [ultra thick,teal] plot [domain=0:4] (\x,{4-\x});
						\node [above right,teal] at (4,0) {$D$};
						\draw [ultra thick,teal] plot [domain=0:4] (\x,{\x});
						\draw [thick,purple] (0,1.5)--(2.5,1.5)--(2.5,0);
						\draw [thick,purple] (1.5,1.5)--(1.5,0);
						\draw [thick,blue] (0,1)--(3,1)--(3,0);
						\draw [thick,blue] (1,1)--(1,0);
						\draw [<->,ultra thick] (0,5)--(0,0)--(5,0);
						\node [right,teal] at (4,4) {$S$};
						\node [left] at (0,5) {$p$};
						\node [below] at (5,0) {$q$};
						\node [left] at (0,1) {\tiny $p^\peq{m}$};
						\node [left] at (0,1.5) {\tiny $p^\peq{m}+t$};
						\node [below] at (3,0) {\tiny $q^\peq{d}_\peq{0}$};
						\node [below] at (1,0) {\tiny $q^\peq{s}_\peq{0}$};
						\node [below] at (2.5,0) {\tiny $q^\peq{d}_\peq{1}$};
						\node [below] at (1.5,0) {\tiny $q^\peq{s}_\peq{1}$};
						\node [below] at (0,0) {\tiny 0};
						\node [above] at (.8,1.82) {\scriptsize \textbf{A}};
						\node [below] at (2,1.95) {\scriptsize \textbf{B}};
						\node at (.6,1.2) {\scriptsize \textbf{C}};
						\node at (1.35,1.15) {\scriptsize \textbf{D}};
						\node at (2,1.15) {\scriptsize \textbf{E}};
						\node at (2.65,1.15) {\scriptsize \textbf{F}};
						\node at (.2,.6) {\scriptsize \textbf{G}};
						\node at (2.5,5) {\underline{Sin política}};
					\end{tikzpicture}
				\end{subfigure}
				\begin{subfigure}[b]{0.49\textwidth}
					\begin{tikzpicture}[scale=1]
						\draw [fill,opacity=.4,yellow] (1.5,1)--(1.5,1.5)--(2.5,1.5)--(2.5,1);
						\draw [ultra thick,teal] plot [domain=0:4] (\x,{4-\x});
						\node [above right,teal] at (4,0) {$D$};
						\draw [ultra thick,teal] plot [domain=0:4] (\x,{\x});
						\draw [thick,purple] (0,1.5)--(2.5,1.5)--(2.5,0);
						\draw [thick,purple] (1.5,1.5)--(1.5,0);
						\draw [thick,blue] (0,1)--(3,1)--(3,0);
						\draw [thick,blue] (1,1)--(1,0);
						\draw [<->,ultra thick] (0,5)--(0,0)--(5,0);
						\node [right,teal] at (4,4) {$S$};
						\node [left] at (0,5) {$p$};
						\node [below] at (5,0) {$q$};
						\node [left] at (0,1) {\tiny $p^\peq{m}$};
						\node [left] at (0,1.5) {\tiny $p^\peq{m}+t$};
						\node [below] at (3,0) {\tiny $q^\peq{d}_\peq{0}$};
						\node [below] at (1,0) {\tiny $q^\peq{s}_\peq{0}$};
						\node [below] at (2.5,0) {\tiny $q^\peq{d}_\peq{1}$};
						\node [below] at (1.5,0) {\tiny $q^\peq{s}_\peq{1}$};
						\node [below] at (0,0) {\tiny 0};
						\node [above] at (.8,1.82) {\scriptsize \textbf{A}};
						\node [below] at (2,1.95) {\scriptsize \textbf{B}};
						\node at (.6,1.2) {\scriptsize \textbf{C}};
						\node at (1.35,1.15) {\scriptsize \textbf{D}};
						\node at (2,1.15) {\scriptsize \textbf{E}};
						\node at (2.65,1.15) {\scriptsize \textbf{F}};
						\node at (.2,.6) {\scriptsize \textbf{G}};
						\node at (2.5,5) {\underline{Con política}};
					\end{tikzpicture}
				\end{subfigure}
			\end{figure}
		\end{frame}

		\begin{frame}
			\frametitle{Aranceles}
			Excedente total
			\begin{figure}[hbtp!]
				\centering
				\begin{subfigure}[b]{0.49\textwidth}
					\begin{tikzpicture}[scale=1]
						\draw [fill,opacity=.3,purple] (0,0)--(2,2)--(0,4);
						\draw [fill,opacity=.3,purple] (1,1)--(2,2)--(3,1);
						\draw [ultra thick,teal] plot [domain=0:4] (\x,{4-\x});
						\node [above right,teal] at (4,0) {$D$};
						\draw [ultra thick,teal] plot [domain=0:4] (\x,{\x});
						\draw [thick,purple] (0,1.5)--(2.5,1.5)--(2.5,0);
						\draw [thick,purple] (1.5,1.5)--(1.5,0);
						\draw [thick,blue] (0,1)--(3,1)--(3,0);
						\draw [thick,blue] (1,1)--(1,0);
						\draw [<->,ultra thick] (0,5)--(0,0)--(5,0);
						\node [right,teal] at (4,4) {$S$};
						\node [left] at (0,5) {$p$};
						\node [below] at (5,0) {$q$};
						\node [left] at (0,1) {\tiny $p^\peq{m}$};
						\node [left] at (0,1.5) {\tiny $p^\peq{m}+t$};
						\node [below] at (3,0) {\tiny $q^\peq{d}_\peq{0}$};
						\node [below] at (1,0) {\tiny $q^\peq{s}_\peq{0}$};
						\node [below] at (2.5,0) {\tiny $q^\peq{d}_\peq{1}$};
						\node [below] at (1.5,0) {\tiny $q^\peq{s}_\peq{1}$};
						\node [below] at (0,0) {\tiny 0};
						\node [above] at (.8,1.82) {\scriptsize \textbf{A}};
						\node [below] at (2,1.95) {\scriptsize \textbf{B}};
						\node at (.6,1.2) {\scriptsize \textbf{C}};
						\node at (1.35,1.15) {\scriptsize \textbf{D}};
						\node at (2,1.15) {\scriptsize \textbf{E}};
						\node at (2.65,1.15) {\scriptsize \textbf{F}};
						\node at (.2,.6) {\scriptsize \textbf{G}};
						\node at (2.5,5) {\underline{Sin política}};
					\end{tikzpicture}
				\end{subfigure}
				\begin{subfigure}[b]{0.49\textwidth}
					\begin{tikzpicture}[scale=1]
						\draw [fill,opacity=.3,purple] (0,4)--(2.5,1.5)--(0,1.5);
						\draw [fill,opacity=.3,purple] (0,0)--(1.5,1.5)--(0,1.5);
						\draw [fill,opacity=.4,purple] (1.5,1)--(1.5,1.5)--(2.5,1.5)--(2.5,1);
						\draw [ultra thick,teal] plot [domain=0:4] (\x,{4-\x});
						\node [above right,teal] at (4,0) {$D$};
						\draw [ultra thick,teal] plot [domain=0:4] (\x,{\x});
						\draw [thick,purple] (0,1.5)--(2.5,1.5)--(2.5,0);
						\draw [thick,purple] (1.5,1.5)--(1.5,0);
						\draw [thick,blue] (0,1)--(3,1)--(3,0);
						\draw [thick,blue] (1,1)--(1,0);
						\draw [<->,ultra thick] (0,5)--(0,0)--(5,0);
						\node [right,teal] at (4,4) {$S$};
						\node [left] at (0,5) {$p$};
						\node [below] at (5,0) {$q$};
						\node [left] at (0,1) {\tiny $p^\peq{m}$};
						\node [left] at (0,1.5) {\tiny $p^\peq{m}+t$};
						\node [below] at (3,0) {\tiny $q^\peq{d}_\peq{0}$};
						\node [below] at (1,0) {\tiny $q^\peq{s}_\peq{0}$};
						\node [below] at (2.5,0) {\tiny $q^\peq{d}_\peq{1}$};
						\node [below] at (1.5,0) {\tiny $q^\peq{s}_\peq{1}$};
						\node [below] at (0,0) {\tiny 0};
						\node [above] at (.8,1.82) {\scriptsize \textbf{A}};
						\node [below] at (2,1.95) {\scriptsize \textbf{B}};
						\node at (.6,1.2) {\scriptsize \textbf{C}};
						\node at (1.35,1.15) {\scriptsize \textbf{D}};
						\node at (2,1.15) {\scriptsize \textbf{E}};
						\node at (2.65,1.15) {\scriptsize \textbf{F}};
						\node at (.2,.6) {\scriptsize \textbf{G}};
						\node at (2.5,5) {\underline{Con política}};
					\end{tikzpicture}
				\end{subfigure}
			\end{figure}
		\end{frame}

		\begin{frame}
			\frametitle{Aranceles}
			Pérdida social
			
			\vspace{.1cm}
			
			\centering
			\begin{tikzpicture}[scale=1]
				\draw [fill,opacity=.3,black] (1,1)--(1.5,1.5)--(1.5,1);
				\draw [fill,opacity=.3,black] (3,1)--(2.5,1.5)--(2.5,1);
				\draw [ultra thick,teal] plot [domain=0:4] (\x,{4-\x});
				\node [above right,teal] at (4,0) {$D$};
				\draw [ultra thick,teal] plot [domain=0:4] (\x,{\x});
				\draw [thick,purple] (0,1.5)--(2.5,1.5)--(2.5,0);
				\draw [thick,purple] (1.5,1.5)--(1.5,0);
				\draw [thick,blue] (0,1)--(3,1)--(3,0);
				\draw [thick,blue] (1,1)--(1,0);
				\draw [<->,ultra thick] (0,5)--(0,0)--(5,0);
				\node [right,teal] at (4,4) {$S$};
				\node [left] at (0,5) {$p$};
				\node [below] at (5,0) {$q$};
				\node [left] at (0,1) {\tiny $p^\peq{m}$};
				\node [left] at (0,1.5) {\tiny $p^\peq{m}+t$};
				\node [below] at (3,0) {\tiny $q^\peq{d}_\peq{0}$};
				\node [below] at (1,0) {\tiny $q^\peq{s}_\peq{0}$};
				\node [below] at (2.5,0) {\tiny $q^\peq{d}_\peq{1}$};
				\node [below] at (1.5,0) {\tiny $q^\peq{s}_\peq{1}$};
				\node [below] at (0,0) {\tiny 0};
				\node [above] at (.8,1.82) {\scriptsize \textbf{A}};
				\node [below] at (2,1.95) {\scriptsize \textbf{B}};
				\node at (.6,1.2) {\scriptsize \textbf{C}};
				\node at (1.35,1.15) {\scriptsize \textbf{D}};
				\node at (2,1.15) {\scriptsize \textbf{E}};
				\node at (2.65,1.15) {\scriptsize \textbf{F}};
				\node at (.2,.6) {\scriptsize \textbf{G}};
			\end{tikzpicture}
		\end{frame}	

		\begin{frame}
			\frametitle{Aranceles}
			\begin{table}[htbp!]
				\centering
				\resizebox{11cm}{!}{
					\begin{tabular}{l c c c}\hline
												&		Sin política	&	Con política	 &	Cambio					\\  \hline 
									 $EC$ &	$A+B+C+D+E+F$		&			$A+B$			 &	$-(C+D+E+F)$			\\
									 $EP$ &				$G$				&			$C+G$			 &	$+(C)$				\\ 
									 $RF$ &									&			$E$				 &	$+(E)$				\\ \hline
									 $ET$ &	$A+B+C+D+E+F+G$	&		$A+B+C+E+G$	 &	$-(D+F)$						\\ \hline
					\end{tabular}}
			\end{table}
		\end{frame}

		\begin{frame}
			\frametitle{Aranceles}
			Si el arancel es tan alto que $p^\peq{m}+t\geq p^\peq{*}$ la economía se cierra, no se vuelve exportadora...
			
			\centering
			\begin{tikzpicture}[scale=1]
				\draw [ultra thick,teal] plot [domain=0:4] (\x,{4-\x});
				\node [above right,teal] at (4,0) {$D$};
				\draw [ultra thick,teal] plot [domain=0:4] (\x,{\x});
				\draw [thick,purple] (0,3)--(5,3);
				\draw [thick,blue] (0,1)--(5,1);
				\draw [<->,ultra thick] (0,5)--(0,0)--(5,0);
				\node [right,teal] at (4,4) {$S$};
				\node [left] at (0,5) {$p$};
				\node [below] at (5,0) {$q$};
				\node [left] at (0,1) {\tiny $p^\peq{m}$};
				\node [left] at (0,3) {\tiny $p^\peq{m}+t$};
				\node [below] at (0,0) {\tiny 0};
				\draw [decorate,decoration={brace,amplitude=2pt,mirror},xshift=0.2pt,yshift=-0.2pt,purple](-.85,3) -- (-.85,1) node[purple,midway,xshift=-0.2cm] {\tiny t};
			\end{tikzpicture}
		\end{frame}

		\begin{frame}
			\frametitle{Aranceles}
			y la pérdida social es toda la ganancia de la apertura comercial...
			
			\centering
			\begin{tikzpicture}[scale=1]
				\draw [fill,opacity=.3,black] (1,1)--(2,2)--(3,1);
				\draw [ultra thick,teal] plot [domain=0:4] (\x,{4-\x});
				\node [above right,teal] at (4,0) {$D$};
				\draw [ultra thick,teal] plot [domain=0:4] (\x,{\x});
				\draw [thick,purple] (0,3)--(5,3);
				\draw [thick,blue] (0,1)--(5,1);
				\draw [<->,ultra thick] (0,5)--(0,0)--(5,0);
				\node [right,teal] at (4,4) {$S$};
				\node [left] at (0,5) {$p$};
				\node [below] at (5,0) {$q$};
				\node [left] at (0,1) {\tiny $p^\peq{m}$};
				\node [left] at (0,3) {\tiny $p^\peq{m}+t$};
				\node [below] at (0,0) {\tiny 0};
				\draw [decorate,decoration={brace,amplitude=2pt,mirror},xshift=0.2pt,yshift=-0.2pt,purple](-.85,3) -- (-.85,1) node[purple,midway,xshift=-0.2cm] {\tiny t};
			\end{tikzpicture}
		\end{frame}

		\begin{frame}
			\frametitle{Cuotas de Importación}
				\begin{mydef}
					\textbf{Cuota de importación ($\mathbf{\overline{M}}$):} Límite a la cantidad que se puede importar. $$M\rp{p}\leq\overline{M}$$
				\end{mydef}
		\end{frame}

		\begin{frame}
			\frametitle{Cuotas de Importación}
			\begin{itemize}
				\item Si $\overline{M}\geq M\rp{p^\peq{m}}\implies$ la política no tiene efecto.
				\item Si $\overline{M}<M\rp{p^\peq{m}}\implies M\rp{p^\peq{c}}=\overline{M}$ donde $p^\peq{c}$ es el nuevo precio de equilibrio.
			\end{itemize}
		\end{frame}

		\begin{frame}
			\frametitle{Cuotas de Importación}
			\centering
			\begin{tikzpicture}[scale=1]
				\draw [ultra thick,teal] plot [domain=0:4] (\x,{4-\x});
				\node [above right,teal] at (4,0) {$D$};
				\draw [ultra thick,teal] plot [domain=0:4] (\x,{\x});
				\draw [thick,blue] (0,1)--(3,1)--(3,0);
				\draw [thick,blue] (1,1)--(1,0);
				\draw [<->,ultra thick] (0,5)--(0,0)--(5,0);
				\node [right,teal] at (4,4) {$S$};
				\node [left] at (0,5) {$p$};
				\node [below] at (5,0) {$q$};
				\node [left] at (0,1) {\tiny $p^\peq{m}$};
				%\node [left] at (0,1.5) {\tiny $p^\peq{m}+t$};
				%\node [below] at (3,0) {\tiny $q^\peq{d}_\peq{0}$};
				%\node [below] at (1,0) {\tiny $q^\peq{s}_\peq{0}$};
				%\node [below] at (2.5,0) {\tiny $q^\peq{d}_\peq{1}$};
				%\node [below] at (1.5,0) {\tiny $q^\peq{s}_\peq{1}$};
				\node [below] at (0,0) {\tiny 0};
				\draw [thick,decorate,decoration={brace,amplitude=6pt,mirror},xshift=0.2pt,yshift=-0.2pt,blue](1,-.05) -- (3,-.05) node[blue,midway,yshift=-0.4cm] {\tiny \pbox{\textwidth}{$M\rp{p^\peq{m}}$}};
				%\draw [decorate,decoration={brace,amplitude=2pt,mirror},xshift=0.2pt,yshift=-0.2pt,purple](1.5,-.4) -- (2.5,-.4) node[purple,midway,yshift=-0.2cm] {\tiny \pbox{\textwidth}{$M\rp{p^\peq{m}+t}$}};
				%\draw [decorate,decoration={brace,amplitude=2pt,mirror},xshift=0.2pt,yshift=-0.2pt,purple](-.85,1.5) -- (-.85,1) node[purple,midway,xshift=-0.2cm] {\tiny t};
			\end{tikzpicture}
		\end{frame}

		\begin{frame}
			\frametitle{Cuotas de Importación}
			\centering
			\begin{tikzpicture}[scale=1]
				\draw [ultra thick,teal] plot [domain=0:4] (\x,{4-\x});
				\node [above right,teal] at (4,0) {$D$};
				\draw [ultra thick,teal] plot [domain=0:4] (\x,{\x});
				\draw [thick,blue] (0,1)--(3,1)--(3,0);
				\draw [thick,blue] (1,1)--(1,0);
				\draw [ultra thick,dashed,purple] (2,0)--(2,5);
				\draw [<->,ultra thick] (0,5)--(0,0)--(5,0);
				\node [right,teal] at (4,4) {$S$};
				\node [left] at (0,5) {$p$};
				\node [below] at (5,0) {$q$};
				\node [left] at (0,1) {\tiny $p^\peq{m}$};
				%\node [left] at (0,1.5) {\tiny $p^\peq{m}+t$};
				%\node [below] at (3,0) {\tiny $q^\peq{d}_\peq{0}$};
				%\node [below] at (1,0) {\tiny $q^\peq{s}_\peq{0}$};
				%\node [below] at (2.5,0) {\tiny $q^\peq{d}_\peq{1}$};
				%\node [below] at (1.5,0) {\tiny $q^\peq{s}_\peq{1}$};
				\node [below] at (0,0) {\tiny 0};
				\draw [thick,decorate,decoration={brace,amplitude=6pt,mirror},xshift=0.2pt,yshift=-0.2pt,blue](1,-.05) -- (3,-.05) node[blue,midway,yshift=-0.4cm] {\tiny \pbox{\textwidth}{$M\rp{p^\peq{m}}$}};
				\draw [thick,decorate,decoration={brace,amplitude=2pt,mirror},xshift=0.2pt,yshift=-0.2pt,purple](1,.95) -- (2,.95) node[purple,midway,yshift=-0.2cm] {\tiny \pbox{\textwidth}{$\overline{M}$}};
				\draw [thick,decorate,decoration={brace,amplitude=2pt,mirror},xshift=0.2pt,yshift=-0.2pt,purple](2,.95) -- (3,.95) node[purple,midway,yshift=-0.3cm,xshift=0.2cm] {\tiny \pbox{\textwidth}{Exceso\\de demanda}};
			\end{tikzpicture}
		\end{frame}

		\begin{frame}
			\frametitle{Cuotas de Importación}
			\centering
			\begin{tikzpicture}[scale=1]
				\draw [ultra thick,teal] plot [domain=0:4] (\x,{4-\x});
				\node [above right,teal] at (4,0) {$D$};
				\draw [ultra thick,teal] plot [domain=0:4] (\x,{\x});
				\draw [thick,blue] (0,1)--(3,1)--(3,0);
				\draw [thick,blue] (1,1)--(1,0);
				\draw [ultra thick,brown] (0,1)--(2,1)--plot [domain=2:4] (\x,{\x-1});
				\draw [<->,ultra thick] (0,5)--(0,0)--(5,0);
				\node [right,teal] at (4,4) {$S$};
				\node [right,brown] at (4,3) {$S^\peq{c}$};
				\node [left] at (0,5) {$p$};
				\node [below] at (5,0) {$q$};
				\node [left] at (0,1) {\tiny $p^\peq{m}$};
				%\node [left] at (0,1.5) {\tiny $p^\peq{m}+t$};
				%\node [below] at (3,0) {\tiny $q^\peq{d}_\peq{0}$};
				%\node [below] at (1,0) {\tiny $q^\peq{s}_\peq{0}$};
				%\node [below] at (2.5,0) {\tiny $q^\peq{d}_\peq{1}$};
				%\node [below] at (1.5,0) {\tiny $q^\peq{s}_\peq{1}$};
				\node [below] at (0,0) {\tiny 0};
				\draw [thick,decorate,decoration={brace,amplitude=6pt,mirror},xshift=0.2pt,yshift=-0.2pt,blue](1,-.05) -- (3,-.05) node[blue,midway,yshift=-0.4cm] {\tiny \pbox{\textwidth}{$M\rp{p^\peq{m}}$}};
				\draw [thick,decorate,decoration={brace,amplitude=2pt,mirror},xshift=0.2pt,yshift=-0.2pt,purple](1,.95) -- (2,.95) node[purple,midway,yshift=-0.2cm] {\tiny \pbox{\textwidth}{$\overline{M}$}};
				%\draw [thick,decorate,decoration={brace,amplitude=2pt,mirror},xshift=0.2pt,yshift=-0.2pt,purple](1,.5) -- (3.5,.5) node[purple,midway,yshift=-0.2cm] {\tiny \pbox{\textwidth}{Exceso de demanda}};
			\end{tikzpicture}
		\end{frame}

		\begin{frame}
			\frametitle{Cuotas de Importación}
			\centering
			\begin{tikzpicture}[scale=1]
				\draw [ultra thick,teal] plot [domain=0:4] (\x,{4-\x});
				\node [above right,teal] at (4,0) {$D$};
				\draw [ultra thick,teal] plot [domain=0:4] (\x,{\x});
				\draw [thick,blue] (0,1)--(3,1)--(3,0);
				\draw [thick,blue] (1,1)--(1,0);
				\draw [ultra thick,brown] (0,1)--(2,1)--plot [domain=2:4] (\x,{\x-1});
				\draw [thick,purple] (0,1.5)--(2.5,1.5)--(2.5,0);
				\draw [thick,purple] (1.5,1.5)--(1.5,0);
				\draw [<->,ultra thick] (0,5)--(0,0)--(5,0);
				\node [right,teal] at (4,4) {$S$};
				\node [right,brown] at (4,3) {$S^\peq{c}$};
				\node [left] at (0,5) {$p$};
				\node [below] at (5,0) {$q$};
				\node [left] at (0,1) {\tiny $p^\peq{m}$};
				\node [left] at (0,1.5) {\tiny $p^\peq{c}$};
				\node [below] at (0,0) {\tiny 0};
				\draw [thick,decorate,decoration={brace,amplitude=6pt,mirror},xshift=0.2pt,yshift=-0.2pt,blue](1,-.35) -- (3,-.35) node[blue,midway,yshift=-0.4cm] {\tiny \pbox{\textwidth}{$M\rp{p^\peq{m}}$}};
				\draw [thick,decorate,decoration={brace,amplitude=2pt,mirror},xshift=0.2pt,yshift=-0.2pt,purple](1.5,-.05) -- (2.5,-.05) node[purple,midway,yshift=-0.2cm] {\tiny \pbox{\textwidth}{$M\rp{p^\peq{c}}=\overline{M}$}};
			\end{tikzpicture}
		\end{frame}

		\begin{frame}
			\frametitle{Cuotas de Importación}
			Excedente del consumidor
			\begin{figure}[hbtp!]
				\centering
				\begin{subfigure}[b]{0.49\textwidth}
					\begin{tikzpicture}[scale=1]
						\draw [fill,opacity=.3,blue] (0,4)--(3,1)--(0,1);
						\draw [ultra thick,teal] plot [domain=0:4] (\x,{4-\x});
						\node [above right,teal] at (4,0) {$D$};
						\draw [ultra thick,teal] plot [domain=0:4] (\x,{\x});
						\draw [thick,purple] (0,1.5)--(2.5,1.5)--(2.5,0);
						\draw [thick,purple] (1.5,1.5)--(1.5,0);
						\draw [thick,blue] (0,1)--(3,1)--(3,0);
						\draw [thick,blue] (1,1)--(1,0);
						\draw [<->,ultra thick] (0,5)--(0,0)--(5,0);
						\node [right,teal] at (4,4) {$S$};
						\node [left] at (0,5) {$p$};
						\node [below] at (5,0) {$q$};
						\node [left] at (0,1) {\tiny $p^\peq{m}$};
						\node [left] at (0,1.5) {\tiny $p^\peq{c}$};
						\node [below] at (3,0) {\tiny $q^\peq{d}_\peq{0}$};
						\node [below] at (1,0) {\tiny $q^\peq{s}_\peq{0}$};
						\node [below] at (2.5,0) {\tiny $q^\peq{d}_\peq{1}$};
						\node [below] at (1.5,0) {\tiny $q^\peq{s}_\peq{1}$};
						\node [below] at (0,0) {\tiny 0};
						\node [above] at (.8,1.82) {\scriptsize \textbf{A}};
						\node [below] at (2,1.95) {\scriptsize \textbf{B}};
						\node at (.6,1.2) {\scriptsize \textbf{C}};
						\node at (1.35,1.15) {\scriptsize \textbf{D}};
						\node at (2,1.15) {\scriptsize \textbf{E}};
						\node at (2.65,1.15) {\scriptsize \textbf{F}};
						\node at (.2,.6) {\scriptsize \textbf{G}};
						\node at (2.5,5) {\underline{Sin política}};
					\end{tikzpicture}
				\end{subfigure}
				\begin{subfigure}[b]{0.49\textwidth}
					\begin{tikzpicture}[scale=1]
						\draw [fill,opacity=.3,blue] (0,4)--(2.5,1.5)--(0,1.5);
						\draw [ultra thick,teal] plot [domain=0:4] (\x,{4-\x});
						\node [above right,teal] at (4,0) {$D$};
						\draw [ultra thick,teal] plot [domain=0:4] (\x,{\x});
						\draw [thick,purple] (0,1.5)--(2.5,1.5)--(2.5,0);
						\draw [thick,purple] (1.5,1.5)--(1.5,0);
						\draw [thick,blue] (0,1)--(3,1)--(3,0);
						\draw [thick,blue] (1,1)--(1,0);
						\draw [<->,ultra thick] (0,5)--(0,0)--(5,0);
						\node [right,teal] at (4,4) {$S$};
						\node [left] at (0,5) {$p$};
						\node [below] at (5,0) {$q$};
						\node [left] at (0,1) {\tiny $p^\peq{m}$};
						\node [left] at (0,1.5) {\tiny $p^\peq{c}$};
						\node [below] at (3,0) {\tiny $q^\peq{d}_\peq{0}$};
						\node [below] at (1,0) {\tiny $q^\peq{s}_\peq{0}$};
						\node [below] at (2.5,0) {\tiny $q^\peq{d}_\peq{1}$};
						\node [below] at (1.5,0) {\tiny $q^\peq{s}_\peq{1}$};
						\node [below] at (0,0) {\tiny 0};
						\node [above] at (.8,1.82) {\scriptsize \textbf{A}};
						\node [below] at (2,1.95) {\scriptsize \textbf{B}};
						\node at (.6,1.2) {\scriptsize \textbf{C}};
						\node at (1.35,1.15) {\scriptsize \textbf{D}};
						\node at (2,1.15) {\scriptsize \textbf{E}};
						\node at (2.65,1.15) {\scriptsize \textbf{F}};
						\node at (.2,.6) {\scriptsize \textbf{G}};
						\node at (2.5,5) {\underline{Con política}};
					\end{tikzpicture}
				\end{subfigure}
			\end{figure}
		\end{frame}

		\begin{frame}
			\frametitle{Cuotas de Importación}
			Excedente del productor
			\begin{figure}[hbtp!]
				\centering
				\begin{subfigure}[b]{0.49\textwidth}
					\begin{tikzpicture}[scale=1]
						\draw [fill,opacity=.3,green] (0,0)--(1,1)--(0,1);
						\draw [ultra thick,teal] plot [domain=0:4] (\x,{4-\x});
						\node [above right,teal] at (4,0) {$D$};
						\draw [ultra thick,teal] plot [domain=0:4] (\x,{\x});
						\draw [thick,purple] (0,1.5)--(2.5,1.5)--(2.5,0);
						\draw [thick,purple] (1.5,1.5)--(1.5,0);
						\draw [thick,blue] (0,1)--(3,1)--(3,0);
						\draw [thick,blue] (1,1)--(1,0);
						\draw [<->,ultra thick] (0,5)--(0,0)--(5,0);
						\node [right,teal] at (4,4) {$S$};
						\node [left] at (0,5) {$p$};
						\node [below] at (5,0) {$q$};
						\node [left] at (0,1) {\tiny $p^\peq{m}$};
						\node [left] at (0,1.5) {\tiny $p^\peq{c}$};
						\node [below] at (3,0) {\tiny $q^\peq{d}_\peq{0}$};
						\node [below] at (1,0) {\tiny $q^\peq{s}_\peq{0}$};
						\node [below] at (2.5,0) {\tiny $q^\peq{d}_\peq{1}$};
						\node [below] at (1.5,0) {\tiny $q^\peq{s}_\peq{1}$};
						\node [below] at (0,0) {\tiny 0};
						\node [above] at (.8,1.82) {\scriptsize \textbf{A}};
						\node [below] at (2,1.95) {\scriptsize \textbf{B}};
						\node at (.6,1.2) {\scriptsize \textbf{C}};
						\node at (1.35,1.15) {\scriptsize \textbf{D}};
						\node at (2,1.15) {\scriptsize \textbf{E}};
						\node at (2.65,1.15) {\scriptsize \textbf{F}};
						\node at (.2,.6) {\scriptsize \textbf{G}};
						\node at (2.5,5) {\underline{Sin política}};
					\end{tikzpicture}
				\end{subfigure}
				\begin{subfigure}[b]{0.49\textwidth}
					\begin{tikzpicture}[scale=1]
						\draw [fill,opacity=.3,green] (0,0)--(1.5,1.5)--(0,1.5);
						\draw [ultra thick,teal] plot [domain=0:4] (\x,{4-\x});
						\node [above right,teal] at (4,0) {$D$};
						\draw [ultra thick,teal] plot [domain=0:4] (\x,{\x});
						\draw [thick,purple] (0,1.5)--(2.5,1.5)--(2.5,0);
						\draw [thick,purple] (1.5,1.5)--(1.5,0);
						\draw [thick,blue] (0,1)--(3,1)--(3,0);
						\draw [thick,blue] (1,1)--(1,0);
						\draw [<->,ultra thick] (0,5)--(0,0)--(5,0);
						\node [right,teal] at (4,4) {$S$};
						\node [left] at (0,5) {$p$};
						\node [below] at (5,0) {$q$};
						\node [left] at (0,1) {\tiny $p^\peq{m}$};
						\node [left] at (0,1.5) {\tiny $p^\peq{c}$};
						\node [below] at (3,0) {\tiny $q^\peq{d}_\peq{0}$};
						\node [below] at (1,0) {\tiny $q^\peq{s}_\peq{0}$};
						\node [below] at (2.5,0) {\tiny $q^\peq{d}_\peq{1}$};
						\node [below] at (1.5,0) {\tiny $q^\peq{s}_\peq{1}$};
						\node [below] at (0,0) {\tiny 0};
						\node [above] at (.8,1.82) {\scriptsize \textbf{A}};
						\node [below] at (2,1.95) {\scriptsize \textbf{B}};
						\node at (.6,1.2) {\scriptsize \textbf{C}};
						\node at (1.35,1.15) {\scriptsize \textbf{D}};
						\node at (2,1.15) {\scriptsize \textbf{E}};
						\node at (2.65,1.15) {\scriptsize \textbf{F}};
						\node at (.2,.6) {\scriptsize \textbf{G}};
						\node at (2.5,5) {\underline{Con política}};
					\end{tikzpicture}
				\end{subfigure}
			\end{figure}
		\end{frame}

		\begin{frame}
			\frametitle{Cuotas de Importación}
			Excedente del importador y/o recaudación fiscal
			\begin{figure}[hbtp!]
				\centering
				\begin{subfigure}[b]{0.49\textwidth}
					\begin{tikzpicture}[scale=1]
						\draw [ultra thick,teal] plot [domain=0:4] (\x,{4-\x});
						\node [above right,teal] at (4,0) {$D$};
						\draw [ultra thick,teal] plot [domain=0:4] (\x,{\x});
						\draw [thick,purple] (0,1.5)--(2.5,1.5)--(2.5,0);
						\draw [thick,purple] (1.5,1.5)--(1.5,0);
						\draw [thick,blue] (0,1)--(3,1)--(3,0);
						\draw [thick,blue] (1,1)--(1,0);
						\draw [<->,ultra thick] (0,5)--(0,0)--(5,0);
						\node [right,teal] at (4,4) {$S$};
						\node [left] at (0,5) {$p$};
						\node [below] at (5,0) {$q$};
						\node [left] at (0,1) {\tiny $p^\peq{m}$};
						\node [left] at (0,1.5) {\tiny $p^\peq{c}$};
						\node [below] at (3,0) {\tiny $q^\peq{d}_\peq{0}$};
						\node [below] at (1,0) {\tiny $q^\peq{s}_\peq{0}$};
						\node [below] at (2.5,0) {\tiny $q^\peq{d}_\peq{1}$};
						\node [below] at (1.5,0) {\tiny $q^\peq{s}_\peq{1}$};
						\node [below] at (0,0) {\tiny 0};
						\node [above] at (.8,1.82) {\scriptsize \textbf{A}};
						\node [below] at (2,1.95) {\scriptsize \textbf{B}};
						\node at (.6,1.2) {\scriptsize \textbf{C}};
						\node at (1.35,1.15) {\scriptsize \textbf{D}};
						\node at (2,1.15) {\scriptsize \textbf{E}};
						\node at (2.65,1.15) {\scriptsize \textbf{F}};
						\node at (.2,.6) {\scriptsize \textbf{G}};
						\node at (2.5,5) {\underline{Sin política}};
					\end{tikzpicture}
				\end{subfigure}
				\begin{subfigure}[b]{0.49\textwidth}
					\begin{tikzpicture}[scale=1]
						\draw [fill,opacity=.4,yellow] (1.5,1)--(1.5,1.5)--(2.5,1.5)--(2.5,1);
						\draw [ultra thick,teal] plot [domain=0:4] (\x,{4-\x});
						\node [above right,teal] at (4,0) {$D$};
						\draw [ultra thick,teal] plot [domain=0:4] (\x,{\x});
						\draw [thick,purple] (0,1.5)--(2.5,1.5)--(2.5,0);
						\draw [thick,purple] (1.5,1.5)--(1.5,0);
						\draw [thick,blue] (0,1)--(3,1)--(3,0);
						\draw [thick,blue] (1,1)--(1,0);
						\draw [<->,ultra thick] (0,5)--(0,0)--(5,0);
						\node [right,teal] at (4,4) {$S$};
						\node [left] at (0,5) {$p$};
						\node [below] at (5,0) {$q$};
						\node [left] at (0,1) {\tiny $p^\peq{m}$};
						\node [left] at (0,1.5) {\tiny $p^\peq{c}$};
						\node [below] at (3,0) {\tiny $q^\peq{d}_\peq{0}$};
						\node [below] at (1,0) {\tiny $q^\peq{s}_\peq{0}$};
						\node [below] at (2.5,0) {\tiny $q^\peq{d}_\peq{1}$};
						\node [below] at (1.5,0) {\tiny $q^\peq{s}_\peq{1}$};
						\node [below] at (0,0) {\tiny 0};
						\node [above] at (.8,1.82) {\scriptsize \textbf{A}};
						\node [below] at (2,1.95) {\scriptsize \textbf{B}};
						\node at (.6,1.2) {\scriptsize \textbf{C}};
						\node at (1.35,1.15) {\scriptsize \textbf{D}};
						\node at (2,1.15) {\scriptsize \textbf{E}};
						\node at (2.65,1.15) {\scriptsize \textbf{F}};
						\node at (.2,.6) {\scriptsize \textbf{G}};
						\node at (2.5,5) {\underline{Con política}};
					\end{tikzpicture}
				\end{subfigure}
			\end{figure}
		\end{frame}

		\begin{frame}
			\frametitle{Cuotas de Importación}
			Excedente total
			\begin{figure}[hbtp!]
				\centering
				\begin{subfigure}[b]{0.49\textwidth}
					\begin{tikzpicture}[scale=1]
						\draw [fill,opacity=.3,purple] (0,0)--(2,2)--(0,4);
						\draw [fill,opacity=.3,purple] (1,1)--(2,2)--(3,1);
						\draw [ultra thick,teal] plot [domain=0:4] (\x,{4-\x});
						\node [above right,teal] at (4,0) {$D$};
						\draw [ultra thick,teal] plot [domain=0:4] (\x,{\x});
						\draw [thick,purple] (0,1.5)--(2.5,1.5)--(2.5,0);
						\draw [thick,purple] (1.5,1.5)--(1.5,0);
						\draw [thick,blue] (0,1)--(3,1)--(3,0);
						\draw [thick,blue] (1,1)--(1,0);
						\draw [<->,ultra thick] (0,5)--(0,0)--(5,0);
						\node [right,teal] at (4,4) {$S$};
						\node [left] at (0,5) {$p$};
						\node [below] at (5,0) {$q$};
						\node [left] at (0,1) {\tiny $p^\peq{m}$};
						\node [left] at (0,1.5) {\tiny $p^\peq{c}$};
						\node [below] at (3,0) {\tiny $q^\peq{d}_\peq{0}$};
						\node [below] at (1,0) {\tiny $q^\peq{s}_\peq{0}$};
						\node [below] at (2.5,0) {\tiny $q^\peq{d}_\peq{1}$};
						\node [below] at (1.5,0) {\tiny $q^\peq{s}_\peq{1}$};
						\node [below] at (0,0) {\tiny 0};
						\node [above] at (.8,1.82) {\scriptsize \textbf{A}};
						\node [below] at (2,1.95) {\scriptsize \textbf{B}};
						\node at (.6,1.2) {\scriptsize \textbf{C}};
						\node at (1.35,1.15) {\scriptsize \textbf{D}};
						\node at (2,1.15) {\scriptsize \textbf{E}};
						\node at (2.65,1.15) {\scriptsize \textbf{F}};
						\node at (.2,.6) {\scriptsize \textbf{G}};
						\node at (2.5,5) {\underline{Sin política}};
					\end{tikzpicture}
				\end{subfigure}
				\begin{subfigure}[b]{0.49\textwidth}
					\begin{tikzpicture}[scale=1]
						\draw [fill,opacity=.3,purple] (0,4)--(2.5,1.5)--(0,1.5);
						\draw [fill,opacity=.3,purple] (0,0)--(1.5,1.5)--(0,1.5);
						\draw [fill,opacity=.4,purple] (1.5,1)--(1.5,1.5)--(2.5,1.5)--(2.5,1);
						\draw [ultra thick,teal] plot [domain=0:4] (\x,{4-\x});
						\node [above right,teal] at (4,0) {$D$};
						\draw [ultra thick,teal] plot [domain=0:4] (\x,{\x});
						\draw [thick,purple] (0,1.5)--(2.5,1.5)--(2.5,0);
						\draw [thick,purple] (1.5,1.5)--(1.5,0);
						\draw [thick,blue] (0,1)--(3,1)--(3,0);
						\draw [thick,blue] (1,1)--(1,0);
						\draw [<->,ultra thick] (0,5)--(0,0)--(5,0);
						\node [right,teal] at (4,4) {$S$};
						\node [left] at (0,5) {$p$};
						\node [below] at (5,0) {$q$};
						\node [left] at (0,1) {\tiny $p^\peq{m}$};
						\node [left] at (0,1.5) {\tiny $p^\peq{c}$};
						\node [below] at (3,0) {\tiny $q^\peq{d}_\peq{0}$};
						\node [below] at (1,0) {\tiny $q^\peq{s}_\peq{0}$};
						\node [below] at (2.5,0) {\tiny $q^\peq{d}_\peq{1}$};
						\node [below] at (1.5,0) {\tiny $q^\peq{s}_\peq{1}$};
						\node [below] at (0,0) {\tiny 0};
						\node [above] at (.8,1.82) {\scriptsize \textbf{A}};
						\node [below] at (2,1.95) {\scriptsize \textbf{B}};
						\node at (.6,1.2) {\scriptsize \textbf{C}};
						\node at (1.35,1.15) {\scriptsize \textbf{D}};
						\node at (2,1.15) {\scriptsize \textbf{E}};
						\node at (2.65,1.15) {\scriptsize \textbf{F}};
						\node at (.2,.6) {\scriptsize \textbf{G}};
						\node at (2.5,5) {\underline{Con política}};
					\end{tikzpicture}
				\end{subfigure}
			\end{figure}
		\end{frame}

		\begin{frame}
			\frametitle{Cuotas de Importación}
			Pérdida social
			
			\vspace{.1cm}
			
			\centering
			\begin{tikzpicture}[scale=1]
				\draw [fill,opacity=.3,black] (1,1)--(1.5,1.5)--(1.5,1);
				\draw [fill,opacity=.3,black] (3,1)--(2.5,1.5)--(2.5,1);
				\draw [ultra thick,teal] plot [domain=0:4] (\x,{4-\x});
				\node [above right,teal] at (4,0) {$D$};
				\draw [ultra thick,teal] plot [domain=0:4] (\x,{\x});
				\draw [thick,purple] (0,1.5)--(2.5,1.5)--(2.5,0);
				\draw [thick,purple] (1.5,1.5)--(1.5,0);
				\draw [thick,blue] (0,1)--(3,1)--(3,0);
				\draw [thick,blue] (1,1)--(1,0);
				\draw [<->,ultra thick] (0,5)--(0,0)--(5,0);
				\node [right,teal] at (4,4) {$S$};
				\node [left] at (0,5) {$p$};
				\node [below] at (5,0) {$q$};
				\node [left] at (0,1) {\tiny $p^\peq{m}$};
				\node [left] at (0,1.5) {\tiny $p^\peq{c}$};
				\node [below] at (3,0) {\tiny $q^\peq{d}_\peq{0}$};
				\node [below] at (1,0) {\tiny $q^\peq{s}_\peq{0}$};
				\node [below] at (2.5,0) {\tiny $q^\peq{d}_\peq{1}$};
				\node [below] at (1.5,0) {\tiny $q^\peq{s}_\peq{1}$};
				\node [below] at (0,0) {\tiny 0};
				\node [above] at (.8,1.82) {\scriptsize \textbf{A}};
				\node [below] at (2,1.95) {\scriptsize \textbf{B}};
				\node at (.6,1.2) {\scriptsize \textbf{C}};
				\node at (1.35,1.15) {\scriptsize \textbf{D}};
				\node at (2,1.15) {\scriptsize \textbf{E}};
				\node at (2.65,1.15) {\scriptsize \textbf{F}};
				\node at (.2,.6) {\scriptsize \textbf{G}};
			\end{tikzpicture}
		\end{frame}	

		\begin{frame}
			\frametitle{Cuotas de Importación}
			\begin{table}[htbp!]
				\centering
				\resizebox{11cm}{!}{
					\begin{tabular}{l c c c}\hline
												&		Sin política	&	Con política	 &	Cambio					\\  \hline 
									 $EC$ &	$A+B+C+D+E+F$		&			$A+B$			 &	$-(C+D+E+F)$			\\
									 $EP$ &				$G$				&			$C+G$			 &	$+(C)$				\\ 
								$EI/RF$ &									&			$E$				 &	$+(E)$				\\ \hline
									 $ET$ &	$A+B+C+D+E+F+G$	&		$A+B+C+E+G$	 &	$-(D+F)$						\\ \hline
					\end{tabular}}
			\end{table}
		\end{frame}

		\begin{frame}
			\frametitle{Cuotas de Importación}
			¿Qué podemos decir respecto del excedente del importador/recaudación fiscal?
			\begin{itemize}
				\item Supongamos que el gobierno cobra $\$p_\peq{M}$ por el derecho a importar una unidad.
			\end{itemize}
		\end{frame}

		\begin{frame}
			\frametitle{Cuotas de Importación}
			\begin{itemize}
				\item El gobierno vende derechos para importar $\overline{M}$ unidades, recaudando $\$p_\peq{M}$ por unidad: $$RF=p_\peq{M}\cdot\overline{M}$$
			\end{itemize}
		\end{frame}
		
		\begin{frame}
			\frametitle{Cuotas de Importación}
			\begin{itemize}
				\item Los individuos que compran derechos de importación pagan $\$p_\peq{M}$ por cada unidad importada que compran a $p^\peq{m}$ y pueden vender a $\$p^\peq{c}$: $$EI=\rp{p^\peq{c}-\sqp{p_\peq{M}+p^\peq{m}}}\cdot\overline{M}$$
			\end{itemize}
		\end{frame}

		\begin{frame}
			\frametitle{Cuotas de Importación}
			\begin{align*}
				RF+EI&=\rp{p_\peq{M}+p^\peq{c}-\sqp{p_\peq{M}+p^\peq{m}}}\cdot\overline{M}\\
						 &=\rp{p^\peq{c}-p^\peq{m}}\cdot\overline{M}
			\end{align*}
		\end{frame}

		\begin{frame}
			\frametitle{Cuotas de Importación}
			Notar que
			\begin{itemize}
				\item Si el gobierno regala los derechos de importación ($p_\peq{M}$=0), los importadores se quedan con todo el excedente de la importación: $$EI=\rp{p^\peq{c}-p^\peq{m}}\cdot\overline{M}$$
			\end{itemize}
		\end{frame}

		\begin{frame}
			\frametitle{Cuotas de Importación}
			\begin{itemize}
				\item Si el gobierno cobra $P_\peq{M}=p^\peq{c}-p^\peq{m}$ por derecho de importación,
					\begin{itemize}
						\item Los importadores no obtienen excedentes: $$EI=\rp{p^\peq{c}-\sqp{\rp{p^\peq{c}-p^\peq{m}}+p^\peq{m}}}\cdot\overline{M}=0$$
						\item El gobierno recauda $$RF=\rp{p^\peq{c}-p^\peq{m}}\cdot\overline{M}$$
						\item La política es equivalente a un arancel de monto $$t=p^\peq{c}-p^\peq{m}$$
					\end{itemize}
			\end{itemize}
		\end{frame}

		\begin{frame}
			\frametitle{Argumentos para Restringir el Comercio}
			\begin{itemize}
				\item El comercio destruye puestos de trabajo.
				\item Hay industrias estratégicas para la seguridad nacional.
				\item Las industrias nacientes necesitan protección para madurar y ser rentables.
				\item Competencia delseal cuando otros países implementan políticas proteccionistas.
				\item Las restricciones al comercio sirven como arma de negociación para incentivar a otros países a levantar sus restricciones.
			\end{itemize}
		\end{frame}

		\begin{frame}
			\frametitle{Otros beneficios del Comercio}
			\begin{itemize}
				\item Mayor variedad de bienes.
				\item Reducción de costos gracias a las economías de escala.
				\item Aumento de la competencia.
				\item Mayor flujo de ideas.
			\end{itemize}
		\end{frame}

\end{document}