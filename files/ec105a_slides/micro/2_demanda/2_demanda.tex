
\documentclass{beamer}
%\documentclass{beamer}
%\usepackage{beamerthemesplit} 
%\usetheme{Berkeley}
%\usecolortheme{dolphin}
\usetheme{Szeged}

\usepackage{amsfonts}
\usepackage{txfonts}
\usepackage[spanish]{babel}
%\usepackage[latin1]{inputenc}
\usepackage[utf8]{inputenc}
%\usepackage[dvips]{graphicx}
\usepackage{cancel}
%\usepackage{bm}
\usepackage{ae,aecompl,amsmath,amsbsy}
\setbeamertemplate{navigation symbols}{}

\beamertemplateballitem

\usepackage{tikz}
\usepackage{pbox}
%\usepackage{subfigure}
\usepackage{subcaption}
\usepackage{centernot}

\usetikzlibrary{babel,decorations.pathreplacing,decorations.markings}

\newtheorem{mydef}{Definición}
\newcommand{\peq}[1]{{\scriptscriptstyle{#1}}} 
\newcommand{\rp}[1]{\left(#1\right)}

\title{EAE105A \\ Introducción a la Economía}
\subtitle{II. Microeconomía: Teoría de la Demanda} 
\author{Pinjas Albagli}
\institute{Instituto de Economía \\ Pontificia Universidad Católica de Chile}
\date{Primer Semestre de 2018}

\begin{document}

	\maketitle 
	
	\section{Introducción}

		\begin{frame}
			\frametitle{La Curva de Demanda}
			\begin{mydef}
				\textbf{Cantidad demandada ($\mathbf{q^\peq{d}}$):} Cantidad de un bien que los compradores están dispuestos y tienen la capacidad de comprar.
			\end{mydef}
			\begin{mydef}
				\textbf{Función de demanda:} Función que relaciona la cantidad demandada de un bien con su precio, manteniendo constantes los precios de otros bienes relacionados, las preferencias del consumidor y su ingreso. $$q^\peq{d}_\peq{x}=f\left(p_\peq{x},\overline{p_\peq{y}},\overline{p_\peq{z}},\overline{m},\overline{\text{otros factores...}}\right)$$
			\end{mydef}
		\end{frame}
		
		\begin{frame}
			\frametitle{La Curva de Demanda}
			\begin{mydef}
				\textbf{Ley de la demanda:} Con todo lo demás constante, la cantidad demandada de un bien ($q^\peq{d}$) disminuye cuando su precio ($p$) aumenta.
			\end{mydef}
		\end{frame}
		
		\begin{frame}
			\frametitle{La Curva de Demanda}
			Tradicionalmente la graficamos al revés: $$p^\peq{d}_\peq{x}=f^\peq{-1}\left(q_\peq{x},\overline{p_\peq{y}},\overline{p_\peq{z}},\overline{m},\overline{\text{otros factores...}}\right)$$
			\begin{figure}[htbp!]
				\centering
				\begin{tikzpicture}[scale=3]
					\draw plot [domain=0.05:1.4](\x,{1-ln(1+\x)});
					\draw[<->, thick] (0,1.2) -- (0,0) -- (1.6,0);
					\node [right] at (1.4,.1245312626) {$D$};
					\node [below] at (1.6,0) {$q$};
					\node [left] at (0,1.2) {$p$};
				\end{tikzpicture}
				%\caption{Curva de demanda}
			\end{figure}
		\end{frame}
		
		\begin{frame}
			\frametitle{La Curva de Demanda}
			De esta manera, la curva de demanda refleja el máximo precio que un consumidor estaría dispuesto a pagar por una unidad adicional. Es decir, la \textbf{valoración marginal del consumo}. $$\text{Ley de la demanda}\implies\text{la curva de demanda tiene pendiente negativa}$$
		\end{frame}
		
		\begin{frame}
			\frametitle{La Curva de Demanda}
			Es \textbf{muy} importante distinguir entre:
			\begin{mydef}
				\begin{itemize}
					\item \textbf{Cambio en la cantidad demandada:} Cuando cambia $p$ y se mantienen constantes otros determinantes de la demanda, se produce un movimiento en/sobre/através de la curva de demanda.
					\item \textbf{Cambio en la demanda:} Cuando cambia alguno de los determinantes de la demanda que no están representados en los ejes del gráfico (es decir, cualquiera distinto de $p$), cambia la cantidad demandada a cada precio, por lo que se desplaza la curva de demanda.
				\end{itemize}
			\end{mydef}
		\end{frame}
		
	\section{Fundamento Teórico}

		\begin{frame}
			\frametitle{Teoría del Consumidor}
			Estudiaremos la elección de un consumidor tomador de precios.
			
			Modelaremos al consumidor como un individuo que consume 2 bienes cuyas cantidades denotaremos por $x$ e $y$. Paga $p_\peq{x}$ y $p_\peq{y}$ por cada unidad de $x$ e $y$ respectivamente. Suponemos que el objetivo del consumidor es maximizar su utilidad $u\rp{x,y}$.
		\end{frame}	

		\begin{frame}
			\frametitle{Teoría del Consumidor}
			Definamos algunos conceptos:
			\begin{mydef}
				\begin{itemize}
					\item \textbf{Canasta de consumo $\mathbf{\rp{x,y}}$:} Combinación compuesta por las cantidades $x$ e $y$ de ambos bienes.
					\item \textbf{Restricción presupuestaria:} El límite en las canastas de consumo que el consumidor puede comprar. $$p_\peq{x}\cdot x + p_\peq{y}\cdot y \leq m$$
				\end{itemize}
			\end{mydef}
		\end{frame}	

		\begin{frame}
			\frametitle{Teoría del Consumidor}
			\centering
			\begin{tikzpicture}[scale=.05]
				\draw [ultra  thick, teal] plot [domain=0:100] (\x,{100-\x});
				\draw [<->,thick] (0,120)--(0,0)--(120,0);
				\draw [fill,blue] (0,100) circle [radius=2];
				\draw [fill,blue] (100,0) circle [radius=2];
				\node [left] at (0,120) {$y$};
				\node [below] at (120,0) {$x$};
				\node [left] at (0,100) {\tiny $\frac{m}{p_\peq{y}}$};
				\node [below] at (100,0) {\tiny $\frac{m}{p_\peq{x}}$};
				\node [below] at (0,0) {\tiny 0};
			\end{tikzpicture}
		\end{frame}	

		\begin{frame}
			\frametitle{Teoría del Consumidor}
			\centering
			\begin{tikzpicture}[scale=.05]
				\draw [fill,teal,opacity=.3] (100,0) -- (0,0) -- (0,100);
				\draw [ultra  thick, teal] plot [domain=0:100] (\x,{100-\x});
				\draw [<->,thick] (0,120)--(0,0)--(120,0);
				\draw [fill,blue] (0,100) circle [radius=2];
				\draw [fill,blue] (100,0) circle [radius=2];
				\node [left] at (0,120) {$y$};
				\node [below] at (120,0) {$x$};
				\node [left] at (0,100) {\tiny $\frac{m}{p_\peq{y}}$};
				\node [below] at (100,0) {\tiny $\frac{m}{p_\peq{x}}$};
				\node [below] at (0,0) {\tiny 0};
			\end{tikzpicture}
		\end{frame}	

		\begin{frame}
			\frametitle{Teoría del Consumidor}
				\begin{itemize}
					\item \textbf{Función de utilidad ($\mathbf{u\rp{\cdot}}$):} Función que representa las preferencias del consumidor.$$\rp{x_\peq{A},y_\peq{A}}\succsim \rp{x_\peq{B},y_\peq{B}}\iff u\rp{x_\peq{A},y_\peq{A}}\geq u\rp{x_\peq{B},y_\peq{B}}$$
					\item \textbf{Utilidad marginal ($\mathbf{UMg}$):} Cambio en la función de utilidad al aumentar el consumo en una unidad.$$UMg_\peq{x}=u\rp{x+1,y}-u\rp{x,y}$$ $$UMg_\peq{y}=u\rp{x,y+1}-u\rp{x,y}$$
				\end{itemize}
		\end{frame}	

		\begin{frame}
			\frametitle{Teoría del Consumidor}
			\centering
			\begin{tikzpicture}[scale=.055]
				\draw [dashed,help lines] (20,0)--(20,44.72135954)--(0,44.72135954);
				\draw [dashed,help lines] (30,0)--(30,54.77225575)--(0,54.77225575);
				\draw [dashed,help lines] (80,0)--(80,89.44271908)--(0,89.44271908);
				\draw [dashed,help lines] (90,0)--(90,94.86832980)--(0,94.86832980);
				\draw [ultra  thick, teal] plot [domain=0:95] (\x,{10*sqrt(\x)});
				\draw [<->,thick] (0,100)--(0,0)--(100,0);
				\node [left] at (0,100) {$y$};
				\node [below] at (100,0) {$x$};
				\node [below] at (0,0) {\tiny 0};
				\node [left] at (0,44.72135954) {\tiny $u\rp{x_\peq{1},y_\peq{0}}$};
				\node [left] at (0,54.77225575) {\tiny $u\rp{x_\peq{2},y_\peq{0}}$};
				\node [left] at (0,89.44271908) {\tiny $u\rp{x_\peq{3},y_\peq{0}}$};
				\node [left] at (0,94.86832980) {\tiny $u\rp{x_\peq{4},y_\peq{0}}$};
				\node [below] at (20,0) {\tiny $x_\peq{1}$};
				\node [below] at (30,0) {\tiny $x_\peq{2}$};
				\node [below] at (80,0) {\tiny $x_\peq{3}$};
				\node [below] at (90,0) {\tiny $x_\peq{4}$};
				\node [right,teal] at (95,97.46794345) {$u\rp{x,y_\peq{0}}$};
			\end{tikzpicture}
		\end{frame}	

		\begin{frame}
			\frametitle{Teoría del Consumidor}
				\begin{itemize}
					\item \textbf{Curva de indiferencia ($\mathbf{I}$):} Curva que muestra las canastas de consumo que le proporcionan al consumidor el mismo nivel de utilidad. Hay una para cada nivel de utilidad posible.$$I_\peq{\overline{u}}=\{\rp{x,y}\in\varmathbb{R}_\peq{+}^\peq{2} \ : \ u\rp{x,y} = \overline{u}\}$$
					\item \textbf{Tasa marginal de sustitución ($\mathbf{TMS}$):} Tasa a la cual un consumidor está dispuesto a intercambiar un bien por otro. Es la pendiente de la curva de indiferencia. $$TMS=\frac{UMg_\peq{x}}{UMg_\peq{y}}$$
				\end{itemize}
		\end{frame}	

		\begin{frame}
			\frametitle{Teoría del Consumidor}
			\centering
			\begin{tikzpicture}[scale=.05]
				\draw [<->,thick] (0,100)--(0,0)--(100,0);
				\draw [ultra  thick, teal] plot [domain=10:100] (\x,{1000/\x});
				\draw [ultra  thick, teal] plot [domain=15:100] (\x,{1500/\x});
				\draw [ultra  thick, teal] plot [domain=20:100] (\x,{2000/\x});
				\draw [ultra  thick, teal] plot [domain=25:100] (\x,{2500/\x});
				\node [left] at (0,100) {$y$};
				\node [below] at (100,0) {$x$};
				\node [right,teal] at (100,1000/100) {\tiny $I_\peq{0}$};
				\node [right,teal] at (100,1500/100) {\tiny $I_\peq{1}$};
				\node [right,teal] at (100,2000/100) {\tiny $I_\peq{2}$};
				\node [right,teal] at (100,2500/100) {\tiny $I_\peq{3}$};
				\node [below] at (0,0) {\tiny 0};
			\end{tikzpicture}
		\end{frame}	

		\begin{frame}
			\frametitle{Teoría del Consumidor}
				Propiedades de las curvas de indiferencia:
				\begin{itemize}
					\item Las curvas de indiferencia más altas son preferidas a las más bajas.
					\item Las curvas de indiferencia tienen pendiente negativa.
					\item Las curvas de indiferencia no se cruzan.
					\item Las curvas de indiferencia son convexas al origen.
				\end{itemize}
		\end{frame}	

		\begin{frame}
			\frametitle{Teoría del Consumidor}
			Caso extremo 1: Complementos perfectos
			\centering
			\begin{tikzpicture}[scale=.05]
				\draw [<->,thick] (0,100)--(0,0)--(100,0);
				\draw [ultra  thick, teal] (20,100)--(20,20)--(100,20);
				\draw [ultra  thick, teal] (30,100)--(30,30)--(100,30);
				\draw [ultra  thick, teal] (40,100)--(40,40)--(100,40);
				\node [left] at (0,100) {$y$};
				\node [below] at (100,0) {$x$};
				\node [right,teal] at (100,20) {\tiny $I_\peq{0}$};
				\node [right,teal] at (100,30) {\tiny $I_\peq{1}$};
				\node [right,teal] at (100,40) {\tiny $I_\peq{2}$};
				\node [below] at (0,0) {\tiny 0};
			\end{tikzpicture}
		\end{frame}	

		\begin{frame}
			\frametitle{Teoría del Consumidor}
			Caso extremo 2: Sustitutos perfectos
			\centering
			\begin{tikzpicture}[scale=.05]
				\draw [ultra  thick, teal] plot [domain=0:50] (\x,{50-\x});
				\draw [ultra  thick, teal] plot [domain=0:70] (\x,{70-\x});
				\draw [ultra  thick, teal] plot [domain=0:90] (\x,{90-\x});
				\draw [<->,thick] (0,100)--(0,0)--(100,0);
				\node [left] at (0,100) {$y$};
				\node [below] at (100,0) {$x$};
				\node [above right,teal] at (50,0) {\tiny $I_\peq{0}$};
				\node [above right,teal] at (70,0) {\tiny $I_\peq{1}$};
				\node [above right,teal] at (90,0) {\tiny $I_\peq{2}$};
				\node [below] at (0,0) {\tiny 0};
			\end{tikzpicture}
		\end{frame}	

		\begin{frame}
			\frametitle{Teoría del Consumidor}
			Suponemos que el objetivo del consumidor es escoger la cantidad canasta de consumo $\rp{x^{*},y^{*}}$ que maximiza su utilidad dentro del conjunto de canastas que puede comprar, dado su ingreso y los precios de los bienes. Esto se traduce en alcanzar la curva de indiferencia más lejana al origen cumpliendo la restricción presupuestaria.
		\end{frame}					
		
		\begin{frame}
			\frametitle{Teoría del Consumidor}
			Caracterizando el óptimo:
			\begin{itemize}
				\item Si el consumidor gasta \$1 adicional en $x$, puede comprar $\frac{1}{p_\peq{x}}$ unidades adicionales...
				\item su utilidad aumentará en $\frac{UMg_\peq{x}}{p_\peq{x}}$...
				\item pero tendrá que dejar de comprar $\frac{1}{p_\peq{y}}$ unidades de $y$...
				\item y su utilidad disminuirá en $\frac{UMg_\peq{y}}{p_\peq{y}}$.
			\end{itemize}
		\end{frame}

		\begin{frame}
			\frametitle{Teoría del Consumidor}
			Por lo tanto,
			\begin{itemize}
				\item Si $\frac{UMg_\peq{x}}{p_\peq{x}}>\frac{UMg_\peq{y}}{p_\peq{y}}$, es posible aumentar la utilidad reasignando gasto al bien $x$.
				\item Si $\frac{UMg_\peq{x}}{p_\peq{x}}<\frac{UMg_\peq{y}}{p_\peq{y}}$, es posible aumentar la utilidad reasignando gasto al bien $y$.
			\end{itemize}
			Pero en el óptimo no es posible aumentar la utilidad redistribuyendo el gasto.
		\end{frame}

		\begin{frame}
			\frametitle{Teoría del Consumidor}
			Concluimos que en el óptimo $$\frac{UMg_\peq{x}}{p_\peq{x}}=\frac{UMg_\peq{y}}{p_\peq{y}}$$
			Reordenando, $$\frac{UMg_\peq{x}}{UMg_\peq{y}}=\frac{p_\peq{x}}{p_\peq{y}}$$ $$TMS=\frac{p_\peq{x}}{p_\peq{y}}$$ $$\implies\text{tangencia entre curva de indiferencia y la RP}$$
		\end{frame}

		\begin{frame}
			\frametitle{Teoría del Consumidor}
			\centering
			\begin{tikzpicture}[scale=.5]
				\draw [dashed,help lines] (0,5)--(5,5)--(5,0);
				%\draw [dashed,help lines] (2.5,0)--(2.5,2.5)--(0,2.5);
				\draw [ultra thick,red] plot [domain=0:10] (\x,{10-\x});
				%\draw [ultra thick,teal] plot [domain=1:6] (\x,{6.25/\x});
				\draw [ultra thick,teal] plot [domain=3:8.2] (\x,{25/\x});
				\draw [ultra thick,teal] plot [domain=3.5:8.3] (\x,{30/\x});
				\draw [ultra thick,teal] plot [domain=2.5:8.1] (\x,{20/\x});
				\draw [<->,thick] (0,10.7)--(0,0)--(10.7,0);
				\draw [fill,blue] (5,5) circle [radius=.2];
				\node [right,blue] at (5,5) {\textbf{A}};
				%\draw [fill,blue] (2.5,2.5) circle [radius=.2];
				\node [left] at (0,10.7) {$y$};
				\node [below] at (10.7,0) {$x$};
				%\node [left] at (0,2.5) {\tiny $y_\peq{A}$};
				%\node [below] at (2.5,0) {\tiny $x_\peq{A}$};
				\node [left] at (0,5) {\tiny $y^\peq{*}$};
				\node [below] at (5,0) {\tiny $x^\peq{*}$};
				%\node [right,teal] at (6,6.25/6) {\tiny $I_\peq{0}$};
				%\node [right,teal] at (8,25/8) {\tiny $I_\peq{1}$};
				%\node [above,red] at (5,0) {\tiny $RP_\peq{0}$};
				%\node [above,red] at (10,0) {\tiny $RP_\peq{1}$};
				\node [below] at (0,0) {\tiny 0};
			\end{tikzpicture}
		\end{frame}	

		\begin{frame}
			\frametitle{Teoría del Consumidor}
			Cuando (\emph{ceteris paribus}) aumenta el ingreso, la restricción presupuestaria se desplaza hacia afuera en forma paralela.
			\begin{mydef}
				\begin{itemize}
					\item \textbf{Bien normal o superior:} Un incremento en el ingreso aumenta la cantidad demandada.
					\item \textbf{Bien inferior:} Un incremento en el ingreso reduce la cantidad demandada.
				\end{itemize}
			\end{mydef}
		\end{frame}	

		\begin{frame}
			\frametitle{Teoría del Consumidor}
			Ejemplo 1: $x$ e $y$ normales
			
			\vspace{.3cm}
			
			\centering
			\begin{tikzpicture}[scale=.5]
				\draw [dashed,help lines] (0,5)--(5,5)--(5,0);
				\draw [dashed,help lines] (2.5,0)--(2.5,2.5)--(0,2.5);
				\draw [ultra thick,red] plot [domain=0:10] (\x,{10-\x});
				\draw [ultra thick,red] plot [domain=0:5] (\x,{5-\x});
				\draw [ultra thick,teal] plot [domain=1:6] (\x,{6.25/\x});
				\draw [ultra thick,teal] plot [domain=3:8] (\x,{25/\x});
				\draw [<->,thick] (0,10.7)--(0,0)--(10.7,0);
				\draw [fill,blue] (5,5) circle [radius=.2];
				\node [above right,blue] at (5,5) {\textbf{B}};
				\draw [fill,blue] (2.5,2.5) circle [radius=.2];
				\node [above right,blue] at (2.5,2.5) {\textbf{A}};
				\node [left] at (0,10.7) {$y$};
				\node [below] at (10.7,0) {$x$};
				\node [left] at (0,2.5) {\tiny $y_\peq{A}$};
				\node [below] at (2.5,0) {\tiny $x_\peq{A}$};
				\node [left] at (0,5) {\tiny $y_\peq{B}$};
				\node [below] at (5,0) {\tiny $x_\peq{B}$};
				\node [right,teal] at (6,6.25/6) {\tiny $I_\peq{0}$};
				\node [right,teal] at (8,25/8) {\tiny $I_\peq{1}$};
				\node [above,red] at (5,0) {\tiny $RP_\peq{0}$};
				\node [above,red] at (10,0) {\tiny $RP_\peq{1}$};
				\node [below] at (0,0) {\tiny 0};
			\end{tikzpicture}
		\end{frame}	

		\begin{frame}
			\frametitle{Teoría del Consumidor}
			Ejemplo 2: $x$ inferior, $y$ normal
			
			\vspace{.3cm}
			
			\centering
			\begin{tikzpicture}[scale=2]
				\draw [dashed,help lines] (0,9)--(1,9)--(1,8);
				\draw [dashed,help lines] (0,9.8)--(.6,9.8)--(.6,8);
				\draw [ultra thick,red] plot [domain=0:2] (\x,{10-\x});
				\draw [ultra thick,red] plot [domain=0:2.4] (\x,{52/5-\x});
				\draw [ultra thick,teal] plot [domain=.6:2] (\x,{-1*exp(.5*ln(\x-.5)+.3465735903)+10});
				\draw [ultra thick,teal] plot [domain=.500009:2.5] (\x,{-1*exp(.5*ln(\x-.5)-.4581453655)+10});
				\draw [<->,thick] (0,10.7)--(0,8)--(2.7,8);
				\draw [fill,blue] (1,9) circle [radius=.05];
				\node [above right,blue] at (1,9) {\textbf{A}};
				\draw [fill,blue] (.6,9.8) circle [radius=.05];
				\node [above right,blue] at (.6,9.8) {\textbf{B}};
				\node [left] at (0,10.7) {$y$};
				\node [below] at (2.7,8) {$x$};
				\node [left] at (0,9) {\tiny $y_\peq{A}$};
				\node [below] at (1,8) {\tiny $x_\peq{A}$};
				\node [left] at (0,9.8) {\tiny $y_\peq{B}$};
				\node [below] at (.6,8) {\tiny $x_\peq{B}$};
				\node [right,teal] at (2,8.267949192) {\tiny $I_\peq{0}$};
				\node [right,teal] at (2.5,9.128220211) {\tiny $I_\peq{1}$};
				\node [below] at (0,8) {\tiny 0};
				\node [above,red] at (2,8) {\tiny $RP_\peq{0}$};
				\node [above,red] at (2.4,8) {\tiny $RP_\peq{1}$};
			\end{tikzpicture}
		\end{frame}	

		\begin{frame}
			\frametitle{Teoría del Consumidor}
			Cuando (\emph{ceteris paribus}) cambia el precio relativo, cambia la pendiente de la restricción presupuestaria. Por ejemplo, $$\Delta^\peq{+}p_\peq{x}\implies\Delta^\peq{+}\rp{\frac{p_\peq{x}}{p_\peq{y}}}$$ $$\Delta^\peq{+}p_\peq{y}\implies\Delta^\peq{-}\rp{\frac{p_\peq{x}}{p_\peq{y}}}$$
		\end{frame}	

		\begin{frame}
			\frametitle{Teoría del Consumidor}
			Ejemplo: $\Delta^\peq{-}p_\peq{y}$
			
			\vspace{.3cm}
			
			\centering
			\begin{tikzpicture}[scale=.5]
				\draw [dashed,help lines] (0,5)--(5,5)--(5,0);
				\draw [dashed,help lines] (6,0)--(6,8/3)--(0,8/3);
				\draw [ultra thick,red] plot [domain=0:10] (\x,{10-\x});
				\draw [ultra thick,red] plot [domain=0:10] (\x,{-2/3*\x+20/3});
				\draw [ultra thick,teal] plot [domain=3:8] (\x,{-4*\x^(1/2)*5^(1/2)+\x+20});
				\draw [ultra thick,teal] plot [domain=3:8] (\x,{-10/3*\x^(1/2)*6^(1/2)+\x+50/3});
				\draw [<->,thick] (0,10.7)--(0,0)--(10.7,0);
				\draw [fill,blue] (5,5) circle [radius=.2];
				\node [right,blue] at (5,5) {\textbf{C}};
				\draw [fill,blue] (6,8/3) circle [radius=.2];
				\node [right,blue] at (6,8/3) {\textbf{A}};
				\node [left] at (0,10.7) {$y$};
				\node [below] at (10.7,0) {$x$};
				\node [left] at (0,8/3) {\tiny $y^\peq{*}_\peq{A}$};
				\node [below] at (6,0) {\tiny $x^\peq{*}_\peq{A}$};
				\node [left] at (0,5) {\tiny $y^\peq{*}_\peq{C}$};
				\node [below] at (5,0) {\tiny $x^\peq{*}_\peq{C}$};
				\node [right,teal] at (8,1.57265590) {\tiny $I_\peq{0}$};
				\node [right,teal] at (8,2.70177872) {\tiny $I_\peq{1}$};
				\node [right,red] at (0,10) {\tiny $RP_\peq{1}$};
				\node [right,red] at (0,20/3) {\tiny $RP_\peq{0}$};
				\node [below] at (0,0) {\tiny 0};
			\end{tikzpicture}
		\end{frame}	

		\begin{frame}
			\frametitle{Teoría del Consumidor}
			Podemos descomponer el efecto...
			
			\vspace{.3cm}
			
			\centering
			\begin{tikzpicture}[scale=.5]
				\draw [dashed,help lines] (0,5)--(5,5)--(5,0);
				\draw [dashed,help lines] (6,0)--(6,8/3)--(0,8/3);
				\draw [dashed,help lines] (25/6,0)--(25/6,25/6)--(0,25/6);
				\draw [ultra thick,purple,dashed] plot [domain=0:25/3] (\x,{-\x+25/3});
				\draw [ultra thick,red] plot [domain=0:10] (\x,{10-\x});
				\draw [ultra thick,red] plot [domain=0:10] (\x,{-2/3*\x+20/3});
				\draw [ultra thick,teal] plot [domain=3:8] (\x,{-4*\x^(1/2)*5^(1/2)+\x+20});
				\draw [ultra thick,teal] plot [domain=3:8] (\x,{-10/3*\x^(1/2)*6^(1/2)+\x+50/3});
				\draw [<->,thick] (0,10.7)--(0,0)--(10.7,0);
				\draw [fill,blue] (5,5) circle [radius=.2];
				\node [right,blue] at (5,5) {\textbf{C}};
				\draw [fill,blue] (6,8/3) circle [radius=.2];
				\node [right,blue] at (6,8/3) {\textbf{A}};
				\draw [fill,purple] (25/6,25/6) circle [radius=.2];
				\node [right,purple] at (25/6,25/6) {\textbf{B}};
				\node [left] at (0,10.7) {$y$};
				\node [below] at (10.7,0) {$x$};
				\node [left] at (0,8/3) {\tiny $y^\peq{*}_\peq{A}$};
				\node [below] at (6,0) {\tiny $x^\peq{*}_\peq{A}$};
				\node [left] at (0,5) {\tiny $y^\peq{*}_\peq{C}$};
				\node [below] at (5,0) {\tiny $x^\peq{*}_\peq{C}$};
				\node [left] at (0,25/6) {\tiny $y^\peq{*}_\peq{B}$};
				\node [below] at (25/6,0) {\tiny $x^\peq{*}_\peq{B}$};
				\node [right,teal] at (8,1.57265590) {\tiny $I_\peq{0}$};
				\node [right,teal] at (8,2.70177872) {\tiny $I_\peq{1}$};
				\node [right,red] at (0,10) {\tiny $RP_\peq{1}$};
				\node [right,red] at (0,20/3) {\tiny $RP_\peq{0}$};
				\node [below] at (0,0) {\tiny 0};
			\end{tikzpicture}
		\end{frame}	

		\begin{frame}
			\frametitle{Teoría del Consumidor}
			\begin{mydef}
				\begin{itemize}
						\item \textbf{Efecto sustitución:} Cambio en el consumo cuando un cambio en el precio mueve al consumidor a lo largo de una curva de indiferencia determinada, hasta un punto con una nueva tasa marginal de sustitución.
							\begin{itemize}
								\item Mantenemos el nivel de utilidad artificialmente constante y nos preguntamos qué canasta escogería el consumidor para mantener el mismo nivel de utilidad con el nuevo precio relativo. Esto requiere ``quitarle'' ingreso.
							\end{itemize}
				\end{itemize}
			\end{mydef}
		\end{frame}	


		\begin{frame}
			\frametitle{Teoría del Consumidor}
			\begin{mydef}
				\begin{itemize}
						\item \textbf{Efecto ingreso: } Cambio en el consumo cuando un cambio en el precio mueve al consumidor a una curva de indiferencia distinta.
							\begin{itemize}
								\item Le ``devolvemos'' el ingreso que le habíamos ``quitado'', permitiéndole alcanzar un nivel de utilidad más alto.  
							\end{itemize}
				\end{itemize}
			\end{mydef}
		\end{frame}	

		%\begin{frame}
			%\frametitle{Teoría del Consumidor}
			%Variando $p_\peq{x}$ y manteniendo $p_\peq{y}$ y $m$ constantes, podemos obtener la curva de demanda por $x$:
			%\begin{figure}[htbp!]
				%\centering
				%\begin{subfigure}[b]{0.49\textwidth}
					%\resizebox{\linewidth}{!}{
						%\begin{tikzpicture}[scale=.5]
							%\draw [dashed,help lines] (0,5)--(2.5,5)--(2.5,0);
							%\draw [dashed,help lines] (5,0)--(5,5)--(2.5,5);
							%\draw [ultra thick,red] plot [domain=0:10] (\x,{10-\x});
							%\draw [ultra thick,red] plot [domain=0:5] (\x,{10-2*\x});
							%\draw [ultra thick,teal] plot [domain=1.5:5.5] (\x,{25/(2*\x)});
							%\draw [ultra thick,teal] plot [domain=2.8:7.5] (\x,{25/\x});
							%\draw [<->,thick] (0,12)--(0,0)--(12,0);
							%\draw [fill,blue] (2.5,5) circle [radius=.2];
							%\node [above right,blue] at (2.5,5) {\textbf{A}};
							%\draw [fill,blue] (5,5) circle [radius=.2];
							%\node [above right,blue] at (5,5) {\textbf{B}};
							%\node [left] at (0,12) {$y$};
							%\node [below] at (12,0) {$x$};
							%\node [left] at (0,5) {\tiny $y^\peq{*}_\peq{A}=y^\peq{*}_\peq{B}$};
							%\node [below] at (2.5,0) {\tiny $x^\peq{*}_\peq{A}$};
							%\node [below] at (5,0) {\tiny $x^\peq{*}_\peq{B}$};
							%\node [right,teal] at (5.5,25/11) {\tiny $I_\peq{0}$};
							%\node [right,teal] at (7.5,25/7.5) {\tiny $I_\peq{1}$};
							%\node [above right,red] at (10,0) {\tiny $RP_\peq{1}$};
							%\node [above right,red] at (5,0) {\tiny $RP_\peq{0}$};
							%\node [below] at (0,0) {\tiny 0};
						%\end{tikzpicture}
					%}
				%\end{subfigure}
				%\begin{subfigure}[b]{0.45\textwidth}
					%\resizebox{\linewidth}{!}{
						%\begin{tikzpicture}[scale=.5]
							%\draw [dashed,help lines] (0,10/2.5)--(2.5,10/2.5)--(2.5,0);
							%\draw [dashed,help lines] (0,10/5)--(5,10/5)--(5,0);
							%\draw [ultra thick,teal] plot [domain=1:10] (\x,{10/\x});
							%\draw [<->,thick] (0,12)--(0,0)--(12,0);
							%\draw [fill,blue] (5,10/5) circle [radius=.2];
							%\node [above right,blue] at (5,10/5) {\textbf{B}};
							%\draw [fill,blue] (2.5,10/2.5) circle [radius=.2];
							%\node [above right,blue] at (2.5,10/2.5) {\textbf{A}};
							%\node [left] at (0,12) {$p_\peq{x}$};
							%\node [below] at (12,0) {$q^\peq{d}_\peq{x}$};
							%\node [left] at (0,10/2.5) {\tiny $p^\peq{0}_\peq{x}$};
							%\node [below] at (2.5,0) {\tiny $x^\peq{*}_\peq{A}$};
							%\node [left] at (0,10/5) {\tiny $p^\peq{1}_\peq{x}$};
							%\node [below] at (5,0) {\tiny $x^\peq{*}_\peq{B}$};
							%\node [right,teal] at (10,1) {$D_\peq{x}$};
							%\node [below] at (0,0) {\tiny 0};
						%\end{tikzpicture}
					%}
				%\end{subfigure}
			%\end{figure}	
		%\end{frame}	

		\begin{frame}
			\frametitle{Teoría del Consumidor}
			Variando $p_\peq{x}$ y manteniendo $p_\peq{y}$ y $m$ constantes, podemos obtener la curva de demanda por $x$:
			\begin{figure}[htbp!]
				\centering
				\begin{subfigure}[b]{0.49\textwidth}
					\resizebox{\linewidth}{!}{
						\begin{tikzpicture}[scale=.5]
							\draw [dashed,help lines] (0,20/3)--(5/3,20/3)--(5/3,0);
							\draw [dashed,help lines] (5,0)--(5,5)--(0,5);
							\draw [ultra thick,red] plot [domain=0:10] (\x,{10-\x});
							\draw [ultra thick,red] plot [domain=0:5] (\x,{10-2*\x});
							\draw [ultra thick,teal] plot [domain=2:9.2] (\x,{-4*sqrt(5*\x)+\x+20});
							\draw [ultra thick,teal] plot [domain=.8:9] (\x,{-2*sqrt(15*\x)+\x+15});
							\draw [<->,thick] (0,12)--(0,0)--(12,0);
							\draw [fill,blue] (5/3,20/3) circle [radius=.2];
							\node [above right,blue] at (5/3,20/3) {\textbf{A}};
							\draw [fill,blue] (5,5) circle [radius=.2];
							\node [above right,blue] at (5,5) {\textbf{B}};
							\node [left] at (0,12) {$y$};
							\node [below] at (12,0) {$x$};
							\node [left] at (0,5) {\tiny $y^\peq{*}_\peq{B}$};
							\node [left] at (0,20/3) {\tiny $y^\peq{*}_\peq{A}$};
							\node [below] at (5/3,0) {\tiny $x^\peq{*}_\peq{A}$};
							\node [below] at (5,0) {\tiny $x^\peq{*}_\peq{B}$};
							\node [right,teal] at (8.8,.76209992) {\tiny $I_\peq{0}$};
							\node [right,teal] at (9,2.07068007) {\tiny $I_\peq{1}$};
							\node [above right,red] at (10,0) {\tiny $RP_\peq{1}$};
							\node [above right,red] at (5,0) {\tiny $RP_\peq{0}$};
							\node [below] at (0,0) {\tiny 0};
						\end{tikzpicture}
					}
				\end{subfigure}
				\begin{subfigure}[b]{0.49\textwidth}
					\resizebox{\linewidth}{!}{
						\begin{tikzpicture}[scale=.5]
							\draw [dashed,help lines] (0,8.43908891)--(5/3,8.43908891)--(5/3,0);
							\draw [dashed,help lines] (0,3.41640786)--(5,3.41640786)--(5,0);
							\draw [ultra thick,teal] plot [domain=1.2:11] (\x,{10/\x*(-\x+sqrt(\x^2+4*\x))});
							\draw [<->,thick] (0,12)--(0,0)--(12,0);
							\draw [fill,blue] (5,3.41640786) circle [radius=.2];
							\node [above right,blue] at (5,3.41640786) {\textbf{B}};
							\draw [fill,blue] (5/3,8.43908891) circle [radius=.2];
							\node [above right,blue] at (5/3,8.43908891) {\textbf{A}};
							\node [left] at (0,12) {$p_\peq{x}$};
							\node [below] at (12,0) {$q^\peq{d}_\peq{x}$};
							\node [left] at (0,8.43908891) {\tiny $p^\peq{0}_\peq{x}$};
							\node [below] at (5/3,0) {\tiny $x^\peq{*}_\peq{A}$};
							\node [left] at (0,3.41640786) {\tiny $p^\peq{1}_\peq{x}$};
							\node [below] at (5,0) {\tiny $x^\peq{*}_\peq{B}$};
							\node [right,teal] at (11,1.67748416) {$D_\peq{x}$};
							\node [below] at (0,0) {\tiny 0};
						\end{tikzpicture}
					}
				\end{subfigure}
			\end{figure}	
		\end{frame}

		\begin{frame}
			\frametitle{Aplicación 1: Comportamiento Giffen}
			\begin{mydef}
						\textbf{Bien Giffen:} Un bien para el cual un aumento en el precio incrementa la cantidad demandada.
			\end{mydef}
			En otras palabras, un bien cuya demanda tiene pendiente positiva:
			\begin{figure}[htbp!]
				\centering
				\begin{tikzpicture}[scale=1]
					\draw[<->, thick] (0,4) -- (0,0) -- (4,0);
					\draw plot [domain=0:ln(20)](\x,{exp(\x)/5});
					\node [right] at (3,4) {$D$};
					\node [below] at (4,0) {$q^\peq{d}$};
					\node [left] at (0,4) {$p$};
				\end{tikzpicture}
			\end{figure}
		\end{frame}	
			
		\begin{frame}
			\frametitle{Aplicación 1: Comportamiento Giffen}
			Ejemplo: Elección entre carne y arroz para un consumidor pobre
			\begin{itemize}
				\item Bien $x$: Arroz
				\item Bien $y$: Carne
				\item Calorías por peso gastado: $\frac{c_\peq{x}}{p_\peq{x}}>\frac{c_\peq{y}}{p_\peq{y}}$
				\item Restricción presupuestaria: $p_\peq{x}\cdot x + p_\peq{y}\cdot y\leq m$
				\item Restricción de ingesta calórica: $c_\peq{x}\cdot x + c_\peq{y}\cdot y\geq c^\peq{*}$
			\end{itemize}
		\end{frame}			

		\begin{frame}
			\frametitle{Aplicación 1: Comportamiento Giffen}
			\centering
			\begin{tikzpicture}[scale=.5]
				\draw [fill,opacity=.2,blue] (0,5)--(0,0)--(5,0);
				%\draw [dashed,help lines] (0,2.5)--(2.5,2.5)--(2.5,0);
				%\draw [dashed,help lines] (5/3,0)--(5/3,10/3)--(0,10/3);
				\draw [ultra thick,red] plot [domain=0:5] (\x,{5-\x});
				%\draw [ultra thick,purple] plot [domain=0:10/4] (\x,{10-4*\x});
				%\draw [ultra thick,teal] plot [domain=1.5:4] (\x,{6.25/\x});
				%\draw [ultra thick,teal] plot [domain=3:8.2] (\x,{50/9/\x});
				\draw [<->,thick] (0,10.7)--(0,0)--(10.7,0);
				%\draw [fill,blue] (2.5,2.5) circle [radius=.2];
				%\node [right,blue] at (2.5,2.5) {\textbf{A}};
				\node [left] at (0,10.7) {$y$};
				\node [below] at (10.7,0) {$x$};
				%\node [left] at (0,2.5) {\tiny $y^\peq{*}$};
				%\node [below] at (2.5,0) {\tiny $x^\peq{*}$};
			\end{tikzpicture}
		\end{frame}	

		\begin{frame}
			\frametitle{Aplicación 1: Comportamiento Giffen}
			\centering
			\begin{tikzpicture}[scale=.5]
				\draw [fill,opacity=.2,blue] (0,5)--(0,0)--(5,0);
				\draw [dashed,help lines] (0,2.5)--(2.5,2.5)--(2.5,0);
				%\draw [dashed,help lines] (5/3,0)--(5/3,10/3)--(0,10/3);
				\draw [ultra thick,red] plot [domain=0:5] (\x,{5-\x});
				%\draw [ultra thick,purple] plot [domain=0:10/4] (\x,{10-4*\x});
				\draw [ultra thick,teal] plot [domain=1.5:4] (\x,{6.25/\x});
				%\draw [ultra thick,teal] plot [domain=3:8.2] (\x,{50/9/\x});
				\draw [<->,thick] (0,10.7)--(0,0)--(10.7,0);
				\draw [fill,blue] (2.5,2.5) circle [radius=.2];
				\node [right,blue] at (2.5,2.5) {\textbf{A}};
				\node [left] at (0,10.7) {$y$};
				\node [below] at (10.7,0) {$x$};
				\node [left] at (0,2.5) {\tiny $y^\peq{*}$};
				\node [below] at (2.5,0) {\tiny $x^\peq{*}$};
			\end{tikzpicture}
		\end{frame}	

		\begin{frame}
			\frametitle{Aplicación 1: Comportamiento Giffen}
			\centering
			\begin{tikzpicture}[scale=.5]
				\draw [fill,opacity=.2,blue] (10/3,5/3)--(10/2.5,0)--(5,0);
				%\draw [dashed,help lines] (0,2.5)--(2.5,2.5)--(2.5,0);
				%\draw [dashed,help lines] (10/3,0)--(10/3,5/3)--(0,5/3);
				\draw [ultra thick,red] plot [domain=0:5] (\x,{5-\x});
				\draw [ultra thick,purple] plot [domain=0:10/2.5] (\x,{10-2.5*\x});
				%\draw [ultra thick,teal] plot [domain=1.5:4] (\x,{6.25/\x});
				%\draw [ultra thick,teal] plot [domain=3:8.2] (\x,{50/9/\x});
				\draw [<->,thick] (0,10.7)--(0,0)--(10.7,0);
				%\draw [fill,blue] (2.5,2.5) circle [radius=.2];
				%\node [right,blue] at (2.5,2.5) {\textbf{A}};
				\node [left] at (0,10.7) {$y$};
				\node [below] at (10.7,0) {$x$};
				%\node [left] at (0,2.5) {\tiny $y^\peq{*}$};
				%\node [below] at (2.5,0) {\tiny $x^\peq{*}$};
			\end{tikzpicture}
		\end{frame}	

		\begin{frame}
			\frametitle{Aplicación 1: Comportamiento Giffen}
			\centering
			\begin{tikzpicture}[scale=.5]
				\draw [fill,opacity=.2,blue] (10/3,5/3)--(10/2.5,0)--(5,0);
				\draw [dashed,help lines] (0,2.5)--(2.5,2.5)--(2.5,0);
				%\draw [dashed,help lines] (10/3,0)--(10/3,5/3)--(0,5/3);
				\draw [ultra thick,red] plot [domain=0:5] (\x,{5-\x});
				\draw [ultra thick,purple] plot [domain=0:10/2.5] (\x,{10-2.5*\x});
				%\draw [ultra thick,teal] plot [domain=1.5:4] (\x,{6.25/\x});
				%\draw [ultra thick,teal] plot [domain=3:8.2] (\x,{50/9/\x});
				\draw [<->,thick] (0,10.7)--(0,0)--(10.7,0);
				\draw [fill,blue] (2.5,2.5) circle [radius=.2];
				\node [right,blue] at (2.5,2.5) {\textbf{A}};
				\node [left] at (0,10.7) {$y$};
				\node [below] at (10.7,0) {$x$};
				%\node [left] at (0,2.5) {\tiny $y^\peq{*}$};
				%\node [below] at (2.5,0) {\tiny $x^\peq{*}$};
			\end{tikzpicture}
		\end{frame}	

		\begin{frame}
			\frametitle{Aplicación 1: Comportamiento Giffen}
			\centering
			\begin{tikzpicture}[scale=.5]
				\draw [fill,opacity=.2,blue] (10/3,5/3)--(10/2.5,0)--(5,0);
				%\draw [dashed,help lines] (0,2.5)--(2.5,2.5)--(2.5,0);
				\draw [dashed,help lines] (10/3,0)--(10/3,5/3)--(0,5/3);
				\draw [ultra thick,red] plot [domain=0:5] (\x,{5-\x});
				\draw [ultra thick,purple] plot [domain=0:10/2.5] (\x,{10-2.5*\x});
				%\draw [ultra thick,teal] plot [domain=1.5:4] (\x,{6.25/\x});
				\draw [ultra thick,teal] plot [domain=2:5] (\x,{50/9/\x});
				\draw [<->,thick] (0,10.7)--(0,0)--(10.7,0);
				\draw [fill,blue] (10/3,5/3) circle [radius=.2];
				\node [right,blue] at (10/3,5/3) {\textbf{B}};
				\node [left] at (0,10.7) {$y$};
				\node [below] at (10.7,0) {$x$};
				\node [left] at (0,5/3) {\tiny $y^\peq{*}$};
				\node [below] at (10/3,0) {\tiny $x^\peq{*}$};
			\end{tikzpicture}
		\end{frame}	

		\begin{frame}
			\frametitle{Aplicación 1: Comportamiento Giffen}
			\begin{figure}[htbp!]
				\centering
				\begin{tikzpicture}[scale=.45]
							\draw [dashed,help lines] (0,2.8)--(3.2,2.8)--(3.2,0);
							\draw [dashed,help lines] (3.764705882,0)--(3.764705882,1.294117647)--(0,1.294117647);
							\draw [ultra thick,red] plot [domain=0:6] (\x,{6-\x});
							\draw [ultra thick,purple] plot [domain=0:17/4] (\x,{34/3-8/3*\x});
							\draw [ultra thick,red] plot [domain=0:24/5] (\x,{-5/4*\x+6});
							\draw [ultra thick,teal] plot [domain=1:5] (\x,{\x-6.924348870*\x^(1/2)+11.98665182});
							\draw [ultra thick,teal] plot [domain=1:5] (\x,{\x-6.155755836*\x^(1/2)+9.473332478});
							\draw [<->,thick] (0,12)--(0,0)--(12,0);
							\draw [fill,blue] (3.764705882,1.294117647) circle [radius=.2];
							\node [right,blue] at (3.764705882,1.294117647) {\textbf{B}};
							\draw [fill,blue] (3.2,2.8) circle [radius=.2];
							\node [right,blue] at (3.2,2.8) {\textbf{A}};
							\node [left] at (0,12) {$y$};
							\node [right,purple] at (0,34/3) {\tiny $RC$};
							\node [below] at (12,0) {$x$};
							\node [left] at (0,2.8) {\tiny $y^\peq{*}_\peq{A}$};
							\node [left] at (0,1.294117647) {\tiny $y^\peq{*}_\peq{B}$};
							\node [below] at (3.764705882,0) {\tiny $x^\peq{*}_\peq{B}$};
							\node [below] at (3.2,0) {\tiny $x^\peq{*}_\peq{A}$};
							\node [right,teal] at (5,1.50333705) {\tiny $I_\peq{0}$};
							\node [right,teal] at (5,.70864398) {\tiny $I_\peq{1}$};
							\node [below,red] at (6,0) {\tiny $RP_\peq{0}$};
							\node [below,red] at (24/5,0) {\tiny $RP_\peq{1}$};
							\node [below] at (0,0) {\tiny 0};
						\end{tikzpicture}
			\end{figure}
		\end{frame}			

		\begin{frame}
			\frametitle{Aplicación 1: Comportamiento Giffen}
			\begin{figure}[htbp!]
				\centering
				\begin{tikzpicture}[scale=.9]
							\draw [dashed,help lines] (0,2.8)--(3.2,2.8)--(3.2,0);
							\draw [dashed,help lines] (3.764705882,0)--(3.764705882,1.294117647)--(0,1.294117647);
							\draw [ultra thick,red] plot [domain=0:6] (\x,{6-\x});
							\draw [ultra thick,purple] plot [domain=13/8:17/4] (\x,{34/3-8/3*\x});
							\draw [ultra thick,red] plot [domain=0:24/5] (\x,{-5/4*\x+6});
							\draw [ultra thick,teal] plot [domain=1:5] (\x,{\x-6.924348870*\x^(1/2)+11.98665182});
							\draw [ultra thick,teal] plot [domain=1:5] (\x,{\x-6.155755836*\x^(1/2)+9.473332478});
							\draw [<->,thick] (0,7)--(0,0)--(7,0);
							\draw [fill,blue] (3.764705882,1.294117647) circle [radius=.15];
							\node [right,blue] at (3.764705882,1.294117647) {\textbf{B}};
							\draw [fill,blue] (3.2,2.8) circle [radius=.15];
							\node [right,blue] at (3.2,2.8) {\textbf{A}};
							\node [left] at (0,7) {$y$};
							\node [right,purple] at (13/8,7) {\tiny $RC$};
							\node [below] at (7,0) {$x$};
							\node [left] at (0,2.8) {\tiny $y^\peq{*}_\peq{A}$};
							\node [left] at (0,1.294117647) {\tiny $y^\peq{*}_\peq{B}$};
							\node [below] at (3.764705882,0) {\tiny $x^\peq{*}_\peq{B}$};
							\node [below] at (3.2,0) {\tiny $x^\peq{*}_\peq{A}$};
							\node [right,teal] at (5,1.50333705) {\tiny $I_\peq{0}$};
							\node [right,teal] at (5,.70864398) {\tiny $I_\peq{1}$};
							\node [above right,red] at (6,0) {\tiny $RP_\peq{0}$};
							\node [above right,red] at (24/5,0) {\tiny $RP_\peq{1}$};
							\node [below] at (0,0) {\tiny 0};
						\end{tikzpicture}
			\end{figure}
		\end{frame}	

		\begin{frame}
			\frametitle{Aplicación 1: Comportamiento Giffen}
			\begin{figure}[htbp!]
				\centering
				\begin{subfigure}[b]{0.49\textwidth}
					\resizebox{\linewidth}{!}{
						\begin{tikzpicture}[scale=.5]
							\draw [dashed,help lines] (0,2.8)--(3.2,2.8)--(3.2,0);
							\draw [dashed,help lines] (3.764705882,0)--(3.764705882,1.294117647)--(0,1.294117647);
							\draw [ultra thick,red] plot [domain=0:6] (\x,{6-\x});
							\draw [ultra thick,purple] plot [domain=0:17/4] (\x,{34/3-8/3*\x});
							\draw [ultra thick,red] plot [domain=0:24/5] (\x,{-5/4*\x+6});
							\draw [ultra thick,teal] plot [domain=1:5] (\x,{\x-6.924348870*\x^(1/2)+11.98665182});
							\draw [ultra thick,teal] plot [domain=1:5] (\x,{\x-6.155755836*\x^(1/2)+9.473332478});
							\draw [<->,thick] (0,12)--(0,0)--(12,0);
							\draw [fill,blue] (3.764705882,1.294117647) circle [radius=.2];
							\node [right,blue] at (3.764705882,1.294117647) {\textbf{B}};
							\draw [fill,blue] (3.2,2.8) circle [radius=.2];
							\node [right,blue] at (3.2,2.8) {\textbf{A}};
							\node [left] at (0,12) {$y$};
							\node [right,purple] at (0,34/3) {\tiny $RC$};
							\node [below] at (12,0) {$x$};
							\node [left] at (0,2.8) {\tiny $y^\peq{*}_\peq{A}$};
							\node [left] at (0,1.294117647) {\tiny $y^\peq{*}_\peq{B}$};
							\node [below] at (3.764705882,0) {\tiny $x^\peq{*}_\peq{B}$};
							\node [below] at (3.2,0) {\tiny $x^\peq{*}_\peq{A}$};
							\node [right,teal] at (5,1.50333705) {\tiny $I_\peq{0}$};
							\node [right,teal] at (5,.70864398) {\tiny $I_\peq{1}$};
							\node [below,red] at (6,0) {\tiny $RP_\peq{0}$};
							\node [below,red] at (24/5,0) {\tiny $RP_\peq{1}$};
							\node [below] at (0,0) {\tiny 0};
						\end{tikzpicture}
					}
				\end{subfigure}
				\begin{subfigure}[b]{0.49\textwidth}
					\resizebox{\linewidth}{!}{
						\begin{tikzpicture}[scale=.5]
							\draw [dashed,help lines] (0,5.2)--(3.2,5.2)--(3.2,0);
							\draw [dashed,help lines] (0,5.764705882)--(3.764705882,5.764705882)--(3.764705882,0);
							\draw [ultra thick,teal] plot [domain=0:10] (\x,{2+\x});
							\draw [<->,thick] (0,12)--(0,0)--(12,0);
							\draw [fill,blue] (3.2,5.2) circle [radius=.2];
							\node [right,blue] at (3.2,5.2) {\textbf{A}};
							\draw [fill,blue] (3.764705882,5.764705882) circle [radius=.2];
							\node [right,blue] at (3.764705882,5.764705882) {\textbf{B}};
							\node [left] at (0,12) {$p_\peq{x}$};
							\node [below] at (12,0) {$q^\peq{d}_\peq{x}$};
							\node [left] at (0,5.2) {\tiny $p^\peq{0}_\peq{x}$};
							\node [below] at (3.2,0) {\tiny $x^\peq{*}_\peq{A}$};
							\node [left] at (0,5.764705882) {\tiny $p^\peq{1}_\peq{x}$};
							\node [below] at (3.764705882,0) {\tiny $x^\peq{*}_\peq{B}$};
							\node [right,teal] at (10,12) {$D_\peq{x}$};
							\node [below] at (0,0) {\tiny 0};
						\end{tikzpicture}
					}
				\end{subfigure}
			\end{figure}	
		\end{frame}

		\begin{frame}
			\frametitle{Aplicación 2: Oferta de Trabajo}
			\begin{mydef}
				\begin{itemize}
					\item \textbf{Cantidad ofrecida de trabajo ($\mathbf{L^\peq{s}}$):} Cantidad de horas que los trabajadores quieren y pueden trabajar.
					\item \textbf{Oferta de trabajo:} Relación entre $L^\peq{s}$ y el salario por hora de trabajo ($w$), manteniendo otros determiantes de $L^\peq{s}$ constantes.
				\end{itemize}
			\end{mydef}
		\end{frame}	

		\begin{frame}
			\frametitle{Aplicación 2: Oferta de Trabajo}
			Modelo neoclásico de elección ocio-consumo
			\begin{itemize}
				\item Preferencias sobre consumo y ocio: $u\rp{C,O}$
				\item Restricción presupuestaria: $C\leq w\cdot L + Z$
				\item Restricción de tiempo: $T=L+O$
				\item Combinando las restricciones: $C\leq\rp{w\cdot T+Z} - w\cdot O$
			\end{itemize}
		\end{frame}	
			
		\begin{frame}
			\frametitle{Aplicación 2: Oferta de Trabajo}
			\centering
			\begin{tikzpicture}[scale=.05]
				\draw [fill,blue,opacity=.2] (0,0)--(0,100)--(80,20)--(80,0);
				\draw [ultra thick,dashed,teal] (0,20)--(80,20);
				\draw [ultra thick,teal] (80,0)--(80,20);
				\draw [ultra thick,teal] plot [domain=0:80] (\x,{100-\x});
				\draw [<->,thick] (0,120)--(0,0)--(120,0);
				\draw [fill,blue] (0,100) circle [radius=2];
				\draw [fill,blue] (80,20) circle [radius=2];
				\node [left] at (0,120) {$C$};
				\node [below] at (120,0) {$O$};
				\node [left] at (0,100) {\tiny $w\cdot T+Z$};
				\node [left] at (0,20) {\tiny $Z$};
				\node [below] at (80,0) {\tiny $T$};
				\node [below] at (0,0) {\tiny 0};
			\end{tikzpicture}
		\end{frame}				

		\begin{frame}
			\frametitle{Aplicación 2: Oferta de Trabajo}
			\centering
			\begin{tikzpicture}[scale=.05]
				%\draw [fill,blue,opacity=.2] (0,0)--(0,100)--(80,20)--(80,0);
				\draw [dashed] (0,20)--(80,20);
				\draw [dashed] (80,0)--(80,20);
				\draw [dashed,help lines] (0,50)--(50,50)--(50,0);
				\draw [ultra thick,teal] plot [domain=0:80] (\x,{100-\x});
				\draw [ultra thick,purple] plot [domain=30:80](\x,{2500/\x});
				\draw [<->,thick] (0,120)--(0,0)--(120,0);
				%\draw [fill,blue] (0,100) circle [radius=2];
				%\draw [fill,blue] (80,20) circle [radius=2];
				\draw [fill,blue] (50,50) circle [radius=2];
				\node [above right,blue] at (50,50) {\textbf{A}};
				\node [left] at (0,120) {$C$};
				\node [below] at (120,0) {$O$};
				\node [left] at (0,100) {\tiny $w\cdot T+Z$};
				\node [left] at (0,20) {\tiny $Z$};
				\node [below] at (80,0) {\tiny $T$};
				\node [below] at (0,0) {\tiny 0};
				\node [below] at (50,0) {\tiny $O^\peq{*}_\peq{A}$};
				\node [left] at (0,50) {\tiny $C^\peq{*}_\peq{A}$};
			\end{tikzpicture}
		\end{frame}
					
		\begin{frame}
			\frametitle{Aplicación 2: Oferta de Trabajo}
			\centering
			\begin{tikzpicture}[scale=.05]
				\draw [dashed,help lines] (0,20)--(80,20);
				\draw [dashed,help lines] (0,40)--(80,40)--(80,0);
				\draw [dashed,help lines] (0,50)--(50,50)--(50,0);
				\draw [dashed,help lines] (0,60)--(60,60)--(60,0);
				\draw [ultra thick,teal] plot [domain=0:80] (\x,{100-\x});
				\draw [ultra thick,teal] plot [domain=0:80] (\x,{120-\x});
				\draw [ultra thick,purple] plot [domain=30:80](\x,{2500/\x});
				\draw [ultra thick,purple] plot [domain=40:80](\x,{3600/\x});
				\draw [<->,thick] (0,130)--(0,0)--(120,0);
				%\draw [fill,blue] (0,100) circle [radius=2];
				%\draw [fill,blue] (80,20) circle [radius=2];
				\draw [fill,blue] (50,50) circle [radius=2];
				\node [above right,blue] at (50,50) {\textbf{A}};
				\draw [fill,blue] (60,60) circle [radius=2];
				\node [above right,blue] at (60,60) {\textbf{B}};
				\node [left] at (0,130) {$C$};
				\node [below] at (120,0) {$O$};
				\node [left] at (0,100) {\tiny $w\cdot T+Z_\peq{0}$};
				\node [left] at (0,120) {\tiny $w\cdot T+Z_\peq{1}$};
				\node [left] at (0,20) {\tiny $Z_\peq{0}$};
				\node [left] at (0,40) {\tiny $Z_\peq{1}$};
				\node [below] at (80,0) {\tiny $T$};
				\node [below] at (0,0) {\tiny 0};
				\node [below] at (50,0) {\tiny $O^\peq{*}_\peq{A}$};
				\node [left] at (0,50) {\tiny $C^\peq{*}_\peq{A}$};
				\node [below] at (60,0) {\tiny $O^\peq{*}_\peq{B}$};
				\node [left] at (0,60) {\tiny $C^\peq{*}_\peq{B}$};
			\end{tikzpicture}
		\end{frame}

		\begin{frame}
			\frametitle{Aplicación 2: Oferta de Trabajo}
			\centering
			\begin{tikzpicture}[scale=2.4]
				\draw [dashed,help lines] (0,9)--(1,9)--(1,8);
				\draw [dashed,help lines] (0,9.8)--(.6,9.8)--(.6,8);
				\draw [dashed,help lines] (1.8,8)--(1.8,8.2)--(0,8.2);
				\draw [dashed,help lines] (1.8,8.2)--(1.8,8.6)--(0,8.6);
				\draw [ultra thick,teal] plot [domain=0:1.8] (\x,{10-\x});
				\draw [ultra thick,teal] plot [domain=0:1.8] (\x,{52/5-\x});
				\draw [ultra thick,purple] plot [domain=.6:1.5] (\x,{-1*exp(.5*ln(\x-.5)+.3465735903)+10});
				\draw [ultra thick,purple] plot [domain=.500009:1.5] (\x,{-1*exp(.5*ln(\x-.5)-.4581453655)+10});
				\draw [<->,thick] (0,10.7)--(0,8)--(2.7,8);
				\draw [fill,blue] (1,9) circle [radius=.05];
				\node [above right,blue] at (1,9) {\textbf{A}};
				\draw [fill,blue] (.6,9.8) circle [radius=.05];
				\node [above right,blue] at (.6,9.8) {\textbf{B}};
				\node [left] at (0,10.7) {$C$};
				\node [below] at (2.7,8) {$O$};
				\node [left] at (0,9) {\tiny $C^\peq{*}_\peq{A}$};
				\node [below] at (1,8) {\tiny $O^\peq{*}_\peq{A}$};
				\node [left] at (0,9.8) {\tiny $C^\peq{*}_\peq{B}$};
				\node [below] at (.6,8) {\tiny $O^\peq{*}_\peq{B}$};
				\node [left] at (0,8.2) {\tiny $Z_\peq{0}$};
				\node [left] at (0,8.6) {\tiny $Z_\peq{1}$};
				\node [below] at (1.8,8) {\tiny $T$};
				\node [left] at (0,52/5) {\tiny $w\cdot T+Z_\peq{1}$};
				\node [left] at (0,10) {\tiny $w\cdot T+Z_\peq{0}$};
				\node [below] at (0,8) {\tiny 0};
			\end{tikzpicture}
		\end{frame}	

\begin{frame}
			\frametitle{Aplicación 2: Oferta de Trabajo}
			\centering
			\begin{tikzpicture}[scale=.5]
				\draw [dashed,help lines] (0,5)--(5,5)--(5,-2);
				\draw [dashed,help lines] (6,-2)--(6,8/3)--(0,8/3);
				\draw [dashed,help lines] (0,0)--(10,0)--(10,-2);
				\draw [ultra thick,teal] plot [domain=0:10] (\x,{10-\x});
				\draw [ultra thick,teal] plot [domain=0:10] (\x,{-2/3*\x+20/3});
				\draw [ultra thick,purple] plot [domain=3:8] (\x,{-4*\x^(1/2)*5^(1/2)+\x+20});
				\draw [ultra thick,purple] plot [domain=3:8] (\x,{-10/3*\x^(1/2)*6^(1/2)+\x+50/3});
				\draw [<->,thick] (0,10.7)--(0,-2)--(11.7,-2);
				\draw [fill,blue] (5,5) circle [radius=.2];
				\node [right,blue] at (5,5) {\textbf{B}};
				\draw [fill,blue] (6,8/3) circle [radius=.2];
				\node [right,blue] at (6,8/3) {\textbf{A}};
				\node [left] at (0,10.7) {$C$};
				\node [below] at (11.7,-2) {$O$};
				\node [left] at (0,8/3) {\tiny $C^\peq{*}_\peq{A}$};
				\node [below] at (6,-2) {\tiny $O^\peq{*}_\peq{A}$};
				\node [left] at (0,5) {\tiny $C^\peq{*}_\peq{B}$};
				\node [left] at (0,10) {\tiny $w_\peq{1}\cdot T+Z$};
				\node [left] at (0,20/3) {\tiny $w_\peq{0}\cdot T+Z$};
				\node [left] at (0,0) {\tiny $Z$};
				\node [below] at (5,-2) {\tiny $O^\peq{*}_\peq{B}$};
				\node [below] at (10,-2) {\tiny $T$};
				\node [below] at (0,-2) {\tiny 0};
			\end{tikzpicture}
		\end{frame}	

		\begin{frame}
			\frametitle{Aplicación 2: Oferta de Trabajo}
			\begin{figure}[htbp!]
				\centering
				\begin{subfigure}[b]{0.49\textwidth}
					\resizebox{\linewidth}{!}{
						\begin{tikzpicture}[scale=.5]
							\draw [dashed,help lines] (0,5)--(5,5)--(5,-2);
							\draw [dashed,help lines] (6,-2)--(6,8/3)--(0,8/3);
							\draw [dashed,help lines] (0,0)--(10,0)--(10,-2);
							\draw [ultra thick,teal] plot [domain=0:10] (\x,{10-\x});
							\draw [ultra thick,teal] plot [domain=0:10] (\x,{-2/3*\x+20/3});
							\draw [ultra thick,purple] plot [domain=3:8] (\x,{-4*\x^(1/2)*5^(1/2)+\x+20});
							\draw [ultra thick,purple] plot [domain=3:8] (\x,{-10/3*\x^(1/2)*6^(1/2)+\x+50/3});
							\draw [<->,thick] (0,10.7)--(0,-2)--(11.7,-2);
							\draw [fill,blue] (5,5) circle [radius=.2];
							\node [right,blue] at (5,5) {\textbf{B}};
							\draw [fill,blue] (6,8/3) circle [radius=.2];
							\node [right,blue] at (6,8/3) {\textbf{A}};
							\node [left] at (0,10.7) {$C$};
							\node [below] at (11.7,-2) {$O$};
							\node [left] at (0,8/3) {\tiny $C^\peq{*}_\peq{A}$};
							\node [below] at (6,-2) {\tiny $O^\peq{*}_\peq{A}$};
							\node [left] at (0,5) {\tiny $C^\peq{*}_\peq{B}$};
							\node [left] at (0,10) {\tiny $w_\peq{1}\cdot T+Z$};
							\node [left] at (0,20/3) {\tiny $w_\peq{0}\cdot T+Z$};
							\node [left] at (0,0) {\tiny $Z$};
							\node [below] at (5,-2) {\tiny $O^\peq{*}_\peq{B}$};
							\node [below] at (10,-2) {\tiny $T$};
							\node [below] at (0,-2) {\tiny 0};
						\end{tikzpicture}
					}
				\end{subfigure}
				\begin{subfigure}[b]{0.49\textwidth}
					\resizebox{\linewidth}{!}{
						\begin{tikzpicture}[scale=.5]
							\draw [dashed,help lines] (0,4)--(2,4)--(2,0);
							\draw [dashed,help lines] (0,8)--(6,8)--(6,0);
							\draw [ultra thick,teal] plot [domain=0:10] (\x,{2+\x});
							\draw [<->,thick] (0,12)--(0,0)--(12,0);
							\draw [fill,blue] (2,4) circle [radius=.2];
							\node [right,blue] at (2,4) {\textbf{A}};
							\draw [fill,blue] (6,8) circle [radius=.2];
							\node [right,blue] at (6,8) {\textbf{B}};
							\node [left] at (0,12) {$w$};
							\node [below] at (12,0) {$L^\peq{s}$};
							\node [left] at (0,4) {\tiny $w_\peq{0}$};
							\node [below] at (2,0) {\tiny $\rp{T-O^\peq{*}_\peq{A}}$};
							\node [left] at (0,8) {\tiny $w_\peq{1}$};
							\node [below] at (6,0) {\tiny $\rp{T-O^\peq{*}_\peq{B}}$};
							\node [right,teal] at (10,12) {$S_\peq{L}$};
							\node [below] at (0,0) {\tiny 0};
						\end{tikzpicture}
					}
				\end{subfigure}
			\end{figure}	
		\end{frame}

		\begin{frame}
			\frametitle{Aplicación 2: Oferta de Trabajo}
			\begin{itemize}
				\item Si domina el efecto sustitución $\implies$ oferta laboral tiene pendiente positiva
				\item Si domina el efecto ingreso $\implies$ oferta laboral tiene pendiente negativa
				\item Decisión de participación laboral está determinada por el salario de reserva...
			\end{itemize}
		\end{frame}	

		\begin{frame}
			\frametitle{Aplicación 2: Oferta de Trabajo}
			\begin{mydef}
				\begin{itemize}
					\item \textbf{Salario de reserva ($\mathbf{\tilde{w}}$):} Salario que deja al consumidor indiferente entre trabajar y no trabajar. $$L^\peq{*}\left\{
								\begin{array}{ll}
									\geq 0 & \mbox{si } w \geq \tilde{w} \\
									 = 0	 & \mbox{si } w < \tilde{w} 
								\end{array}
							\right.$$
					\item \textbf{Utilidad de reserva ($\mathbf{\tilde{u}}$):} Nivel de utilidad que el individuo obtiene si decide no trabajar. $$\tilde{u}=u(Z,T)$$
				\end{itemize}
			\end{mydef}
		\end{frame}	

\begin{frame}
			\frametitle{Aplicación 2: Oferta de Trabajo}
			\centering
			\begin{tikzpicture}[scale=.45]
				%\draw [dashed,help lines] (0,5)--(5,5)--(5,-2);
				%\draw [dashed,help lines] (6,-2)--(6,8/3)--(0,8/3);
				\draw [dashed,help lines] (0,2)--(10,2)--(10,-2);
				%\draw [ultra thick,teal] (0,4)--(10,2);
				%\draw [ultra thick,teal] plot [domain=0:10] (\x,{10-\x});
				%\draw [ultra thick,teal] plot [domain=0:10] (\x,{-2/3*\x+20/3});
				%\draw [ultra thick,teal] plot [domain=0:10] (\x,{6.472135950-.4472135950*\x});
				%\draw [ultra thick,purple] plot [domain=3:13] (\x,{-4*\x^(1/2)*5^(1/2)+\x+20});
				%\draw [ultra thick,purple] plot [domain=3:13] (\x,{-10/3*\x^(1/2)*6^(1/2)+\x+50/3});
				\draw [ultra thick,purple] plot [domain=7:12.5] (\x,{-2*sqrt(\x)*sqrt(5)*sqrt(2)+4*sqrt(5)-2*sqrt(\x)*sqrt(2)+\x+12});
				%\draw [ultra thick,purple] plot [domain=3:13] (\x,{-2*sqrt(\x)*sqrt(10)+\x+10});
				\draw [<->,thick] (0,10.7)--(0,-2)--(13,-2);
				\draw [fill,blue] (10,2) circle [radius=.2];
				\node [above right,blue] at (10,2) {\textbf{R}};
				%\draw [fill,blue] (6,8/3) circle [radius=.2];
				%\node [right,blue] at (6,8/3) {\textbf{A}};
				\node [left] at (0,10.7) {$C$};
				\node [below] at (13,-2) {$O$};
				%\node [left] at (0,8/3) {\tiny $C^\peq{*}_\peq{A}$};
				%\node [below] at (6,-2) {\tiny $O^\peq{*}_\peq{A}$};
				%\node [left] at (0,5) {\tiny $C^\peq{*}_\peq{B}$};
				%\node [left] at (0,10) {\tiny $w_\peq{1}\cdot T+Z$};
				%\node [left] at (0,20/3) {\tiny $w_\peq{0}\cdot T+Z$};
				\node [left] at (0,2) {\tiny $Z$};
				%\node [below] at (5,-2) {\tiny $O^\peq{*}_\peq{B}$};
				\node [below] at (10,-2) {\tiny $T$};
				\node [right,purple] at (12.5,1.08359214) {\tiny $\tilde{u}$};
				\node [below] at (0,-2) {\tiny 0};
			\end{tikzpicture}
		\end{frame}	

		\begin{frame}
			\frametitle{Aplicación 2: Oferta de Trabajo}
			\centering
			\begin{tikzpicture}[scale=.45]
				%\draw [dashed,help lines] (0,5)--(5,5)--(5,-2);
				%\draw [dashed,help lines] (6,-2)--(6,8/3)--(0,8/3);
				\draw [dashed,help lines] (0,2)--(10,2)--(10,-2);
				%\draw [ultra thick,teal] (0,4)--(10,2);
				%\draw [ultra thick,teal] plot [domain=0:10] (\x,{10-\x});
				%\draw [ultra thick,teal] plot [domain=0:10] (\x,{-2/3*\x+20/3});
				\draw [ultra thick,teal] plot [domain=0:10] (\x,{6.472135950-.4472135950*\x});
				%\draw [ultra thick,purple] plot [domain=3:13] (\x,{-4*\x^(1/2)*5^(1/2)+\x+20});
				%\draw [ultra thick,purple] plot [domain=3:13] (\x,{-10/3*\x^(1/2)*6^(1/2)+\x+50/3});
				\draw [ultra thick,purple] plot [domain=7:12.5] (\x,{-2*sqrt(\x)*sqrt(5)*sqrt(2)+4*sqrt(5)-2*sqrt(\x)*sqrt(2)+\x+12});
				%\draw [ultra thick,purple] plot [domain=3:13] (\x,{-2*sqrt(\x)*sqrt(10)+\x+10});
				\draw [<->,thick] (0,10.7)--(0,-2)--(13,-2);
				\draw [fill,blue] (10,2) circle [radius=.2];
				\node [above right,blue] at (10,2) {\textbf{R}};
				%\draw [fill,blue] (6,8/3) circle [radius=.2];
				%\node [right,blue] at (6,8/3) {\textbf{A}};
				\node [left] at (0,10.7) {$C$};
				\node [below] at (13,-2) {$O$};
				%\node [left] at (0,8/3) {\tiny $C^\peq{*}_\peq{A}$};
				%\node [below] at (6,-2) {\tiny $O^\peq{*}_\peq{A}$};
				%\node [left] at (0,5) {\tiny $C^\peq{*}_\peq{B}$};
				\node [left] at (0,6.472135950) {\tiny $\tilde{w}\cdot T+Z$};
				%\node [left] at (0,20/3) {\tiny $w_\peq{0}\cdot T+Z$};
				\node [left] at (0,2) {\tiny $Z$};
				%\node [below] at (5,-2) {\tiny $O^\peq{*}_\peq{B}$};
				\node [below] at (10,-2) {\tiny $T$};
				\node [right,purple] at (12.5,1.08359214) {\tiny $\tilde{u}$};
				\node [below] at (0,-2) {\tiny 0};
			\end{tikzpicture}
		\end{frame}	

		\begin{frame}
			\frametitle{Aplicación 2: Oferta de Trabajo}
			\centering
			\begin{tikzpicture}[scale=.45]
				%\draw [dashed,help lines] (0,5)--(5,5)--(5,-2);
				%\draw [dashed,help lines] (6,-2)--(6,8/3)--(0,8/3);
				\draw [dashed,help lines] (0,2)--(10,2)--(10,-2);
				\draw [ultra thick,teal] (0,4)--(10,2);
				\draw [ultra thick,teal] (0,10)--(10,2);
				\draw [ultra thick,teal] plot [domain=0:10] (\x,{6.472135950-.4472135950*\x});
				%\draw [ultra thick,purple] plot [domain=3:13] (\x,{-4*\x^(1/2)*5^(1/2)+\x+20});
				%\draw [ultra thick,purple] plot [domain=3:13] (\x,{-10/3*\x^(1/2)*6^(1/2)+\x+50/3});
				\draw [ultra thick,purple] plot [domain=7:12.5] (\x,{-2*sqrt(\x)*sqrt(5)*sqrt(2)+4*sqrt(5)-2*sqrt(\x)*sqrt(2)+\x+12});
				\draw [ultra thick,purple] plot [domain=4.5:10] (\x,{-3*sqrt(\x)*sqrt(10)+\x+45/2});
				%\draw [ultra thick,purple] plot [domain=3:13] (\x,{-2*sqrt(\x)*sqrt(10)+\x+10});
				\draw [<->,thick] (0,10.7)--(0,-2)--(13,-2);
				\draw [fill,blue] (10,2) circle [radius=.2];
				\node [above right,blue] at (10,2) {\textbf{R}};
				\draw [fill,blue] (125/18,40/9) circle [radius=.2];
				\node [above right,blue] at (125/18,40/9) {\textbf{A}};
				\node [left] at (0,10.7) {$C$};
				\node [below] at (13,-2) {$O$};
				%\node [left] at (0,8/3) {\tiny $C^\peq{*}_\peq{A}$};
				%\node [below] at (6,-2) {\tiny $O^\peq{*}_\peq{A}$};
				%\node [left] at (0,5) {\tiny $C^\peq{*}_\peq{B}$};
				\node [left] at (0,6.472135950) {\tiny $\tilde{w}\cdot T+Z$};
				\node [left] at (0,4) {\tiny $w_\peq{0}\cdot T+Z$};
				\node [left] at (0,10) {\tiny $w_\peq{1}\cdot T+Z$};
				\node [left] at (0,2) {\tiny $Z$};
				%\node [below] at (5,-2) {\tiny $O^\peq{*}_\peq{B}$};
				\node [below] at (10,-2) {\tiny $T$};
				\node [right,purple] at (12.5,1.08359214) {\tiny $\tilde{u}$};
				\node [below] at (0,-2) {\tiny 0};
			\end{tikzpicture}
		\end{frame}	

		\begin{frame}
			\frametitle{Aplicación 3: Consumo Intertemporal}
			\begin{itemize}
				\item $c_\peq{t}$: consumo en el peroido $t$
				\item $m_\peq{t}$: ingreso en el peroido $t$
				\item 2 periodos: $t\in\{0,1\}$
				\item El consumidor puede ahorrar o endeudarse: $s=m_\peq{0}-c_\peq{0}$
					\begin{itemize}
						\item $s > 0 \implies \text{ahorro}$
						\item $s < 0 \implies \text{deuda}$
					\end{itemize}
			\end{itemize}
		\end{frame}	

		\begin{frame}
			\frametitle{Aplicación 3: Consumo Intertemporal}
			\begin{itemize}
				\item $r$: tasa de interés. 
					\begin{itemize}
						\item Si ahorra en $t=0$, puede consumir $\$\rp{1+r}s$ adicionales en $t=1$.
						\item Si se endeuda en $t=0$, debe pagar $\$\rp{1+r}s$ en $t=1$.
					\end{itemize}
				\item Restricciones: $$c_\peq{0}+s=m_\peq{0}$$ $$c_\peq{1}\leq m_\peq{1}+\rp{1+r}s$$
				\item Combinando: $$c_\peq{0}+\frac{c_\peq{1}}{\rp{1+r}}\leq m_\peq{0}+\frac{m_\peq{1}}{\rp{1+r}}$$
			\end{itemize}
		\end{frame}	

\begin{frame}
			\frametitle{Aplicación 3: Consumo Intertemporal}
			\centering
			\begin{tikzpicture}[scale=.5]
				\draw [dashed,help lines] (0,4.5)--(5,4.5)--(5,0);
				\draw [fill,opacity=.2,blue] (0,10)--(0,0)--(9.090909091,0);
				\draw [ultra thick,teal] plot [domain=0:9.090909091] (\x,{10-1.1*\x});
				\draw [<->,thick] (0,12)--(0,0)--(13,0);
				\node [left] at (0,12) {$c_\peq{1}$};
				\node [below] at (13,0) {$c_\peq{0}$};
				\draw [fill,blue] (5,4.5) circle [radius=.2];
				\node [above right,blue] at (5,4.5) {\textbf{D}};
				\node [left] at (0,10) {\tiny $\rp{1+r}m_\peq{0}+m_\peq{1}$};
				\node [below] at (9.090909091,0) {\tiny $m_\peq{0}+\frac{m_\peq{1}}{\rp{1+r}}$};
				\node [left] at (0,4.5) {\tiny $m_\peq{1}$};
				\node [below] at (5,0) {\tiny $m_\peq{0}$};
				\node [below] at (0,0) {\tiny 0};
			\end{tikzpicture}
		\end{frame}	

		\begin{frame}
			\frametitle{Aplicación 3: Consumo Intertemporal}
				\begin{itemize}
					\item La pendiente de la restricción presupuestaria es $-\rp{1+r}$ y representa el precio relativo del consumo presente en términos del consumo futuro:
					\begin{itemize}
						\item Consumir $\$1$ adicional en el presente significa dejar de consumir $\$\rp{1+r}$ en el futuro.
					\end{itemize}
				\item Alternativamente, $\frac{1}{\rp{1+r}}$ es el precio relativo del consumo futuro en términos del consumo presente:
					\begin{itemize}
						\item Consumir $\$1$ en el futuro significa dejar de consumir $\$\frac{1}{\rp{1+r}}$ en el presente.
					\end{itemize}
				\end{itemize}
		\end{frame}	

		\begin{frame}
			\frametitle{Aplicación 3: Consumo Intertemporal}
				\begin{itemize}
					\item La restricción presupuestaria intertemporal indica que el valor presente de la corriente de flujos de consumo no puede superar el valor presente de la corriente de flujos de ingreso.
					\item Dicho de otra forma, durante toda su vida, el individuo no puede gastar más que el valor de su riqueza.\footnote{ Ver definición de valor presente en Wikipedia: \url{https://en.wikipedia.org/wiki/Present_value}}
				\end{itemize}
		\end{frame}	

		\begin{frame}
			\frametitle{Aplicación 3: Consumo Intertemporal}
			\centering
			\begin{tikzpicture}[scale=.45]
				\draw [dashed,help lines] (0,4.5)--(5,4.5)--(5,0);
				\draw [dashed,help lines] (3,4.5)--(3,0);
				\draw [ultra thick,teal] plot [domain=0:6] (\x,{9-1.5*\x});
				\draw [ultra thick,teal] plot [domain=0:8] (\x,{12-1.5*\x});
				\draw [<->,thick] (0,13)--(0,0)--(13,0);
				\node [left] at (0,13) {$c_\peq{1}$};
				\node [below] at (13,0) {$c_\peq{0}$};
				\draw [fill,blue] (5,4.5) circle [radius=.2];
				\node [above right,blue] at (5,4.5) {$\text{\textbf{D}}_\peq{\text{\textbf{B}}}$};
				\draw [fill,blue] (3,4.5) circle [radius=.2];
				\node [above right,blue] at (3,4.5) {$\text{\textbf{D}}_\peq{\text{\textbf{A}}}$};
				\node [left] at (0,12) {\tiny $\rp{1+r}w_\peq{B}$};
				\node [below] at (8,0) {\tiny $w_\peq{B}$};
				\node [left] at (0,9) {\tiny $\rp{1+r}w_\peq{A}$};
				\node [below] at (6,0) {\tiny $w_\peq{A}$};
				\node [left] at (0,4.5) {\tiny $m_\peq{1}$};
				\node [below] at (5,0) {\tiny $m_\peq{0}^\peq{B}$};
				\node [below] at (3,0) {\tiny $m_\peq{0}^\peq{A}$};
				\node [below] at (0,0) {\tiny 0};
				\node [left] at (-4,13) {$\Delta^\peq{+}m_\peq{0}:$};
			\end{tikzpicture}
		\end{frame}	

		\begin{frame}
			\frametitle{Aplicación 3: Consumo Intertemporal}
			\centering
			\begin{tikzpicture}[scale=.9]
				\draw [dashed,help lines] (0+2,4.5)--(5,4.5)--(5,0+3);
				\draw [dashed,help lines] (0+2,6.582392777)--(3.611738149,6.582392777)--(3.611738149,0+3);
				\draw [dashed,help lines] (0+2,4.936794582)--(2.708803612,4.936794582)--(2.708803612,0+3);
				\draw [dashed,help lines] (3,4.5)--(3,0+3);
				\draw [ultra thick,teal] plot [domain=0+2:4] (\x,{9-1.5*\x});
				\draw [ultra thick,teal] plot [domain=0+2:6] (\x,{12-1.5*\x});
				\draw [ultra thick,purple] plot [domain=2:4] (\x,{1.234567901*\x-9.001346237*\x^(1/2)+16.40740740});
				\draw [ultra thick,purple] plot [domain=2.5:5] (\x,{1.234567901*\x-10.39385935*\x^(1/2)+21.87654321});
				\draw [<->,thick] (0+2,10)--(0+2,0+3)--(9,0+3);
				\node [left] at (0+2,10) {$c_\peq{1}$};
				\node [below] at (9,0+3) {$c_\peq{0}$};
				\draw [fill,blue] (5,4.5) circle [radius=.1];
				\node [above right,blue] at (5,4.5) {$\text{\textbf{D}}_\peq{\text{\textbf{B}}}$};
				\draw [fill,blue] (3,4.5) circle [radius=.1];
				\node [above right,blue] at (3,4.5) {$\text{\textbf{D}}_\peq{\text{\textbf{A}}}$};
				\draw [fill,blue] (2.708803612,4.936794582) circle [radius=.1];
				\node [above right,blue] at (2.708803612,4.936794582) {\textbf{A}};
				\draw [fill,blue] (3.611738149,6.582392777) circle [radius=.1];
				\node [above right,blue] at (3.611738149,6.582392777) {\textbf{B}};
				\node [left] at (0+2,9) {\tiny $\rp{1+r}w_\peq{B}$};
				\node [below] at (9/1.5,0+3) {\tiny $w_\peq{B}$};
				\node [left] at (0+2,6) {\tiny $\rp{1+r}w_\peq{A}$};
				\node [below] at (6/1.5,0+3) {\tiny $w_\peq{A}$};
				\node [left] at (0+2,4.5) {\tiny $m_\peq{1}$};
				\node [left] at (0+2,4.936794582) {\tiny $c_\peq{1}^\peq{A}$};
				\node [left] at (0+2,6.582392777) {\tiny $c_\peq{1}^\peq{B}$};
				\node [below] at (5,0+3) {\tiny $m_\peq{0}^\peq{B}$};
				\node [below] at (3,0+3) {\tiny $m_\peq{0}^\peq{A}$};
				\node [below] at (2.5,0+3) {\tiny $c_\peq{0}^\peq{A}$};
				\node [below] at (3.611738149,0+3) {\tiny $c_\peq{0}^\peq{B}$};
				\node [below] at (0+2,0+3) {\tiny 0};
				\node [left] at (-2+2,10) {$\Delta^\peq{+}m_\peq{0}:$};
			\end{tikzpicture}
		\end{frame}	

		\begin{frame}
			\frametitle{Aplicación 3: Consumo Intertemporal}
			\centering
			\begin{tikzpicture}[scale=.45]
				\draw [dashed,help lines] (0,4.5)--(5,4.5)--(5,0);
				\draw [dashed,help lines] (5,1.5)--(0,1.5);
				\draw [ultra thick,teal] plot [domain=0:6] (\x,{9-1.5*\x});
				\draw [ultra thick,teal] plot [domain=0:8] (\x,{12-1.5*\x});
				\draw [<->,thick] (0,13)--(0,0)--(13,0);
				\node [left] at (0,13) {$c_\peq{1}$};
				\node [below] at (13,0) {$c_\peq{0}$};
				\draw [fill,blue] (5,4.5) circle [radius=.2];
				\node [above right,blue] at (5,4.5) {$\text{\textbf{D}}_\peq{\text{\textbf{B}}}$};
				\draw [fill,blue] (5,1.5) circle [radius=.2];
				\node [above right,blue] at (5,1.5) {$\text{\textbf{D}}_\peq{\text{\textbf{A}}}$};
				\node [left] at (0,12) {\tiny $\rp{1+r}w_\peq{B}$};
				\node [below] at (8,0) {\tiny $w_\peq{B}$};
				\node [left] at (0,9) {\tiny $\rp{1+r}w_\peq{A}$};
				\node [below] at (6,0) {\tiny $w_\peq{A}$};
				\node [left] at (0,4.5) {\tiny $m_\peq{1}^\peq{B}$};
				\node [left] at (0,1.5) {\tiny $m_\peq{1}^\peq{A}$};
				\node [below] at (5,0) {\tiny $m_\peq{0}$};
				\node [below] at (0,0) {\tiny 0};
				\node [left] at (-4,13) {$\Delta^\peq{+}m_\peq{1}:$};
			\end{tikzpicture}
		\end{frame}	

\begin{frame}
			\frametitle{Aplicación 3: Consumo Intertemporal}
			\centering
			\begin{tikzpicture}[scale=.45]
				\draw [dashed,help lines] (0,4.5)--(5,4.5)--(5,0);
				\draw [dashed,help lines] (5,1.5)--(0,1.5);
				\draw [dashed,help lines] (0,6.582392777)--(3.611738149,6.582392777)--(3.611738149,0);
				\draw [dashed,help lines] (0,4.936794582)--(2.708803612,4.936794582)--(2.708803612,0);
				\draw [ultra thick,teal] plot [domain=0:6] (\x,{9-1.5*\x});
				\draw [ultra thick,teal] plot [domain=0:8] (\x,{12-1.5*\x});
				\draw [ultra thick,purple] plot [domain=1.8:4] (\x,{1.234567901*\x-9.001346237*\x^(1/2)+16.40740740});
				\draw [ultra thick,purple] plot [domain=2.5:5] (\x,{1.234567901*\x-10.39385935*\x^(1/2)+21.87654321});
				\draw [<->,thick] (0,13)--(0,0)--(13,0);
				\node [left] at (0,13) {$c_\peq{1}$};
				\node [below] at (13,0) {$c_\peq{0}$};
				\draw [fill,blue] (5,4.5) circle [radius=.2];
				\node [above right,blue] at (5,4.5) {$\text{\textbf{D}}_\peq{\text{\textbf{B}}}$};
				\draw [fill,blue] (5,1.5) circle [radius=.2];
				\node [above right,blue] at (5,1.5) {$\text{\textbf{D}}_\peq{\text{\textbf{A}}}$};
				\draw [fill,blue] (2.708803612,4.936794582) circle [radius=.2];
				\node [above right,blue] at (2.708803612,4.936794582) {\textbf{A}};
				\draw [fill,blue] (3.611738149,6.582392777) circle [radius=.2];
				\node [above right,blue] at (3.611738149,6.582392777) {\textbf{B}};
				\node [left] at (0,12) {\tiny $\rp{1+r}w_\peq{B}$};
				\node [below] at (8,0) {\tiny $w_\peq{B}$};
				\node [left] at (0,9) {\tiny $\rp{1+r}w_\peq{A}$};
				\node [below] at (6,0) {\tiny $w_\peq{A}$};
				\node [left] at (0,4.5) {\tiny $m_\peq{1}^\peq{B}$};
				\node [left] at (0,1.3) {\tiny $m_\peq{1}^\peq{A}$};
				\node [below] at (5,0) {\tiny $m_\peq{0}$};
				\node [below] at (0,0) {\tiny 0};
				\node [below] at (2.708803612,0) {\tiny $c_\peq{0}^\peq{A}$};
				\node [below] at (3.611738149,0) {\tiny $c_\peq{0}^\peq{B}$};
				\node [left] at (0,5.2) {\tiny $c_\peq{1}^\peq{A}$};
				\node [left] at (0,6.582392777) {\tiny $c_\peq{1}^\peq{B}$};
				\node [left] at (-4,13) {$\Delta^\peq{+}m_\peq{1}:$};
			\end{tikzpicture}
		\end{frame}

		\begin{frame}
			\frametitle{Aplicación 3: Consumo Intertemporal}
			\centering
			\begin{tikzpicture}[scale=.45]
				\draw [dashed,help lines] (0,4.5)--(5,4.5)--(5,0);
				\draw [ultra thick,teal] plot [domain=0:9.477611940] (\x,{9.525-1.005*\x});
				\draw [ultra thick,teal] plot [domain=0:8] (\x,{12-1.5*\x});
				\draw [<->,thick] (0,13)--(0,0)--(13,0);
				\node [left] at (0,13) {$c_\peq{1}$};
				\node [below] at (13,0) {$c_\peq{0}$};
				\draw [fill,blue] (5,4.5) circle [radius=.2];
				\node [above right,blue] at (5,4.5) {\textbf{D}};
				\node [left] at (0,12) {\tiny $\rp{1+r_\peq{B}}m_\peq{0}+m_\peq{1}$};
				\node [below] at (7.5,0) {\tiny $m_\peq{0}+\frac{m_\peq{1}}{\rp{1+r_\peq{B}}}$};
				\node [left] at (0,9.525) {\tiny $\rp{1+r_\peq{A}}m_\peq{0}+m_\peq{1}$};
				\node [below] at (10.5,0) {\tiny $m_\peq{0}+\frac{m_\peq{1}}{\rp{1+r_\peq{A}}}$};
				\node [left] at (0,4.5) {\tiny $m_\peq{1}$};
				\node [below] at (5,0) {\tiny $m_\peq{0}$};
				\node [below] at (0,0) {\tiny 0};
				\node [left] at (-4,13) {$\Delta^\peq{+}r:$};
			\end{tikzpicture}
		\end{frame}	

		\begin{frame}
			\frametitle{Aplicación 3: Consumo Intertemporal}
			\centering
			\begin{tikzpicture}[scale=.7]
				\draw [dashed,help lines] (0+3,4.5)--(5,4.5)--(5,0);
				\draw [dashed,help lines] (0+3,3.272727273)--(5.818181818,3.272727273)--(5.818181818,0);
				\draw [dashed,help lines] (0+3,1.911608392)--(7.570538914,1.911608392)--(7.570538914,0);
				\draw [ultra thick,teal] plot [domain=0+3:9.477611940] (\x,{9.525-1.005*\x});
				\draw [ultra thick,teal] plot [domain=0+3:8] (\x,{12-1.5*\x});
				\draw [ultra thick,purple] plot [domain=4.5:9] (\x,{4*\x-26.53299833*\x^(1/2)+44.00000002});
				\draw [ultra thick,purple] plot [domain=5:9] (\x,{4*\x-27.54212694*\x^(1/2)+47.41054729});
				\draw [<->,thick] (0+3,9)--(0+3,0)--(13,0);
				\node [left] at (0+3,9) {$c_\peq{1}$};
				\node [left] at (-2+3,9) {$\Delta^\peq{+}r$};
				\node [left] at (-2+3,8.5) {Deudor};
				\node [below] at (13,0) {$c_\peq{0}$};
				\draw [fill,blue] (5,4.5) circle [radius=.12];
				\node [below left,blue] at (5,4.5) {\textbf{D}};
				\draw [fill,blue] (5.818181818,3.272727273) circle [radius=.12];
				\node [below left,blue] at (5.818181818,3.272727273) {\textbf{B}};
				\draw [fill,blue] (7.570538914,1.911608392) circle [radius=.12];
				\node [above right,blue] at (7.570538914,1.911608392) {\textbf{A}};
				\node [left] at (0+3,7.5) {\tiny $\rp{1+r_\peq{B}}w_\peq{B}$};
				\node [below] at (8.4,0) {\tiny $w_\peq{B}$};
				\node [left] at (0+3,6.51) {\tiny $\rp{1+r_\peq{A}}w_\peq{A}$};
				\node [below] at (9.477611940,0) {\tiny $w_\peq{A}$};
				\node [left] at (0+3,3.272727273) {\tiny $c^\peq{*}_\peq{1B}$};
				\node [below] at (5.818181818,0) {\tiny $c^\peq{*}_\peq{0B}$};
				\node [left] at (0+3,1.911608392) {\tiny $c^\peq{*}_\peq{1A}$};
				\node [below] at (7.570538914,0) {\tiny $c^\peq{*}_\peq{0A}$};
				\node [left] at (0+3,4.5) {\tiny $m_\peq{1}$};
				\node [below] at (5,0) {\tiny $m_\peq{0}$};
				\node [below] at (0+3,0) {\tiny 0};
			\end{tikzpicture}
		\end{frame}	

		\begin{frame}
			\frametitle{Aplicación 3: Consumo Intertemporal}
			\centering
			\begin{tikzpicture}[scale=.9]
				\draw [dashed,help lines] (0+2,4.5)--(5,4.5)--(5,0+3);
				\draw [dashed,help lines] (0+2,7.141995425)--(3.238669716,7.141995425)--(3.238669716,0+3);
				\draw [dashed,help lines] (0+2,4.724031435)--(4.772108025,4.724031435)--(4.772108025,0+3);
				\draw [ultra thick,teal] plot [domain=0+2:6.492537313] (\x,{9.525-1.005*\x});
				\draw [ultra thick,teal] plot [domain=0+2:6] (\x,{12-1.5*\x});
				\draw [ultra thick,purple] plot [domain=2.5:4.5] (\x,{1.020304051*\x-9.071231770*\x^(1/2)+20.16243241});
				\draw [ultra thick,purple] plot [domain=3.5:6.5] (\x,{1.020304051*\x-8.848616233*\x^(1/2)+19.18496971});
				\draw [<->,thick] (0+2,10)--(0+2,0+3)--(8,0+3);
				\node [left] at (0+2,10) {$c_\peq{1}$};
				\node [below] at (8,0+3) {$c_\peq{0}$};
				\draw [fill,blue] (5,4.5) circle [radius=.1];
				\node [above right,blue] at (5.1,4.5) {\textbf{D}};
				\draw [fill,blue] (3.238669716,7.141995425) circle [radius=.1];
				\node [above right,blue] at (3.238669716,7.141995425) {\textbf{B}};
				\draw [fill,blue] (4.772108025,4.724031435) circle [radius=.1];
				\node [above right,blue] at (4.772108025,4.724031435) {\textbf{A}};
				\node [left] at (0+2,9) {\tiny $\rp{1+r_\peq{B}}w_\peq{B}$};
				\node [below] at (6,0+3) {\tiny $w_\peq{B}$};
				\node [left] at (0+2,7.515) {\tiny $\rp{1+r_\peq{A}}w_\peq{A}$};
				\node [below] at (6.492537313,0+3) {\tiny $w_\peq{A}$};
				\node [left] at (0+2,4.5-.125) {\tiny $m_\peq{1}$};
				\node [left] at (0+2,7.141995425) {\tiny $c^\peq{*}_\peq{1B}$};
				\node [left] at (0+2,4.724031435+.125) {\tiny $c^\peq{*}_\peq{1A}$};
				\node [below] at (5.125,0+3) {\tiny $m_\peq{0}$};
				\node [below] at (3.238669716,0+3) {\tiny $c^\peq{*}_\peq{0B}$};
				\node [below] at (4.772108025-.125,0+3) {\tiny $c^\peq{*}_\peq{0A}$};
				\node [below] at (0+2,0+3) {\tiny 0};
				\node [left] at (-2+2,10) {$\Delta^\peq{+}r$};
				\node [left] at (-2+2,9.5) {Ahorrador};
			\end{tikzpicture}
		\end{frame}	

		\begin{frame}
			\frametitle{Aplicación 3: Consumo Intertemporal}
			En general:
			\begin{table}[htbp!]
				\centering
				\resizebox{10cm}{!}{
					\begin{tabular}{|l|c c c|c c c|}\hline
												&\multicolumn{3}{c}{Ahorrador}	\vline&\multicolumn{3}{c}{Deudor}\vline \\
												&$\Delta c_\peq{0}$&$\Delta c_\peq{1}$&$\Delta s$&$\Delta c_\peq{0}$&$\Delta c_\peq{1}$&$\Delta s$ \\  \hline 
		 Efecto sustitución &			$-$	 			 &			$+$				&		$+$	 &			$-$	 			 &			$+$				&		$+$	  \\ 
		 Efecto ingreso 		&			$+$	 			 &			$+$				&		$-$	 &			$-$	 			 &			$-$				&		$+$	  \\  \hline 
		 Efecto neto 				&			$?$	 			 &			$+$				&		$?$	 &			$-$	 			 &			$?$				&		$+$	  \\  \hline
					\end{tabular}}
			\end{table}
		\end{frame}	

	\section{La Curva de Demanda}

		\begin{frame}
			\frametitle{Determinantes de la Demanda}
				\begin{itemize}
					\item \textbf{Ingreso del consumidor:} 
						\begin{itemize}
							\item Bien normal: $\Delta^\peq{+}m\implies\Delta^\peq{+}q^\peq{d}$ 
							\item Bien neutro: $\Delta m\centernot\implies\Delta q^\peq{d}$
							\item Bien inferior: $\Delta^\peq{+}m\implies\Delta^\peq{-}q^\peq{d}$ 
						\end{itemize}
					\item \textbf{Precios de bienes relacionados:}
						\begin{itemize}
							\item Bienes sustitutos: $\Delta^\peq{+}p_\peq{y}\implies\Delta^\peq{+}q^\peq{d}_\peq{x}$ 
							\item Bienes complementarios: $\Delta^\peq{+}p_\peq{z}\implies\Delta^\peq{-}q^\peq{d}_\peq{x}$
						\end{itemize}
				\end{itemize}
		\end{frame}	

		\begin{frame}
			\frametitle{Determinantes de la Demanda}
				\begin{itemize}
					\item \textbf{Preferencias:} Las suponemos estables. Pero si aumenta el gusto por $X$, es de esperar que aumente $q^\peq{d}_\peq{x}$ para cada $p_\peq{x}$.
					\item \textbf{Expectativas:} Cambios contemporáneos en alguno de los determinantes pueden alterar la demanda futura.
					\item \textbf{Cantidad de compradores:} Si hay más compradores, aumenta la demanda de mercado.
				\end{itemize}
		\end{frame}	
		
		\begin{frame}
			\frametitle{Demanda Individual y Demanda de Mercado}
			\begin{mydef}
				La \textbf{demanda de mercado} es la suma (horizontal) de las demandas individuales $$Q^\peq{d}\rp{p}=\sum_\peq{i=1}^{n}{q_\peq{i}^\peq{d}\rp{p}}$$
			\end{mydef}
		\end{frame}	
		
		\begin{frame}
			\frametitle{Demanda Individual y Demanda de Mercado}
			Ejemplo 1: 2 consumidores con demandas idénticas $$q^\peq{d}_\peq{i}=6-2p$$
			\begin{table}[htbp!]
				\centering
				\resizebox{4.5cm}{!}{
					\begin{tabular}{|l|c|c|c|}\hline
						$p$&$q^\peq{d}_\peq{A}$&$q^\peq{d}_\peq{B}$&$Q^\peq{d}$\\ [1ex] \hline 
						0	 &				6	 				 &				6	 				 &		12		 \\ \hline
					  0,5&				5 				 &				5	 				 &		10		 \\ \hline
						1	 &				4					 &				4	 				 &		8			 \\ \hline
					  1,5&				3					 &				3	 				 &		6			 \\ \hline
						2	 &				2					 &				2	 				 &		4			 \\ \hline
					  2,5&				1					 &				1	 				 &		2			 \\ \hline
						3	 &				0					 &				0	 				 &		0 		 \\ \hline
					\end{tabular}}
			\end{table}
		\end{frame}

		\begin{frame}
			\frametitle{Demanda Individual y Demanda de Mercado}
			\begin{figure}[htbp!]
				\centering
				\begin{subfigure}[b]{0.32\textwidth}
					%\resizebox{\linewidth}{!}{
						\begin{tikzpicture}[scale=.2]
							\draw [<->,thick] (12,0)--(0,0)--(0,12);
							\draw [red,ultra thick] (0,11)--(0,3);
							\draw [red,ultra thick] plot [domain=0:6](\x,{3-.5*\x});
							\node [left] at (0,12) {$p$};
							\node [below] at (12,0) {$q$};
							\node [below] at (6,0) {\tiny 6};
							\node [left] at (0,3) {\tiny 3};
							\node [above right,red] at (6,0) {$D_\peq{A}$};
						\end{tikzpicture}
					%}
				\end{subfigure}
				\begin{subfigure}[b]{0.32\textwidth}
					%\resizebox{\linewidth}{!}{
						\begin{tikzpicture}[scale=.2]
							\draw [<->,thick] (12,0)--(0,0)--(0,12);
							\draw [red,ultra thick] (0,11)--(0,3);
							\draw [red,ultra thick] plot [domain=0:6](\x,{3-.5*\x});
							\node [left] at (0,12) {$p$};
							\node [below] at (12,0) {$q$};
							\node [below] at (6,0) {\tiny 6};
							\node [left] at (0,3) {\tiny 3};
							\node [above right,red] at (6,0) {$D_\peq{B}$};
						\end{tikzpicture}
					%}
				\end{subfigure}
				\begin{subfigure}[b]{0.32\textwidth}
					%\resizebox{\linewidth}{!}{
						\begin{tikzpicture}[scale=.2]
							\draw [<->,thick] (13,0)--(0,0)--(0,12);
							\draw [red,ultra thick] (0,11)--(0,3);
							\draw [red,ultra thick] plot [domain=0:12](\x,{3-\x/4});
							\node [left] at (0,12) {$p$};
							\node [below] at (13,0) {$q$};
							\node [below] at (12,0) {\tiny 12};
							\node [left] at (0,3) {\tiny 3};
							\node [above right,red] at (12,0) {$D_\peq{A+B}$};
						\end{tikzpicture}
					%}
				\end{subfigure}
			\end{figure}	
		\end{frame}	

		\begin{frame}
			\frametitle{Demanda Individual y Demanda de Mercado}
			Ejemplo 2: 2 consumidores con demandas distintas 
			\begin{align*}
				q_\peq{A}^\peq{d}&=\left\{
					\begin{array}{ll}
						6-2p  & \mbox{si } p \leq 3 \\
						0  & \mbox{si } p > 3 
					\end{array}
				\right. \\
				& \\
			  q_\peq{B}^\peq{d}&=\left\{
					\begin{array}{ll}
						5-p   & \mbox{si } p \leq 5 \\
						0  & \mbox{si } p > 5 
					\end{array}
				\right.
			\end{align*}
		\end{frame}
		
		\begin{frame}
			\frametitle{Demanda Individual y Demanda de Mercado}
			\begin{table}[htbp!]
				\centering
				\resizebox{4.5cm}{!}{
					\begin{tabular}{|l|c|c|c|}\hline
						$p$&$q^\peq{d}_\peq{A}$&$q^\peq{d}_\peq{B}$&$Q^\peq{d}$\\ [1ex] \hline 
						0	 &				6	 				 &				5	 				 &		 11		 \\ \hline
						1	 &				4					 &				4	 				 &			8		 \\ \hline
						2	 &				2					 &				3	 				 &		  5		 \\ \hline
						3	 &			  0					 &			  2	 				 &		  2		 \\ \hline
						4	 &			  0					 &			  1	 				 &		  1		 \\ \hline
						5	 &			  0					 &			  0	 				 &		  0		 \\ \hline
						6	 &			  0					 &			  0	 				 &		  0		 \\ \hline
					\end{tabular}}
			\end{table}
		\end{frame}

		\begin{frame}
			\frametitle{Demanda Individual y Demanda de Mercado}
			La demanda de mercado está descrita por la función
			\begin{equation*}
				Q^\peq{d}=\left\{
					\begin{array}{ll}
						0  		& \mbox{si } p > 5 \\
						5-p 	& \mbox{si } 3 < p \leq 5 \\
						11-3p	& \mbox{si } p \leq 3
					\end{array}
				\right.
			\end{equation*}
		\end{frame}

		\begin{frame}
			\frametitle{Demanda Individual y Demanda de Mercado}
			\begin{figure}[htbp!]
				\centering
				\begin{subfigure}[b]{0.32\textwidth}
					%\resizebox{\linewidth}{!}{
						\begin{tikzpicture}[scale=.2]
							\draw [<->,thick] (12,0)--(0,0)--(0,12);
							\draw [red,ultra thick] (0,11)--(0,3);
							\draw [red,ultra thick] plot [domain=0:6](\x,{3-.5*\x});
							\node [left] at (0,12) {$p$};
							\node [below] at (12,0) {$q$};
							\node [below] at (6,0) {\tiny 6};
							\node [left] at (0,3) {\tiny 3};
							\node [above right,red] at (6,0) {$D_\peq{A}$};
						\end{tikzpicture}
					%}
				\end{subfigure}
				\begin{subfigure}[b]{0.32\textwidth}
					%\resizebox{\linewidth}{!}{
						\begin{tikzpicture}[scale=.2]
							\draw [<->,thick] (12,0)--(0,0)--(0,12);
							\draw [red,ultra thick] (0,11)--(0,5);
							\draw [red,ultra thick] plot [domain=0:5](\x,{5-\x});
							\node [left] at (0,12) {$p$};
							\node [below] at (12,0) {$q$};
							\node [below] at (5,0) {\tiny 5};
							\node [left] at (0,5) {\tiny 5};
							\node [above right,red] at (5,0) {$D_\peq{B}$};
						\end{tikzpicture}
					%}
				\end{subfigure}
				\begin{subfigure}[b]{0.32\textwidth}
					%\resizebox{\linewidth}{!}{
						\begin{tikzpicture}[scale=.2]
							\draw [dashed] (0,3)--(2,3)--(2,0);
							\draw [<->,thick] (12,0)--(0,0)--(0,12);
							\draw [red,ultra thick] (0,11)--(0,5);
							\draw [red,ultra thick] plot [domain=0:2](\x,{5-\x});
							\draw [red,ultra thick] plot [domain=2:11](\x,{(11-\x)/3});
							\node [left] at (0,12) {$p$};
							\node [below] at (12,0) {$q$};
							\node [below] at (2,0) {\tiny 2};
							\node [below] at (11,0) {\tiny 11};
							\node [left] at (0,3) {\tiny 3};
							\node [left] at (0,5) {\tiny 5};
							\node [above right,red] at (11,0) {$D_\peq{A+B}$};
						\end{tikzpicture}
					%}
				\end{subfigure}
			\end{figure}	
		\end{frame}	

		\begin{frame}
			\frametitle{Elasticidad de la Demanda}
			\begin{mydef}
				\textbf{Elasticidad precio de la demanda:} Una medida de cuánto responde $q^\peq{d}$ respecto al cambio en $p$. $$\eta_\peq{q,p}\equiv\frac{\Delta\%q^\peq{d}}{\Delta\%p}$$
			\end{mydef}
		\end{frame}

		\begin{frame}
			\frametitle{Elasticidad de la Demanda}
			Notar que
			\begin{align*}
				\frac{\Delta\%q^\peq{d}}{\Delta\%p}&=\frac{\Delta q^\peq{d}/q^\peq{d}}{\Delta p/p} \\
																					 &=\frac{1}{\Delta p / \Delta q^\peq{d}}\cdot\frac{p}{q^\peq{d}}								
			\end{align*}
		\end{frame}

		\begin{frame}
			\frametitle{Elasticidad de la Demanda}
			$\text{Ley de la demanda }\implies\eta_\peq{q,p}\leq0$
			\begin{itemize}
				\item $\left|\eta_\peq{q,p}\right|>1\implies$ demanda elástica $\rp{\left|\Delta\%q^\peq{d}\right|>\left|\Delta\%p\right|}$
				\item $\left|\eta_\peq{q,p}\right|=1\implies$ elasticidad unitaria $\rp{\left|\Delta\%q^\peq{d}\right|=\left|\Delta\%p\right|}$
				\item $\left|\eta_\peq{q,p}\right|<1\implies$ demanda inelástica $\rp{\left|\Delta\%q^\peq{d}\right|<\left|\Delta\%p\right|}$
			\end{itemize}
		\end{frame}

		\begin{frame}
			\frametitle{Elasticidad de la Demanda}
			Si calculamos el cambio porcentual como $$\Delta\%x=\frac{x_\peq{final}-x_\peq{inicial}}{x_\peq{inicial}}$$ obtendremos resultados distintos dependiendo del punto de partida. Por eso usamos el método del punto medio $$\Delta\%x=\frac{x_\peq{final}-x_\peq{inicial}}{\rp{\frac{x_\peq{inicial}+x_\peq{final}}{2}}}$$
		\end{frame}

		\begin{frame}
			\frametitle{Elasticidad de la Demanda}
			Determinantes de la elasticidad precio de la demanda:
			\begin{itemize}
				\item Disponibilidad de sustitutos cercanos (mayor sustitución $\implies$ más elasticidad)
				\item Necesidades frente a lujos (menos elasticidad en caso de necesidades)
				\item Definición del mercado (más estrecha $\implies$ más elasticidad)
				\item Horizonte temporal (más amplio $\implies$ más elasticidad)
			\end{itemize}
		\end{frame}

		\begin{frame}
			\frametitle{Elasticidad de la Demanda}
			Ejemplos: (a) Demanda perfectamente inelástica
						\begin{figure}[htbp!]
							\centering
								\begin{tikzpicture}[scale=.9]
									\draw [dashed] (0,1)--(100/30,1);
									\draw [dashed] (0,5)--(100/30,5);
									\draw [<->,thick] (0,6) -- (0,0) -- (180/30,0);
									\draw [teal,ultra thick] (100/30,0)--(100/30,5.9);
									\node [below] at (180/30,0) {$q$};
									\node [left] at (0,6) {$p$};
									\node [teal,right] at (100/30,5.9) {$D$};
									\draw [fill,blue] (100/30,1) circle [radius=.08];
									\node [above right,blue] at (100/30,1) {\small $A$};
									\draw [fill,blue] (100/30,5) circle [radius=.08];
									\node [above right,blue] at (100/30,5) {\small $B$};
									\node [below] at (100/30,0) {\tiny 100};
									\node [left] at (0,1) {\tiny 4};
									\node [left] at (0,5) {\tiny 5};
									\node [right] at (180/30,5.5) {$\Delta\%p=22,\overline{2}\%$};
									\node [right] at (180/30,4.5) {$\Delta\%q^\peq{d}=0\%$};
									\node [right] at (180/30,3.5) {$\eta_\peq{q,p}=0$};
								\end{tikzpicture}
						\end{figure}	
		\end{frame}

		\begin{frame}
			\frametitle{Elasticidad de la Demanda}
			(b) demanda inelástica
						\begin{figure}[htbp!]
							\centering
								\begin{tikzpicture}[scale=.9]
									\draw [<->,thick] (0,6) -- (0,0) -- (6,0);
									\draw [dashed] (0,1)--(3,1)--(3,0);
									\draw [dashed] (0,5)--(2.447213595,5)--(2.447213595,0);
									\draw [teal,ultra thick] plot [domain=(2+.4082482906):5] (\x,{1/(\x-2)^2});
									\node [below] at (210/30,0) {$q$};
									\node [left] at (0,6) {$p$};
									\node [teal,above right] at (5,1/25) {$D$};
									\draw [fill,blue] (3,1) circle [radius=.08];
									\node [above right,blue] at (3,1) {\small $A$};
									\draw [fill,blue] (2.447213595,5) circle [radius=.08];
									\node [above right,blue] at (2.447213595,5) {\small $B$};
									\node [below] at (2.447213595,0) {\tiny 90};
									\node [below] at (3,0) {\tiny 100};
									\node [left] at (0,1) {\tiny 4};
									\node [left] at (0,5) {\tiny 5};
									\node [right] at (180/30,5.5) {$\Delta\%p=22,\overline{2}\%$};
									\node [right] at (180/30,4.5) {$\Delta\%q^\peq{d}=-10.52\%$};
									\node [right] at (180/30,3.5) {$\eta_\peq{q,p}=-0.47$};
								\end{tikzpicture}
						\end{figure}	
		\end{frame}

		\begin{frame}
			\frametitle{Elasticidad de la Demanda}
			(c) demanda elástica unitaria
						\begin{figure}[htbp!]
							\centering
								\begin{tikzpicture}[scale=.9]
									\draw [dashed] (0,1)--(3,1)--(3,0);
									\draw [dashed] (0,5)--(2.2,5)--(2.2,0);
									\draw [<->,thick] (0,6) -- (0,0) -- (180/30,0);
									\draw [teal,ultra thick] plot [domain=(2+1/6):5] (\x,{1/(\x-2)});
									\node [below] at (180/30,0) {$q$};
									\node [left] at (0,6) {$p$};
									\node [teal,right] at (5,1/3) {$D$};
									\draw [fill,blue] (3,1) circle [radius=.08];
									\node [above right,blue] at (3,1) {\small $A$};
									\draw [fill,blue] (2.2,5) circle [radius=.08];
									\node [above right,blue] at (2.2,5) {\small $B$};
									\node [below] at (2.2,0) {\tiny 80};
									\node [below] at (3,0) {\tiny 100};
									\node [left] at (0,1) {\tiny 4};
									\node [left] at (0,5) {\tiny 5};
									\node [right] at (180/30,5.5) {$\Delta\%p=22,\overline{2}\%$};
									\node [right] at (180/30,4.5) {$\Delta\%q^\peq{d}=-22,\overline{2}\%$};
									\node [right] at (180/30,3.5) {$\eta_\peq{q,p}=-1$};
								\end{tikzpicture}
						\end{figure}	
		\end{frame}

		\begin{frame}
			\frametitle{Elasticidad de la Demanda}
			(d) demanda elástica
						\begin{figure}[htbp!]
							\centering
								\begin{tikzpicture}[scale=.9]
									\draw [dashed] (0,1)--(3,1)--(3,0);
									\draw [dashed] (0,5)--(2.04,5)--(2.04,0);
									\draw [<->,thick] (0,6) -- (0,0) -- (6,0);
									\draw [teal,ultra thick] plot [domain=(2+1/36):5] (\x,{1/sqrt(\x-2)});
									\node [below] at (180/30,0) {$q$};
									\node [left] at (0,6) {$p$};
									\node [teal,right] at (5,.4472135954) {$D$};
									\draw [fill,blue] (3,1) circle [radius=.08];
									\node [above right,blue] at (3,1) {\small $A$};
									\draw [fill,blue] (2.04,5) circle [radius=.08];
									\node [above right,blue] at (2.04,5) {\small $B$};
									\node [below] at (2.04,0) {\tiny 50};
									\node [below] at (3,0) {\tiny 100};
									\node [left] at (0,1) {\tiny 4};
									\node [left] at (0,5) {\tiny 5};
									\node [right] at (180/30,5.5) {$\Delta\%p=22,\overline{2}\%$};
									\node [right] at (180/30,4.5) {$\Delta\%q^\peq{d}=-66,\overline{6}\%$};
									\node [right] at (180/30,3.5) {$\eta_\peq{q,p}=-3$};
								\end{tikzpicture}
						\end{figure}	
		\end{frame}

		\begin{frame}
			\frametitle{Elasticidad de la Demanda}
			(e) demanda perfectamente elástica
						\begin{figure}[htbp!]
							\centering
								\begin{tikzpicture}[scale=.9]
									\draw [<->,thick] (0,6) -- (0,0) -- (180/30,0);
									\draw [teal,ultra thick] (0,4)--(180/30,4);
									\node [below] at (180/30,0) {$q$};
									\node [left] at (0,6) {$p$};
									\node [teal,right] at (180/30,4) {$D$};
									\node [left] at (0,4) {\tiny 5};
									\node [right] at (200/30,5.5) {Si $p>5$, $q^\peq{d}=0$};
									\node [right] at (200/30,4) {Si $p=5$, $q^\peq{d}$ indefinida};
									\node [right] at (200/30,2.5) {Si $p<5$, $q^\peq{d}\to\infty$};
								\end{tikzpicture}
						\end{figure}	
		\end{frame}

		\begin{frame}
			\frametitle{Elasticidad de la Demanda}
			Ejemplo: demanda lineal $$p=a-b\cdot q^\peq{d}$$ con $a,b>0$. 
			
			\vspace{.3cm}
			¿Qué podemos decir sobre $\eta_\peq{q,p}$?
			\begin{itemize}
				\item $q<\frac{a}{2b}\implies\left|\eta_\peq{q,p}\right|>1$
				\item $q=\frac{a}{2b}\implies\left|\eta_\peq{q,p}\right|=1$
				\item $q>\frac{a}{2b}\implies\left|\eta_\peq{q,p}\right|<1$
			\end{itemize}
		\end{frame}
		
	\begin{frame}
			\frametitle{Elasticidad de la Demanda}
						\begin{figure}[htbp!]
							\centering
								\begin{tikzpicture}[scale=.9]
									\draw [dashed] (0,2.5)--(2.5,2.5)--(2.5,0);
									\draw [<->,thick] (0,6) -- (0,0) -- (180/30,0);
									\draw [teal,ultra thick] plot [domain=0:5] (\x,{5-\x});
									\node [below] at (6,0) {$q$};
									\node [left] at (0,6) {$p$};
									\node [teal,above right] at (5,0) {$D$};
									\draw [fill,blue] (2.5,2.5) circle [radius=.1];
									\node [above right,blue] at (2.5,2.5) {\small elasticidad unitaria};
									\node [below] at (2.5,0) {\tiny $\frac{a}{2b}$};
									\node [below] at (5,0) {\tiny $\frac{a}{b}$};
									\node [left] at (0,2.5) {\tiny $\frac{a}{2}$};
									\node [left] at (0,5) {\tiny $a$};
									\draw [decorate,decoration={brace,amplitude=10pt},xshift=0.2pt,yshift=-0.2pt,color=blue, thick](0,5) -- (2.5,2.5) node[above right,midway,blue,yshift=6pt,xshift=6pt] {\small \pbox{\textwidth}{demanda \\ elástica}};
									\draw [decorate,decoration={brace,amplitude=10pt},xshift=0.2pt,yshift=-0.2pt,color=blue, thick](2.5,2.5) -- (5,0) node[above right,midway,blue,yshift=6pt,xshift=6pt] {\small \pbox{\textwidth}{demanda \\ inelástica}};
								\end{tikzpicture}
						\end{figure}	
		\end{frame}		

	\begin{frame}
			\frametitle{Elasticidad de la Demanda}
						\begin{figure}[htbp!]
							\centering
								\begin{tikzpicture}[scale=.9]
									\draw [dashed] (0,6.25)--(2.5,6.25)--(2.5,0);
									\draw [<->,thick] (0,7) -- (0,0) -- (180/30,0);
									\draw [teal,ultra thick] plot [domain=0:5] (\x,{(5-\x)*\x});
									\node [below] at (6,0) {$q$};
									\node [left] at (0,7) {$p\cdot q$};
									\node [teal,above right] at (5,0) {$IT$};
									\draw [fill,blue] (2.5,6.25) circle [radius=.1];
									\node [below] at (2.5,0) {\tiny $\frac{a}{2b}$};
									\node [left] at (0,6.25) {\tiny $\frac{a^2}{4b}$};
								\end{tikzpicture}
						\end{figure}	
		\end{frame}		

		\begin{frame}
			\frametitle{Elasticidad de la Demanda}
			\begin{mydef}
				\textbf{Elasticidad ingreso de la demanda:} Una medida de cuánto responde $q^\peq{d}$ respecto al cambio en $m$. $$\eta_\peq{q,m}\equiv\frac{\Delta\%q^\peq{d}}{\Delta\%m}$$
			\end{mydef}
		\end{frame}

\begin{frame}
			\frametitle{Elasticidad de la Demanda}
			\begin{itemize}
				\item $\eta_\peq{q,m}>0\implies$ bien normal
					\begin{itemize}
						\item $\eta_\peq{q,m}>1\implies$ ``bien de lujo''
						\item $\eta_\peq{q,m}\in\rp{0,1}\implies$ ``necesidad''
					\end{itemize}
				\item $\eta_\peq{q,m}=0\implies$ bien neutro
				\item $\eta_\peq{q,m}<0\implies$ bien inferior
			\end{itemize}
		\end{frame}
		
		\begin{frame}
			\frametitle{Elasticidad de la Demanda}
			Ejemplo: Pizza
			\begin{itemize}
				\item $m_\peq{0}=975$, $m_\peq{1}=1025$
				\item $q^\peq{d}_\peq{0}=9$, $q^\peq{d}_\peq{1}=11$
				\item $\eta_\peq{q,m}=4\implies$ pizza es un bien normal y elástico al ingreso
			\end{itemize}
		\end{frame}		

		\begin{frame}
			\frametitle{Elasticidad de la Demanda}
			\begin{mydef}
				\textbf{Elasticidad precio-cruzada de la demanda:} Una medida de cuánto responde $q^\peq{d}_\peq{x}$ respecto al cambio en $q_\peq{y}$. $$\eta_\peq{q_\peq{x},p_\peq{y}}\equiv\frac{\Delta\%q^\peq{d}_\peq{x}}{\Delta\%p_\peq{y}}$$
			\end{mydef}
		\end{frame}

\begin{frame}
			\frametitle{Elasticidad de la Demanda}
			\begin{itemize}
				\item $\eta_\peq{q,m}>0\implies$ bienes sustitutos
				\item $\eta_\peq{q,m}=0\implies$ bien no relacionados
				\item $\eta_\peq{q,m}<0\implies$ bienes complementarios
			\end{itemize}
		\end{frame}
		
		\begin{frame}
			\frametitle{Elasticidad de la Demanda}
			Ejemplo 1: Pizza y bebida
			\begin{itemize}
				\item $p_\peq{y,0}=1.5$, $p_\peq{y,1}=2.5$
				\item $q^\peq{d}_\peq{x,0}=11$, $q^\peq{d}_\peq{x,1}=9$
				\item $\eta_\peq{q_\peq{x},p_\peq{y}}=-0.4\implies$ pizza y bebida son complementos
			\end{itemize}
		\end{frame}		
		
		\begin{frame}
			\frametitle{Elasticidad de la Demanda}
			Ejemplo 2: Pizza y hamburguesa
			\begin{itemize}
				\item $p_\peq{y,0}=1.5$, $p_\peq{y,1}=2.5$
				\item $q^\peq{d}_\peq{x,0}=9$, $q^\peq{d}_\peq{x,1}=11$
				\item $\eta_\peq{q_\peq{x},p_\peq{y}}=0.4\implies$ pizza y hamburguesa son sustitutos
			\end{itemize}
		\end{frame}		

	\section{Análisis de Bienestar}

		\begin{frame}
			\frametitle{Excedente del Consumidor}
			\begin{mydef}
				\textbf{Excedente del consumidor ($\mathbf{EC}$):} Diferencia entre la dispocisión del comprador y lo que efectivamente paga.
			\end{mydef}
		\end{frame}	
	
		\begin{frame}
			\frametitle{Excedente del Consumidor}
			Ejemplo:
			\begin{itemize}
				\item Sea $v_\peq{i}$ la valoración que asigna el individuo $i$ al consumo de una unidad del bien $x$.
				\item Cada individuo compra a lo más una unidad $\implies$ la decisión es comprar o no.
				\item Sea $p$ el precio al que se puede comprar una unidad.
			\end{itemize}
		\end{frame}	
			
		\begin{frame}
			\frametitle{Excedente del Consumidor}
			Decisión racional: $$\text{Si}\left\{
					\begin{array}{ll}
						p>v_\peq{i} & \text{no compra} \\
						p=v_\peq{i} & \text{está indiferente} \\
						p<v_\peq{i}	& \text{compra}
					\end{array}
				\right.$$
			\textit{Nota:}  supondremos que, en caso de indiferencia, compra. Este supuesto es irrelevante para nuestras conclusiones.
		\end{frame}	
		
		\begin{frame}
			\frametitle{Excedente del Consumidor}
			Supongamos que hay 4 compradores:
			\begin{table}[htbp!]
				\centering
				\resizebox{4cm}{!}{
					\begin{tabular}{|c|c|}\hline
						$i$&$v_\peq{i}$ \\  \hline 
						 A & \$100/u 		\\ 
						 B & \$80/u 		\\ 
						 C & \$70/u 		\\ 
						 D & \$50/u 		\\ \hline
					\end{tabular}}
			\end{table}
		\end{frame}	

		\begin{frame}
			\frametitle{Excedente del Consumidor}
			Caso 1: Se subasta una unidad del bien $x$ (se venderá al comprador que ofrezca el mayor precio).
			\begin{itemize}
				\item Precio sube rápidamente hasta que $A$ ofrece $\$\rp{80+\delta}$ con $\delta>0$ y pequeño. (¿Por qué no hasta $v_\peq{A}=\$100$?)
				\item $A$ compra una unidad de $x$ a $\$80$ y estaba dispuesto a pagar $\$100\implies$ su excedente es $EC_\peq{A}^\peq{caso 1}=\$20/u\cdot1u=\$20$.
			\end{itemize}
			\textit{Nota:} pensemos en $\delta$ muy pequeño, es decir $\delta\rightarrow0$.
		\end{frame}	

		\begin{frame}
			\frametitle{Excedente del Consumidor}
			Caso 2: Se subastan dos unidades, ambas al mismo precio (pero a distintos compradores).
			\begin{itemize}
				\item Subasta termina cuando $A$ y/o $B$ ofrece $\$\rp{70+\delta}$ con $\delta>0$ y pequeño. 
				\item $EC_\peq{A}^\peq{caso 2}=\rp{\$100/u-\$70/u}\cdot1u=\$30$.
				\item $EC_\peq{B}^\peq{caso 2}=\rp{\$80/u-\$70/u}\cdot1u=\$10$.
				\item $EC_\peq{Total}^\peq{caso 2}=\$30+\$10=\$40$.
			\end{itemize}
			\textit{Nota:} Estamos pensando de nuevo en $\delta\rightarrow0$.
		\end{frame}	

		\begin{frame}
			\frametitle{Excedente del Consumidor}
			Caso general: Nos abstraemos de la subasta, dejando que el precio varíe libremente. De esta manera, obtenemos la demanda del mercado.
			\begin{table}[htbp!]
				\centering
				\resizebox{8cm}{!}{
					\begin{tabular}{|c|c|c|}\hline
						$p$							&Vendedores		 &$Q^\peq{d}$ \\  \hline 
						$100< p<\infty$ & $\emptyset$  &	$0$ \\ 
						$80 < p\leq100$ & $\{A\}$ 		 &	$1$ \\ 
						$70 < p\leq80$ 	& $\{A,B\}$ 	 &	$2$ \\ 
						$50 < p\leq70$ 	& $\{A,B,C\}$  &	$3$ \\
						$0\leq p\leq50$ & $\{A,B,C,D\}$&	$4$ \\ \hline
					\end{tabular}}
			\end{table}
		\end{frame}	

		\begin{frame}
			\frametitle{Excedente del Consumidor}
						\begin{figure}[htbp!]
							\centering
								\begin{tikzpicture}[scale=.54]
									\draw [dashed,help lines] (2,0)--(2,10);
									\draw [dashed,help lines] (4,0)--(4,8);
									\draw [dashed,help lines] (0,8)--(2,8);
									\draw [dashed,help lines] (6,0)--(6,7);
									\draw [dashed,help lines] (4,7)--(0,7);
									\draw [dashed,help lines] (8,0)--(8,5);
									\draw [dashed,help lines] (0,5)--(6,5);
									\draw [<->,thick] (0,11.5) -- (0,0) -- (10,0);
									\draw [red,ultra thick] (0,11)--(0,10);
									\draw [red,ultra thick,dashed] (0,10)--(2,10);
									\draw [red,ultra thick] (2,10)--(2,8);
									\draw [red,ultra thick,dashed] (2,8)--(4,8);
									\draw [red,ultra thick] (4,8)--(4,7);
									\draw [red,ultra thick,dashed] (4,7)--(6,7);
									\draw [red,ultra thick] (6,7)--(6,5);
									\draw [red,ultra thick,dashed] (6,5)--(8,5);
									\draw [red,ultra thick] (8,5)--(8,0);
									\node [below] at (10,0) {$Q$};
									\node [left] at (0,11.5) {$p$};
									\node [red,above right] at (8,0) {$D$};
									\node [left] at (0,10) {\tiny 100};
									\node [left] at (0,8) {\tiny 80};
									\node [left] at (0,7) {\tiny 70};
									\node [left] at (0,5) {\tiny 50};
									\node [below] at (2,0) {\tiny 1};
									\node [below] at (4,0) {\tiny 2};
									\node [below] at (6,0) {\tiny 3};
									\node [below] at (8,0) {\tiny 4};
								\end{tikzpicture}
						\end{figure}	
		\end{frame}

		\begin{frame}
			\frametitle{Excedente del Consumidor}
						\begin{figure}[htbp!]
							\centering
								\begin{tikzpicture}[scale=.54]
									\draw [fill=green,opacity=0.2] (0,10) -- (2,10) -- (2,8) -- (0,8);
									\draw [dashed,help lines] (2,0)--(2,10);
									\draw [dashed,help lines] (4,0)--(4,8);
									\draw [dashed,help lines] (0,8)--(2,8);
									\draw [dashed,help lines] (6,0)--(6,7);
									\draw [dashed,help lines] (4,7)--(0,7);
									\draw [dashed,help lines] (8,0)--(8,5);
									\draw [dashed,help lines] (0,5)--(6,5);
									\draw [<->,thick] (0,11.5) -- (0,0) -- (10,0);
									\draw [red,ultra thick] (0,11)--(0,10);
									\draw [red,ultra thick,dashed] (0,10)--(2,10);
									\draw [red,ultra thick] (2,10)--(2,8);
									\draw [red,ultra thick,dashed] (2,8)--(4,8);
									\draw [red,ultra thick] (4,8)--(4,7);
									\draw [red,ultra thick,dashed] (4,7)--(6,7);
									\draw [red,ultra thick] (6,7)--(6,5);
									\draw [red,ultra thick,dashed] (6,5)--(8,5);
									\draw [red,ultra thick] (8,5)--(8,0);
									\node [below] at (10,0) {$Q$};
									\node [left] at (0,11.5) {$p$};
									\node [red,above right] at (8,0) {$D$};
									\node [left] at (0,10) {\tiny 100};
									\node [left] at (0,8) {\tiny 80};
									\node [left] at (0,7) {\tiny 70};
									\node [left] at (0,5) {\tiny 50};
									\node [below] at (2,0) {\tiny 1};
									\node [below] at (4,0) {\tiny 2};
									\node [below] at (6,0) {\tiny 3};
									\node [below] at (8,0) {\tiny 4};
									\node at (1,9) {\tiny $EC^\peq{caso 1}_\peq{A}$};
								\end{tikzpicture}
						\end{figure}	
		\end{frame}

		\begin{frame}
			\frametitle{Excedente del Consumidor}
						\begin{figure}[htbp!]
							\centering
								\begin{tikzpicture}[scale=.54]
									\draw [fill=blue,opacity=0.2] (0,10) -- (2,10) -- (2,7) -- (0,7);
									\draw [fill=purple,opacity=0.2] (2,7) -- (4,7) -- (4,8) -- (2,8);
									\draw [dashed,help lines] (2,0)--(2,10);
									\draw [dashed,help lines] (4,0)--(4,8);
									\draw [dashed,help lines] (0,8)--(2,8);
									\draw [dashed,help lines] (6,0)--(6,7);
									\draw [dashed,help lines] (4,7)--(0,7);
									\draw [dashed,help lines] (8,0)--(8,5);
									\draw [dashed,help lines] (0,5)--(6,5);
									\draw [<->,thick] (0,11.5) -- (0,0) -- (10,0);
									\draw [red,ultra thick] (0,11)--(0,10);
									\draw [red,ultra thick,dashed] (0,10)--(2,10);
									\draw [red,ultra thick] (2,10)--(2,8);
									\draw [red,ultra thick,dashed] (2,8)--(4,8);
									\draw [red,ultra thick] (4,8)--(4,7);
									\draw [red,ultra thick,dashed] (4,7)--(6,7);
									\draw [red,ultra thick] (6,7)--(6,5);
									\draw [red,ultra thick,dashed] (6,5)--(8,5);
									\draw [red,ultra thick] (8,5)--(8,0);
									\node [below] at (10,0) {$Q$};
									\node [left] at (0,11.5) {$p$};
									\node [red,above right] at (8,0) {$D$};
									\node [left] at (0,10) {\tiny 100};
									\node [left] at (0,8) {\tiny 80};
									\node [left] at (0,7) {\tiny 70};
									\node [left] at (0,5) {\tiny 50};
									\node [below] at (2,0) {\tiny 1};
									\node [below] at (4,0) {\tiny 2};
									\node [below] at (6,0) {\tiny 3};
									\node [below] at (8,0) {\tiny 4};
									\node at (1,8.5) {\tiny $EC^\peq{caso 2}_\peq{A}$};
									\node at (3,7.5) {\tiny $EC^\peq{caso 2}_\peq{B}$};
								\end{tikzpicture}
						\end{figure}	
		\end{frame}

		\begin{frame}
			\frametitle{Excedente del Consumidor}
						\begin{figure}[htbp!]
							\centering
								\begin{tikzpicture}[scale=.54]
									\draw [fill=green,opacity=0.2] (0,10) -- (2,10) -- (2,8) -- (0,8);
									\draw [fill=blue,opacity=0.2] (0,8) -- (2,8) -- (2,7) -- (0,7);
									\draw [fill=purple,opacity=0.2] (2,7) -- (4,7) -- (4,8) -- (2,8);
									\draw [dashed,help lines] (2,0)--(2,10);
									\draw [dashed,help lines] (4,0)--(4,8);
									\draw [dashed,help lines] (0,8)--(2,8);
									\draw [dashed,help lines] (6,0)--(6,7);
									\draw [dashed,help lines] (4,7)--(0,7);
									\draw [dashed,help lines] (8,0)--(8,5);
									\draw [dashed,help lines] (0,5)--(6,5);
									\draw [<->,thick] (0,11.5) -- (0,0) -- (10,0);
									\draw [red,ultra thick] (0,11)--(0,10);
									\draw [red,ultra thick,dashed] (0,10)--(2,10);
									\draw [red,ultra thick] (2,10)--(2,8);
									\draw [red,ultra thick,dashed] (2,8)--(4,8);
									\draw [red,ultra thick] (4,8)--(4,7);
									\draw [red,ultra thick,dashed] (4,7)--(6,7);
									\draw [red,ultra thick] (6,7)--(6,5);
									\draw [red,ultra thick,dashed] (6,5)--(8,5);
									\draw [red,ultra thick] (8,5)--(8,0);
									\node [below] at (10,0) {$Q$};
									\node [left] at (0,11.5) {$p$};
									\node [red,above right] at (8,0) {$D$};
									\node [left] at (0,10) {\tiny 100};
									\node [left] at (0,8) {\tiny 80};
									\node [left] at (0,7) {\tiny 70};
									\node [left] at (0,5) {\tiny 50};
									\node [below] at (2,0) {\tiny 1};
									\node [below] at (4,0) {\tiny 2};
									\node [below] at (6,0) {\tiny 3};
									\node [below] at (8,0) {\tiny 4};
									\node at (1,7.5) {\tiny $\Delta EC_\peq{A}$};
									\node at (3,7.5) {\tiny $\Delta EC_\peq{B}$};
									\node at (1,9) {\tiny $EC^\peq{caso 1}_\peq{A}$};
								\end{tikzpicture}
						\end{figure}	
		\end{frame}
		
		\begin{frame}
			\frametitle{Excedente del Consumidor}
			\begin{mydef}
				\textbf{Comprador marginal:} El primer comprador que saldría del mercado si el precio fuera más alto.
			\end{mydef}
		\end{frame}

		\begin{frame}
			\frametitle{Excedente del Consumidor}
			Notar que:
			\begin{itemize}
				\item Para cualquier cantidad, el precio dado por la curva de demanda corresponde a la valoración del comprador marginal.
				\item El excedente del consumidor es el área entre el precio de mercado y la curva de demanda.
			\end{itemize}
		\end{frame}

		\begin{frame}
			\frametitle{Excedente del Consumidor}
			\begin{itemize}
				\item Cuando se reduce el precio de $p_\peq{0}$ a $p_\peq{1}$, $EC$ aumenta por dos motivos:
					\begin{enumerate}
						\item Las unidades ($q_\peq{0}$) que antes se compraban a $p_\peq{0}$ ahora se venden a $p_\peq{1}<p_\peq{0}$.
						\item Se compran más unidades ($q_\peq{1}-q_\peq{0}$) con $p_\peq{1}<\text{disposición a pagar}$.
					\end{enumerate}
			\end{itemize}
		\end{frame}

		\begin{frame}
			\frametitle{Excedente del Consumidor}
			Lo anterior es válido en general para cualquier curva de demanda...
						\begin{figure}[htbp!]
							\centering
								\begin{tikzpicture}[scale=.5]
									\draw [fill=blue,opacity=0.2] (0,7) -- (2,7) -- (0,9);
									\draw [fill=red,opacity=0.2] (0,4) -- (2,4) -- (2,7) -- (0,7);
									\draw [fill=green,opacity=0.2] (2,4) -- (5,4) -- (2,7);
									\draw [teal,ultra thick] plot [domain=0:9] (\x,{9-\x});
									\draw [<->,thick] (0,10) -- (0,0) -- (10,0);
									\draw [red] (5,0)--(5,4)--(0,4);
									\draw [blue] (2,0)--(2,7)--(0,7);
									\node [below] at (10,0) {$q$};
									\node [left] at (0,10) {$p$};
									\node [teal,above right] at (9,0) {$D$};
									\draw [fill,red] (5,4) circle [radius=.2];
									\draw [fill,blue] (2,7) circle [radius=.2];
									\node [left] at (0,4) {$p_\peq{1}$};
									\node [left] at (0,7) {$p_\peq{0}$};
									\node [below] at (5,0) {$q_\peq{1}$};
									\node [below] at (2,0) {$q_\peq{0}$};
									\node [below] at (0,0) {$0$};
									\draw [decorate,decoration={brace,amplitude=6pt,mirror},xshift=0.2pt,yshift=-0.2pt](0,-.8) -- (2,-.8) node[black,midway,yshift=-0.5cm,xshift=-5pt] {\tiny \pbox{\textwidth}{$q_\peq{0}$ unidades \\ más baratas}};
									\draw [decorate,decoration={brace,amplitude=6pt,mirror},xshift=0.2pt,yshift=-0.2pt](2,-.8) -- (5,-.8) node[black,midway,yshift=-0.5cm,xshift=5pt] {\tiny \pbox{\textwidth}{$\rp{q_\peq{1}-q_\peq{0}}$ unidades \\ adicionales}};
								\end{tikzpicture}
						\end{figure}	
		\end{frame}
	
\end{document}